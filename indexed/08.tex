\documentclass[output=paper]{langscibook}
\ChapterDOI{10.5281/zenodo.5792965}
\author{Lars-Olof Delsing\affiliation{Lund University}}
\title[From ‘big’ to ‘much’]{From ‘big’ to ‘much’: On the grammaticalization of two gradable adjectives in Swedish}
\abstract{In this paper, I give a short description of a language change that can be viewed as an instance of grammaticalization, namely the transition of the two adjectives \mbox{\textit{mycken/t}} and \textit{liten/t} into quantifiers. Data from the corpus of Swedish drama dialogue reveal that \textit{liten/t} became a quantifier as early as the 1700s, whereas \textit{mycken/t} seems to have gone through the same change roughly 150 years later. Inherent plurals (such as \textit{pengar}, ‘money’) appear to be a promising context for the starting point of the transition. I further illustrate how both quantifiers have weak and strong versions in present-day Swedish, and I argue that weak \textit{mycket} (\textit{myke}) has turned into a \is{negative polarity item (NPI)}negative polarity item that is found in negated clauses (but not for example in questions and conditionals), whereas weak \textit{lite(t)} has turned into a positive polarity item and is found elsewhere. If we assume that weak versions of quantifiers are more frequent than strong ones, and that positive polarity contexts are more frequent than negative ones, we expect the frequency of \textit{mycken/t} to drop, whereas the frequency of \textit{liten/t} should rise over time. A preliminary study that seems to confirm this prediction is presented here.

\keywords{grammaticalization, quantifiers, adjectives, negative polarity, positive polarity, language change, Swedish}
}
\IfFileExists{../localcommands.tex}{
  \addbibresource{localbibliography.bib}
  % add all extra packages you need to load to this file

\usepackage{tabularx,multicol}
\usepackage{url}
\urlstyle{same}

\usepackage{listings}
\lstset{basicstyle=\ttfamily,tabsize=2,breaklines=true}

\usepackage{langsci-basic}
\usepackage{langsci-optional}
\usepackage{langsci-lgr}
\usepackage{langsci-gb4e}

\usepackage{todonotes}

\usepackage[linguistics]{forest}
\usepackage{soul}
\usepackage{subfigure}
\usepackage{longtable}
\usepackage{enumitem}

  \newcommand*{\orcid}{}
%\newcommand{\keywords}[1]{\textbf{#1}}


\makeatletter
\let\theauthor\@author
\makeatother

\newcommand{\keywords}[1]{\textbf{Keywords:} #1}


\DeclareNewSectionCommand
  [
    counterwithin = chapter,
    afterskip = 2.3ex plus .2ex,
    beforeskip = -3.5ex plus -1ex minus -.2ex,
    indent = 0pt,
    font = \usekomafont{section},
    level = 1,
    tocindent = 1.5em,
    toclevel = 1,
    tocnumwidth = 2.3em,
    tocstyle = section,
    style = section
  ]
  {appendixsection}

\renewcommand*\theappendixsection{\Alph{appendixsection}}
\renewcommand*{\appendixsectionformat}{\appendixname~\theappendixsection\autodot\enskip}
\renewcommand*{\appendixsectionmarkformat}{\appendixname~\theappendixsection\autodot\enskip}
 
  %% hyphenation points for line breaks
%% Normally, automatic hyphenation in LaTeX is very good
%% If a word is mis-hyphenated, add it to this file
%%
%% add information to TeX file before \begin{document} with:
%% %% hyphenation points for line breaks
%% Normally, automatic hyphenation in LaTeX is very good
%% If a word is mis-hyphenated, add it to this file
%%
%% add information to TeX file before \begin{document} with:
%% %% hyphenation points for line breaks
%% Normally, automatic hyphenation in LaTeX is very good
%% If a word is mis-hyphenated, add it to this file
%%
%% add information to TeX file before \begin{document} with:
%% \include{localhyphenation}
\hyphenation{
anaph-o-ra
Dor-drecht
mono-mor-phe-mic
Swed-ish
sche-mat-ic
Viska-da-li-an
An-ders-son
dia-lekt-forsk-ning
dra-ma-språk
bref-vex-ling
Ak-tu-ell
folk-livs-forsk-ning
Þor-björg
Ak-ti-ons-art
Upp-sala
myck-en
}

\hyphenation{
anaph-o-ra
Dor-drecht
mono-mor-phe-mic
Swed-ish
sche-mat-ic
Viska-da-li-an
An-ders-son
dia-lekt-forsk-ning
dra-ma-språk
bref-vex-ling
Ak-tu-ell
folk-livs-forsk-ning
Þor-björg
Ak-ti-ons-art
Upp-sala
myck-en
}

\hyphenation{
anaph-o-ra
Dor-drecht
mono-mor-phe-mic
Swed-ish
sche-mat-ic
Viska-da-li-an
An-ders-son
dia-lekt-forsk-ning
dra-ma-språk
bref-vex-ling
Ak-tu-ell
folk-livs-forsk-ning
Þor-björg
Ak-ti-ons-art
Upp-sala
myck-en
}
 
  \togglepaper[8]%%chapternumber
}{}

\begin{document}
\SetupAffiliations{mark style=none}
 \maketitle

\pagebreak\section{Introduction}\label{sec:delsing:1}\largerpage[2]


The \isi{quantifier} \textit{mycket} (‘much’) in present-day Swedish is primarily used to quantify mass nouns and plurals, roughly like \textit{much} in Modern \ili{English}:\footnote{With countable nouns in the plural, \textit{många} ’many’ is normally used (see \ref{ex:delsing:fni}), but \textit{mycket} may also be used (see \ref{ex:delsing:fnii}), and then the plurality is seen more collectively.

    \ea\label{ex:delsing:fni}
    \ea
    \gll många    böcker      \\
    many      books           \\
    \glt ‘many books’

    \ex\label{ex:delsing:fnii}
    \gll mycket   böcker\\
    much     books\\
    \glt  ‘lots of books’
    \z
    \z
}


\ea\label{ex:delsing:1}
(present-day Swedish)\\
\gll mycket  mjölk/socker/pengar\\
much      milk.\textsc{c}.\textsc{sg}/sugar.\textsc{n.sg}/money.\textsc{pl}\\
\glt `much milk/sugar/money'
\z

As can be seen from \REF{ex:delsing:1}, the \isi{quantifier} is uninflected for number and \isi{gender}, and the form is the same, whether the noun is common \isi{gender} (glossed \textsc{c}) or \isi{neuter} (\textsc{n}), or whether it is singular or plural. As in \ili{English} it is also used with comparative adjectives:\footnote{In Swedish, \textit{mycket} is also used with the meaning ‘very’ to denote degrees of positive adjectives. The \ili{Old Swedish} distinction between \textit{miok} ’very’ and \textit{mykit} ‘much’ was levelled in the 15\textsuperscript{th} century. This development, however, is beyond the scope of this paper.}


\ea\label{ex:delsing:2}
(present-day Swedish)\\
\gll mycket  längre\\
much      longer\\
\z

In older Swedish, the \isi{adjective} \textit{mycken} (‘big’) agreed with its noun in \isi{gender} and number, as in \REF{ex:delsing:3}. Around 1600, it consistently agreed with the head noun in \isi{gender} and number.


\ea(\ili{Early Modern Swedish})\\\label{ex:delsing:3}
\ea\label{ex:delsing:3a}
\gll mycken      glädje\\
much\textsc{.c.sg}    joy \\
\ex\label{ex:delsing:3b}
\gll mycket          oljud\\
    much\textsc{.n.sg}    noise \\
\ex\label{ex:delsing:3c}
\gll myckna      tårar \\
    much\textsc{.pl}  tears \\
\z
\z

Comparing the examples in \REF{ex:delsing:1} and \REF{ex:delsing:3}, we note that the \isi{agreement} pattern has changed. The former \isi{neuter} singular form \textit{mycket} (as in \ref{ex:delsing:3b}) has spread, and is now used with all nouns (common \isi{gender} singular of uncountable nouns as well as plural of countable nouns, as shown in \ref{ex:delsing:1}).\largerpage[2.5]



A very similar development has happened with the \isi{adjective} \textit{liten/t} ‘little’, which has been replaced by the uninflected \textit{lite(t)}, based on the former \isi{neuter} form of the \isi{adjective}. Older phrases with the agreeing \isi{adjective} (like in \ref{ex:delsing:4a}) are expressed with the non-agreeing \isi{quantifier} \textit{lite} in present-day Swedish; cf. \REF{ex:delsing:4b}. \textit{Liten/t}, however, is  slightly different from \textit{mycken/t} (see \sectref{sec:delsing:2}).


\ea\label{ex:delsing:4}
\ea\label{ex:delsing:4a}(EMS system)\\
\gll Hon  fick    liten        hjälp  av  sina          grannar\\
she      got  little.\textsc{c.sg}    help  of    \textsc{poss.refl}    neighbours\\
     \glt ‘She got a little help from her neighbours’\\

\ex\label{ex:delsing:4b}(present-day Swedish)\\
\gll Hon    fick    lite        hjälp    av  sina        grannar    \\
      she      got    little.\textsc{n.sg}  help    of  \textsc{poss.refl}  neighbours\\
      \glt ‘She got little/some help from her neighbours’\footnote{The reading ‘some help’ in \REF{ex:delsing:4b} is only available with unstressed \textit{lite}.}\\
\z
\z

In this paper, I make two claims. First, I claim that the adjectives \textit{mycken} ‘big’ and \textit{liten} ‘little’ have turned into quantifiers during the last three hundred years. The shift is most clearly noticeable in the increasing lack of \isi{agreement}, i.e. in the use of the old \isi{neuter} singular form even with common \isi{gender} singular and plural nouns. The development is studied in the corpus of Swedish drama dialogue \citep{MarttalaStromquist2001}, covering the years 1725–2000. The corpus is presented in \sectref{sec:delsing:3}. I will show that \textit{lite} was grammaticalized as a \isi{quantifier} a little earlier than \textit{mycket.} Second, I will propose that the weak forms of the quantifiers \textit{mycket} and \textit{lite(t)} have turned into polarity items in present-day Swedish (\sectref{sec:delsing:5}). This leads to a prediction about the frequency of these words, namely that the \is{positive polarity item (PPI)}positive polarity item \textit{lite} should become more frequent and the negative \isi{polarity item} \textit{mycket} should become less frequent. In \sectref{sec:delsing:6}, this prediction is tested in the drama corpus.


\section{Agreement in gender/number}\label{sec:delsing:2}


In \ili{Old Swedish}, there are basically two adjectives meaning ‘big’, \textit{stor} and \textit{mykil} (with the \isi{masculine} \isi{accusative} form \textit{mykin}).\footnote{The word \textit{diger} ‘big’ is also used, but  is nowadays almost obsolete, and has not interfered with the change studied in this paper. The word \textit{stor} is less frequent in the oldest Swedish sources and is not attested in \ili{Runic Swedish} (800–1225 CE), whereas \textit{mykil} is found at least eight times in \ili{Runic Swedish} \citep{Peterson2006}.}  The first one is mainly used with countable nouns, whereas the second is mainly restricted to mass nouns, and occasionally occurs with plurals. The \isi{adjective} \textit{litil} (with the \isi{masculine} \isi{accusative} form. \textit{litin}) ‘little, small’ is used both with count and mass nouns, but in the plural the suppletive form \textit{smar} ‘little, small’ is normally used.



Adjectives used in Swedish definite noun phrases have a different \isi{inflection} from adjectives used in \isi{indefinite} noun phrases or as predicatives. The definite form is normally seen as a true sign of adjectivehood.\footnote{The form used in definite noun phrases is traditionally called \textit{weak inflection}, and the other, used in indefinites, is called \textit{strong inflection}. This distinction is a traditional morphological distinction, which refers to more regular forms (weak), and more irregular forms (strong). To avoid confusion with \textit{strong} and \textit{weak} referring to stress, I only discuss \textit{definiteness} here (although this is historically less adequate).} The forms (in the \isi{nominative} singular) are given in \tabref{tab:delsing:1} below.


\begin{table}
\caption{Inflection of the adjectives \textit{mykil} ‘big’ and \textit{litil} ‘little’ in Old and Early Modern Swedish}
\label{tab:delsing:1}
\begin{tabular}{lll}
\lsptoprule
Adjectival \isi{inflection} & \ili{Old Swedish} & \ili{Early Modern Swedish}\\
\midrule
\is{indefinite}Indefinite/\isi{predicative} & mykil\slash litil & mycken\slash liten\\
Definite & mykli\slash litli & myckna\slash lilla\\
\lspbottomrule
\end{tabular}
\end{table}

We can note that the final \textit{{}-l} of the \isi{nominative} is replaced with final \textit{{}-n} (from the \isi{accusative} forms) in the history of \ili{Old Swedish}. As for the definite forms, the \ili{Old Swedish} \isi{masculine} \isi{nominative} singular, ending in -\textit{i,} is often exchanged for the \isi{masculine} oblique/\isi{feminine} \isi{nominative} form, ending in -\textit{a}.



In the second half of the 17\textsuperscript{th} century, we find the first occurrence of uninflected \textit{mycket} (originally the \isi{neuter} form) with plural or common \isi{gender} singular nouns, according to the Swedish Academy dictionary (\isi{SAOB}, the entry \textit{mycken}), as illustrated in \REF{ex:delsing:5}.


\ea\label{ex:delsing:5}
(\isi{SAOB}, example from 1676)\\
\gll {Mycket  Lieutenanter      och    andra       Officerare}\\
much.\textsc{n.sg}   lieutenant.\textsc{c}.\textsc{pl}    and    other      officers\\
\glt ‘lots of lieutenants and other (commissioned) officers’ 
\z


As we will see, the word \textit{liten/litet} seems to have turned into a \isi{quantifier} slightly earlier than \textit{mycken/mycket}, and, probably because of this, they differ in spelling today. Both are pronounced without the final -\textit{t}, but the \isi{quantifier} \textit{mycket} is always spelled with a final \textit{{}-t} in present-day Swedish, just like the old \isi{neuter} form, whereas the \isi{quantifier} \textit{lite(t)} normally lacks the final -\textit{t} in the spelling.\footnote{A relevant fact may also be that \textit{mycken/t} as an \isi{adjective} is no longer in use, whereas \textit{liten/t} functions as a regular \isi{adjective} with countable nouns (in the singular) today. In other words, it makes sense to distinguish the \isi{adjective} \textit{litet} from the \isi{quantifier}/degree element \textit{lite}, whereas this is not necessary for \textit{mycket}.}



In the next section, I present a small study of the non-agreeing uses of \textit{mycket} and \textit{lite(t)} in the drama corpus.\footnote{This study is limited both with regard to the number of elements studied and to the size of my corpus, but as far as I know it is new for Swedish. For other languages there are of course more extensive studies in these respects, e.g. \citet{RoehrsSapp2016} is a diachronic study of quantifying elements in the history of \ili{German}.} 


\section{Investigation of the drama corpus (1725–2000)}\label{sec:delsing:3}


The corpus of Swedish drama dialogue consists of 45 original Swedish dramas written between 1725 and 2000. They are partitioned into six periods of 25 years (with 25 years between each period), two in each century, where the three earliest periods contain five dramas each, while the three latter contain ten dramas each. The periods, the number of dramas, and the number of words of the corpus are given in \tabref{tab:delsing:2}. For details on the individual dramas, see \citet{MarttalaStromquist2001} or \citet[38–39 and Appendix 1]{Stroh-Wollin2008}.



\begin{table}
\caption{The bulk of the drama corpus\label{tab:delsing:2}}
\begin{tabular}{lcrr}
\lsptoprule
Corpus section & Period & No. of dramas & No. of words (tokens)\\
\midrule
Period 1 & 1725–1750 & 5 & 92,000\\
Period 2 & 1775–1800 & 5 & 73,000\\
Period 3 & 1825–1850 & 5 & 99,000\\
Period 4 & 1875–1900 & 10 & 178,000\\
Period 5 & 1925–1950 & 10 & 205,000\\
Period 6 & 1975–2000 & 10 & 166,000\\
\lspbottomrule
\end{tabular}
\end{table}

I have studied all occurrences of the adjectives/quantifiers \textit{mycken/mycket} and \textit{liten/litet} with mass nouns and plurals in the drama corpus, noting the \isi{agreement} pattern. Typical excerpted phrases are illustrated in \REF{ex:delsing:6} below. Reference to the specific drama is made with the corpus period number followed by a letter indicating the specific drama. The reference (5B) thus indicates drama B (the second) in the fifth period of the corpus (from 1925–1950).


\ea\label{ex:delsing:6}
\ea
\gll Jag  önskar    Er        lycka,      Cousin,  mycken      lycka! \\
I          wish     you.\textsc{obj}   happiness,  cousin,    much.\textsc{c}.\textsc{sg}  happiness.\textsc{c}.\textsc{sg}\\
\glt ‘I wish you happiness, (my) cousin, much happiness’ (1B)

\ex
\gll Nå,    nu    lär    Ni        väl    föra    mycket      varor med      Er? \\
    well    now    \textsc{aux}  you      \textsc{dm}     bring  much.\textsc{n.sg}  merchandise.\textsc{pl} with     you.\textsc{obj}\\
\glt ‘Well, now you will bring much merchandise with you’ (2C)\\

\ex
\gll Lite    sill            har    jag    gudskelov \\
    little    herring.\textsc{c}.\textsc{sg}   have   I       god.be.praised\\
    \glt ‘Little/some herring, I have, thank god’ (5B)\\
\z
\z



Only cases where there is a head noun and where there are no other determiners are counted. Some determiners are incompatible with mass nouns and plurals, and others may only occur in front of adjectives, but not quantifiers. Examples of excluded noun phrases are given in \REF{ex:delsing:7}.\footnote{The few cases with definite \isi{inflection} are always adjectival and may co-occur with determiners. See \tabref{tab:delsing:1} above.}


\ea\label{ex:delsing:7}
\ea(head noun missing)\\
\gll så    skulle  \textit{mycket}         \textit{ondt}          kunna     undvikas     \\
so        should  much.\textsc{n.sg}   {evil.\textsc{n.sg}}  be.able.to be.avoided \\
    \glt ‘In that way, a lot of evil should possibly be avoided’ (4B)\\
\ex(head noun missing)\\
\gll dessutan    hade  jag  lärt      så    \textit{mycket} af    Fransyskan\\
     additionally  had  I      learned    so    {much.\textsc{n.sg}}   of    \ili{French}\\
\glt ‘Additionally, I had learned enough \ili{French}’ (1B)\\
\ex  (other \isi{determiner})\\
\gll \textit{en}         \textit{liten}          förtrolig        bekanntskap\\
    \textsc{ac.sg}    little\textsc{.c.sg}  intimate\textsc{.c.sg}     relationship.\textsc{c.sg}\\
    \glt ‘a small intimate relationship’ (2C)\\
\z
\z


I have also excluded some other examples. Since \textit{mycket/lite(t)} may be used either to quantify nouns or to signal degrees of adjectives (see footnote 2), some examples are ambiguous between quantifiers and degree adverbials, as in \REF{ex:delsing:8}. If the word after \textit{mycket/lite(t)} is ambiguous between \isi{adjective} and noun, the status of \textit{mycket/lite(t)} cannot be determined, which is the case in \REF{ex:delsing:9}. Consequently, examples like \REF{ex:delsing:8} and \REF{ex:delsing:9} are also excluded from the investigation.


\ea\label{ex:delsing:8}
\gll även    detta  ett  rum    med    mycket    konservativ    möblering \\
also      this      a    room  with  much/very  conservative    furnishing\\
\glt `Also this being a room with \{much of/ very\} conservative furnishing’ (5E)\\
\ex\label{ex:delsing:9}
\gll Var    det         för   lite     salt?\\
was     there/it       too     little   salt/salty\\
\glt ‘Was there too little salt?\slash Was it not salty enough?’ (6E)
\z



The remaining examples are classified into three groups: unambiguous adjectives (which include clearly agreeing \textit{mycken/liten} and the rare cases with definite \isi{inflection}; see \tabref{tab:delsing:2} above and \REF{ex:delsing:10c} below), unambiguous quantifiers, and ambiguous cases, illustrated in (\ref{ex:delsing:10}--\ref{ex:delsing:12}) respectively.


\ea\label{ex:delsing:10}
\ea\label{ex:delsing:10a}\isi{agreement} in common \isi{gender} singular\\
\gll mycken        oro\\
    much.\textsc{c.sg}      unrest.\textsc{c.sg}                  \\
    \glt ‘much unrest’\\
\ex\label{ex:delsing:10b}\isi{agreement} in plural\\
\gll myckna      tårar\\
    much.\textsc{pl}  tears.\textsc{pl}                                \\
    \glt ‘much tears’\\
\ex\label{ex:delsing:10c}definite \isi{inflection} (i.e. \isi{adjective})\\
\gll det    myckna       skrivandet\\
    the    much.\textsc{def}   writing.\textsc{n.sg}\\
    \glt ‘the abundant writing’\\
\z
\ex\label{ex:delsing:11}
\ea lack of \isi{agreement}\\\gll mycket      mjölk\\
    much.\textsc{n.sg}      milk.\textsc{c.sg}\\
    \glt ‘much milk’\\
\ex lack of \isi{agreement}\\
    \gll mycket            pengar\\
    much.\textsc{n.sg}      money.\textsc{pl} \\
    \glt ‘much money’\\
\z
\ex\label{ex:delsing:12}
\ea\label{ex:delsing:12a}\isi{agreement} in \isi{neuter} singular\\
\gll mycket      kött\\
much.\textsc{n.sg}     meat.\textsc{n.sg} \\
    \glt ‘much meat’
\ex \label{ex:delsing:12b}ambiguous in number\\
\gll mycket          folk \\                           
    much.\textsc{n.sg}    people.\textsc{n.sg/pl} \\
    \glt ‘much people’
\z
\z


Note that phrases may be assigned to the ambiguous group (as in \ref{ex:delsing:12}) for two reasons: either because there seems to be \isi{agreement} in the \isi{neuter} (which was found both before and after the change, as in \ref{ex:delsing:12a}), or because the head noun is ambiguous between singular and plural (which is the case with most \isi{neuter} nouns), and therefore might be an instance of \isi{neuter} \isi{agreement} in the singular (as in \ref{ex:delsing:12b}).\largerpage[1.5]



We are now in a position to present the data from the investigation of the drama corpus. In Tables~\ref{tab:delsing:3}--\ref{tab:delsing:4}, the numbers of unambiguous examples of adjectives and quantifiers are given as well as the numbers of ambiguous examples. The \isi{Q-quote} is a percentage of unambiguous quantifiers out of all unambiguous examples (the sum of adjectives and quantifiers in the table). The numbers are quite small, so the percentages should be taken approximately, but in both cases the development is quite clear. We start out with \textit{liten/lite(t)} in \tabref{tab:delsing:3}, and continue with \textit{mycken/mycket} in \tabref{tab:delsing:4}.



\begin{table}
\caption{Agreement and Q-quote for \textit{liten/lite(t)} 1725–2000}
\label{tab:delsing:3}
\begin{tabular}{lrrrr}
\lsptoprule
Period & Adjectives & Ambiguous & Quantifiers & \isi{Q-quote}\\\midrule
1. 1725–1750 & 3 & 6 & 4 & 57\%\\
2. 1775–1800 & 2 & 6 & 8 & 80\%\\
3. 1825–1850 & 1 & 3 & 7 & 88\%\\
4. 1875–1900 & 0 & 7 & 19 & 100\%\\
5. 1925–1950 & 1 & 11 & 25 & 96\%\\
6. 1975–2000 & 0 & 12 & 50 & 100\%\\
\lspbottomrule
\end{tabular}
\end{table}

The development of \textit{liten/t}, illustrated in \tabref{tab:delsing:3}, seems quite clear. The \isi{Q-quote} had already risen to 80\% by the end of the 1700s, and since the end of the 1800s, \textit{liten} is hardly ever used as an \isi{adjective} with mass nouns/plurals.\footnote{The only example from the last 150 years in the corpus is the following:
\ea \gll  Men     en     sten     har   man   mycket   liten       användning   för\\
          but     a     stone   has   one   very     little\textsc{.c.sg}   use\textsc{.c.sg}     for\\
     \glt ‘You only have very little use for a stone’ (5H)\z%
} As we will see, this is earlier than the development for \textit{mycken/mycket}. Consider \tabref{tab:delsing:4}.\pagebreak



\begin{table}
\caption{Agreement and Q-quote for \textit{mycken/mycket} 1725–2000}
\label{tab:delsing:4}
\begin{tabular}{lrrrr}
\lsptoprule
Period & Adjectives & Ambiguous & Quantifiers & \isi{Q-quote}\\
\midrule
1. 1725–1750 & 16 & 12 & 2 & 11\%\\
2. 1775–1800 & 18 & 11 & 4 & 18\%\\
3. 1825–1850 & 11 & 12 & 9 & 45\%\\
4. 1875–1900 & 9 & 12 & 11 & 55\%\\
5. 1925–1950 & 3 & 17 & 22 & 88\%\\
6. 1975–2000 & 0 & 10 & 36 & 100\%\\
\lspbottomrule
\end{tabular}
\end{table}

The development of \textit{mycket}, illustrated in \tabref{tab:delsing:4}, shows that the \isi{Q-quote} reached 80\% in the 1900s, and the development seems to be clearly slower than that of \textit{lite(t)}. One reason that the \isi{Q-quote} is higher for \textit{liten/lite(t)} might be that this \isi{adjective} does not have a plural form (but the suppletive forms \textit{små} or sometimes \textit{få/fåtaliga} ‘few/few-numbery’ are used instead). By contrast, the new \isi{quantifier} \textit{lite(t)} may indeed be used with plurals. Two examples from the corpus (with \textit{lite(t)} followed by plurals) are given in \REF{ex:delsing:13}.


\ea\label{ex:delsing:13}
\ea \gll Junkaren      Tusenskön,     som   har   litet         pengar\\   
          young.man\textsc{.def}  T.  who     has   little.\textsc{n.sg}  money.\textsc{pl}\\
 \glt ‘Young T., who has little money’ (2C)\\
\ex \gll Nu      får     du            lite         böcker\\
  Now  get     you\textsc{.sg}     little.\textsc{n.sg}   book.\textsc{pl}\\
  \glt ‘Now, you will get some books’ (6C)\\
\z
\z


In other words, if the development starts out in the singular, \textit{liten/t} would be earlier than \textit{mycken/t}. The plural cases with \textit{lite(t)} (like in \ref{ex:delsing:13}) are, however, quite rare in the older dramas, only 5 in the three oldest periods. With \textit{mycken/mycket}, on the other hand, plurals are more frequent: there are 13 instances in the three older periods. If we calculate only singulars in the three oldest periods, we arrive at the figures in \tabref{tab:delsing:5}.



\begin{table}
\caption{Agreement and Q-quotes for singulars 1725–1850\label{tab:delsing:5}}
\begin{tabular}{lrrrr}
\lsptoprule
Word & Adjectives & Ambiguous & Quantifiers & \isi{Q-quote}\\
\midrule
\textit{liten/lite(t)} & 6 & 15 & 14 & 70\%\\
\textit{mycken/mycket} & 46 & 38 & 3 & 6\%\\
\lspbottomrule
\end{tabular}
\end{table}

Thus, the difference between the \isi{Q-quote} of the two pronouns is even stronger if we exclude the plurals, which means that we can say (even if the numbers are small) that \textit{lite(t)} is certainly earlier as a \isi{quantifier} than \textit{mycket}.



We should also examine which kinds of nouns are the first to occur in the \isi{quantifier} cases. Initially, it seems as if inherent plurals, i.e. plurals that lack a singular form with the relevant meaning, are frequent; the word \textit{pengar} ‘money’\footnote{The word \textit{pengar} ‘money’ is always plural (in the sense of ‘money’). Occasionally it may be used in the singular, \textit{peng}, but then the meaning is ‘coin’.} especially is over-represented as a quantified noun in the earlier periods. I have counted the nouns used with non-agreeing \textit{mycket} and \textit{lite(t)} during the four oldest periods, and the results are given in \tabref{tab:delsing:6}.



\begin{table}
\caption{Nouns with quantifying \textit{mycket/lite(t)} 1725–1800}
\label{tab:delsing:6}
\begin{tabular}{lrr}
\lsptoprule
Quantified noun & \textit{mycket} & \textit{lite(t)}\\\midrule
inherent plurals & 14 & 4\\
mass nouns & 2 & 14\\
other plurals & 6 & 1\\
\lspbottomrule
\end{tabular}
\end{table}

Out of the 54 quantifiers in the first four periods (numbers given in \tabref{tab:delsing:6}), 18 take inherent plurals (17 are \textit{pengar} ‘money’ and one is \textit{förfriskningar} ‘refreshments’). In the last two periods only nine out of 102 nouns are inherent plurals (five are \textit{pengar} and one is \textit{stålar}, both meaning ‘money’). I think that this is significant; it seems as if \textit{pengar} was the noun that introduced the possibility of using non-agreeing \textit{mycket/lite(t)} with mass nouns. Ordinary plurals seem to come later. The majority of the seven ordinary plural examples in \tabref{tab:delsing:6} are from period 4, and they become abundant in periods 5 and 6.



We have seen that in an initial stage of the change, the \isi{inherent plural} \textit{pengar} ‘money’ is over-represented as a quantified noun. With respect to \textit{mycket}, we find 22 instances with a common \isi{gender} singular or plural noun during the first four periods (1725–1800). Out of these, no fewer than 13 have the word \textit{pengar} in the plural. Out of the other nine instances, six are other plurals, one word, \textit{djur} ‘animal(s)’, is ambiguous between singular and plural, and only two are common \isi{gender} singular, namely \textit{kärlek} ‘love’ and \textit{respekt} ‘respect’. 



To summarize, the \isi{adjective} \textit{liten/t} seems to have already grammaticalized into a \isi{quantifier} in the 1700s, whereas \textit{mycken/t} grammaticalized later, in the late 1800s and early 1900s. The corpus used is admittedly small, but the texts normally come close to the spoken language, and the tendencies are quite clear. There is thus good reason to conclude that \textit{liten/t} was earlier than \textit{mycken/t} in the \isi{grammaticalization} process. 



The \isi{grammaticalization} process from \isi{adjective} to \isi{quantifier} can be described as climbing higher in the syntactic tree, as has been suggested by \textcite{RobertsRoussou1999, RobertsRoussou2003}.\footnote{Related analyses of \isi{grammaticalization} of adjectives that are described as climbing higher in the syntactic tree can be found in \citet{Oxford2017}; see also \textcitetv{chapters/07}.} In this way lexical adjectives are changed into functional elements in a Q- or NumPhrase inside the extended DP.


\section{Weak and strong quantifiers}\label{sec:delsing:5}


In present-day Swedish, our two quantifiers, \textit{mycket} and \textit{lite(t),} have one strong and one weak form, i.e. one stressed and one unstressed form. We find both weak and strong quantifiers with nouns, as well as when they quantify comparative adjectives.\footnote{Quantifying \textit{mycket/litet} may also be used with verb phrases: \textit{Han simmar inte mycket nuförtiden} (‘He doesn’t swim much nowadays’), and this use seems similar to other \isi{quantifier} uses; however, I will leave such cases aside in this paper.} In the examples below, I mark the strong variants with \textit{’}\textsc{my}\textit{ke/’}\textsc{li}\textit{te} (to indicate the stress) and the weak ones with \textit{ˌmyke/ˌlite}, in the latter case signifying that both syllables are deaccentuated (i.e. it is not a \isi{prosodic word}).\footnote{Weak and strong forms of \textit{lite} are also present in present-day \ili{Norwegian}, where they are distinguished in both the spoken and the written language: stressed \textit{lite} and unstressed \textit{litt}. Since they are separate entries in the dictionary, the meaning is well described (see e.g \textit{Norsk ordbok}). Stressed \textit{lite} means ‘a limited amount’, and the opposite is ‘much’, whereas unstressed \textit{litt} means ‘a small amount’, and the opposite is ‘nothing’. I find this to be a good description of the difference between the stressed and unstressed \textit{lite} in Swedish, too.} 



In this section, I try to show that the weak forms of the two quantifiers are polarity items. \citet{Israel1996} mentions two problems with the research on polarity items:

\begin{quote}
[A]s the theorist strives for intimations of universality, the complexity and the subtle variability of the data are easily underestimated or ignored. On the other hand, when one considers the phenomenon in all its glorious messiness, one may quickly despair of ever finding any general explanation. \citep[619]{Israel1996}
\end{quote}

In the literature, it has been mentioned that the \ili{English} correspondents \textit{much} and \textit{a little bit} are polarity sensitive. \citet{Israel1996} gives several examples with \textit{much} (and \textit{a little bit}) to discuss polarity sensitivity, but his focus is to explain the system of polarity sensitive elements, rather than to discuss the properties of the specific lexical items.



\citegen{Israel1996} claim is that most (or all) lexical elements that show polarity sensitivity can be classified in terms of two features. The first feature is quantitative value, which describes whether the item is \isi{high-scalar} or \isi{low-scalar}, i.e. whether it denotes a high degree (like \textit{much} and \textit{as hell}) or a low degree (like \textit{a little bit} and \textit{at all}). The second feature is informative value, which describes whether the item is \isi{emphatic} (like \textit{as hell} and \textit{at all}) or \isi{understating} (like \textit{much} and \textit{a little bit}). The \isi{negative polarity} items (NPIs) that are \isi{high-scalar} (like \textit{much}) are \isi{understating}, whereas the \isi{positive polarity} items (PPIs) that are \isi{high-scalar} (like \textit{as hell}) are \isi{emphatic}. See \figref{fig:delsing:1}, adapted from \citet[628]{Israel1996}, with some of his examples.



\begin{figure}
\fbox{\begin{tabularx}{\textwidth - 3\fboxsep}{lXCXl}
&  & High scalar &  & \\
& \is{understating}Understating &  & \is{emphatic}Emphatic & \\
& \textit{much, long} &  & \textit{totally, as hell} & \\
NPIs &  &  &  & PPIs\\
& \is{emphatic}Emphatic &  & \is{understating}Understating & \\
& \textit{a drop, at all} &  & \textit{a little bit, sorta} & \\
&  & Low scalar &  & \\
\end{tabularx}}
\caption{\citegen{Israel1996} model for polarity items\label{fig:delsing:1}}
\end{figure}


I propose that both weak \textit{mycket} and weak \textit{lite(t)} are \isi{understating}, i.e. they only express a small step on a scale, but that weak \textit{mycket} is an \is{negative polarity item (NPI)}NPI, whereas weak \textit{lite(t)} is a \is{positive polarity item (PPI)}PPI.



Let us now turn to some properties of these alleged polarity items. Consider first the examples below for the strong version of \textit{mycket}, i.e. ‘\textsc{my}ke.


\ea
\gll Han    har    fått         ‘\textsc{my}ke   pengar.\\
he           has       received   much     money\\
\glt `He has received lots of money.’\\
\ex
\gll Hon  har   blivit     ‘\textsc{my}ke   klokare.\\
she         has    become  much    wiser\\
\glt `She has become much wiser.’\\
\ex
\gll Han  har     inte     fått       ‘\textsc{my}ke   pengar (men   han     har   fått \textsc{li}te).\\
he       has       not     received  much     money   but   he   has   received   little\\
\ex
\gll Hon    har     inte     blivit     ‘\textsc{my}ke     klokare   (bara     ’\textsc{li}te).\\
she             has       not     become     much   wiser     only     little\\
\z


As can be seen, the strong variant of \textit{mycket} is felicitous both in affirmative and negative clauses. The same is true of the strong variant of \textit{lite(t)}. All the examples above are focused or contrastive in one way or another, but we cannot detect any polarity effects. When we start looking at the weak variants, on the other hand, we do find polarity effects. Now, consider the different behaviour of weak \textit{ˌmyke} in affirmative and negated clauses. I include a stressed verb to make sure that the \isi{quantifier} is weak.

{\judgewidth{\#}
\ea[\#]{\label{ex:delsing:18}
\gll Han     har \textsc{fått} ˌmyke     pengar.\\
he         has   received   much     money\\}
\ex[\#]{\label{ex:delsing:19}
\gll Hon     har \textsc{bli}vit     ˌmyke   klokare.\\
he         has     become     much   wiser\\}
\ex[]{\label{ex:delsing:20}
\gll Han   har     inte \textsc{fått} ˌmyke   pengar.\\
he       has       not     received   much     money\\
\glt `He has not received very much money.’\\}
\ex[]{\label{ex:delsing:21}
\gll Hon   har inte \textsc{bli}vit     ˌmyke   klokare.\\
she       has   not     become   much       wiser\\
\glt `She hasn’t become very much wiser.'}
\z}

\begin{sloppypar}
As can be seen in (\ref{ex:delsing:18}–\ref{ex:delsing:19}) above, weak \textit{ˌmyke} is infelicitous in affirmative clauses. The ‘\#’ denotes that the clauses are possibly not ungrammatical, but to my mind they do not really mean anything.\footnote{\is{judgements}Judgements are often a bit uncertain, but I have checked my intuitions with a handful of other native speakers and they clearly \isi{agree} on the difference between (\ref{ex:delsing:18}--\ref{ex:delsing:19}) on one hand and (\ref{ex:delsing:20}--\ref{ex:delsing:21}) on the other.}
\end{sloppypar}



With \textit{lite(t)}, the strong version, ‘\textsc{li}\textit{te}, also seems to be allowed in all contexts, whereas the weak version, \textit{ˌlite}, seems to behave in the opposite way to \textit{ˌmyke}, i.e. like a \is{positive polarity item (PPI)}positive polarity item, avoiding negative sentences. This is illustrated in (\ref{ex:delsing:22}--\ref{ex:delsing:25}) below.

{\judgewidth{\#}
\ea[]{\label{ex:delsing:22}
\gll Han     har \textsc{fått} ˌlite   pengar.\\
he         has   received little money\\
\glt ‘He has received some money.’\\}
\ex[]{\label{ex:delsing:23}
\gll Hon   har \textsc{bli}vit     ˌlite   klokare.\\
she       has become   little   wiser\\
\glt ‘She has become a little bit wiser.’\\}
\ex[\#]{\label{ex:delsing:24}
\gll Han     har     inte \textsc{fått} ˌlite   pengar.\\
he               has       not       received   little     money\\
\glt ‘He hasn’t received any money.’\\}
\ex[\#]{\label{ex:delsing:25}
\gll Hon     har     inte \textsc{bli}vit     ˌlite   klokare.\\
she         has     not     become   little     wiser\\
\glt ‘She hasn’t become a little bit wiser.’\\}
\z}


As indicated in (\ref{ex:delsing:24}--\ref{ex:delsing:25}), unstressed \textit{ˌlite} is infelicitous in negative contexts. It may not be ungrammatical, but it is my impression that these sentences may only be used as echo answers, when the concept of ‘a little money’ or ‘a little (bit) wiser’ have just been mentioned.



It is well known that many NPIs do occur not only in negated sentences, but also in other polarity contexts. Those that only occur in negative sentences are normally called strong NPIs, whereas those that may occur in other contexts too are normally called weak NPIs (see e.g. \citealt{Brandtler2010}: 12–14). Other contexts include questions, conditionals, and comparative clauses/phrases after comparative adjectives. If we test our two PI candidates for these kinds of sentences, we get the following results:


\ea\label{ex:delsing:26}\judgewidth{\#}
\ea[\#]{\label{ex:delsing:26a}
\gll Har    hon   fått       ˌmyke   pengar?\\
has           she     received   much     money\\}
\ex[*]{
\gll Vem       kan     skala       ˌmyke     potatis?\\
    who       can     peel       much     potatoes\\}
\ex[*]{
\gll Om     du       har     fått         ˌmyke     pengar,   så…\\
      if         you.\textsc{sg}   have   received   much     money,   then…\\}
\z
\ex\label{ex:delsing:27}
\ea[]{
\gll Har   hon     fått       ˌlite   pengar?\\
    has     she       received   little   money\\
    \glt ‘Has she received some money?’\\}
\ex[]{
\gll Vem     kan     skala     ˌlite     potatis?\\
    who   can       peel     little     potatoes\\
    \glt ‘Who wants to peel some potatoes?’\\}
\ex[]{
\gll Om    du         har       fått         ˌlite     pengar, så…\\
    if         you.\textsc{sg}     have     received   little     money,  then…\\
    \glt ‘If you have received some money, then…\\}
\z
\z


As indicated in \REF{ex:delsing:26} above, weak \textit{ˌmyke} is ungrammatical or strange (as in \ref{ex:delsing:26a}), but may be possible in echo-questions. Weak \textit{ˌlite} on the other hand works fine in these contexts, as illustrated in \REF{ex:delsing:27}. The data presented above suggest that \textit{ˌmyke} is a strong negative \isi{polarity item}, i.e. one that requires an overt negation in the clause, whereas \textit{ˌlite} is used elsewhere.



We may tentatively conclude that weak \textit{ˌmyke} has certain properties of a negative \isi{polarity item}, and that unstressed \textit{ˌlite} has certain properties of a \is{positive polarity item (PPI)}positive polarity item. Needless to say, there are lots of questions that have to be resolved before the claim can be substantiated in full. I leave this for future research.



Our two weak quantifiers are easily incorporated into Israel’s model (see \figref{fig:delsing:1} above). Weak \textit{ˌmyke} is a \isi{high-scalar} \is{negative polarity item (NPI)}NPI, whereas \textit{ˌlite} is a \isi{low-scalar} \is{positive polarity item (PPI)}PPI. Both, however, are \isi{understating} in Israel’s terms. As \citet[51]{Traugott2010} points out, it is a feature of \isi{negative polarity} that words that are \isi{understating} in positive contexts (she mentions \textit{a bit (of)} and \textit{a shred of}) are reversed and become \isi{emphatic} in the sense of \citet{Israel1996} in negative contexts. It seems to work both ways, so a word like \textit{much} is originally \isi{emphatic} in positive contexts, but when it becomes a \is{negative polarity item (NPI)}NPI it becomes \isi{understating}. Intuitively, it is not surprising that \textit{not much} and \textit{little} are used for the same function, namely to denote a small step on a scale. In a sense, then, they have a similar meaning, albeit with complementary distribution.\footnote{The fact that weak \textit{lite} has a bleached meaning, only signifying a step on a scale, is mentioned in the Swedish Academy Grammar (\citealt{TelemanEtAl1999}/2: 406). I claim that weak \textit{mycket} has the same property in negated clauses.}


\section{A prediction}\label{sec:delsing:6}


In the previous sections, we have seen that the adjectives \textit{mycken} ‘much’ and \textit{liten} ‘little’, have successively turned into quantifiers (roughly) over the last 300 years. The former \isi{neuter} singular forms \textit{mycket} and \textit{lite(t)} are now also used with common \isi{gender} mass nouns and plurals. The old \isi{adjective} \textit{mycken/t} is now only used (in its \isi{neuter} form \textit{mycket}) as a \isi{quantifier}, whereas the \isi{adjective} \textit{liten/t} is still used as a regular \isi{adjective} with countable nouns, but has turned into an uninflected \isi{quantifier} with mass nouns and plurals. Second, it seems that the new quantifiers have developed a strong-weak distinction, and that the weak forms have turned into polarity items: \textit{ˌmyke} has turned into a negative \isi{polarity item}, whereas \textit{ˌlite} has turned into the opposite, a \is{positive polarity item (PPI)}positive polarity item.



Now, let us make two assumptions: 1) weak quantifiers are less marked (and thus more frequent) than strong ones, and 2) \isi{positive polarity} environments are less marked (and thus more frequent) than negative ones (this is especially relevant if, as I believe, \textit{ˌmyke} is only found in clauses with negation). Clauses with negation are simply less frequent than clauses without negation. If these two assumptions are on the right track, we predict that the introduction of the polarity sensitivity of our two weak quantifiers (\textit{ˌmyke} and \textit{ˌlite}) would yield a drop in the frequency of the \isi{quantifier} \textit{mycket}, whereas the \isi{quantifier} \textit{lite(t)} would increase in frequency. 



I have counted the occurrences of the two words \textit{mycken/t} and \textit{liten/t}, both as adjectives and as quantifiers, with mass nouns and plurals in the drama corpus (leaving the quantifiers followed by comparatives and other types aside). I have calculated their frequency per 10,000 words. The results are found in \tabref{tab:delsing:7}, where the periods are given in pairs (period 1 and 2 together, etc.).



\begin{table}
\caption{Frequency per 10,000 words of \textit{mycken/t} and \textit{liten/t} with mass nouns/plurals\label{tab:delsing:7}}
\begin{tabular}{lr rrrr}
\lsptoprule
Period & No. of words & \textit{mycken/t} & Frequency & \textit{liten/litet} & Frequency\\\midrule
1725–1800 & 165,000 & 63 & 3.8 & 29 & 1.8\\
1825–1900 & 277,000 & 63 & 2.3 & 35 & 1.3\\
1925–2000 & 371,000 & 87 & 2.3 & 99 & 2.7\\
\lspbottomrule
\end{tabular}
\end{table}

\tabref{tab:delsing:7} indicates a drop in the frequency of the \isi{quantifier} \textit{mycken/t} from 3.8 to 2.3, whereas \textit{liten/t} increases from 1.8 to 2.7, where the first value is from the 1700s and the second from the 1900s. The overall result (small as it is) is fully in accordance with the prediction: there is a clear drop in the (weak form of the) word which turns into a negative \isi{polarity item} (\textit{mycken/t}), whereas the word (\textit{liten/t}) that (in its weak form) turns into a \is{positive polarity item (PPI)}positive polarity item gains in frequency. We would need a larger corpus to see if the drop in the 1800s for \textit{liten/t} is significant and if so what it means. Thus, more research is clearly needed, but these first results of this preliminary investigation seem to support the prediction.\largerpage



The data should, of course, be checked further in other and larger corpora. Important questions for future research include whether the quantifiers \textit{mycket} and \textit{lite} used with comparatives behave in the same way, and, further, how degree adverbials \textit{mycket/lite} with positive adjectives/adverbs behave. Further research is clearly needed. An additional complication might be that the drop in frequency for \textit{mycken/t} is earlier than the rise for \textit{liten/t}, although the previous investigation in \sectref{sec:delsing:3} showed that \textit{liten/t} turned into a \isi{quantifier} earlier than \textit{mycken/t} did. On the other hand, there is no reason to believe that the weak-strong distinction and polarity status emerged immediately after the transition from \isi{adjective} to \isi{quantifier}. More research is needed in this respect as well.



Although many questions remain, I hope to have shown that the adjectives \textit{mycken/liten} turned into quantifiers (roughly) during the last three centuries, and that the weak versions of these quantifiers have turned into polarity items in present-day Swedish. If this is true, they also behave in the predicted way, as the weak \isi{quantifier} that has become a negative \isi{polarity item} (\textit{ˌmyke}) has dropped in frequency, whereas the one that has become a \is{positive polarity item (PPI)}positive polarity item (\textit{ˌlite}) has gained in frequency.


\section{Summary}\label{sec:delsing:7}


In this paper I have given a short description of a language change that can be viewed as an instance of \isi{grammaticalization}, namely the transition of the two adjectives \textit{mycken/t} ‘big’ and \textit{liten/t} ‘little’ into quantifiers. Data from the drama corpus show that \textit{liten/t} was already becoming a \isi{quantifier} in the 1700s, whereas \textit{mycken/t} seems to have gone through the same change roughly 150 years later. Inherent plurals (such as \textit{pengar} ‘money’) appear to be a plausible context for the starting point of the transition.



I have further illustrated how both quantifiers have weak and strong versions in present-day Swedish, and I have argued that weak \textit{mycket} (\textit{ˌmyke}) has turned into a negative \isi{polarity item}, found in negated clauses, whereas weak \textit{lite(t)} has turned into a \is{positive polarity item (PPI)}positive polarity item, found elsewhere. If we assume that weak versions of quantifiers are more frequent than strong ones, and that \isi{positive polarity} contexts are more frequent than negative ones, we would expect the frequency of \textit{mycken/t} to drop, whereas the frequency of \textit{liten/t} should rise. In \sectref{sec:delsing:5}, I presented a small preliminary study that seems to confirm this prediction.


\section*{Acknowledgements}


I am thankful to the audiences of the Grammar seminar and the Scandinavian seminar in Lund, as well as the participants of the workshop Grammar in Focus in Lund 2019, for comments on empirical and theoretical matters. The paper also gained a lot from discussions with Johan Brandtler, whose insights on polarity items and grammar in general were very valuable.


\section*{Abbreviations}
\begin{tabularx}{.5\textwidth}{@{}lQ}
CE  &  Common Era                    \\
EMS  &  \ili{Early Modern Swedish}         \\
\end{tabularx}\begin{tabularx}{.5\textwidth}{lQ@{}}
NPI &   Negative Polarity Item      \\
PPI &   Positive Polarity Item      \\
\end{tabularx}

\section*{Sources}
\begin{description}[font=\normalfont]
\item[SAOB:] \textit{Ordbok över svenska språket, utg. av Svenska Akademien} [Dictionary of the Swedish language, published by The Swedish Academy]. 1893–. Lund. Available here: \url{http://www.saob.se}
\item[The corpus of Swedish drama dialogue.] 45 original Swedish dramas written between 1725 and 2000. Details of the corpus can be found in \citealt{MarttalaStromquist2001} or \citealt{Stroh-Wollin2008}.
\end{description}



{\sloppy\printbibliography[heading=subbibliography,notkeyword=this]}
\end{document}
