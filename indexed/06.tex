\documentclass[output=paper, colorlinks, citecolor=brown]{langscibook} 
\ChapterDOI{10.5281/zenodo.5792961}

\author{Adrian Sangfelt\affiliation{Uppsala University}}
\title{VP word order variation and verbal clusters in Late Modern Swedish}
\abstract{Some Germanic languages (e.g. German) have a VP structure where multiple verbs behave as an inseparable unit, i.e. a verbal cluster, and some (e.g. English) do not. This seems to be at least partially connected to OV versus VO word order. In this article, I use the Basic Branching Constraint (e.g. \citealt{Haider2013}) and Late Modern Swedish data to argue that clustering is universal if a main verb (V) precedes an auxiliary (Aux), but a language-specific property at most if Aux precedes V. VP word order in the history of Swedish indicates that there is no immediate connection between OV and clustering; on the contrary, as OV disappeared, evidence for cluster breaking clearly dropped in frequency.

\keywords{verbal clusters, OV word order, word order variation, Late Modern Swedish, the Basic Branching Constraint}}

\begin{document}
\SetupAffiliations{mark style=none}
\multicolsep=.25\baselineskip
\maketitle 


\section{Introduction}\label{sec:sangfelt:1}

Varieties within the \ili{Germanic} language family differ with regard to the possibility of having intervening syntactic material between two verbs. As shown in \REF{ex:sangfelt:1a}, \ili{English} rather freely accepts adverbials surfacing between two verbs, while in \ili{German} (see \ref{ex:sangfelt:1b}), \isi{intervening constituents} are ruled out.\footnote{Throughout the article, relevant verbs in the language examples are put in italics.} The difference seems to be connected to the linear order of verbs and objects. According to \citet[17–19, 33–35, 287–293]{Haider2010}, verbs in an OV language (e.g. \ili{German}) form unbreakable verbal clusters that do not allow intervening material, while \isi{clustering} is not an option in VO languages (e.g. \ili{English}).

\ea
\label{ex:sangfelt:1}
\ea \ili{English}\label{ex:sangfelt:1a}\\
The new law {certainly} \textit{may} {possibly} \textit{have} {indeed} \textit{been} {badly} \textit{formulated} \\
\ex \ili{German}\label{ex:sangfelt:1b}\\
\gll dass das neue Gesetz {wohl} {wirklich} {schlecht} \textit{formuliert}(*) \textit{worden}(*) \textit{sein}(*) \textit{mag} \\
that the new law possibly indeed badly \textit{formulated} been be may \\
\glt ‘that presumably the new law indeed may have been badly formulated’ (from \citealt{Haider2010}: 17)
\z
\z 


\ili{Present-day Swedish} is a VO language, so we expect its grammar to allow \isi{intervening constituents}. Older \ili{Swedish}, however, did not show a strict linear order for verbs and objects. It is then reasonable to expect the possibility of \isi{intervening constituents} to be more constrained in older \ili{Swedish} than in its present-day counterpart.


The literature on cluster-related word order in \ili{Swedish} is sparse, but we do find some rather surprising claims. Both \citet[171–172]{Falk1993} and \citet[157]{Petzell2011} state that older \ili{Swedish} allowed constituents to surface between VP-internal verbs, but that this is no longer possible in present-day \ili{Swedish}. It is also striking that we simultaneously find evidence for OV and for non-\isi{clustering} in older \ili{Swedish}. See the examples in (\ref{ex:sangfelt:2}a–b), where the object intervenes between two verbs; \REF{ex:sangfelt:2a} is from the 14\textsuperscript{th} century, and \REF{ex:sangfelt:2b} from the 17\textsuperscript{th} century.\footnote{Throughout the article, I use abbreviations (listed in the Appendix) to refer to older \ili{Swedish} texts. When referring to texts from the corpus of \ili{Swedish} drama dialogue (see \sectref{sec:sangfelt:4.1}), a combination of numbers and letters is used. First symbol = period; second symbol = drama; third symbol = act; fourth and fifth symbol = scene.}


\ea
\label{ex:sangfelt:2}
\ea\label{ex:sangfelt:2a}
\gll at enghin skal \textit{gita} {tik} \textit{lækt} \\
that nobody shall be.able you healed\\
\glt ‘that nobody will be able to heal you’ (MB1B, OS, p. 331)\\

\ex \label{ex:sangfelt:2b}
\gll som iag af honom \textit{ha}’\textit{r} {månge} {wackre} {meddelningar} \textit{ehrhållit}\\
 as I from him have many beautiful messages obtained \\
\glt ‘as I’ve received many beautiful messages from him’ (Columbus, EMS, p. 26)\\
\z
\z


Examples like these indicate that the correlation between OV–\isi{VO word order} and verbal clusters is not as straightforward as \citet{Haider2010} suggests. In this article, I investigate the evidence for non-\isi{clustering} in the history of \ili{Swedish} by looking at cases where an arbitrary \isi{constituent} intervenes between an \isi{auxiliary} and a main verb. My main focus is on the period when \isi{OV word order} finally disappeared, \ili{Late Modern Swedish}. Given \citegen{Haider2010} assumptions, we expect to find growing evidence for non-\isi{clustering} as OV becomes less frequent, but as we will see, this is not really the case.


Behind \citegen{Haider2010} proposal of a connection between verbal clusters and \isi{OV word order} lies an even bigger idea: that all human grammars only contain right-branching structures. This is called the \textit{Basic Branching Constraint} (\citealt{Haider2010, Haider2013}), or the \isi{BBC} for short. The implications of the \isi{BBC} and the meaning of right-branching will be explained and illustrated below. This article seeks to demonstrate, in light of the \isi{BBC}, how \isi{VP} word order variation in \ili{Late Modern Swedish} – compared to other periods in \ili{Swedish} language history and \ili{Germanic} languages in general – may be of interest for a general theory of phrase structure in human languages.


This article is organized as follows: In \sectref{sec:sangfelt:2}, I introduce \isi{the Basic Branching Constraint} and compare it to another proposed restriction on phrase structure in human languages, the \textit{Final-over-Final Condition} (\citealt{BiberauerEtAl2014,SheehanEtAl2017}). \sectref{sec:sangfelt:3} provides some background to clause structure and \isi{VP} word order in \ili{Swedish} and other \ili{Germanic} languages. In \sectref{sec:sangfelt:4}, I present an empirical investigation of \isi{VP} word order in \ili{Late Modern Swedish}, and briefly compare the findings to data from earlier periods. In \sectref{sec:sangfelt:5}, the \ili{Swedish} data is put in a comparative perspective and interpreted in the light of \isi{the Basic Branching Constraint}. \sectref{sec:sangfelt:6} summarizes the findings.


\section{Verbal clusters and the Basic Branching Constraint}\label{sec:sangfelt:2}

\isi{Clustering verbs} is a particular way of structuring a \isi{VP}. If verbs form a cluster, they are considered a syntactic \isi{constituent} that includes multiple verbal heads, which roughly corresponds to the structure [\textsubscript{V°} V° V°]. This is in contrast to an \isi{auxiliary} taking a \isi{VP} as its \isi{complement}, which creates embedded VPs. It is important to note that while verbal clusters exclude intervening material, the converse does not hold: the fact that two verbs are adjacent does not necessarily mean that they form a cluster (see e.g. \citealt{Sheehan2017Final}: 101).\footnote{Thus, a \isi{cluster analysis} is in some sense unverifiable, due to the lack of positive evidence. It is certainly falsifiable, however, since the analysis straightforwardly predicts that no non-verbal material should intervene between verbs that form a cluster.}


Verbal clusters are an important concept for understanding how \isi{the Basic Branching Constraint} can explain limits on word order variation in languages of the world, as we are about to see. I begin this section by introducing the \isi{BBC} and some of its implications (\sectref{sec:sangfelt:2.1}). In \sectref{sec:sangfelt:2.2}, I discuss whether or not verbal clusters (or what appear to be verbal clusters) can be seen as a side effect of \isi{the Final-over-Final Condition}.


\subsection{The Basic Branching Constraint}\label{sec:sangfelt:2.1}

A version of \isi{the Basic Branching Constraint} first saw the light of day in \citet{Haider1992}. More recently, \citet[3]{Haider2013} defines the \isi{BBC} as in \REF{ex:sangfelt:3} below (emphasis in the original):

\ea
\label{ex:sangfelt:3}
\isi{The Basic Branching Constraint} (\isi{BBC})\\
The structural build-up (\textit{merger}) of phrases (and their functional extensions) is 
universally \textit{right branching}
\z 


The key to understanding the \isi{BBC}, as stated above, is the concept of \textit{right-branching}. A structure is right-branching if merger of phrases occurs to the left, i.e. if a phrase that enters the structure precedes the already existing structure (see \citealt{Haider2013}: 3–4). The order of a lexical head (e.g. V°, N°) and a \isi{complement} is not subject to this restriction, but both head-\isi{complement} and complement-head order are possible. To illustrate this, let us assume that α is a (non-lexical) head in the phrase αP, taking βP as its \isi{complement}. βP consists of a (lexical) head β and a \isi{complement}. In \REF{ex:sangfelt:4a}, α is merged to the left of a \isi{head-initial} βP, which leads to a right-branching structure. In \REF{ex:sangfelt:4b}, α is merged to the right of a \isi{head-initial} βP, which leads to a left-branching structure. (4c–d) show that the same applies if βP is \isi{head-final}, with Comp-β word order instead of β{}-Comp.


\ea
\label{ex:sangfelt:4} 
\columnsep=-1ex
\begin{multicols}{4}
\ea \label{ex:sangfelt:4a}
\begin{forest}
	[αP
		 [α]
		  [βP 
		  	 [β] 
		  	 [Comp]
  	 	 ]
	 ]
\end{forest}

\ex\label{ex:sangfelt:4b} 
\begin{forest}
	[αP 
		[βP 
			[β] 
			[Comp]
			]
		[α] 
	]	
\end{forest}

\ex \label{ex:sangfelt:4c} 
\begin{forest}
	[αP
		[α]
		[βP
			[Comp] 
			[β]
		]
	]
\end{forest}

\ex \label{ex:sangfelt:4d} 
\begin{forest}
		[αP
			[βP
				[Comp]
				[β]
			]
			[α]
		]
\end{forest}

\z 
\end{multicols}
\z

         

Given the \isi{BBC} and the structures in (\ref{ex:sangfelt:4}a–d), we are not far from seeing why verbs in an OV language like \ili{German} must form verbal clusters. Let us first state that an \isi{auxiliary} Aux corresponds to α in \REF{ex:sangfelt:4}, and a main verb V corresponds to β, provided that auxiliaries are functional extensions of a lexical \isi{VP} (see \citealt{Haider2013}: 68–73). This means that the \ili{German} word order \isi{V-Aux} cannot instantiate an \isi{auxiliary} that takes a \isi{VP} as its \isi{complement}, according to the \isi{BBC}; this would correspond to a left-branching structure (see \ref{ex:sangfelt:4b}, \ref{ex:sangfelt:4d}). Instead, V and Aux adjoin to each other and form a \isi{verbal cluster}.

An immediate consequence of a \isi{cluster structure} is that no syntactic material should be able to intervene between two verbs if both remain in a non-derived position. This is true for \ili{German}. Usually, verbal complements and adjuncts precede all verbs. If a \isi{constituent} is extraposed, as PPs can be, for example, they must follow both the main verb and auxiliaries. This is shown in (\ref{ex:sangfelt:5}a–b). As seen in \REF{ex:sangfelt:5b}, the PP cannot be placed between V and Aux.


\ea \ili{German}\label{ex:sangfelt:5}
\ea[]{\gll dass er nicht \textit{gesprochen} \textit{haben} \textit{kann} {mit} {ihr} \\
that he not spoken have can with her\\
\glt ‘that he cannot have spoken with her’\label{ex:sangfelt:5a}}

\ex[*]{\gll dass er nicht \textit{gesprochen} {mit} {ihr} \textit{haben} \textit{kann}\\
 that he not spoken with her have can \citep[18]{Haider2010}\\\label{ex:sangfelt:5b}}
\z 
\z 

This leads \textcites{Haider2003}[17–18, 335–343]{Haider2010}[90–93, 132–135]{Haider2013} to assume that a \isi{VP} in an OV language is usually structured as in \REF{ex:sangfelt:6a}, where X can be a \isi{complement} or an \isi{adjunct} to the verb. The word order X-\isi{V-Aux} is not an example of an \isi{auxiliary} taking a \isi{VP} \isi{complement} to its left, but of X being in a left-hand sister position to the cluster. The analysis is also extended to OV languages where Aux can precede V, with X-Aux-V as the VP-internal word order, like \ili{Dutch} (see \cites[341–343]{Haider2010}[133]{Haider2013}). The only difference between \REF{ex:sangfelt:6a} and \REF{ex:sangfelt:6b} is the cluster-internal ordering of V and Aux.


\ea
\label{ex:sangfelt:6}
\ea {[\textsubscript{VP} X [\textsubscript{V°} V+Aux]]} \jambox{X-\isi{V-Aux} – V and Aux form a verbal cluster}\label{ex:sangfelt:6a}
\ex {[\textsubscript{VP} X [\textsubscript{V°} Aux+V]]} \jambox{X-Aux-V – Aux and V form a verbal cluster}\label{ex:sangfelt:6b}
\z
\z 


 Haider’s position raises at least two types of questions. First, we must ask to what extent there is empirical evidence for a one-to-one correlation between \isi{OV word order} and verbal clusters in the languages of the world. This \isi{question} will be discussed throughout the article, mostly with reference to older \ili{Swedish} and the other \ili{Germanic} languages, but also including some typological observations in \sectref{sec:sangfelt:5}.

Secondly, we should clarify what the \isi{BBC} requires when it comes to verbal clusters, and what might be unexpected given \citeauthor{Haider2010}’s (\citeyear{Haider2010, Haider2013}) analysis, but not necessarily impossible. In this case, it is obvious that the word orders \isi{V-X-Aux} and \isi{Aux-X-V} have different status. \isi{V-X-Aux} is an inherently left-branching structure if V and X are inside a \isi{complement} to Aux. The word order \isi{Aux-X-V} would typically correspond to a right-branching, BBC-compatible structure, since the extended verbal projection has its \isi{complement} to the right. This important difference is shown in (\ref{ex:sangfelt:7}a–b).


\ea
\label{ex:sangfelt:7}
\ea{[[\textsubscript{VP1} [\textsubscript{VP2} V X]] Aux]\label{ex:sangfelt:7a}}\jambox*{Left-branching}
\ex{[\textsubscript{VP1} Aux [\textsubscript{VP2} X V]\label{ex:sangfelt:7b}}\jambox*{Right-branching}
\z 
\z 

To sum up, the \isi{BBC} excludes the right-branching structure [[V X] Aux] and forces a \isi{cluster analysis} if a main verb precedes an \isi{auxiliary}. \isi{V-Aux} is certainly associated with \isi{OV word order}, since both are examples of head finality, but in the light of the \isi{BBC}, it is the \isi{V-Aux} word order itself that necessitates the \isi{verbal cluster}. Aux-V, on the other hand, is associated with \isi{VO word order}, but regardless of the order of verb and object, a \isi{cluster analysis} is not a structural necessity.

\subsection{A note on the BBC and the FOFC}\label{sec:sangfelt:2.2}

The \isi{BBC} is a hypothesis which makes clear and falsifiable predictions about syntactic structure in human languages (see \citealt{Haider2013}). If correct, it is part of universal grammar. Despite its merits, the \isi{BBC} does not seem to have been extensively investigated by anyone but Haider himself. A possible reason for this is that there is another, arguably more well-known, proposal about phrase structure configurations in human languages, which makes several predictions overlapping those of the \isi{BBC}, namely the \textit{Final-over-Final Condition} (\isi{FOFC}), first introduced by \citet{Holmberg2000}.


\citet[171]{BiberauerEtAl2014} informally state the \isi{FOFC} as in \REF{ex:sangfelt:8}: 

\ea \isi{The Final-over-Final Condition} (\isi{FOFC})\label{ex:sangfelt:8}\\
A \isi{head-final} phrase αP cannot immediately dominate a \isi{head-initial} phrase βP, where
α and β are heads in the same extended projection.
\z 

To illustrate the implications of the \isi{FOFC}, let us repeat the phrase structure configurations in (\ref{ex:sangfelt:4}a–d) above; see (\ref{ex:sangfelt:9}a–d). In \REF{ex:sangfelt:9a}, the \isi{head-initial} phrase αP dominates the \isi{head-initial} phrase βP, a configuration accepted by the \isi{FOFC}. In \REF{ex:sangfelt:9b}, the \isi{head-initial} βP is dominated by a \isi{head-final} αP; this is the Final-over-Initial structure that the \isi{FOFC} is formulated to exclude from human grammars. The structures in (\ref{ex:sangfelt:9}c–d) are not excluded, since a \isi{head-initial} phrase can dominate a \isi{head-final} phrase \REF{ex:sangfelt:9c}, and a \isi{head-final} phrase can dominate another \isi{head-final} phrase \REF{ex:sangfelt:9d}.

\settowidth\jamwidth{Initial-over-Initial}
\ea\label{ex:sangfelt:9}
\ea[]{[\textsubscript{αP} α [\textsubscript{βP} β Comp]] \jambox{Initial-over-Initial}\label{ex:sangfelt:9a}} 
\ex[*]{[\textsubscript{αP} [\textsubscript{βP} β Comp] α] \jambox{*Final-over-Initial}\label{ex:sangfelt:9b}}
\ex[]{[\textsubscript{αP} α [\textsubscript{βP} Comp β]] \jambox{Initial-over-Final}\label{ex:sangfelt:9c}}
\ex[]{[\textsubscript{αP} [\textsubscript{βP} Comp β] α] \jambox{Final-over-Final}\label{ex:sangfelt:9d}}
\z 
\z 


Like the \isi{BBC}, the \isi{FOFC} excludes \isi{V-X-Aux}, given that V is β, X is Comp, and α is Aux. The difference primarily concerns what could be a possible structure for the word order X-\isi{V-Aux}. The \isi{FOFC} does not exclude the left-branching/Final-over-Final structure in \REF{ex:sangfelt:9d}, but the \isi{BBC} does. Consequently, the \isi{FOFC} does not force a \isi{cluster analysis} of \isi{V-Aux} (i.e. [X [V+Aux]]), and the \isi{adjacency} requirement of V and Aux becomes more of an epiphenomenon. \citet[132--133]{Haider2013} claims that the \isi{BBC} is empirically superior to the \isi{FOFC}, when it comes to predicting possible word orders with verbs and complements\slash adjuncts in \ili{Germanic} languages. According to \citet[133]{Haider2013}, the \isi{FOFC} predicts the existence of the structure in \REF{ex:sangfelt:10}, where an \isi{adjunct} but not a \isi{complement} may intervene between V and Aux.


\ea {}[\textsubscript{VP1} [\textsubscript{VP2} {Comp} {V} {Adjunct}] {Aux}]\label{ex:sangfelt:10}\z


That the \isi{FOFC} would tolerate the structure in \REF{ex:sangfelt:10} could be true, but only if the constraint is formulated exclusively with regard to complementation and not adjunction. It is certainly true that proponents of the \isi{FOFC} have focused on the former rather than the latter \citep[97]{Sheehan2017Final}, but \citet{Sheehan2017Final} explicitly suggests that the \isi{FOFC} is a constraint that involves both types of merger. Thus, if the \isi{FOFC} is formulated with respect to both complementation and adjunction, the \isi{FOFC} and the \isi{BBC} make identical empirical predictions regarding possible orders of V, Aux, and X.

Despite overlapping empirical predictions, I consider the \isi{BBC} to be a more straightforward constraint than the \isi{FOFC} for deriving the limitations of \isi{VP} word order in \ili{Germanic} languages.\footnote{This is not to say that the \isi{BBC} and the \isi{FOFC} generally make identical predictions, despite a significant overlap. For an overview of the various cross-linguistic implications of the \isi{BBC} and the \isi{FOFC}, see \citet[10–17, 65–94]{Haider2013} and \citet[173–205]{BiberauerEtAl2014}, respectively.} My reasoning goes as follows: The ban on \isi{V-X-Aux} follows directly from the properties of the \isi{BBC} as a universal constraint on phrase structure configurations. The same cannot be said of the \isi{FOFC}. Rather, the \isi{FOFC} is an independently formulated constraint that needs to be interpreted within a more general model of phrase structure (see \citealt{Haider2013}: 132–135, \citealt[205–215]{BiberauerEtAl2014}). In \citet{BiberauerEtAl2014}, the \isi{FOFC} is implemented within \citegen{Kayne1994} \textit{Linear Correspondence Axiom}.\footnote{For other implementations of the \isi{FOFC}, see the overview in \citet{Holmberg2017}.}  Since the \isi{LCA} postulates that all phrases in all languages have an underlying Spec-Head-Comp structure, the \isi{FOFC} becomes a restriction on \isi{movement} within a model that could easily derive the \isi{V-X-Aux} word order. What the \isi{BBC} handles in one step, the \isi{FOFC} handles in two. For the \isi{FOFC} to be justified as an independent constraint, it needs to be shown why it is more empirically adequate than the \isi{BBC}, and not just equally adequate.


\section{Clause structure and VP word order variation in Germanic languages}\label{sec:sangfelt:3}

In this section, I provide some background to clause structure and \isi{VP} word order variation in the \ili{Germanic} languages. This description serves two purposes. First, it gives a basic empirical introduction to the subject of the article. Secondly, it clarifies in what syntactic environments we should look when trying to discriminate between possible verbal clusters and multiple embedded VPs. \sectref{sec:sangfelt:3.1} deals with the basic clause structure in the \ili{Germanic} language family, \sectref{sec:sangfelt:3.2} with \isi{VP} word order in \ili{Swedish}, and \sectref{sec:sangfelt:3.3} with \isi{VP} word order in the West \ili{Germanic} OV languages. In \sectref{sec:sangfelt:3.4}, I give an intermediate summary before the empirical investigation of \isi{VP} word order in \ili{Late Modern Swedish}.

\subsection{Clause structure in Germanic languages}\label{sec:sangfelt:3.1}

A basic property of \ili{Germanic} clause structure, shared by all varieties except \ili{English}, is the requirement for \isi{V2 word order}. In a \isi{declarative} \isi{main clause}, the finite verb must be spelled out as the second \isi{constituent}. Since \citet{denBesten1983}, the standard generative analysis of V2 has been that a finite verb moves from its base position in V° to the head of the highest functional projection of the clause: C°. The first \isi{constituent} of the clause is found in spec-\isi{CP}, and this position can be occupied by several phrasal types (e.g. NP, PP, AP, AdvP, or \isi{VP}) with different syntactic functions. In (\ref{ex:sangfelt:11}a–b), the V2 property is exemplified with parallel present-day \ili{Swedish} and \ili{German} sentences.

\ea
\label{ex:sangfelt:11}
\begin{multicols}{2}\raggedcolumns
\ea \ili{Present-day Swedish}\\
\gll Igår brann huset. \\
 Yesterday burned house.\textsc{def}\\\label{ex:sangfelt:11a}\columnbreak
 
\ex \ili{German}\\
\gll Gestern brannte das Haus.\\
 Yesterday burned the house\\
\glt ‘Yesterday, the house was on fire.’\label{ex:sangfelt:11b}
\z 
\end{multicols}
\z 

Verb \isi{movement} to C° is confined to finite verbs in main clauses and a restricted set of (finite) embedded clauses, while other verbs seem to stay in V°. For \ili{Swedish}, this can be shown by negating a clause, assuming that negation must be (externally) merged above the base position of all verbs (see e.g. \citealt{Zeijlstra2013}). As seen in (\ref{ex:sangfelt:12}a–b), a negation precedes the non-finite verb in a \isi{main clause}, and the finite verb in an \isi{embedded clause}.


\ea \ili{Present-day Swedish}\label{ex:sangfelt:12}
\begin{multicols}{2}\raggedcolumns
\ea \label{ex:sangfelt:12a}
\gll Du får inte läsa boken. \\
 You may not read book.\textsc{def}\\
\glt ‘You may not read the book.’\\

[\textsubscript{CP} Du får\textsubscript{v} inte [\textsubscript{VP} t\textsubscript{v} läsa boken]]\columnbreak

\ex\label{ex:sangfelt:12b}
\gll om du inte läser boken \\ 
 if you not read book.\textsc{def}\\
\glt ‘if you don’t read the book’\\

[\textsubscript{CP} om du inte [\textsubscript{VP} läser boken]]
\z 
\end{multicols}
\z 

The placement of negation and similar \isi{sentence adverbials} can always be used to determine the position of a verb in present-day \ili{Swedish}, since they must precede V°. Other adjuncts (e.g. adverbials of time, manner and place) optionally precede a verb, while complements are strictly post-verbal (disregarding well-established instances of leftward \isi{movement}, like \isi{topicalization}). In the OV language \ili{German}, other categories than \isi{sentence adverbials} can be used as a diagnostic tool to decide whether a verb is in C° or in V°, including objects. As seen in (\ref{ex:sangfelt:13}a–b), objects precede non-finite verbs in main clauses and finite verbs in embedded clauses.\footnote{The properties of \ili{Swedish} are representative of present-day North \ili{Germanic} languages, and the properties of \ili{German} are representative of continental West \ili{Germanic} languages, but two notable exceptions should be mentioned. \ili{Icelandic} requires a finite verb to move in embedded clauses as well, to a functional projection between \isi{VP} and \isi{CP} (\isi{IP}/\isi{AgrP}/TP). \ili{Yiddish} also displays \isi{movement} in embedded clauses, and is a West \ili{Germanic} language with both OV and \isi{VO word order}. For further information and discussion, see e.g. \citet[3–18]{Vikner2001}.}

% language information on top
\ea \ili{German}\label{ex:sangfelt:13}
\ea\label{ex:sangfelt:13a}
\gll Du kannst das Buch lesen.\\
 You may the book read\\
\glt ‘You may read the book.’\\

[\textsubscript{CP} Du kannst\textsubscript{v} [\textsubscript{VP} {das} {Buch} lesen t\textsubscript{v}]]

\ex\label{ex:sangfelt:13b}
\gll wenn du das Buch liest \\
 if you the book read\\

[\textsubscript{CP} wenn du [\textsubscript{VP} {das} {Buch} liest]]\\

\glt ‘if you read the book’
\z 
\z 

An immediate consequence of \ili{Germanic} \isi{V-to-C} \isi{movement} is that we cannot use main clauses with only a finite \isi{auxiliary} to study the existence of verbal clusters. In the \ili{Swedish} example in \REF{ex:sangfelt:14a}, we cannot know whether the \isi{auxiliary} has moved from a position to the left or the right of the \isi{adverb}, if both possibilities can be shown to exist. The ambiguity is illustrated in (\ref{ex:sangfelt:14}b).

\ea \ili{Present-day Swedish}\label{ex:sangfelt:14}
\ea \label{ex:sangfelt:14a}
\gll Huvudvärk kan {snabbt} försvinna.  \\
 Headache can quickly disappear \\
\glt ‘A headache can quickly disappear.’
\ex\label{ex:sangfelt:14b}
[\textsubscript{CP} Huvudvärk kan\textsubscript{v} [\textsubscript{VP} snabbt t\textsubscript{v} försvinna]]

[\textsubscript{CP} Huvudvärk kan\textsubscript{v} [\textsubscript{VP} t\textsubscript{v} snabbt försvinna]]
\z 
\z

To study VP-internal word order variation in the \ili{Germanic} V2 languages, we must turn to environments where an \isi{auxiliary} is realized VP-internally. Roughly, this means clauses with at least one non-finite \isi{auxiliary}, or embedded clauses with a finite \isi{auxiliary} (if we regard auxiliaries as part of (the extended) \isi{VP}, as I do throughout the article; see \sectref{sec:sangfelt:2.1}).\footnote{I will not discuss the possibility that embedded clauses in West \ili{Germanic} are instances of I°-final structures. For conceptual criticism and empirical evidence against such a view, see \citet[54–68]{Haider2010}.} In the remaining parts of \sectref{sec:sangfelt:3}, I turn to these syntactic environments.

\subsection{VP word order in Swedish}\label{sec:sangfelt:3.2}

As mentioned in the introduction, not much has been said about the grammaticality of adverbials intervening between an \isi{auxiliary} and a main verb in present-day \ili{Swedish}. \citet[157]{Petzell2011} has claimed, however, that the sentence in \REF{ex:sangfelt:15a}, where an \isi{adverbial} PP intervenes between VP-internal verbs, is ungrammatical. \REF{ex:sangfelt:15b} is fine, on the other hand, with the \isi{adverbial} preceding both verbs.\footnote{It is actually noteworthy that, as shown in example \REF{ex:sangfelt:15b}, pre-\isi{verbal adjunct} PPs are acceptable in present-day \ili{Swedish}, and not only pre-verbal adverbs\slash \isi{adverbial} phrases. There are claims in the literature that a head of an \isi{adjunct} modifying a \isi{head-initial} phrase must be (linearly) phrase-final, if the \isi{adjunct} precedes the head of the phrase it modifies (see e.g. \citet{Haider_left-left_nodate} for recent discussion). This would exclude the word order PP-V if the \isi{VP} is \isi{head-initial}, as are present-day \ili{Swedish} VPs. The restriction does seem to hold, however, for attributive adjectives modifying a noun; see (i). 
\ea[*]{
      \gll En  snabbare  än  dig  person\\
      a  faster    than  you  person\\\jambox*{(\ili{Present-day Swedish})}
      \glt Intended reading: ‘a faster person than you’}
\z}

\ea \ili{Present-day Swedish}\footnote{The judgement of the sentence in \REF{ex:sangfelt:15a} reflects \citegen{Petzell2011} view and not my own. Personally, I find this sentence acceptable, albeit a bit awkward. In contrast to \citet[157]{Petzell2011}, I have added a negation between the subject and the finite verb to disambiguate from embedded \isi{V-to-C} (see further \sectref{sec:sangfelt:4.1}).}
\label{ex:sangfelt:15}
\ea[*]{\label{ex:sangfelt:15a}
\gll att han inte \textit{hade} {under} {eftermiddagen} \textit{cyklat} två  mil\\
 that he not had during afternoon.\textsc{def} bicycled two miles\\
\glt ‘that he hadn’t bicycled two (\ili{Swedish}) miles during the afternoon’ \\}

\ex[]{\label{ex:sangfelt:15b}
\gll att han {under} {eftermiddagen} \textit{hade} \textit{cyklat} två mil\\
that he during afternoon.\textsc{def} had bicycled two miles\\}
\z 
\z 


The ungrammaticality of the word order in \REF{ex:sangfelt:15a} is not, however, as straightforward as \citet{Petzell2011} would have it. Teleman et al. (1999/3: 488) note that an \isi{adverbial} of manner can separate two VP-internal verbs; the examples in (16a–b) illustrate this possibility.


\ea \ili{Present-day Swedish}\label{ex:sangfelt:16}
\ea\label{ex:sangfelt:16a}
\gll [d]e som inte \textit{har} {medvetet} \textit{upplevat } 1930- och 1940-talen \\
 those that not have consciously experienced 1930- and 1940-number\textsc{.pl.def}\\
\glt ‘those who haven’t consciously experienced the 1930s and the 1940s’

\ex\label{ex:sangfelt:16b}
\gll genom att \textit{låta} {kritiskt} \textit{granska} förslagen \\
 by to let critically examine proposal.\textsc{pl.def}\\
 \glt ‘by letting the proposals be critically examined’ (\citealt{TelemanEtAl1999}/3: 488)\\
\z
\z

Thus, \isi{Aux-X-V}, where X is an \isi{adverbial}, seems to be an option in present-day \ili{Swedish}, at least marginally and for some speakers. This conforms with my \isi{intuition} as a native \isi{speaker} of (Central) \ili{Swedish}: \isi{Aux-X-V} is a possible but sometimes not entirely natural word order. At the same time, it is clear that not much is known about possible restrictions on the word order. A detailed study of \isi{Aux-X-V} is beyond the scope of this article, but in \sectref{sec:sangfelt:4.3}, I make some more observations regarding the word order in present-day \ili{Swedish}.

It is not difficult to confirm that \isi{Aux-X-V} exists in older \ili{Swedish}. The examples in (\ref{ex:sangfelt:17}a–d), all from texts written around 1500, indicate that \isi{Aux-X-V} is possible for adjuncts (\ref{ex:sangfelt:17}a–b) – adverbs as well as PPs – for NP objects \REF{ex:sangfelt:17c} and for PP complements \REF{ex:sangfelt:17d}. Their existence has been noted by \citet[171--172]{Falk1993}, for example, but I know of no investigation that focuses on the distribution of \isi{Aux-X-V} in older \ili{Swedish}.


\ea \ili{Old Swedish}\label{ex:sangfelt:17}
\ea\label{ex:sangfelt:17a}
\gll Thiänaren sagdhe sik thz \textit{haffwa} {offta} \textit{giort} \\
 Servant.\textsc{def} said \textsc{refl} it have often done\\
\glt ‘The servant said that he had often done it’ (LinLeg, p. 306)
\ex\label{ex:sangfelt:17b}
\gll at han skulde \textit{haffwa} i {thino} {farahws} \textit{giort} skada \\
 that he should have in your sheep.house done harm\\
\glt ‘that he should have done harm in your sheep house’ (SpecV, p. 486)

\ex\label{ex:sangfelt:17c}
\gll at wij skuldom \textit{haffua} {priis} \textit{fongit} \\
 that we should have price got\\
\glt ‘that we should have got a price' (Di, p. 200)

\ex\label{ex:sangfelt:17d}
\gll at wi skullom \textit{haffwa} {om} {jomfrunnar} \textit{taladh} \\
 that we should have about virgin.\textsc{pl.def} spoken\\
\glt ‘that we should have spoken about the virgins’ (SpecV, p. 334)
\z 
\z 

In other words, the possibility for non-verbal material to intervene between verbs seems to have been quite unrestricted in older \ili{Swedish}. One might suspect that this would mean that we might also find examples of the \isi{V-X-Aux} word order, contrary to what the \isi{BBC} predicts. To the best of my knowledge, though, no one has ever made such a claim. \citet[155, 158–160]{Petzell2011}, who studies different types of \isi{OV word order} in older \ili{Swedish}, notes that out of four possible combinations of an \isi{auxiliary}, a main verb, and an argument (including NP objects and predicatives), with the argument preceding at least one of the verbs, only three can be found: Arg-Aux-V, Aux-Arg-V and Arg-\isi{V-Aux}, but not V-Arg-Aux. He does not specify whether or not this restriction holds for adjuncts as well as for arguments, but as we will see in \sectref{sec:sangfelt:4}, this is likely to be the case.

\subsection{VP word order in West Germanic languages}\label{sec:sangfelt:3.3}

As already shown in §§\ref{sec:sangfelt:1} and \ref{sec:sangfelt:2}, \ili{German} does not accept constituents between a main verb and an \isi{auxiliary} if the verbs remain in \isi{VP}. The same restriction seems to apply to \ili{Dutch} (\citealt[290--291]{Haider2010}, see also \citealt{Wurmbrand2004}). As shown in (\ref{ex:sangfelt:18}a–b), a \isi{constituent} cannot intervene between an \isi{auxiliary} and a main verb \REF{ex:sangfelt:18a}, or between two VP-internal auxiliaries \REF{ex:sangfelt:18b}. Importantly, this is true regardless of the number and order of verbal elements. Both examples have Aux-V word order, and in \REF{ex:sangfelt:18b} the selecting \isi{auxiliary} precedes the selected one.

\ea \ili{Dutch}\label{ex:sangfelt:18}
\ea[*]{\label{ex:sangfelt:18a}
\gll dat hij graag \textit{wilde} {kraanvogels} \textit{fotograferen} \\
 that he gladly wanted cranes photograph \\
 \glt Intended reading: ‘that he gladly wanted to photograph cranes’}

\ex[*]{\label{ex:sangfelt:18b}
\gll dat hij \textit{zal} {naar} {huis} \textit{willen} \textit{gaan} \\
 that he will to home want go\\
\glt Intended reading: ‘that he’ll want to go home’ (from \citealt[291]{Haider2010})}
\z 
\z

There are nevertheless combinations of three verbs in standard \ili{German} that show that this restriction is not as clear-cut as the \ili{Dutch} data might suggest. In what are known as IPP constructions,\footnote{IPP = \textit{Infinitivus Pro Participio} (infinitival instead of \isi{participle}) – an \isi{auxiliary} \textit{haben} taking another \isi{auxiliary} in infinitival form instead of a \isi{participle} (see e.g. \citealt[46–48]{Wurmbrand2004}).} the unmarked order is not V-Aux2-Aux1, as is usually the case, but Aux1-V-Aux2.\footnote{As is common practice, Aux1 denotes the highest \isi{auxiliary}, which has all other verbs inside its \isi{complement}; Aux2 denotes the second highest, and so on. For further discussion and information on the relative order of verbs in different West \ili{Germanic} varieties, see e.g. \citet{Wurmbrand2004,Sapp2011,Culicover2014}.} Here, constituents can be placed between the finite \isi{auxiliary} and the two non-finite verbs, as seen in \REF{ex:sangfelt:19}. In \REF{ex:sangfelt:19a}, an object and a PP are found between the auxiliaries, and \REF{ex:sangfelt:19b} has an \isi{adverb} in the corresponding position. Thus, if a VP-internal \isi{auxiliary} takes a \isi{VP} \isi{complement} on its right-hand side, the \isi{adjacency} requirement of VP-internal verbs disappears (see \citealt[97–102]{Sheehan2017Final}, see also \cites[]{Haider2003}[132–135]{Haider2013}).


\ea \ili{German}
\label{ex:sangfelt:19}
\ea\label{ex:sangfelt:19a}
\gll dass er für ihn nicht \textit{hatte} {die} {Firma} am Leben \textit{halten} \textit{wollen} \\
 that he for him not have.\textsc{pst} the company at.\textsc{def} life keep.\textsc{inf} want.\textsc{inf} \\
\glt ‘that he had not wanted to keep the company alive for him’ (from \citealt{Haider2013}, 128)

\ex\label{ex:sangfelt:19b}
\gll dass er das Buch \textit{hätte} {genau} \textit{durchsehen} \textit{sollen}\\
 that he the book have.\textsc{pst} carefully through.seen.\textsc{inf} shall.\textsc{inf} \\
\glt ‘that he should have looked through the book carefully’ (from \citealt{Sheehan2017Final}: 101)
\z 
\z

Furthermore, dialects of both \ili{German} and \ili{Dutch} show that verbal complexes with two VP-internal verbs can indeed have \isi{intervening constituents}, if the \isi{auxiliary} precedes the main verb (see e.g. \citealt{Sapp2011}: 124–129). This is illustrated in (20a–c), with objects intervening between the verbs. \REF{ex:sangfelt:20a} is representative of the \ili{German} spoken, for example, in Vienna \citep[128]{Haider2013}, \REF{ex:sangfelt:20b} is Swiss \ili{German}, and \REF{ex:sangfelt:20c} West Flemish.


\ea 
\label{ex:sangfelt:20}
\ea Dialectal \ili{German}\label{ex:sangfelt:20a}\\
\gll Man hätte \textit{müssen} {die} {Polizei} \textit{verständigen} \\
 One have.\textsc{pst} must.\textsc{inf} the police call.\textsc{inf} \\
\glt ‘People were forced to call the police’ (from \citealt{Haider2013}: 128)

\ex Swiss \ili{German}\label{ex:sangfelt:20b}\\
\gll das si am Grendel \textit{wöt} {sine} {verlore} {chlause} \textit{zruggeh}\\
 that she to.\textsc{def} Grendel wanted his lost claw return \\
\glt ‘that she wanted to return his lost claw to Grendel’ (from \citealt{Haider2013}: 128)

\ex West Flemish\label{ex:sangfelt:20c}\\
\gll da Jan vuor Marie \textit{wil} {da} {boek} \textit{kuopen} \\
 that Jan for Marie wants that book buy \\
\glt ‘that Jan wants to buy the book for Marie’ (from \citealt{Haegeman1992}: 181)\\
\z
\z

To conclude, West \ili{Germanic} OV languages conform to what has already been indicated for older \ili{Swedish}. Verbs must be adjacent if a verb that is selected by another verb precedes the selecting verb. Typically, though not exclusively, this applies to a main verb selected by an \isi{auxiliary}. Even though standard \ili{Dutch} is also in line with this generalization, it is something of an exception; intervening material with either \isi{V-Aux} or Aux-V word order is not accepted.

\subsection{Intermediate summary}\label{sec:sangfelt:3.4}

It has now been established that \isi{Aux-X-V} is a possible word order in several \ili{Germanic} varieties, with both OV and VO clause structure. If we disregard word orders where X is preceded by both Aux and V, then \isi{Aux-X-V} seems to be in competition with X-Aux-V and X-\isi{V-Aux}. The word order \isi{V-X-Aux} does not exist in West \ili{Germanic} varieties and possibly not in older \ili{Swedish} either. However, not much is known about the diachronic development of \isi{Aux-X-V} in older \ili{Swedish}. This is the focus of the upcoming section. As mentioned, I focus on \ili{Late Modern Swedish}, i.e. the time when \isi{OV word order} finally disappeared from older \ili{Swedish} texts (see \citealt{Platzack1983,Petzell2011,Sangfelt2019}). The development leading up to \ili{Late Modern Swedish} is nevertheless of interest, and I therefore make some observations about \ili{Early Modern Swedish}, using data from \citet{Sangfelt2019}.

\section{The development of Aux-X-V word order in the history of Swedish}\label{sec:sangfelt:4}

In this section, I analyse the development of \isi{Aux-X-V} word order in the history of \ili{Swedish}. I begin by presenting the data sources and discussing how \isi{Aux-X-V} should be more narrowly defined (\sectref{sec:sangfelt:4.1}). I then present the development of \isi{Aux-X-V} in \ili{Late Modern Swedish} (\sectref{sec:sangfelt:4.2}) and go on to look at Early Modern and present-day \ili{Swedish} (\sectref{sec:sangfelt:4.3}). In \sectref{sec:sangfelt:4.4}, I summarize and discuss the findings.

\subsection{Defining the Aux-X-V word order in Late Modern Swedish data sources}\label{sec:sangfelt:4.1}

To study the diachronic development of \isi{Aux-X-V} word order, I use the corpus of \ili{Swedish} drama dialogue (see \citealt{MarttalaStromquist2001} for a description of the corpus). The drama corpus is a suitable source of data with regard to the aim of this article for at least two reasons. First, the corpus contains language use from a \isi{genre} that should reflect relatively closely the spoken language in \isi{Central Sweden} during the 18\textsuperscript{th} and 19\textsuperscript{th} centuries. As the name indicates, the texts mostly consist of dialogue. Extensive use of archaic and/or formulaic patterns, not representative of the grammar of the time, is therefore not expected (see e.g. \citealt{Fischer2007}: 12–14).

Secondly, the corpus is well suited to diachronic research, because it is divided into six periods of 25 years each. The first three periods each contain five dramas, derived from texts written during the years 1725–1750, 1775–1800, and 1825–1850. The corpus is digitized but not syntactically annotated, and the extraction has therefore been carried out manually.


The investigation aims to study the frequency and development of \isi{Aux-X-V}, i.e. cases where an arbitrary \isi{constituent} intervenes between a VP-internal \isi{auxiliary} and a main verb. We therefore need to know what verbs count as auxiliaries, how we can determine that an \isi{auxiliary} has not left its VP-internal position, and how we can estimate the frequency of the word order. With respect to what verbs are included among the auxiliaries, I follow \citet[162--163]{Delsing1999}. This implies that the category “\isi{auxiliary}” is lexically specified, and includes the verbs in \REF{ex:sangfelt:21}.\footnote{The decision to follow \citet{Delsing1999} could be questioned, since his reasoning is empirically grounded on \ili{Old Swedish} and \ili{Early Modern Swedish} data. However, relevant examples with other potential auxiliaries, like \textit{behöva} ‘need’ and \textit{börja} ‘start’, are very scarce in the corpus. Hence, the decision only has minor effects on the quantitative results. It should also be noted that \citet{Delsing1999} includes the verbs \textit{gita} ‘be able to’, \textit{ägha} ‘be obliged to’, \textit{mona} ‘intend to, be going to’, and \textit{plägha} ‘tend to’, but these verbs are not used as auxiliaries in \ili{Late Modern Swedish}.}


\ea
\label{ex:sangfelt:21}
\textit{hava} ‘have’,
\textit{kunna} ‘can, be able to’,
\textit{vilja} ‘want’,
\textit{skola} ‘shall, be going to’, 
\textit{få} ‘may, get’,
\textit{magha} ‘may, be able to’,
\textit{måste} ‘must, have to’
\z 


The \isi{question} of when an \isi{auxiliary} has not left its VP-internal position has already been given a partial answer in \sectref{sec:sangfelt:3.1}. As stated there, main clauses with only a finite \isi{auxiliary} are of no interest, since the \isi{auxiliary} has moved to C°. If, however, a \isi{main clause} also contains a non-finite \isi{auxiliary}, we will be able to determine whether or not a non-verbal \isi{constituent} is situated above or below the base position of the said \isi{auxiliary}. Thus, in \REF{ex:sangfelt:22a}, it is clear that the position of the PP \isi{adverbial} \textit{på skämt} (‘as a joke’) between the \isi{auxiliary} and the main verb is not due to verb \isi{movement}. Similarly, in \REF{ex:sangfelt:22b}, the \isi{adverb} \textit{omöjeligen} (‘impossibly’) is unambiguously in a position above the base position of the non-finite \isi{auxiliary}.

\ea\label{ex:sangfelt:22}
\ea\label{ex:sangfelt:22a}
\gll Herr Magistern måtte \textit{vilja} {på} {skämt} \textit{försöka} mig \\ %language information missing?
 Mr. teacher.\textsc{def} must want on joke examine me\\
\glt ‘Mr. teacher must want to examine me as a joke’ (2D108)\\
\ex\label{ex:sangfelt:22b}
\gll Jag skulle {omöjeligen} \textit{kunna} \textit{inbilla} mig det \\
 I would impossibly be.able imagine \textsc{refl} it\\
\glt ‘I couldn’t possibly imagine that’ (1A503)
\z
\z


The assumption that non-finite auxiliaries remain in situ in main clauses can be extended to all types of clauses, finite as well as non-finite. It should be noted that the \isi{movement} of non-finite verbs to a functional head is not universally prohibited. In \ili{Icelandic}, for example, the highest non-finite verb precedes a clausal negation in \isi{control infinitives} with the \isi{infinitive marker} \textit{að} (\citealt{Thrainsson2007}: 417–421) (see \ref{ex:sangfelt:23a}), indicating verb \isi{movement} to a functional position. A few examples of this type of word order can in fact be found in \ili{Old Swedish} texts from the 13\textsuperscript{th} and 14\textsuperscript{th} centuries, but it seems to have disappeared long before the \ili{Late Modern Swedish} period (see \cites[161]{Delsing1999}[]{Falk2010Studier}[141]{Kalm2016Satsekvivalenta}). Consequently, I assume that the position of the \isi{adverbial} \textit{så lätt} (‘so easily’) in \REF{ex:sangfelt:23b} is not due to \isi{movement} of the \isi{auxiliary}. The same assumption applies to \isi{control infinitives} without an \isi{infinitive marker} and to \isi{ECM} infinitives.


\ea \label{ex:sangfelt:23}
\ea \ili{Icelandic}\label{ex:sangfelt:23a}\\
\gll María lofaði að lesa {ekki} bókina \\
 Mary promised to read not book.\textsc{def}\\
\glt ‘Mary promised not to read the book’ (from \citealt{Thrainsson2007}: 421)
\ex\label{ex:sangfelt:23b} \gll Rosorna på hennes kinder äro för friska, för att \textit{kunna} så lätt \textit{förblekna} \\
 Rose.\textsc{pl.def} on her cheeks are too healthy for to be.able so easily pale\\
\glt ‘The roses on her cheeks are too sweet to be able to pale so easily’ (3C707)
\z 
\z

A rather delicate \isi{question} that remains is how to handle embedded clauses with a finite \isi{auxiliary} and a non-finite main verb. As stated in \sectref{sec:sangfelt:3}, a finite verb typically stays in V° in embedded environments in present-day \ili{Swedish}. This means that today, an \isi{embedded clause} with a \isi{constituent} between a finite \isi{auxiliary} and a non-finite main verb is a candidate for a proper instance of \isi{Aux-X-V} word order.


The situation in older \ili{Swedish} is different, however. \ili{Old Swedish} embedded clauses are usually assumed to display obligatory \isi{movement} of a finite verb to a functional projection above \isi{VP} but below \isi{CP}, here called \isi{IP}. This is indicated by the fact that a finite verb tends to precede a negation in all types of embedded clauses (see \citealt{Platzack1988emergence,Falk1993,Hakansson2013}). This sort of verb \isi{movement} was, however, lost during the history of \ili{Swedish}. According to \citet{Platzack1988emergence}, the last instances of \isi{V-to-I} \isi{movement} are found in the first half of the 17\textsuperscript{th} century (see also \citealt{Falk1993}: 174–177). If this is correct, \isi{V-to-I} should not affect the relative position of an \isi{auxiliary} and a non-verbal \isi{constituent} in my data sources.



Determining when \isi{V-to-I} takes place is sometimes difficult, since finite verbs preceding negation could be the result of so-called embedded \isi{V-to-C} (or \isi{embedded V2}). In present-day \ili{Swedish}, such \isi{main clause} word order is grammatical in embedded but assertion-friendly environments, and typically appears in an \isi{embedded clause} introduced by the \isi{complementizer} \textit{att} (‘that’) (see e.g. \citealt{Petersson2014}).\footnote{\citet[4]{Gartner2016} characterizes an environment as assertion-friendly when the content “counts as something the \isi{speaker} commits to and as meant to enrich the common ground”.} The \isi{question} of what types of clauses allowed embedded \isi{V-to-C} in older \ili{Swedish} has not been fully investigated (but see \citealt{Falk1993}: 168–177).\footnote{We should not assume a priori that embedded \isi{V-to-C} has the same characteristics in present-day and older \ili{Swedish}; there are North \ili{Germanic} languages, for example \ili{Icelandic} and \ili{Faroese}, that seem to show less restrictive \isi{V-to-C}, as pointed out by e.g. \citet{Gartner2016}.} It is thus impossible to distinguish between \isi{V-to-I}, embedded \isi{V-to-C}, and \isi{Aux-X-V} in a large number of embedded clauses with a finite \isi{auxiliary} and a non-finite main verb in the history of \ili{Swedish}. This includes \ili{Late Modern Swedish}, at least to some extent.



To bypass this problem, I have excluded embedded clauses with a finite \isi{auxiliary} and a non-finite main verb if the finite \isi{auxiliary} is the first \isi{constituent} after the subject of the \isi{embedded clause}. The \isi{embedded clause} in \REF{ex:sangfelt:24a} is therefore seen as ambiguous between embedded \isi{V-to-C} (or, less likely, \isi{V-to-I}) and \isi{Aux-X-V} proper. By contrast, the \isi{embedded clause} in \REF{ex:sangfelt:24b} is not ambiguous, since a \isi{constituent} in addition to the subject precedes the finite \isi{auxiliary}, indicating that verb \isi{movement} has not taken place.


\ea
\label{ex:sangfelt:24}
\ea\label{ex:sangfelt:24a}
\gll Tänck om gamla Gref {Hurtig (…)} \textit{skulle} {nu} \textit{komma} ur sin graf \\ %language information missing
 Think if old count Hurtig would now come out his grave\\
\glt ‘What if old count Hurtig were now to step out of his grave’ (1B103)

\ex\label{ex:sangfelt:24b}
\gll at I {intet} \textit{skolen} {et} {ögnablick} \textit{wara} ifrån henne \\
 that you not shall a moment be from her\\
\glt ‘that you won’t be away from her for a single moment’ (1A307)
\z
\z


To sum up, \isi{Aux-X-V} word order includes (i) clauses where a \isi{constituent} is placed between a non-finite \isi{auxiliary} and a non-finite main verb, and (ii) embedded clauses where a \isi{constituent} is placed between a finite \isi{auxiliary} and a non-finite main verb if the \isi{auxiliary} is preceded by a \isi{constituent} that indicates that verb \isi{movement} has not taken place.


The last thing to be settled is how to estimate the frequency of \isi{Aux-X-V}. There are of course several possible options regarding how to do this. The one I have chosen, and arguably the most adequate, is to compare \isi{Aux-X-V} with the word orders where X precedes both the \isi{auxiliary} and the main verb: X-Aux-V and X-\isi{V-Aux}. This largely means comparing \isi{Aux-X-V} with X-Aux-V, as \isi{V-Aux} was already rather infrequent at the beginning of the \ili{Late Modern Swedish} period and disappeared along with \isi{OV word order} (see \citealt{Platzack1983,Petzell2011,Sangfelt2019}).



To be counted as X-Aux-V (or X-\isi{V-Aux}), X has to be spelled out in a position below C°. This excludes main clauses like \REF{ex:sangfelt:25a}, where X has been fronted to spec-\isi{CP}. I also exclude embedded clauses where only one \isi{constituent}, excluding the subject, precedes a finite \isi{auxiliary} and a non-finite main verb; see \REF{ex:sangfelt:25b}. If the \isi{constituent} had followed the finite \isi{auxiliary}, it would not have been considered an unambiguous example of \isi{Aux-X-V}, due to the possibility of embedded \isi{V-to-C}.


\ea
\label{ex:sangfelt:25} % langauge information?
\ea\label{ex:sangfelt:25a} 
\gll {den} {människan} har jag aldrig \textit{kunnat} \textit{fördra} \\
 this human.\textsc{def} have I never been.able tolerate\\
\glt ‘I’ve never been able to tolerate this person’ (2D305)

\ex\label{ex:sangfelt:25b} 
\gll När Gubben {detta} \textit{fick} \textit{höra} \\
 when old.man.\textsc{def} this got hear\\
\glt ‘when the old man got to hear this’ (1D301)
\z
\z

\begin{sloppypar}
The category X roughly includes three types of constituents: objects, predicatives, and different types of adverbials, including adverbial-like complements. The category “\isi{adverbial}” is the most heterogeneous and necessitates a comment. \isi{Sentence adverbials}, including negation, are usually thought to be base-generated in a position above the finite verb, as explained in \sectref{sec:sangfelt:3}. If correct, this means that \isi{sentence adverbials} are not expected to occur in \isi{Aux-X-V} proper.
\end{sloppypar}

Despite this, I have included all kinds of adverbials in the category X for two reasons. Firstly, this is a way of testing the adequacy of the definition of \isi{Aux-X-V} word order and the basic assumptions about clause structure in this article, since we expect \isi{sentence adverbials} not to occur in this position. Secondly, we can avoid the problem of consistently identifying \isi{sentence adverbials} vs. other adverbials, which is not always easy in historical texts.



In the presentation of the data, I distinguish the numbers for negation, a frequent and easy-to-identify sentence \isi{adverbial}. Other adverbials have been coded for size, making a distinction between single adverbs and multiple-word adverbials (MW adverbials). MW adverbials mostly consist of PPs, but some instances of \isi{adverbial} phrases and NP adverbials are also placed in this subcategory.


\subsection{Aux-X-V in Late Modern Swedish}\label{sec:sangfelt:4.2}

\tabref{tab:sangfelt:1} shows the development of \isi{Aux-X-V} word order in \ili{Late Modern Swedish}, both in absolute numbers and in percentages. The percentages are calculated by dividing the absolute number of instances of \isi{Aux-X-V} by all instances of \isi{Aux-X-V}, X-Aux-V and X-\isi{V-Aux} word order combined. Consequently, \tabref{tab:sangfelt:1} shows how often we get \isi{Aux-X-V} word order, when X precedes the non-finite main verb, as explained in \sectref{sec:sangfelt:4.1}.



\begin{table}
\caption{The development of Aux-X-V word order in Late Modern Swedish\label{tab:sangfelt:1}}
\begin{tabular}{crcc}
\lsptoprule
Period & {Aux-X-V} & Total & {Aux-X-V} \%\\
\midrule
1725–1750 & 33 & 450 & 7\\
1775–1800 & 5 & 134 & 4\\
1825–1850 & 2 & 128 & 2\\
\lspbottomrule
\end{tabular}
\end{table}

As is clear from \tabref{tab:sangfelt:1}, most examples of \isi{Aux-X-V} word order are found during the first period. Of a total of 40 instances of \isi{Aux-X-V}, 33 come from texts written during the time span 1725–1750. However, this is partly due to sample size, it seems – period 1 contains more than three times as many instances of pre-verbal constituents in general, compared to periods 2 and 3. When this is controlled for, the percentages suggest that there is a rather minor decrease in frequency of \isi{Aux-X-V} during the period of \ili{Late Modern Swedish}. The numbers drop from 7\% to 4\% and then to 2\% between periods 1 and 3.

It should also be said that the differences in sample size might lead to questions about how representative the data for periods 2 and 3 are. Consequently, I will interpret the minor decrease in frequency with caution. As a matter of fact, one could argue that the sparse occurrence of \isi{Aux-X-V} in all periods seems to be a more substantial finding – the word order was obviously rather infrequent already by the beginning of \ili{Late Modern Swedish}.

As described in \sectref{sec:sangfelt:4.1}, the category X can be divided into five types of constituents: negation, object, \isi{predicative}, \isi{adverb}, and multiple-word (MW) \isi{adverbial}. The numbers for each subcategory are shown in \tabref{tab:sangfelt:2}, again divided into three periods.



\begin{table}
\caption{The development of Aux-X-V word order for five types of constituents. (Percentages based on less than 25 instances are given in parentheses.)\label{tab:sangfelt:2}}
\begin{tabular}{ll ccc}
\lsptoprule
\multicolumn{2}{l}{\isi{Constituent} type} & 1725–1750 & 1775–1800 & 1825–1850\\\midrule
Neg & Total & 0/108 & 0/46 & 0/36\\
    & \% & 0 & 0 & 0\\
Obj & Total & 5/11 & 0 & 0\\
    & \% & (45) & -- & --\\
Pred & Total & 0/5 & 1/6 & 0/3\\
     & \% & (0) & (17) & (0)\\
\isi{Adverb} & Total & 15/197 & 1/49 & 1/61\\
       & \% & 8 & 2 & 2\\
MWadv & Total & 13/129 & 3/33 & 1/27\\
      & \% & 10 & 9 & 4\\
\lspbottomrule
\end{tabular}
\end{table}
 
 
Negation clearly stands out from the other categories; it is the only one not represented in \isi{Aux-X-V} word order. As pointed out in \sectref{sec:sangfelt:4.1}, this is to be expected, given the assumption that \isi{sentential negation} is base-generated above the highest verb of a clause. In other words, the absence of negation in \isi{Aux-X-V} word order is in line with the basic assumptions of clause structure and the criteria used to identify \isi{Aux-X-V} proper.


Objects are the only type of \isi{constituent} not represented in all three periods, counting both \isi{Aux-X-V} and word orders where X precedes Aux and V. In period 1, a rather high proportion of objects show \isi{Aux-X-V} word order (5/11), although the absolute number is small. In periods 2 and 3, none of the sentences included contains a pre-verbal object. However, the disappearance of pre-verbal objects has nothing to do with the differences between \isi{Aux-X-V}, X-Aux-V, and X-\isi{V-Aux}, but presumably instead has to do with the general loss of \isi{OV word order}. In periods 2 and 3, I find no examples at all in the corpus of objects that precede their main verb, even outside the more narrowly defined word orders \isi{Aux-X-V}, X-Aux-V, and X-\isi{V-Aux}.\footnote{I exclude word orders that are still possible in present-day \ili{Swedish}, like fronted objects in spec-\isi{CP}.} The possibility of having pre-verbal objects thus seems to have disappeared between 1750 and 1775 in the history of \ili{Swedish}, at least in the rather informal \isi{genre} of drama dialogue.



The numbers for the \isi{predicative} subcategory do not lend themselves to a thorough diachronic analysis. Of a total of only 14 examples in all periods, one is found with \isi{Aux-X-V} word order. While it seems reasonable to include predicatives among the syntactic categories that are allowed in \isi{Aux-X-V}, the data are too limited to draw further conclusions.\footnote{Just like objects, resultative predicatives and predicatives that occur with a copula are not allowed to precede a verb in present-day \ili{Swedish}. It might thus seem surprising that there are predicatives preceding a verb in all three periods. However, all the examples in \isi{question} are instances of \isi{adjunct} predicatives (see the example in (i)), which are allowed to precede a verb in present-day \ili{Swedish}.  
\ea \gll at okänd \textit{få} \textit{sluta} mina  dagar  här  i  denna  skogspark\\
         to unknown  get  end  my  days  here  in  this  forest.park\\
     \glt ‘to have to die unknown here in this forest park’ (2E101)
\z}



Finally, we turn to the development of adverbials. Both adverbs and MW adverbials are well represented in the corpus in comparison with objects and predicatives. The two \isi{adverbial} categories also show a similar decrease in frequency over time, falling from around 10\% to under 5\% over the course of the three periods.\largerpage[3]



As pointed out in \sectref{sec:sangfelt:4.1}, the two \isi{adverbial} categories are internally heterogeneous, since they include different types of adjuncts, \isi{sentence adverbials}, and complement-like adverbials. Notably, there are no potential examples of a sentence \isi{adverbial} within the \isi{Aux-X-V} word order in my data. \isi{Sentence adverbials} in general thus pattern with negation, which is to be expected if they too are externally merged above the highest verb of the clause. Otherwise, it is difficult to establish any restrictions on what types of adverbials can occur in the \isi{Aux-X-V} word order. Instances of \isi{Aux-X-V} include, for example, adverbials of manner (see \REF{ex:sangfelt:23b} above, repeated in \REF{ex:sangfelt:26a}), place \REF{ex:sangfelt:26b}, time (see \REF{ex:sangfelt:24b} above, repeated in \REF{ex:sangfelt:26c}) and complement-like adverbials \REF{ex:sangfelt:26d}. Among the MW adverbials, we find \isi{adverbial} phrases \REF{ex:sangfelt:26a}, prepositional phrases \REF{ex:sangfelt:26b}, and NP adverbials \REF{ex:sangfelt:26c}.


\ea
\label{ex:sangfelt:26}
\ea \label{ex:sangfelt:26a}
\gll Rosorna på hennes kinder äro för friska, för att \textit{kunna} {så} {lätt} \textit{förblekna}\\ % language information
 Rose.\textsc{def.pl} on her cheeks are too healthy for to be.able so easily pale\\
\glt ‘The roses on her cheeks are too sweet to be able to pale so easily’ (3C707)

\ex\label{ex:sangfelt:26b}
\gll när winet likwäl \textit{skulle} {wid} {tullen} {eller} {uplastningen} \textit{proberas}\\
 when wine.\textsc{def} nevertheless would at customs.\textsc{def} or unloading.\textsc{def} be.investigated\\
\glt ‘when the wine would nevertheless be investigated at the customs or at the unloading’ (1E101)

\ex\label{ex:sangfelt:26c}
\gll at I intet \textit{skolen} {et} {ögnablick} \textit{wara} ifrån henne \\
 that you not shall a moment be from her\\
\glt ‘that you won’t be away from her for a single moment’ (1A307)

\ex\label{ex:sangfelt:26d}
\gll at I ej länge \textit{få} {derutinnan} \textit{framhärda} \\
 that you not long may in.this.thing persevere\\
\glt ‘that you don’t need to persevere for a long time in this matter’ (1B502)
\z
\z 


 To sum up, setting aside \isi{sentence adverbials}, most types of adverbials appear to have been permitted as X in \isi{Aux-X-V} configurations in \ili{Late Modern Swedish}, despite the rarity of this word order compared to X-Aux-V.

\subsection{Earlier periods and present-day Swedish}\label{sec:sangfelt:4.3}\largerpage[2]

 Since \isi{Aux-X-V} existed before the beginning of \ili{Late Modern Swedish}, and to some extent still exists today, some notes about its development beyond the 18\textsuperscript{th} and 19\textsuperscript{th} centuries are in order. As concluded in \sectref{sec:sangfelt:3.2}, there are no previous studies explicitly focusing on \isi{Aux-X-V} word order in the history of \ili{Swedish}, and this applies to all historical stages of the language. Some relevant data are nevertheless found in \citet{Sangfelt2019}, who presents numbers for \isi{Aux-X-V} in \ili{Early Modern Swedish}, where X is either an object or a prepositional phrase.\footnote{As in the present study, \citet{Sangfelt2019} compares \isi{Aux-X-V} to X-Aux-V and X-\isi{V-Aux}. In contrast to the present study, the relevant data are exclusively taken from clauses that contain a non-finite \isi{auxiliary}.}


A comparison of the data in \citet{Sangfelt2019} with the results from the present study indicates that the \isi{Aux-X-V} word order is somewhat more frequent before the beginning of \ili{Late Modern Swedish}, at least with regard to prepositional phrases. PPs show percentages between 15 and 20\% of \isi{Aux-X-V} in \ili{Early Modern Swedish} \citep[119]{Sangfelt2019}. The percentage is also slightly higher at the beginning of the era than later on. As we saw in the previous section, the percentages for multiple-word adverbials, roughly equivalent to PPs, fall from 10\% to 2\% during the \ili{Late Modern Swedish} period. In other words, the data suggest that there was a slow but steady decline of \isi{Aux-X-V} throughout the \ili{Modern Swedish} period.



Regarding objects, the percentages fluctuate between 24 and 42 in \ili{Early Modern Swedish} (see \citealt{Sangfelt2019}: 116), which is somewhat more frequent than the corresponding numbers for PPs. It is possible that the discrepancy in frequency between objects and PPs/MW adverbials is preserved as long as pre-verbal objects are grammatically acceptable, with 5 out of 11 pre-verbal objects in \ili{Late Modern Swedish} showing the \isi{Aux-X-V} word order.



In the last of the three \ili{Late Modern Swedish} periods in the present study, \isi{Aux-X-V} apparently became a rather infrequent word order, only possible with certain types of adverbials. This may be quite like the situation in present-day \ili{Swedish}. In \sectref{sec:sangfelt:3.2}, I concluded that \isi{Aux-X-V} appears to be a marginal but existing word order possibility in present-day \ili{Swedish}, if X is an \isi{adverbial}. To confirm my own intuitions and the observations in Teleman et al. (1999/3: 488–489), I conducted minor searches for \isi{Aux-X-V} by using the \ili{Swedish} corpus infrastructure \isi{Korp} \citep{BorinEtAl2012}.\footnote{\isi{Korp} is available here: \url{https://spraakbanken.gu.se/korp/}}  The results clearly indicate that \isi{Aux-X-V} is quite easily found in present-day language use, at least in these large corpora. If nothing else, this is true where X is an \isi{adverbial} of time or manner.{\interfootnotelinepenalty=10000\footnote{In passing, we should also note that we find focusing adverbs, typically \textit{bara} ‘only’, between an \isi{auxiliary} and a main verb. Out of the first 50 instances of \isi{Aux-X-V} word order found in \isi{Korp} (subcorpus \textit{sociala medier} ‘social media’), 13 examples contain \textit{bara} as the \isi{adverb}. This arguably constitutes evidence that verbs do not form a cluster in present-day \ili{Swedish}, but I am not sure to what extent this type of sentence should be compared with other instances of \isi{Aux-X-V}. As discussed in \citet{BrandtlerHakansson2017}, such adverbs can also be placed in front of a finite main verb (giving rise to a focused interpretation of the finite verb), which at least superficially breaks the V2 requirement of present-day \ili{Swedish}.   

\ea
\gll  Han  bara  grät  av  glädje  när  han  fick  se  dem.\\
      He  only  cried  of  joy  when  he  got  see  them\\
\glt ‘He just \textit{wept} for joy when he got to see them.’ (from \citealt{BrandtlerHakansson2017}: 12)
\z}} Two examples are given in (27a–b), one with an \isi{adverb} of manner \REF{ex:sangfelt:27a} and one with an \isi{adverb} of time \REF{ex:sangfelt:27b}.


\ea \ili{Present-day Swedish}\label{ex:sangfelt:27}
\ea\label{ex:sangfelt:27a}
\gll För mig som alltid bloggar via ipad och iPhone så är det rätt jobbigt att inte \textit{kunna} {enkelt} \textit{slänga} in ett inlägg\\
 For me that always blog through ipad and iPhone so is it quite annoying to not be.able easily throw in a post\\
\glt ‘For me who always blogs using an iPad or iPhone, it’s quite annoying not to be able to post easily’ (Bloggmix 2015)
\ex \label{ex:sangfelt:27b}\gll 99 \% av morgonnyheterna är DÅLIGA – skulle man inte \textit{kunna} {alltid} \textit{börja} med en bra nyhet \\
                                99 \% of morning.news.\textsc{def.pl} are BAD {} would one not be.able always start with a good piece.of.news \\
\glt ‘99\% of the morning news is BAD – if only you were able to always start with a good piece of news’ (Bloggmix 2011)\\
\z 
\z 

It goes without saying that these data are not sufficient to fully understand the mechanisms governing \isi{Aux-X-V} in present-day \ili{Swedish}. We could nevertheless hypothesize that not much has happened since the middle of the 19\textsuperscript{th} century; \isi{Aux-X-V} is an existing but uncommon word order pattern, and appears to have been so since the end of the \ili{Late Modern Swedish} period.

\subsection{Intermediate conclusions and discussion}\label{sec:sangfelt:4.4}
Since the beginning of \ili{Early Modern Swedish}, the \isi{Aux-X-V} word order seems to have decreased in frequency. Over time, we find word order restrictions that are diachronically stable, and other things that have changed. It appears to be invariant that \isi{sentence adverbials} are excluded in \isi{Aux-X-V}, which is not surprising if X needs to be merged below the edge of \isi{VP} for the word order to be generated. One thing that has changed concerns the ability of objects, and presumably complements in general, to occur in \isi{Aux-X-V}. If X is a pre-verbal object, \isi{Aux-X-V} is rather frequently used in \ili{Early Modern Swedish} and at the beginning of \ili{Late Modern Swedish}, but from the end of the 18\textsuperscript{th} century and onwards, objects are nonexistent in \isi{Aux-X-V}. However, this is not ultimately related to the properties of \isi{Aux-X-V} per se, but to a general loss of \isi{OV word order}. Given the data in this article, it actually seems difficult to detect any kind of abrupt grammatical change exclusively related to \isi{Aux-X-V} word order in the history of \ili{Swedish}. The decline of \isi{Aux-X-V} can be described as very slow, and the word order is still present to a certain extent in present-day \ili{Swedish}.


Regarding the connection between inseparable verbal clusters and \isi{OV word order} (see \citealt[17–19, 33–35]{Haider2010}), the data from older \ili{Swedish} do not really support such a conclusion. Whether or not an object precedes its main verb does not seem to be a crucial factor when it comes to allowing non-verbal material to intervene between VP-internal verbs.



Furthermore, the diachronic development of older \ili{Swedish} would be rather curious if there were such a straightforward correlation. \isi{Aux-X-V} became somewhat less frequent when OV decreased in frequency during \ili{Early Modern Swedish}, and even more infrequent when OV disappeared completely during \ili{Late Modern Swedish}. There are certainly ways of getting around this if we want to maintain the idea that OV forces verbs to cluster (see \cites[290–292]{Haider2010}[132–135]{Haider2013}), but, as will be argued in \sectref{sec:sangfelt:5}, data from the history of \ili{Swedish} indicate that such an account is conceptually undesirable.



One possible restriction on intervening syntactic material between verbs remains to be commented on. As emphasized in \sectref{sec:sangfelt:3.2}, \citet[155, 159]{Petzell2011}  claims that the word order V-Obj-Aux is not found in the history of \ili{Swedish}. This is the case despite the fact that both \isi{V-Aux} and VO are readily attested word orders, but is in line with restrictions put forward by the \isi{BBC} (and, for that matter, by the \isi{FOFC}; see \sectref{sec:sangfelt:2}). \citet{Petzell2011} does not, however, specify whether this generalization can be extended to \isi{V-X-Aux} as a whole, with X including not only objects, but all types of non-verbal constituents. The \ili{Late Modern Swedish} data are admittedly far from ideal for answering this \isi{question}. In fact, only two sentences in the data sample show \isi{V-Aux} word order. These are given in (28a–b).


\ea\label{ex:sangfelt:28}
\ea\label{ex:sangfelt:28a}
\gll at jag det så \textit{hafwa} \textit{wil} \\ % iangauge information
 that I it so have want\\
\glt ‘that I want it that way’ (1A201)

\ex\label{ex:sangfelt:28b}
\gll hwarmed I {för} {en} {bort-faren} {Wänn} {skul,} {eder} {emot} {Konungen} \textit{förbrutit} \textit{hafwen}\\
 whereby you for an away-traveled friend sake \textsc{refl} against king.\textsc{def} committed.crime have\\
\glt ‘whereby you have committed a crime against the king, for the sake of a friend that has traveled away’ (1A507)
\z 
\z 


As seen in (\ref{ex:sangfelt:28}a–b), all non-verbal constituents occur to the left of the verbs, so V and Aux remain adjacent. The reason why \isi{V-X-Aux} is not found could of course be the highly limited number of sentences, but the additional data in \citet{Sangfelt2019} suggest that the absence of \isi{intervening constituents} is not due to the small number of sentences. The data collected in \citet[133]{Sangfelt2019} contain 487 instances of a main verb preceding an \isi{auxiliary} in older \ili{Swedish} texts written between the 13\textsuperscript{th} and 18\textsuperscript{th} centuries. Despite the large number of clauses and the broad time span, \isi{V-X-Aux} is unattested. In other words, in clauses with \isi{V-Aux} word order, V and Aux are always adjacent. One should acknowledge that the evidence invoked here is negative by its nature, and that the ungrammaticality of \isi{V-X-Aux} does not follow as a logical consequence from these data. Nevertheless, this now appears to be a well-supported hypothesis. In the remainder of the article, I will assume that \citegen{Petzell2011} conclusion with regard to objects can be generalized to all non-verbal constituents, in full \isi{agreement} with the \isi{BBC}.

\section{VP structure and verbal clusters in Late Modern Swedish and beyond}\label{sec:sangfelt:5}

A prohibition on \isi{V-X-Aux} might seem like a rather idiosyncratic word order restriction in older \ili{Swedish} and the other \ili{Germanic} languages. However, the restriction is far from language-specific, as I will show in this section. In \sectref{sec:sangfelt:5.1}, I comment on comparative data that indicate that there is indeed something that bans \isi{V-X-Aux} in the languages of the world. In \sectref{sec:sangfelt:5.2}, I discuss the existence or non-existence of \isi{Aux-X-V} in some of the world’s languages, and whether or not X-Aux-V can contain instances of verbal clusters. In \sectref{sec:sangfelt:5.3}, I present my conclusion on verbal clusters in the history of \ili{Swedish}, and discuss its relevance for our understanding of verbal clusters in languages worldwide.

\subsection{*V-X-Aux in the languages of the world}\label{sec:sangfelt:5.1}

In \sectref{sec:sangfelt:2}, I stated that the absence of \isi{V-X-Aux} in the \ili{Germanic} languages was to be expected, given the limits on syntactic representations that follow from \isi{the Basic Branching Constraint}. In the discussion so far, the empirical evidence exclusively comes from the \ili{Germanic} language family. In the following, it will however be clear that this is not an idiosyncratic property of the \ili{Germanic} languages, but a possible language universal in need of explanation.


\citet{BiberauerEtAl2014} observe that it is extremely difficult to find languages where an object intervenes between a main verb and an \isi{auxiliary}, if the word order is \isi{V-Aux} and not Aux-V. Languages that are perfectly designed to test this generalization would be languages where VO varies with OV, and Aux-V with \isi{V-Aux}. Two non-\ili{Indo-European} languages that meet this criterion are Basque and \ili{Finnish}; in certain syntactic environments, both varieties exhibit what looks like free variation in the position of Aux, V, and objects. Despite this, V-Obj-Aux is not a grammatical possibility, as shown by the examples in (29a–b).


\ea
\label{ex:sangfelt:29}
\ea Basque\label{ex:sangfelt:29a}\\
\gll * Jon-ek \textit{esan} {Miren-i} {egia} \textit{dio}. \\
 {} Jon-\textsc{erg} said Miren-\textsc{dat} truth \textsc{aux}\\
\glt Intended reading: ‘John has told Miren the truth’

\ex \ili{Finnish}\label{ex:sangfelt:29b}\\
\gll * Milloin Jussi \textit{kirjoittanut} {romaanin} \textit{olisi}? \\
 {} when Jussi written novel would.have\\
\glt Intended reading: ‘When would Jussi have written a novel?’ (from \citealt{BiberauerEtAl2014}: 177)\\
\z 
\z 


The discussion in \citet{BiberauerEtAl2014} focuses on complementation, which leaves open the \isi{question} of whether or not we could find instances of adjuncts that interrupt a \isi{V-Aux} sequence (see \sectref{sec:sangfelt:2.2}). For verbal clusters to be obligatory with \isi{V-Aux} word order (and for the \isi{BBC} to be correct), it must equally be the case that adjuncts cannot intervene between V and Aux. This \isi{question} is discussed by \citet{Sheehan2017Final}. At a first glance, the ban on V-Adv-Aux does not seem quite as straightforward as that on V-Obj-Aux. As a minor pattern, the linear string V-Adv-Aux can be found, for example, in Hindi and Turkish with a small class of adverbs. Looking more closely, however, \citet{Sheehan2017Final} concludes that these are only apparent counterexamples to a general prohibition on elements intervening in a proper \isi{V-Aux} sequence. Rather than being instances of [[\textsubscript{VP} V X] Aux], they are instances where the \isi{adverb} is a projecting head, and where AdvP contains the main verb as an embedded subpart (see \citealt{Sheehan2017Final}: 102–120).


The structure [[\textsubscript{AdvP} V Adv] Aux] is not an example of a right-branching structure, if the AdvP is an \isi{adjunct} of the higher \isi{VP}/AuxP (see \citealt{Haider2013}). Therefore, I will assume that universal grammar forces a \isi{cluster analysis} upon \isi{V-Aux}, in accordance with the \isi{BBC}. 


\subsection{The structure of Aux-X-V and X-Aux-V in the languages of the world}\label{sec:sangfelt:5.2}

In §§\ref{sec:sangfelt:3} and~\ref{sec:sangfelt:4}, we saw that there are cases where \isi{Aux-X-V} is combined with \isi{OV word order}. This is an interesting conclusion, since it could be expected from \citet{Haider2010} that verbal clusters are a direct consequence of \isi{OV word order}. Let us first note, however, that Haider is well aware of the fact that there are \ili{German} VPs where an element can intervene between two verbs (see e.g. \citealt{Haider2013}: 128, 134), as I also noted in \sectref{sec:sangfelt:3}. Despite this, Haider argues that all verbs start out as clusters, but that an \isi{auxiliary} can leave its base position and target a verbal head above the lowest \isi{VP} \parencites[290–291]{Haider2010}[134]{Haider2013}. The basic idea is schematically illustrated in \REF{ex:sangfelt:30a}. The alternative would be to assume that the structure involves no \isi{movement}, but only an \isi{auxiliary} taking the lower \isi{VP} as its right-hand \isi{complement}, as in \REF{ex:sangfelt:30b}. Note that both \REF{ex:sangfelt:30a} and \REF{ex:sangfelt:30b} are compatible with the \isi{BBC}; the complex \isi{VP} is clearly right-branching.

\settowidth\jamwidth{\isi{Movement} of Aux from the verbal cluster}
\ea
\label{ex:sangfelt:30}
\ea {\label{ex:sangfelt:30a}[\textsubscript{VP1} Aux\textsubscript{i} [\textsubscript{VP2} X [V+t\textsubscript{i}]]]} \jambox{\isi{Movement} of Aux from the verbal cluster}
\ex {\label{ex:sangfelt:30b}[\textsubscript{VP1} Aux [\textsubscript{VP2} X V]]} \jambox{A \isi{VP} inside another VP}
\z 
\z

From Haider’s discussion, it does not seem at all clear to me why one would assume the structure in \REF{ex:sangfelt:30a} rather than the one in \REF{ex:sangfelt:30b}. I also note that \isi{sentence adverbials}, which are the category usually employed to discriminate between verb \isi{movement} and V-in-situ, never intervene between Aux and V, as we saw in \sectref{sec:sangfelt:4}. If we accept the premise that the \isi{movement} analysis bears the onus of proof rather than the base-generation analysis, then \REF{ex:sangfelt:30b} is actually preferable, as far as I can tell. Be that as it may, both analyses capture the general conclusion that \isi{Aux-X-V} can never be (just) a \isi{verbal cluster}, and there is no doubt that it can be found with both OV and VO clause structure (see \citealt{Haider2010}: 291).


As is the case with \isi{Aux-X-V}, X-Aux-V is clearly a word order that occurs in both OV and VO languages. In varieties (or grammars) where X-Aux-V varies with \isi{Aux-X-V}, it is hard to see any reasons why X-Aux-V should be analysed as a cluster variant. In (31a–b), I show two authentic dialectal \ili{German} examples (West Central, according to \citealt{Sapp2011}: 125) where a \isi{constituent} intervenes between a finite \isi{auxiliary} and a main verb. As seen in the examples, there are also constituents that intervene between the subject and the \isi{auxiliary}: in \REF{ex:sangfelt:31a} an NP \isi{adverbial} and in \REF{ex:sangfelt:31b} an \isi{adverb} (see also examples (20b–c) in \sectref{sec:sangfelt:3.3}). Hence, we see that X-Aux-V order does not rely on the \isi{adjacency} of the verbs or, by extension, on a \isi{cluster analysis}.



\ea dialectal \ili{German}
\label{ex:sangfelt:31}
\ea\label{ex:sangfelt:31a}
\gll dass er {jeden} {Augenblick} \textit{musste}  {hinter} {eine} {Hecke} \textit{laufen} \\
that he any moment must behind a hedge run\\
\glt ‘that he had to run behind a hedge at any moment’

\ex\label{ex:sangfelt:31b}
\gll dass sie {da} \textit{müssen} {einen} {ordentlichen} {Korb} \textit{kochen}\\
that they there must a decent basket cook\\
\glt ‘that they have to cook a decent basketful of food’ (from \citealt[126]{Sapp2011})\\
\z 
\z 

The situation in a language like \ili{Dutch} could be analysed differently, however (see \sectref{sec:sangfelt:3.3}). If we look at \ili{Dutch} data in isolation, there is no doubt that a \isi{cluster analysis} would explain why Aux and V must be adjacent regardless of order. Among the West \ili{Germanic} varieties, \ili{Dutch} seems to be the odd one out, though (see e.g. \citealt{Haegeman1992}, \citealt{Sapp2011}: 124–129). I will refrain from commenting on how common or uncommon this property is in a typological perspective, but I note that \ili{Persian} seems to be an OV language that behaves very much like \ili{Dutch}. As reported by \citet[100]{Sheehan2017Final}, most \ili{Persian} auxiliaries occur with \isi{V-Aux} word order. The future \isi{auxiliary} \textit{xâhad} ‘will’ is an exception, however, since it always precedes a main verb. Despite this, an \isi{adverb} can interrupt neither a \isi{V-Aux} \REF{ex:sangfelt:32a} nor an Aux-V sequence \REF{ex:sangfelt:32b}. This is expected if \ili{Persian} verbs cluster obligatorily, regardless of linear order.

\ea \ili{Persian}
\label{ex:sangfelt:32}
\ea[*]{\label{ex:sangfelt:32a}
\gll ali gitâr \textit{zade} {hamishe} \textit{ast} \\
 Ali guitar played always is\\
\glt Intended reading: ‘Ali has always played the guitar’}

\ex[*]{\label{ex:sangfelt:32b}
\gll ali gitâr \textit{xâhad} {hamishe} \textit{zad}\\
 Ali guitar will always play\\
\glt Intended reading: ‘Ali will always play the guitar’ (from \citealt{Sheehan2017Final}: 100)}
\z 
\z


As I see it, the \ili{Persian} and \ili{Dutch} data give us two different options when it comes to analysing instances of X-Aux-V word order. The first possibility is a \isi{cluster analysis}. This would immediately explain the \isi{adjacency} requirement and capture the fact that the requirement holds regardless of VP-internal word order. In addition, this analysis would be in line with considerations of economy in some sense; we still have to assume that \ili{Dutch} and \ili{Persian} employ verbal clusters with \isi{V-Aux} word order, in accordance with the \isi{BBC}. 

At the same time, it would also be possible to perceive the need for \isi{adjacency} as some kind of language-specific principle, independent of verbal clusters. For one thing, we know that the \isi{cluster analysis} is not forced by the principles of UG, since [\textsubscript{VP1} Aux [\textsubscript{VP2} V]] is a perfectly acceptable structure of a complex \isi{VP}. A \isi{cluster analysis} of Aux-V also runs the risk of leading to inconvenient questions about \isi{clustering} in VO languages. \citet[343]{Haider2010} is of the view that strict VO languages can never cluster, since the structure with embedded VPs is already a \isi{perfect} right-branching structure if the object follows the main verb (see \sectref{sec:sangfelt:2}). But this is not to say that it must be possible for constituents to intervene between two VP-internal verbs in a VO language. If we were indeed to find VO languages with a general \isi{adjacency} requirement for Aux and V (which is certainly conceivable within the \isi{BBC}), a problem arises: what reason would there be to assume a \isi{cluster structure} for Aux-V sequences in \ili{Dutch}, for example, if we find both OV and VO languages where Aux and V must always be adjacent (see \citealt{Sheehan2017Final}: 99–101)?



A way forward in the discussion of X-Aux-V would be to carefully examine whether we can find VO languages that exhibit a complete ban on \isi{Aux-X-V}, like the OV languages \ili{Dutch} and \ili{Persian}. However, such an investigation is beyond the scope of this article. The general conclusion drawn is that the \isi{BBC} forces a \isi{cluster analysis} on a \isi{V-Aux} sequence to avoid left-branching structures within a complex \isi{VP}. The \isi{question} of the possibility for an Aux-V sequence to instantiate a \isi{verbal cluster} is left partially open. This could be a way of accounting for \isi{VP} properties in a language like \ili{Dutch}, but so far, we lack decisive evidence in favour of such an analysis. Importantly though, this \isi{question} has no effect on how we should understand the development of verbal clusters in the history of \ili{Swedish}, as I will show in the following section.


\subsection{Verbal clusters in the history of Swedish}\label{sec:sangfelt:5.3}

In the history of \ili{Swedish}, we find a great deal of word order flexibility within the \isi{VP} up until the middle of the 18\textsuperscript{th} century. When it comes to the categories Aux, V and X, it almost seems that they are allowed to occur in any order. Throughout the article, I have discussed instances of X-\isi{V-Aux}, X-Aux-V, and \isi{Aux-X-V}. I have not focused on the characteristic \isi{VO word order} Aux-V-X, although, statistically speaking, this has been the main option since at least the middle of the 14\textsuperscript{th} century (see \citealt{Delsing1999,Petzell2011,Sangfelt2019}). The pattern V-Aux-X, where X is placed to the right of a \isi{verbal cluster}, is a minor one, but it can be found with objects and adverbials basically as long as \isi{OV word order} is possible (see \citealt{Sangfelt2019}: 225–227).


One imaginable word order is missing, however, as predicted by the \isi{BBC}: \isi{V-X-Aux} is not found in the history of \ili{Swedish}. This should be taken as rather strong evidence that the \isi{BBC} is at work. Older \ili{Swedish} (like Basque and \ili{Finnish}) allows several types of constituents to intervene between an \isi{auxiliary} and a main verb, and V-X is, furthermore, a very common linearization pattern, since older \ili{Swedish} exhibits a mix between VO and \isi{OV word order}. Despite having all the properties that should facilitate \isi{V-X-Aux}, this order is, as noted, not found in older \ili{Swedish}.



The lack of \isi{V-X-Aux} in older \ili{Swedish} and other languages has been analysed as an effect of \isi{V-Aux} enforcing a \isi{cluster structure}, where two (or more) verbal heads form a complex \isi{constituent}. Thus, older \ili{Swedish} employed verbal clusters as long as \isi{V-Aux} and \isi{OV word order} were possible. Despite this, the main empirical conclusion in the present study is that evidence for non-\isi{clustering} dropped in frequency over the history of \ili{Swedish}. By the middle of the 19\textsuperscript{th} century, \isi{Aux-X-V} was without doubt a rarely attested word order. This is important, since it clearly shows that the idea of a bi-directional correlation between \isi{OV word order} and \isi{clustering} is misleading.



In the final parts of the paper, I have entertained the idea that we should differentiate between two types of languages or grammars with regard to \isi{clustering}. On the one hand, we clearly find languages where \isi{clustering} is exclusively related to \isi{V-Aux} word order, and where embedded VPs are employed when Aux precedes V (or when a selecting \isi{auxiliary} Aux1 precedes a selected \isi{auxiliary} Aux2). All the evidence points to the conclusion that older \ili{Swedish} belongs to this type, together with most other \ili{Germanic} languages. On the other hand, we have languages like \ili{Dutch}, where \isi{clustering} could be seen as a property of the \isi{VP} itself, since the \isi{adjacency} requirement of Aux and V applies regardless of the order of the verbal elements. It remains to be shown, however, whether this property is best explained by a \isi{cluster analysis} or is rather a consequence of some other, cluster-independent principle.


\section{Conclusion}\label{sec:sangfelt:6}

In this article, I have studied VP-internal word order variation and discussed the presence of verbal clusters in the history of \ili{Swedish}. The main conclusion is that \isi{clustering} is exclusively related to the order of the verbs; \isi{V-X-Aux} has never been a possible word order in the history of \ili{Swedish}. However, in the case of the reversed order of verbs (\isi{Aux-X-V}), there is plentiful evidence for non-\isi{clustering}; such examples are attested throughout the history of \ili{Swedish}, where X can be an object, a \isi{predicative}, or a non-sentential \isi{adverbial}. The diachronic development furthermore suggests that there is no bi-directional correlation between verbal clusters and \isi{OV word order}; \isi{Aux-X-V} dropped in frequency despite the fact that VO became more common over the history of \ili{Swedish}, and finally became the only available option in \ili{Late Modern Swedish}.


The history of verbal clusters in \ili{Swedish} is hardly guided by some idiosyncratic, language-specific principle. Rather, I have argued, it follows from a universal principle, which states that syntactic material is never allowed to intervene between verbs in a \isi{V-Aux} sequence, while the same restriction does not hold within an Aux-V sequence. In as much as \isi{V-Aux} word order is employed in \isi{head-final} and not \isi{head-initial} languages, the existence of verbal clusters is of course not completely independent of OV and \isi{VO word order}. Given the data from the history of \ili{Swedish} and other languages, it is nevertheless clear that the picture is too complex to allow us to assert that OV structures have verbal clusters and VO structures do not.


\section*{Acknowledgments}


For valuable criticism and both major and minor suggestions of clarification on earlier drafts of this paper, I thank three anonymous reviewers and, especially, the two editors Ida Larsson and Erik M. Petzell. All errors and inconsistencies are my own.


\section*{Abbreviations}
\begin{multicols}{2}
\begin{tabbing}
\isi{FOFC} \hspace{.5ex} \= Final-Over-Final Condition\kill
\isi{BBC}  \>   Basic Branching Constraint\\
EMS  \>   \ili{Early Modern Swedish}\\
\isi{FOFC} \>    Final-Over-Final Condition\\
IPP  \>   Infinitivus Pro Participio\\
\isi{LCA}  \>   \isi{Linear Correspondence Axiom}\\
MW   \>  Multiple-Word adverbials\\
OS   \>  \ili{Old Swedish}\\
OV   \>  Object-Verb word order\\
V    \> Main verb\\
VO   \>  Verb-Object word order
\end{tabbing}
\end{multicols}

\section*{Sources}

\subsection*{Investigated texts from the corpus of \ili{Swedish} drama dialogue }
\subsubsection*{Period 1 (1725--1750)}

\begin{description}[font=\normalfont]\sloppy
\item[1A:] Lagerström, Magnus (b. 1691). \textit{Le Tartuffe eller den skenhelige}  [Le Tartuffe or the hypocrite]. Stockholm, 1731. 
\item[1B:] Gyllenborg, Carl (b. 1679). \textit{Swenska sprätthöken} [The \ili{Swedish} dandy]. Stockholm, 1740. Available through LB.
\item[1C:] Dalin, Olof von (b. 1708). \textit{Den afwundsiuke} [The jealous one]. Stockholm, 1739. Available through LB. 
\item[1D:] Modée, Reinhold Gustaf (b. 1698). \textit{Håkan Smulgråt} [Håkan Cheapskate]. Stockholm, 1739. Available through LB. 
\item[1E:] Stagnell, Johan (b. 1711). \textit{Den lyckelige banqueroutieren} [The happy bankrupter]. Stockholm, 1753. Available through LB.
\end{description}


\subsubsection*{Period 2 (1775–1800)}
\begin{description}[font=\normalfont]\sloppy
\item[2A:] Kexél, Olof (b. 1748). \textit{Sterbhus-kammereraren Mulpus eller Caffe-huset i Stora Kyrkobrinken} [The chief accountant of the estate Mulpus or the coffee house in the main church hill]. Stockholm, 1776. Available through LB.
\item[2B:] Ristell, Adolf Fredrik. (b. 1744). \textit{Några mil från Stockholm} [A few miles from Stockholm]. Manuscript from 1787. Edited by Gösta Langenfeldt \& Bo Thörnqvist. Stockholm: Department of Scandinavian languages, 1974. 
\item[2C:] Envallson, Carl (b. 1756). \textit{Kusinerna eller Fruntimmers-sqvallret} [The cousins or the gossip of the women]. Stockholm, 1807. Available through LB.
\item[2D:] Enbom, Per (b. 1759). \textit{Fabriks-flickan} [The factory girl]. Stockholm, 1796. Available through LB.
\item[2E:] Stridsberg, Carl (b. 1755). \textit{Friman eller Den enslige och de resande fruntimren} [Friman or the loner and the travelling women]. Stockholm, 1798. Available through LB.
\end{description}


\subsubsection*{Period 3 (1825–1850)}
\begin{description}[font=\normalfont]\sloppy
\item[3A:] Wetterbergh, Carl Anton (b. 1804). \textit{Pröfningen} [The test]. Stockholm, 1842. Available through LB.
\item[3B:] Blanche, August (b. 1811). \textit{Hittebarnet} [The foundling]. Stockholm, 1848. Available through LB.
\item[3C:] Jolin, Johan (b. 1818). \textit{Barnhusbarnen eller Verldens dom} [The children of the orphanage or the judgement of the world]. Stockholm, 1849. Available through LB.
\item[3D:] Ridderstad, Carl Fredrik (b. 1807). \textit{Syskonen eller Hattarnas och Mössornas sista strid} [The siblings or the last battle of the hats and the caps]. Linköping, 1849. Available through LB.
\item[3E:] Blink, Carl and Malméen, Georges. 1850. \textit{Tidens Strid eller Det Bästa Kapitalet} [The battle of time or the best capital]. Stockholm. Available through LB.
\end{description}
 
\section*{Other cited texts}

\begin{description}[font=\normalfont]\sloppy
\item[Bloggmix] [A selection of \ili{Swedish} blogs] 1998–2017. Available through \isi{Korp}.
\item[Columbus:]Columbus, Samuel (b. 1642). \textit{Mål-Roo eller roo-mål} [Language amusement or amusing language]. Ca. 1675. Edited by Bengt Hesselman. (Nordiska texter och undersökningar 6.) Stockholm: Hugo Geebers förlag, 1935. Available through FTB/\isi{Korp}.
\item[Di:]\textit{Sagan om Didrik af Bern} [The story of Didrik of Bern]. Ca. 1450. Edited by Gunnar Olof Hyltén-Cavallius. (Samlingar utgivna av Svenska fornskriftsällskapet 10.) Stockholm: Norstedts, 1850–1854. Available through FTB/\isi{Korp}.
\item[LinLeg:]Linköpingslegendariet [The collection of legends from Linköping]. Ca. 1500. In \textit{Ett fornsvenskt legendarium}. Edited by Georg Stephens. (Samlingar utgivna av Svenska fornskriftsällskapet 7:1–3). Stockholm: Norstedts, 1847–1874. Available through FTB/\isi{Korp}.
\item[MB1B:]\textit{Fem Moseböcker på svenska enligt Cod. Holm.}, \textit{A 1} [Five books of Moses in \ili{Swedish} according to Cod. Holm., A1]. Ca. 1330. Edited by Olof Thorell. (Samlingar utgivna av Svenska Fornskriftsällskapet 60.) Uppsala: Almqvist \& Wiksell, 1959. Available through FTB/\isi{Korp}.
\item[SpecV:]\textit{Speculum Virginum}. Ca. 1480. Edited by Robert Geete. (Samlingar utgivna av Svenska fornskriftsällskapet 31.) Stockholm: Norstedts, 1897–1898. Available through FTB/\isi{Korp}.
\end{description}

\section*{Electronic corpora}

\begin{description}[font=\normalfont]\sloppy
\item[FTB:] \isi{Fornsvenska textbanken} [The text bank of \ili{Old Swedish}]:  \url{https://project2.sol.lu.se/fornsvenska} 
\item[\isi{Korp}:] \url{https://spraakbanken.gu.se/korp/?mode=all_hist}
\item[LB:] The \ili{Swedish} literature bank: \url{http://www.litteraturbanken.se}
\end{description}

{\sloppy\printbibliography[heading=subbibliography,notkeyword=this]}
\end{document}
