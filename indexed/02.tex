\documentclass[output=paper]{langscibook}
\ChapterDOI{10.5281/zenodo.5792953}
\author{Cecilia Falk\affiliation{Stockholm University}}
\title{The introduction of object symmetry in passives}
\abstract{This paper investigates the introduction of the possibility to promote the indirect object to subject in passives during the second half of the 19th century. An analysis is proposed where the indirect object occupies a position with inherent case throughout the history of Swedish. Before the change, a passive ditransitive verb had no structural case to assign. Thus, the direct object had to move to the subject position, whereas the indirect object remained in its case position. After the change, passive ditransitives have the capacity to assign one structural case. If the direct object receives this case, the indirect object can escape its case position and be promoted to subject. It is also shown that the change was preceded by shifting preferences in the ordering of arguments in the passive voice.

\keywords{ditransitive, double objects, object symmetry, passive, inherent case, lexical case,  structural case}
}
\IfFileExists{../localcommands.tex}{
  \addbibresource{localbibliography.bib}
  % add all extra packages you need to load to this file

\usepackage{tabularx,multicol}
\usepackage{url}
\urlstyle{same}

\usepackage{listings}
\lstset{basicstyle=\ttfamily,tabsize=2,breaklines=true}

\usepackage{langsci-basic}
\usepackage{langsci-optional}
\usepackage{langsci-lgr}
\usepackage{langsci-gb4e}

\usepackage{todonotes}

\usepackage[linguistics]{forest}
\usepackage{soul}
\usepackage{subfigure}
\usepackage{longtable}
\usepackage{enumitem}

  \newcommand*{\orcid}{}
%\newcommand{\keywords}[1]{\textbf{#1}}


\makeatletter
\let\theauthor\@author
\makeatother

\newcommand{\keywords}[1]{\textbf{Keywords:} #1}


\DeclareNewSectionCommand
  [
    counterwithin = chapter,
    afterskip = 2.3ex plus .2ex,
    beforeskip = -3.5ex plus -1ex minus -.2ex,
    indent = 0pt,
    font = \usekomafont{section},
    level = 1,
    tocindent = 1.5em,
    toclevel = 1,
    tocnumwidth = 2.3em,
    tocstyle = section,
    style = section
  ]
  {appendixsection}

\renewcommand*\theappendixsection{\Alph{appendixsection}}
\renewcommand*{\appendixsectionformat}{\appendixname~\theappendixsection\autodot\enskip}
\renewcommand*{\appendixsectionmarkformat}{\appendixname~\theappendixsection\autodot\enskip}
 
  %% hyphenation points for line breaks
%% Normally, automatic hyphenation in LaTeX is very good
%% If a word is mis-hyphenated, add it to this file
%%
%% add information to TeX file before \begin{document} with:
%% %% hyphenation points for line breaks
%% Normally, automatic hyphenation in LaTeX is very good
%% If a word is mis-hyphenated, add it to this file
%%
%% add information to TeX file before \begin{document} with:
%% %% hyphenation points for line breaks
%% Normally, automatic hyphenation in LaTeX is very good
%% If a word is mis-hyphenated, add it to this file
%%
%% add information to TeX file before \begin{document} with:
%% \include{localhyphenation}
\hyphenation{
anaph-o-ra
Dor-drecht
mono-mor-phe-mic
Swed-ish
sche-mat-ic
Viska-da-li-an
An-ders-son
dia-lekt-forsk-ning
dra-ma-språk
bref-vex-ling
Ak-tu-ell
folk-livs-forsk-ning
Þor-björg
Ak-ti-ons-art
Upp-sala
myck-en
}

\hyphenation{
anaph-o-ra
Dor-drecht
mono-mor-phe-mic
Swed-ish
sche-mat-ic
Viska-da-li-an
An-ders-son
dia-lekt-forsk-ning
dra-ma-språk
bref-vex-ling
Ak-tu-ell
folk-livs-forsk-ning
Þor-björg
Ak-ti-ons-art
Upp-sala
myck-en
}

\hyphenation{
anaph-o-ra
Dor-drecht
mono-mor-phe-mic
Swed-ish
sche-mat-ic
Viska-da-li-an
An-ders-son
dia-lekt-forsk-ning
dra-ma-språk
bref-vex-ling
Ak-tu-ell
folk-livs-forsk-ning
Þor-björg
Ak-ti-ons-art
Upp-sala
myck-en
}
 
  \togglepaper[1]%%chapternumber
}{}

\begin{document}
\SetupAffiliations{mark style=none}
\maketitle


\section{Introduction}


In present-day Swedish, either of the two objects in a \isi{ditransitive} construction may be \isi{promoted to subject} in the \isi{passive voice}:\footnote{In what follows, I shall refer to the two verb complements as “\isi{indirect object}” and “\isi{direct object}”, regardless of their superficial function as subject or object in the \isi{passive voice}. By “\isi{passivized indirect object}” I mean an \isi{indirect object} in the \isi{nominative} case or, if case is not visible, in the \isi{subject position} in a clause with the verb in the \isi{passive voice}.}

\ea%1
    \label{ex:falk:1}
\ea
\gll Han    erbjöds      en  lägenhet\\
      He.\textsc{sbj}  offer.\textsc{pst}.\textsc{pass}  an  apartment\\
\glt ‘He was offered an apartment’

\ex
\gll En    lägenhet      erbjöds      honom\\
      an    apartment    offer.\textsc{pst}.\textsc{pass} him.\textsc{obj}\\
\glt ‘He was offered an apartment’
\z
\z


This possibility of variation is often referred to as \isi{object symmetry} in the \isi{passive voice} (e.g. \citealt{Anagnostopoulou2003}). Earlier stages of Swedish did not show \isi{object symmetry}. Instead, only the \isi{direct object} could be passivized, i.e. \isi{promoted to subject} in a \isi{passive}. In the older Swedish passives given in \REF{ex:falk:2}, either \isi{case morphology} or placement shows that the \isi{indirect object} is not the subject. In (\ref{ex:falk:2}a–b), the indirect objects have \isi{object case} morphology, and in (\ref{ex:falk:2}c–d), the indirect objects lack overt \isi{case morphology} but show up in the object position, after the finite verb in an \isi{embedded clause} \REF{ex:falk:2c}, or after the infinite verb in a \isi{main clause} (2d; indirect objects in italics):\footnote{A third possible way of identifying a \isi{passivized object} is through \isi{verbal agreement}: a \isi{passivized object} always triggered \isi{agreement} in number on the finite verb. \isi{Verb agreement} in number was morphologically marked in written Swedish until the mid-20\textsuperscript{th} century in some verb classes. Below, I gloss \isi{verb agreement} on the finite verb only if it is overt. I found no examples where \isi{verb agreement} was the only indication of subjecthood.}

\ea%2
    \label{ex:falk:2}

\ea\label{ex:falk:2a}
\gll rettferdighet    hwilken \textit{mig} skenckes\\
      justice      which  me.\textsc{obj}  give.\textsc{prs.pass}\\
\glt ‘justice that is given to me’ (\isi{SAOB} \textit{skänka}, 1709)

\ex\label{ex:falk:2b}
\gll      \textit{honom} måtte        tillåtas          een  lägenhet  uthi  Götha  rijkes      hofrett.\\
      him.\textsc{obj}  shall.\textsc{pst}    allow\textsc{.inf.pass}   a    prospect  in    Göta    land.\textsc{poss}  court\\
\glt     ‘he should be guaranteed prospects at the court of appeal in Götaland’ {(}{\isi{SAOB} \textit{tillåta}, 1646)}
\ex\label{ex:falk:2c}
\gll … at  frachten    först  biwdes \textit{borgare}\\
         {} that  cargo.\textsc{def}  first  offer{\textsc{.prs.sg.pass}}    burghers\\
\glt ‘… that the cargo is first offered to burgers’ (\isi{SAOB} \textit{bjuda}, 1529)

\ex\label{ex:falk:2d}
\gll Ett    afskräckande  exempel  måste    gifvas \textit{verlden}\\
      a      warning      example  must    give\textsc{.inf.pass}  world\textsc{.def}\\
\glt `The world must be given a warning example’ (SPF, 1841)
    \z
    \z

\noindent In this paper, I investigate the emergence of \isi{object symmetry}, as shown in \REF{ex:falk:1}.


In a small investigation of the change discussed here, I show that passivized indirect objects were very uncommon before 1850 \citep[167]{Falk1997}. The main focus in this article is therefore the 19\textsuperscript{th} century, but comparisons will be made with earlier periods. The data in my investigation include only the morphological \isi{passive}, formed with the suffix -\textit{s}.\footnote{Swedish also has a \isi{periphrastic passive}, like present-day \ili{English}. I have chosen to investigate only \textit{s}{}-passives, since former investigations of passivized \isi{ditransitive} verbs have concerned \textit{s}{}-passives. Another reason is that further investigations of participles are needed before we can draw safe conclusions on the status of the objects; the issue concerns both verbal vs. adjectival participles and word order possibilities. Finally, \textit{s}{}-passives gained ground during the 19\textsuperscript{th} century at the expense of the \isi{periphrastic passive}, as shown by \citet{Kirri1975}. However, as shown by \citet{Holm1952}, a substantial number of \textit{s}{}-passives are already found in \ili{Early Old Swedish}, making comparisons with earlier stages possible.} Some background is given in \sectref{sec:falk:2}, where I show that the case of the \isi{indirect object} has some atypical properties, both before and after the change, including what appears to be a curious mix of lexical and \isi{structural case}. In \sectref{sec:falk:3}, I present data from the 19\textsuperscript{th} century and earlier periods, and show that the introduction of \isi{object symmetry} was preceded by changes in the order of the arguments. In \sectref{sec:falk:4}, I provide an analysis of the case of the \isi{indirect object} before and after the change. I suggest that the base position of the \isi{indirect object} is an \isi{inherent case} position, and hence that the \isi{indirect object} is licensed in situ both before and after the change. However, the change affected the case assigning properties of \isi{ditransitive} verbs, making it possible for the \isi{indirect object} to escape its base position and move to the \isi{subject position}. The proposal gives a formal account of the mixed properties presented in \sectref{sec:falk:2}, some of which have not been accounted for in previous analyses. In \sectref{sec:falk:5}, I discuss the different developments up until 1900 and some remaining questions. \sectref{sec:falk:6} is a summary.

\section{Background}\label{sec:falk:2}

\subsection{Research on present-day Swedish}\label{sec:falk:2.1}


Investigations of the passivization of \isi{ditransitive} verbs in present-day Swedish often observe that not all types of \isi{ditransitive} verbs passivize equally easily. Selectional as well as morphological properties of the verb have been invoked to explain this phenomenon. \citet{Anward1989} notes that verbs for which double objects are the only possibility passivize easily, whereas the \isi{passive} of verbs with the PP-alternative is less accepted (if not totally prohibited; see \citealt{TelemanEtAl1999}/4: 368, 2c). \citet{HolmbergPlatzack1995} instead take the verb-internal structure as being decisive: bimorphemic \isi{ditransitive} verbs like \textit{till-dela} ‘award’, \textit{er}{}-\textit{bjuda} ‘offer’ passivize easily, whereas \isi{passive} monomorphemic \isi{ditransitive} verbs are “marginal” (1995: 219–220). These generalizations are largely based on their intuitions. \citet{HaddicanHolmberg2019} report on a larger grammaticality judgement test, and the results do indeed show that the informants preferred passivized bimorphemic verbs over monomorphemic. I know of no investigation of the actual use of bimorphemic versus monomorphemic \isi{ditransitive} \textit{s}{}-passives.


These observations concern the degree to which passivization is acceptable. \citet{HaddicanHolmberg2019} also investigated \isi{judgements} on \isi{choice of subject}, and the results showed that the informants preferred passivized indirect objects over passivized direct objects. Similarly, \citet{Lundquist2004} found that passivized indirect objects seem to be the unmarked alternative in actual use, in the sense that direct objects passivize only if they are relativized or questioned, or if they are highly topical and the \isi{indirect object} supplies \isi{new information}.



The main focus in this article is the introduction of \isi{object symmetry} in the \isi{passive voice}, i.e. the possibility of passivizing the \isi{indirect object}, during the 19\textsuperscript{th} century. As will become clear as we proceed, both mono- and bimorphemic \isi{ditransitive} verbs could passivize both before and after the change. I will also discuss the \isi{choice of subject} and argument order as I compare 19\textsuperscript{th} century Swedish with older stages of the language.



The \isi{object symmetry} illustrated in \REF{ex:falk:1} has been analysed e.g. by \citet{HolmbergPlatzack1995}, \citet{Platzack2005,Platzack2006}, and \citet{HaddicanHolmberg2019}. I will return to these proposals in \sectref{sec:falk:3.5}, and briefly compare them to my own analysis. To the extent that older stages in Swedish are mentioned by these authors, the situation is compared with Modern \ili{Icelandic} and Modern \ili{German}, where the \isi{lexical case} of the \isi{indirect object} is preserved under passivization. In a similar vein, \citet{Falk1995, Falk1997} proposes that the change reflects the final loss of \isi{lexical case} in Swedish. However, the case of the \isi{indirect object} before the change had some properties not normally associated with \isi{lexical case}, and after the change the \isi{indirect object} showed some atypical properties for \isi{structural case}. This is the topic of the next section.


\subsection{The case of the indirect object before and after the change: Some superficial properties}\label{sec:falk:2.2}


Preserved morphological case in the \isi{passive voice} is often seen as a property of \isi{lexical case}: it is a lexical property of the verb to assign a certain case, and this is preserved under passivization. \REF{ex:falk:3} shows that \isi{dative} case in \ili{Old Swedish} occurred in active \REF{ex:falk:3a} and \isi{passive} \REF{ex:falk:3b} examples alike. Verb complements without a \isi{lexical case}, on the other hand, turned up in the \isi{nominative} \isi{structural case} in the \isi{passive voice}, as shown with \textit{jak}, ‘I.\textsc{nom}’ in \REF{ex:falk:4b}; cf. the active example in \REF{ex:falk:4a}, where the corresponding argument (\textit{hona}) has an \isi{accusative} ending.

\ea%3
    \label{ex:falk:3}
\ea\label{ex:falk:3a}
\gll þu              böte          siukom\\
      you.\textsc{sg.nom}  cure.\textsc{pst.sg}    ill.\textsc{pl}.\textsc{dat}\\
\glt ‘You cured ill people’ (Leg Bu, EOS, p. 78)

\ex\label{ex:falk:3b}
\gll bötes            mangom\\
      cure.\textsc{pst.sg.pass}  many.\textsc{dat}\\
\glt ‘Many people were cured’ (Leg Bu, EOS, p. 417)
\z
\ex%4
    \label{ex:falk:4}

\ea\label{ex:falk:4a}
\gll huru  guz      ængla  lyptu      hona    gen    himnum\\
      how  God.\textsc{poss}  angels  lift.\textsc{pst}.\textsc{pl}  her.\textsc{acc}  towards  heaven.\textsc{dat}\\
\glt ‘how God’s angels lifted her towards heaven’ (Leg Bil, EOS, p. 272)

\ex\label{ex:falk:4b}
\gll swa  lyptis                                        jak               vij    sinnum  hwar  dagh  a.  xxx  arum  aff guz         ænglum\\
      so    lift.\textsc{pst}.\textsc{sg}.\textsc{pass}  I.\textsc{nom}  seven  times    each  day  in   30  years  by  God.\textsc{poss} angel.\textsc{pl.dat}\\
\glt ‘I was lifted seven times each day for 30 years by God’s angels’ (Leg Bil, EOS, p. 273)
    \z
    \z

In this respect, the case of the \isi{indirect object} looks like a \isi{lexical case} before the change. 


The \isi{indirect object} did not block \isi{movement} of the \isi{direct object} to the \isi{subject position}, and this could also be seen as an effect of the \isi{lexical case}, a lexical property that does not interfere in relationships established in the syntax. However, in other respects the case of the \isi{indirect object} had some atypical properties. Firstly, after the loss of the \ili{Old Swedish} \isi{case system}, the \isi{indirect object} had no \isi{morphologically distinct} form. In other languages with \isi{lexical case}, like \ili{Icelandic}, \ili{German}, or \ili{Old Swedish}, lexical cases typically have a distinctive form, like \isi{dative} or \isi{genitive}. The form \textit{mig} in \REF{ex:falk:2a} is the only object form, regardless of lexical or \isi{structural case}, however.



Secondly, \isi{lexical case} is often thought of as a verb-idiosyncratic property (see e.g. \citealt{Thrainsson2007}: 182). A small number of verbs had this property in \ili{Early Modern Swedish}: the \isi{experiencer} of verbs like \textit{lika} ‘like’, \textit{angra} ‘regret’ received an \isi{oblique case}, and this did not follow from any other property. These verbs lost \isi{lexical case} during the 16\textsuperscript{th} and 17\textsuperscript{th} centuries, leading to the change illustrated in \REF{ex:falk:5} (\citealt{Lindqvist1912,Falk1997}):\footnote{Since \isi{lexical case} with one argument verbs like \textit{hialpa} ‘help’ had already been lost in \ili{Late Old Swedish} \citep{Falk1995}, only \isi{ditransitive} verbs are included in the present study.}


\ea%5
    \label{ex:falk:5}
\ea
\gll hanum    angrar        thz    (OS)\\
      him.\textsc{dat} regret.\textsc{prs.sg}  it\\
\glt ‘He regrets this’
\ex
\gll Han    ångrar    det     (PDS)\\
      he.\textsc{sbj}  regret.\textsc{prs}  it\\
\z
\z

\begin{sloppypar}
After this loss, \isi{ditransitive} verbs were the only verbs with a case resembling \isi{lexical case}. But this actually follows from another property of ditransitives, namely the very fact that they are \isi{ditransitive}: in other words, the case of indirect objects was a verb-type idiosyncratic property. In this respect, it resembled a \isi{structural case}, in the sense of a “case associated with a certain syntactic function” – with the fundamental difference that the \isi{indirect object} was in a way “trapped” in this function, since it could not be \isi{promoted to subject} in passives.
\end{sloppypar}


Turning to the supposed \isi{structural case} of the \isi{indirect object} in present-day Swedish, it also has some unexpected properties. 



Firstly, the possibility of passivizing the \isi{direct object} has not been lost; see \REF{ex:falk:1}. Obviously, this contrasts with a \isi{minimality constraint} on DP \isi{movement}, but somehow the \isi{indirect object} does not intervene in the chain between the \isi{subject position} and the \isi{direct object} position. Compare this with the situation in \ili{English}, an \isi{object-asymmetrical language}, where only the underlying \isi{indirect object} may be passivized:


\ea%6
    \label{ex:falk:6}
\ea[]{He was given the book}
\ex[*]{The book was given him}
\z
\z


Secondly, \isi{definiteness} effects in existential constructions are commonly analysed as a consequence of interpreting the VP-internal DP as a \isi{VP-internal subject} (\isi{associate subject}). \isi{Definiteness} effects are found only on the underlying \isi{direct object} in the \isi{passive voice}. Hence, \REF{ex:falk:7a}, with a definite \isi{indirect object} and an \isi{indefinite} \isi{direct object} as the \isi{associate subject}, is grammatical, whereas a definite \isi{direct object} leads to ungrammaticality even if the \isi{indirect object} is \isi{indefinite}, as in \REF{ex:falk:7b}.

\ea%7
    \label{ex:falk:7}
\ea[]{\label{ex:falk:7a}
\gll Det  erbjöds      Karolina  en  lägenhet\\
      it      offer.\textsc{pst}.\textsc{pass}  Karolina  an  apartment\\
\glt ‘Karolina was offered an apartment’}

\ex[*]{\label{ex:falk:7b}
\gll   Det  erbjöds        en  släkting  lägenheten\\
       it    offer.\textsc{pst}.\textsc{pass}  a  relative  apartment.\textsc{def}\\}
\z
\z

Again, this suggests that the \isi{indirect object} is somehow invisible when establishing a relationship between the \isi{subject position} and the \isi{direct object} position.

\section{Data}\label{sec:falk:3}


In this section, I first give an overview of how data are collected and analysed (§§\ref{sec:falk:3.1}--\ref{sec:falk:3.2}). Patterns from earlier stages, including the earliest examples of passivized indirect objects, are then presented (§§\ref{sec:falk:3.3}--\ref{sec:falk:3.4}), but the focus is the 19\textsuperscript{th} century (§§\ref{sec:falk:3.5}--\ref{sec:falk:3.7}). A final section (\sectref{sec:falk:3.8}) is concerned with changes in argument order before and after 1800.

\subsection{Data sources}
\label{sec:falk:3.1}

The main focus of my investigation is the 19\textsuperscript{th} century, the period during which \isi{object symmetry} first emerged. Data from the 19\textsuperscript{th} century are taken from two different sources. The \isi{SPF corpus} of Swedish prose fiction 1800–1900 (available in \isi{Korp}, \citealt{BorinEtAl2012}) is a corpus of novels from 1800 to 1901.\footnote{\url{https://spraakbanken.gu.se/korp/}}  The other source is recommendations from \isi{normative grammarians}, represented by the first editions of the Swedish Academy word list (\isi{SAOL}).\footnote{\isi{SAOL} is a word-list with contemporary vocabulary, information about morphology, and sometimes information/recommendations on style, usage, etc. \isi{SAOL} has been regularly updated with new editions since 1874. Earlier editions are available at \url{http://spraakdata.gu.se/saolhist/}{} }


In addition, data from \ili{Early Old Swedish} have been collected from the rich sample of examples in \citegen{Holm1952} investigation of the \textit{s}{}-\isi{passive}. I have also manually excerpted an \ili{Early Old Swedish} collection of legends (Leg Bu and Leg Bil). \ili{Late Old Swedish} is represented by passivized \isi{ditransitive} verbs in Söderwall’s dictionary of \ili{Old Swedish} (Sdw); this includes a total of 24 verbs. For later periods, I have manually excerpted 19 texts (see Sources below or \citealt{Falk1993}: 335–338, authors born 1571–1735, for details). To complete the picture, I have collected examples of \isi{ditransitive} verbs from before 1800 in the Swedish Academy dictionary (\isi{SAOB}).\footnote{Available at \url{https://www.saob.se/}}  \isi{SAOB} is a historical dictionary, covering the vocabulary of Swedish from 1526.\footnote{The focus in \isi{SAOB} is on semantics, and less attention is paid to syntax. Of course, this makes \isi{SAOB} a less suitable source for my purpose.}


\subsection{Identifying passivized indirect objects}\label{sec:falk:3.2}


As illustrated in \REF{ex:falk:2}, I have used both morphological and word order criteria to identify which object is passivized. The four-\isi{case system} of \ili{Old Swedish} was lost in \ili{Late Old Swedish}, and the only nominal category that preserved a distinction between subject and \isi{object case} was that of personal pronouns. In the majority of examples, the \isi{indirect object} is pronominal, and its case reveals which object is passivized. For other nominal categories, word order can sometimes identify which object is passivized, but there are also ambiguous examples.


Due to the V2 property of Swedish, it is not possible to tell which object is passivized if the word order is DP\,+\,finite verb\,+\,DP in main clauses and embedded clauses that allow \isi{V2 order}. Compare (\ref{ex:falk:8}a–b).


\ea%8
    \label{ex:falk:8}
\ea \label{ex:falk:8a}
\gll Honom  räckes        en    riktig  kardinalsup\\
      him.\textsc{obj}  hand\textsc{.prs}.\textsc{pass}  a      real   cardinal.glass.of.spirit\\
\glt ‘A very big glass of spirit is handed to him.’ (SPF, 1900)

\ex \label{ex:falk:8b}
\gll denna [armén]    till-fogades      ett  nederlag\\
      this {[the army]}    to-add.\textsc{pst}.\textsc{pass}  a   defeat\\
\glt       ‘the army was defeated’ (SPF, 1900)\footnote{On compound verbs like \textit{till-foga} ‘to-add; inflict on’, see footnote \ref{fn:02:15} below.}

\ex \label{ex:falk:8c}
\gll en    och  annan  beröfvades      sitt      gevär\\
      one  and  another  deprive\textsc{.pst}.\textsc{pass}  \textsc{poss.refl}  gun\\
\glt ‘some men were deprived of their guns’ (SPF, 1900)
\z
\z

\REF{ex:falk:8a} shows a fronted \isi{indirect object} with \isi{object case}, and this is a possible analysis of \REF{ex:falk:8b} as well, where case is ambiguous. Since it is not possible to know which of the objects have been passivized in \REF{ex:falk:8b}, examples such as this have not been included among the examples of passivized indirect objects. In \REF{ex:falk:8c}, the word order is the same as in (\ref{ex:falk:8}a–b), but here the use of the \isi{reflexive} pronoun \textit{sin} reveals that the \isi{indirect object} is passivized.\footnote{The \isi{reflexive} \isi{possessive} must be bound within a binding domain. In the \isi{active voice}, the \isi{indirect object} can – according to some speakers only marginally – bind a \isi{reflexive} \isi{direct object}: 
\ea \gll Jag  gav    honom    sin    docka\\
     I      give.\textsc{pst}  him      \textsc{poss}.\textsc{refl}  doll\\
    \glt ‘I gave him his doll’
\z This possibility is probably not available in the \isi{passive voice}, i.e. it is less probable that the \isi{indirect object} could retain its status as an \isi{indirect object} in the \isi{passive} as the antecedent of a \isi{reflexive} pronoun. The \isi{direct object} would then be analysed as an associate \isi{VP-internal subject}, but definite.}


If the \isi{direct object} occurs after a non-finite verb, the \isi{indirect object} is analysed as having been passivized in examples like the following:\largerpage

\ea%9
    \label{ex:falk:9}
\gll de  i     trångmål  stadda  grupperna    måste  lämnas          allt  möjligt    bistånd\\
    the  in  trouble    being    group\textsc{.pl.def}   must  render\textsc{.inf}.\textsc{pass}  all    possible  help\\
\glt ‘the groups in trouble must be given all possible help’ (SPF, 1900)
\z


When both objects occur post-verbally, the word order \isi{direct object}\,+\,\isi{indirect object} shows a \isi{passivized direct object}:

\ea%10
    \label{ex:falk:10}
    \gll {{\ob}Genom Hansestädernas tullfrihet]} \textit{från-drogos …} betydande   inkomster  svenska    kronan \\
    {{\ob}Since the Hansa-towns were exempted from duty]}   from-draw\textsc{.pst.pl}.\textsc{pass}    important   income\textsc{.pl}  Swedish   crown.\textsc{def}\\
\glt ‘Since the Hansa-towns were exempted from duty, the Swedish crown was deprived of important income’ (\isi{SAOB} \textit{fråndraga}, 1911)
\z


The opposite order does not, however, unambiguously involve a \isi{passivized indirect object}. An \isi{indefinite} DP could be left in situ as an \isi{associate subject}. In present-day Swedish, only the \isi{direct object} may be construed as an \isi{associate subject} in a \isi{passive ditransitive} (cf. \REF{ex:falk:7} above). In \REF{ex:falk:11a} the presence of {the \isi{expletive} subject \textit{det} ‘there’} shows that the \isi{direct object} is an \isi{associate subject} and that the object status of the preposed \isi{indirect object} is preserved. An \isi{expletive} subject \textit{det} is normally obligatory. Earlier – and to some extent still– \textit{det} could be absent if a \isi{locative} was topicalized or if a \isi{locative} \isi{adverb} (\textit{där} ‘there’, \textit{här} ‘here’) occupied the \isi{subject position} immediately following the finite verb. Thus, clauses like \REF{ex:falk:11b} are not taken to be instances of a \isi{passivized indirect object}:

\ea%11
    \label{ex:falk:11}
\ea\label{ex:falk:11a}
\gll Mången  af  oss  fattige  syndare  förunnas      det  icke  en  så    lång  betänketid            som  han  fått\\
many    of  us    poor    sinners    grant\textsc{.prs}.\textsc{pass}     it    not  a  such  long      time.for.consideration  that  he    got\textsc{.sup}\footnotemark{}\\
\glt ‘For many of us, poor sinners, there is not such a long time for consideration granted as he had got’ (SPF, 1880)
\footnotetext{In embedded clauses the temporal \isi{auxiliary} \textit{ha} ‘have’ may be deleted. The \isi{supine} is the active past \isi{particle} in Swedish, used to form anterior tenses. The \isi{supine} may be passivized; see example \REF{ex:falk:12c}}.

\ex\label{ex:falk:11b}
\gll I  en  not  tilldelas            der  Sara  Widebeck  en  örfil\\
      in  a  note  to-share\textsc{.prs}.\textsc{pass}    there  Sara  Widebeck  a  box.on.the.ear\\
\glt ‘In a note, Sara Widebeck is given a box on the ear’ (SPF, 1840)
\z
\z


In embedded clauses that do not allow \isi{main clause} word order, I have analysed the DP in front of the finite verb as the subject; see (\ref{ex:falk:12}a–b). In the relative clauses in (\ref{ex:falk:12}c–d), the post-verbal DP shows its status as an object – the \isi{direct object} in \REF{ex:falk:12c}, and the \isi{indirect object} in \REF{ex:falk:12d}:

\ea%12
    \label{ex:falk:12}
\ea\label{ex:falk:12a}
\gll i  samma  stund    mamsell  Hagman  der    lemnades      inträde\\
      in  same    moment miss    Hagman  there  leave\textsc{.pst}.\textsc{pass}     entry\\
\glt ‘just as Miss Hagman was given permission to enter there’ (SPF, 1841)

\ex \label{ex:falk:12b}
\gll det  mått      af  bildning    och  kunskaper,  våra  barn    kunde bibringas\\
      the  amount  of  education  and  knowledge  our  children  can.\textsc{pst}     impart.\textsc{inf}.\textsc{pass}\\
\glt ‘the amount of education and knowledge that could be imparted to our children’ (SPF, 1886)

\ex \label{ex:falk:12c}
\gll Dessa  orolige  varelser   som  gifwits      talande    tungor\\
      these    anxious  creatures that  give\textsc{.sup.pass}  speaking  tongues\\
\glt ‘These anxious creatures that have been given speaking tongues’ (SPF, 1900)

\ex \label{ex:falk:12d}
\gll Glad        öfver  det  bifall    som  skänktes      detta  hennes  försök\\
      delighted  over  the  applause    that  give\textsc{.pst}.\textsc{pass}  this  her    try\\
\glt ‘Delighted at the applause that was given to this try of hers …’ (SPF, 1840)
\z
\z


A final criterion for identifying passivized indirect objects is when they are left out of coordinations (see \ref{ex:falk:13a}) or remain implicit in control infinitivals (as in \ref{ex:falk:13b}):

\ea%13
    \label{ex:falk:13}
\ea \label{ex:falk:13a}
\gll Hon  var      ganska  lydig      och  snäll,  men  nekades      just    heller ingenting\\
 she  be\textsc{.pst}  quite    obedient  and  kind,  but  deny.\textsc{pst.pass}  really    neither        nothing\\
\glt ‘She was quite obedient and kind, on the other hand, she was never denied anything’ (SPF, 1898)

\ex \label{ex:falk:13b}
\gll jag [var]   utsedd  att  på  en    gång  beröfvas            allt  hvad  för   mit  hjerta utgjort         sällhet    och  fröjd\\
I    was  destined  to  at  one  time  deprive\textsc{.inf.pass}  all    what  for   my  heart    constitute.\textsc{sup}    happiness  and  joy\\
\glt ‘I was destined to be deprived at the same time of everything that had been happiness and joy for me’ (SPF, 1840)
\z
\z


In the following subsections, I will present data from \ili{Old Swedish} (\sectref{sec:falk:3.3}), from the period 1526–1899 (\sectref{sec:falk:3.4}), and from the 19\textsuperscript{th} century as represented in the \isi{SPF corpus} (\sectref{sec:falk:3.5}), in addition to the recommendations in \isi{SAOL} (\sectref{sec:falk:3.6}). \sectref{sec:falk:3.7} discusses the first occurrences of a \isi{passivized indirect object}, and \sectref{sec:falk:3.8} contains comparisons between the data from the \isi{SPF corpus} and the earlier periods.

\subsection{Old Swedish}\label{sec:falk:3.3}


In \ili{Old Swedish}, the most common case pattern with \isi{ditransitive} verbs was to have the \isi{indirect object} in the \isi{dative} and the \isi{direct object} in the \isi{accusative}. In the \isi{passive voice}, the \isi{dative} was preserved, while the underlying \isi{direct object} turned up in the \isi{nominative}. The \isi{dative} often preceded the \isi{nominative} in the linear order. \REF{ex:falk:14} shows this pattern in main clauses:\footnote{In the glosses, I distinguish between \textsc{nom,} \textsc{acc,} and \textsc{dat} also when the case is not unambiguous morphologically (like \textit{mik} in example \ref{ex:falk:14a}), but shares a pattern with unambiguous cases.}

\ea%14
    \label{ex:falk:14}
\ea \label{ex:falk:14a}
\gll Mik    laghdos        tue  andra  costa\\
      me\textsc{.dat}  lay\textsc{.pst.pl.pass}  two  other    choice\textsc{.pl.nom}\\
\glt ‘Two different choices were proposed to me’ (Leg Bu, EOS, p. 143)

\ex \label{ex:falk:14b}
\gll Vitiz              manni    skoghæ      brennæ\\
      accuse\textsc{.prs.sg.pass}  man\textsc{.dat}  forest.\textsc{pl.gen}  fire.\textsc{sg.nom}\\
\glt ‘Someone is accused of causing a forest fire’ (legal text, early 13\textsuperscript{th} century; from \citealt{Holm1952}: 200)
\z
\z

In \REF{ex:falk:14a}, the \isi{dative} is topicalized. In \REF{ex:falk:14b} both nominals are post-verbal, with the \isi{dative} preceding the \isi{nominative}. However, the \isi{nominative} could also be topicalized, as in \REF{ex:falk:15a}. It was also possible to have \isi{nominative}\,+\,\isi{dative}, but this was less common (cf. \ref{ex:falk:15b}):

\ea%15
    \label{ex:falk:15}
\ea \label{ex:falk:15a}
\gll Þiuf      scal          a  þingi    frændum        byuþæs\\
      thief.\textsc{nom}  shall.\textsc{prs.sg}  at  thing.\textsc{dat}  relative.\textsc{pl}.\textsc{dat}  offer.\textsc{inf.pass}\\
\glt ‘The thief shall be offered to the relatives at court (to free him by paying his fine)’ (legal text, early 14\textsuperscript{th} century; from \citealt{Holm1952}\textsc{:} \textsc{252)}

\ex \label{ex:falk:15b}
\gll Tha  giwis          gotz        hans  fore  siäl  hans,  kirkium ok    klostrum\\
    then  give.\textsc{prs.sg.pass}  property.\textsc{nom}  his    for    soul  his    church.\textsc{pl.dat}   and  monastery.\textsc{pl.dat}\\
\glt ‘Then his property is given to churches and monasteries for his soul’ (legal text, early 14\textsuperscript{th} century; from \citealt{Holm1952}: 249)
\z
\z

\REF{ex:falk:16}          shows the two possibilities in embedded clauses:

\ea%16
    \label{ex:falk:16}

\ea
\gll at    them    skulle        witas            thiwffnadher\\
      that  them.\textsc{dat}  shall.\textsc{pst.sg}  accuse.\textsc{inf.pass}    theft.\textsc{sg.nom}\\
\glt ‘in order to accuse them of the theft’ (\isi{Bible} paraphrase, 1330s; from \citealt{Holm1952}: 345)

\ex
\gll før      æn    altara        giordus          sancto      sebastiano j  papia   lombardiestadh\\
      before  than     altar\textsc{.pl.nom}  make.\textsc{pst.pl.pass}  saint.\textsc{dat}  Sebastian.\textsc{dat}       in Pavia    {Lombardic    town}\\
\glt ‘before altars were made in honour of Saint Sebastian in Pavia, a Lombardic town’ (Leg Bil, EOS, p. 481)
\z
\z


The relative weight of the two arguments may have been of importance; pronouns (often the \isi{dative}) tended to precede nouns and full noun phrases (often \isi{nominative}).  


In the sample of 112 \isi{ditransitive} \textit{s}{}-passives in \ili{Old Swedish}, the \isi{dative} precedes the \isi{nominative} in 65 clauses (58\%).\footnote{A \isi{relativized object} is counted as preceding the object in situ, even when it does not correspond to an overt nominal:\label{fn:02:13}

\ea \gll Hin  sum  sakin          gifs\\
         he    that  cause.\textsc{nom.def}  give.\textsc{prs.pass}\\
    \glt ‘The person who is prosecuted’ (legal text, 1280s; from \citealt{Holm1952}: 211)
\ex \gll all    þön   mall  presti      kunnu  witas\\
         all    the    causes  priest.\textsc{dat}    can.\textsc{prs.pl}  accuse.\textsc{inf.pass}\\
    \glt ‘all the causes that a priest can be accused of’ (legal text, 1327; from \citealt{Holm1952}: 241)
\z} Thus, we see a small preference for \isi{dative}\,+\,\isi{nominative}.


\subsection{Early Modern Swedish (1526–1799)}\label{sec:falk:3.4}\largerpage[2]


In \ili{Early Modern Swedish}, a preserved morphological \isi{dative} is found only occasionally. Instead of talking about \isi{dative} and \isi{nominative}, I therefore use the labels indirect and \isi{direct object} to refer to the functions of the constituents in the \isi{active voice}. In the data collected from the period 1526–1799, no clear preference is found: indirect objects precede the \isi{direct object} in 82 of a total of 162 clauses in the \textit{s}{}-\isi{passive} (=\,51\%).


Ten of the 162 clauses show passivized indirect objects (6\%). Four of them are identified through word order, with the \isi{indirect object} preceding the finite verb in an \isi{embedded clause} (see \ref{ex:falk:17a}), the others by other means: a personal pronoun (as in \ref{ex:falk:17b}), a \isi{reflexive} \isi{possessive} pronoun \textit{sin} in the \isi{direct object} \REF{ex:falk:17a}, or deletion in coordination \REF{ex:falk:17d}:


\ea%17
    \label{ex:falk:17}
\ea\label{ex:falk:17a}
\gll där  Ryssen      presenteras       en  sådan  tractat …Och    där    Swerige  anmodes            en  tractat …\\
if    Russian\textsc{.def}  propose\textsc{.prs}.\textsc{pass}    a  such    \isi{agreement}     and    if         Sweden  propose\textsc{.prs}.\textsc{sg}.\textsc{pass}  an  \isi{agreement}\\
\glt ‘if such an \isi{agreement} is proposed to the Russians … and if an \isi{agreement} is         proposed to Sweden …’ {(\isi{SAOB} \textit{anmoda}, 1633)}

\ex\label{ex:falk:17b}
\gll Jag  ville      icke  nu    resa      samma [resa till Lappland] om  jag  bödes           1000  plåtar\\
I    want.\textsc{pst}  not  now  travel\textsc{.inf}  same     trip to Laponia  if  I.\textsc{sbj}   offer.\textsc{pst.sbjv}.\textsc{pass}  1000  crowns\\
\glt ‘I would not want to travel on the same trip, even if I were offered 1000 crowns.’ (\isi{SAOB} \textit{bjuda}, 1732)

\ex\label{ex:falk:17c}
\gll När  et  träd  skall        af-klädas          sin        bark … \\
      when  a  tree  shall.\textsc{prs}  off-dress\textsc{.inf}.\textsc{pass}  \textsc{poss}.\textsc{refl}    bark\\
\glt ‘When a tree shall be debarked …’ (\isi{SAOB} \textit{afkläda}, 1779)

\ex\label{ex:falk:17d}
\gll Hwadh  orätt    och  swårigheeter  som  och  omkostningar  iag    af  denna Människian    lider          och  på-kastas\\
what    wrong  and  troubles       as     also  costs          I.\textsc{sbj}  by  this            person.\textsc{def}    suffer.\textsc{prs.sg}    and  on-throw.\textsc{prs}.\textsc{pass}\\
\glt ‘Such troubles as well as costs that I suffer and that are thrown on   me’ {(\isi{SAOB} \textit{påkasta}, 1704)}
\z
\z

I have taken the word order in \REF{ex:falk:17a} as an indication that the \isi{indirect object} is passivized (cf. \ref{ex:falk:12}a–b). Certainly, in present-day Swedish this order indicates that the \isi{indirect object} is passivized. It is less clear here, though. Compare the following contemporary example with a pre-verbal \isi{dative} \isi{indirect object} in a \isi{periphrastic passive}:\largerpage

\ea%18
    \label{ex:falk:18}
\gll När  människiom      är        något      aff-stulit …. \\
    when  human.beings\textsc{.dat}  be.\textsc{prs.sg}  something  from-steal.\textsc{ptcp}\\
\glt ‘When something is stolen from human beings …’ (\isi{SAOB} \textit{avstjäla}, 1629)
\z


A word order like that in \REF{ex:falk:18} is probably a remnant of a more frequent pattern in \ili{Old Swedish}, \isi{stylistic fronting}, and the same could be the case in clauses like \REF{ex:falk:17a}.\footnote{\isi{Stylistic fronting} is a construction in which any type of \isi{constituent} can occupy the position in front of the finite verb in embedded clauses without an overt subject. \isi{Stylistic fronting} also appears when a subject/\isi{nominative} (\isi{indefinite}) is left in situ in the verb phrase, as is the case in \REF{ex:falk:17a}. See \citet[326]{Falk1993} for statistics on the diachronic development of this construction type.}

\subsection{Passivized indirect objects 1800–1900}\label{sec:falk:3.5}


I have investigated passivized \isi{ditransitive} verbs in 19\textsuperscript{th} century texts using the \isi{SPF corpus}. I have divided the corpus into three parts, and investigated 30 \isi{ditransitive} verbs in total,\footnote{A \isi{productive} way to form \isi{ditransitive} verbs was compounding with a \isi{prepositional prefix}. I have chosen six different prepositional affixes and counted them as only one verb each; see Appendix~\ref{falk:appendix:1} for the selected verbs.\label{fn:02:15}} all attested in the corpus in the \textit{s}{}-\isi{passive} (see Appendix~\ref{falk:appendix:1}). However, not all 30 verbs are attested in all three parts of the corpus, as shown in \tabref{tab:falk:1}.

\begin{table}
\caption{Size in tokens and attested ditransitive passive \textit{s}{}-verbs in three parts of the SPF corpus\label{tab:falk:1}}
\begin{tabular}{lcc}
\lsptoprule
years & size in tokens & attested verbs\\\midrule
1800–1843 & 2,203,451 & 26\\
1860–1880 & 4,231,554 & 25\\
1898–1901 & 9,837,169 & 30\\
\lspbottomrule
\end{tabular}
\end{table}

As is evident, the corpus includes considerably more texts from later periods, which makes direct comparisons difficult: an unusual construction type like a passivized \isi{ditransitive verb} is more likely to turn up in a larger corpus. However, even with this in mind, a tendency of growing possibilities to passivize the \isi{indirect object} can be detected. 


I used two different means to measure the change. First, I counted all instances of passivized indirect objects per million words; second, I counted how many of the attested verbs have a \isi{passivized indirect object}. 


\begin{table}
\caption{Passivized io/million words and number of verbs with a passivized io (pass. io = passivized indirect object)\label{tab:falk:2}}
\begin{tabular}{lccr@{}c}
\lsptoprule
years & \multicolumn{4}{c}{pass. io} \\\cmidrule(lr){2-5}
      & total & per mil. words & \multicolumn{2}{c}{verb with pass.io/}\\
      &       &                & \multicolumn{2}{c}{attested \isi{passive} verbs}\\\midrule
1800–1843 & 13 & 5.9 & 7/26  & ca. 1 in 4 \\
1860–1880 & 20 & 4.7 & 12/25 & ca. 5 in 10\\
1898–1901 & 69 & 7.0 & 19/30 & ca. 6 in 10\\
\lspbottomrule
\end{tabular}
\end{table}

The number of verbs that occur with a \isi{passivized indirect object} grows over time. However, the number of passivized indirect objects per million words is lower in the mid-period, probably because the corpus is too small to reveal the full picture. It is clear, though, that a more general possibility of passivizing the \isi{indirect object} is found towards the end of the 19\textsuperscript{th} century.

\subsection{Passivized indirect objects and normative grammar}\label{sec:falk:3.6}


The use of passivized indirect objects may have been influenced by statements or recommendations from \isi{normative grammarians}. In the first edition of the influential \textit{Riktig svenska} (‘Proper/Appropriate Swedish’), \citet{Wellander1939} advises against passivized indirect objects. However, he recognizes that it could sometimes be a flexible (“\textit{smidig}”) construction, for instance in coordinations (\citeyear[291]{Wellander1939}; see example \REF{ex:falk:13a} above). In the 4\textsuperscript{th} and last edition, he accepts the construction: “Den ökade friheten i konstruktionen gör otvivelaktigt språket smidigare, lätthanterligare” (‘Without doubt, the greater freedom in the construction makes the language more flexible, easier to handle’; \citeyear[148–149]{Wellander1973}).


At the same time, recommendations like this show that a certain amount of variation is found in language use; otherwise, a \isi{recommendation} would not be necessary. And even in the first edition, Wellander gave more than three pages of examples with passivized indirect objects with a variety of different verbs (\citeyear[297–301]{Wellander1939}).  



The recommendations provided in \isi{SAOL} are also illustrative. In the first edition (1874), passivized indirect objects are sometimes called incorrect (“\textit{origtigt}”, “\textit{orätt}”, “\textit{felaktigt}”). This judgement is given in connection with 13 of the 30 verbs investigated here (see Appendix~\ref{falk:appendix:1}).\footnote{I find it less likely that the editors accepted passivized indirect objects for all of the other 17 verbs, such as \textit{gifva} ‘give’, \textit{skänka} ‘give’, but perhaps they fully accepted it with the verb \textit{beröva} ‘deprive of’.} Statements of this kind show that passivized direct objects were found at this time – no statements on “incorrectness” are necessary for non-existent alternatives. “Incorrect” was replaced with a \isi{recommendation} to passivize the \isi{direct object} rather than (“\textit{hellre än}”) the \isi{indirect object} in \isi{SAOL} 7 (1900). In still later editions, \textit{även} ‘also’ indicates the \isi{passivized indirect object} as a marked alternative. The two alternatives are not given as equals for \textit{tilldela} ‘award’, indicated by \textit{eller} ‘or’, until \isi{SAOL} 11 (1986); for \textit{erbjuda} ‘offer’, no comments on \isi{choice of subject} are given in \isi{SAOL} 11.\footnote{A complete investigation of comments on use in the editions of \isi{SAOL} remains to be done. \textit{Tilldela} (with the two alternatives given as equally possible) and \textit{erbjuda} (without comment) from \isi{SAOL} 11 are presumably representative.}



The resistance from \isi{normative grammarians} may have influenced the use of passivized indirect objects in the written language. However, the situation in the written language shows that passivized indirect objects were not completely avoided: of the 12 verbs for which \isi{SAOL} 1 (1874) judged a \isi{passivized indirect object} as “incorrect”, seven are attested in the \isi{SPF corpus} with a \isi{passivized indirect object}. As for the five verbs not attested, the \isi{judgements} in \isi{SAOL} reveal that they were used to some extent, even though not found in the \isi{SPF corpus}.



In the spoken language, \isi{passive ditransitive} verbs have probably always been quite uncommon. Whether they belong only to the written language or not, language users today have intuitions about them (cf. the investigation in \citealt{HaddicanHolmberg2019} described in \sectref{sec:falk:2.1} above). It is simply impossible to detect the intuitions of earlier generations – we are left with the fact that a more general possibility of passivizing indirect objects can be detected by the 19\textsuperscript{th} century, possibly the later part.


\subsection{The first instances of passivized indirect objects}\label{sec:falk:3.7}\largerpage


In this section, the focus is on the first instances of a \isi{passivized indirect object} with the individual verbs. The \isi{question} is whether the verb types that are attested early with a \isi{passivized indirect object} can tell us something about the change. To complete the picture of the 19\textsuperscript{th} century, I have used \isi{SAOB} to search for older examples than those found in the \isi{SPF corpus}. I have also used a corpus of Swedish novels written 1830–1942 (\textit{Äldre svenska romaner}), but without finding any relevant examples (i.e. older examples of passivized indirect objects).\footnote{\textit{Äldre svenska romaner} is available at \url{https://spraakbanken.gu.se/korp/}. The corpus is smaller than the \isi{SPF corpus} (4.2 million words vs. 16.3 million words) and provides hardly any examples of passivized indirect objects.}


It is possible to distinguish different groups of verbs based on formal properties. As was shown in \sectref{sec:falk:2.1}, native \isi{speaker} intuitions about passivized \isi{ditransitive} verbs reveal that the formal properties of the verb are relevant in present-day Swedish: verbs with a PP as an alternative to the \isi{indirect object} are less acceptable in the \isi{passive voice} (e.g. \citealt{Anward1989}); monomorphemic verbs are also less acceptable than bimorphemic verbs in the \isi{passive voice} (\citealt{HolmbergPlatzack1995,HaddicanHolmberg2019}).



A division can also be made based on semantic properties. We can distinguish between \isi{ditransitive} verbs denoting some kind of \isi{transfer} \textit{to} or \isi{transfer} \textit{from} the referent denoted by the \isi{indirect object} (\isi{“to” verbs}, e.g. \textit{giva} ‘give’, and \isi{“from” verbs}, e.g. \textit{beröva} ‘deprive of’).\footnote{“\isi{Transfer}” should be understood in a wide sense: \isi{transfer} of a gift, an offer, an experience, a right, etc. Cf. \citealt{TelemanEtAl1999}/3: 315–318, \citet{Valdeson2021} for a more fine-grained semantic analysis.} As a third type, I distinguish a hindered \isi{transfer} (“hindered” verbs, e.g. \textit{bespara} ‘spare’; \citealt{Valdeson2021}). \isi{“To” verbs} are the typical class of \isi{ditransitive} verbs, while the two others are less typical, with only a few members in present-day Swedish.\footnote{All verbs investigated by Haddican \& Holmberg are \isi{“to” verbs}. \citet{TelemanEtAl1999} exemplify \isi{“from” verbs} and hindered \isi{transfer} verbs together (“\textit{berövas eller förvägras}” – ‘be deprived of’, ‘be refused’; \citealt{TelemanEtAl1999}/3: 316).} 



The first instances of passivized indirect objects are given in Appendix~\ref{falk:appendix:2}, together with the source and an analysis of formal and semantic properties. Four verbs are not attested with a \isi{passivized indirect object} at all in the material investigated (\textit{förlåta} ‘forgive’, \textit{förmena} ‘deny’, \textit{servera} ‘serve with’, \textit{visa} ‘show’). Complementary searches in \textit{Äldre} \textit{svenska romaner} yielded no examples; however, other corpora with fiction from the 20\textsuperscript{th} century show that it is indeed possible to passivize the \isi{indirect object} with these verbs (at least later on).\footnote{\textit{Bonniersromaner I}, novels edited at the publishing house Bonniers 1977–1978, available at \url{https://spraakbanken.gu.se/korp/}} 



Looking first at the semantics, only a few of the investigated verbs, four in total, are \isi{“from” verbs}, denoting that something is taken from somebody (or something). Five verbs denote hindered \isi{transfer}, i.e. the subject referent hinders a \isi{transfer} to somebody. The rest are \isi{“to” verbs}, i.e. denoting a successful or offered \isi{transfer} to somebody. 



Three of the four \isi{“from” verbs} are attested with a \isi{passivized indirect object} as early as 1850 or before, including three different \textit{av}{}- verbs (‘off’) that are found before 1800. But \isi{“to” verbs} are also found among the early examples, showing that typical \isi{ditransitive} semantics was compatible with a \isi{passivized indirect object} early on; see examples \REF{ex:falk:12a} and \REF{ex:falk:13b} above. Only one of the semantically atypical verbs of “hindered \isi{transfer}”, \textit{bespara} ‘spare’, is represented among the earliest examples:


\ea%19
    \label{ex:falk:19}
\gll Lycklig  derföre    den  …  som  besparades      den  svåra    kampen\\
    happy  therefore  any  {}    that  spare.\textsc{pst}.\textsc{pass}  the  difficult  struggle.\textsc{def}\\
\glt ‘Therefore, anyone who was spared the difficult struggle ought to be happy’ (SPF, 1840)
\z

As for the formal properties of the verbs in \isi{question}, six verbs represent the word formation pattern with a \isi{prepositional prefix}. Of the rest, 13 are bimorphemic with another kind of \isi{prefix} and 11 are monomorphemic. Of the six verb types formed by a \isi{prepositional prefix}, three are attested early with a \isi{passivized indirect object}, two of them (\textit{av}{}- ‘off’ and \textit{på-} ‘on’) even before 1800. Both mono- and bimorphemic verbs have early examples of passivized indirect objects. It can be noted, though, that almost all of the verbs for which \isi{SAOL} 1 (1874) explicitly rejects passivized indirect objects are bimorphemic; the notion “first attested 1874” is somewhat misleading here, since these remarks in \isi{SAOL} reflect an earlier use.


To sum up, my material on first occurrences does not show any clear patterns so far. I will return to the different verb types in \sectref{sec:falk:4} below.


\subsection{Choice of subject and argument order before and after 1800}\label{sec:falk:3.8}


We have seen that \isi{object symmetry}, in the sense that both objects may be passivized, became a possibility during the 19\textsuperscript{th} century. Nevertheless, passivized direct objects were more common than passivized indirect objects in the period investigated. In this section, I present data on the \isi{choice of subject}, before and after 1800.

\tabref{tab:falk:3} shows the tokens of passivized \isi{ditransitive} verbs for each period in the \isi{SPF corpus}.

\begin{table}
\caption{Tokens of ditransitive passive verbs in the SPF corpus\label{tab:falk:3}}
\begin{tabular}{lrcrcrcc}
\lsptoprule
          & \multicolumn{2}{c}{pass. io} & \multicolumn{2}{c}{pass. do} & \multicolumn{2}{c}{amb} & \\\cmidrule(lr){2-3}\cmidrule(lr){4-5}\cmidrule(lr){6-7}
years     & $n$ & \%     & $n$ & \%   & $n$ & \% & Σ\\\midrule
1800–1843 & 13  &  13  & 82  & 79 & 9  & 9 & 104\\
1860–1880 & 20  &  13  & 129 & 83 & 6  & 4 & 155\\
1899–1901 & 69  &  21  & 239 & 74 & 13 & 4 & 321\\
Σ         & 102 &  18  & 450 & 78 & 28 & 5 & 580\\
\lspbottomrule
\end{tabular}
\end{table}

The table shows a growing tendency to choose the \isi{indirect object} as the subject towards the end of the century, but the \isi{passivized direct object} is still the preferred alternative in the most recent period. Statistical data for the individual verbs are given in Appendix~\ref{falk:appendix:3} (but not divided into different periods since the numbers are low). With two verbs, passivized indirect objects are preferred over passivized direct objects, \textit{beröva} ‘deprive of’ and \textit{bibringa} ‘impart to’, but the latter is not very common. Even or close to even preferences are found with the less common verbs \textit{anförtro} ‘entrust to’, \textit{lova} ‘promise’, \textit{unna} ‘grant’, and compound verbs with \textit{å}{}- ‘on-’. However, with the majority of verbs, passivized direct objects are clearly preferred.\largerpage


The figures in \tabref{tab:falk:3} differ sharply from the situation in present-day Swedish, where we find a clear preference for passivized indirect objects (see \sectref{sec:falk:2.1}). For this reason, it is of interest to investigate word-order patterns in clauses with passivized direct objects during the 19\textsuperscript{th} century further. I will return to present-day Swedish in \sectref{sec:falk:5.3} below. Here, I will focus on the development from \ili{Old Swedish} to the end of the 19\textsuperscript{th} century.



Recall that the \isi{dative} tended to precede the \isi{nominative} in \ili{Old Swedish} (\sectref{sec:falk:3.3}), while no clear preferences were found in 1526–1799 (\sectref{sec:falk:3.4}). The \isi{choice of subject} as given in \tabref{tab:falk:3} does not fully correspond to the linear order of the arguments, though, as will be discussed further below. To allow a full comparison between the different stages of Swedish, I have analysed data from the \isi{SPF corpus} according to the same principles as in earlier stages, that is considering the ordering of the arguments. In \tabref{tab:falk:4} the arguments are labelled according to their syntactic function in the \isi{active voice}, that is as \isi{indirect object} (io) and \isi{direct object} (do). To give a more detailed picture, I have subdivided the periods further. \ili{Old Swedish} is divided into three groups, two covering \ili{Early Old Swedish} (EOS), the provincial laws representing the most archaic language, and one covering \ili{Late Old Swedish} (LOS). Turning to Early and \ili{Late Modern Swedish}, the period 1526–1799 is divided into two, with 1526–1699 as the first period, since this is the period during which Swedish lost \isi{lexical case} and non-referential subjects were introduced \citep{Falk1993}.


\begin{table}
\caption{Argument order 1225–1901, \textit{s}{}-passives}
\label{tab:falk:4}
\begin{tabular}{lcrrc}
\lsptoprule
& \multicolumn{1}{c}{io\,+\,do} & \multicolumn{1}{c}{do\,+\,io} & \multicolumn{1}{c}{Σ} & \multicolumn{1}{c}{\% io\,+\,do}\\\midrule
Provincial laws & 18 & 7 & 25 & 72\\
Other EOS sources & 27 & 16 & 43 & 63\\
LOS & 20 & 24 & 44 & 45\\
1526–1699 & 45 & 40 & 85 & 53\\
1700–1799 & 37 & 40 & 77 & 48\\
1800–1844 & 39 & 65 & 104 & 38\\
1860–1880 & 36 & 119 & 155 & 23 \\
1898–1901 & 95 & 226 & 321 & 30\\
\lspbottomrule
\end{tabular}
\end{table}

As seen in \tabref{tab:falk:4}, a decrease in the order io\,+\,do is already apparent in \ili{Old Swedish}, although the absolute figures are small. In \ili{Late Old Swedish} this ordering is actually less common than during the periods 1526–1799. For some reason, topicalized direct objects are more common during this period than in any other of the periods investigated, giving the low percentage for the order io\,+\,do. The percentage for io\,+\,do drops further during the 19\textsuperscript{th} century, to increase again towards the end of the century; this increase is an effect of the more common pattern of passivizing the \isi{indirect object}, often leading to the order io\,+\,do.


The decreasing preference for io\,+\,do is not an effect of the form of the \isi{indirect object}. Over time, pronominal indirect objects became more common, from about one-third in the laws, via about half in other \ili{Old Swedish}, to about two-thirds during 1526–1799. In the oldest SPF sample, pronominal indirect objects are somewhat less common (ca. 60\%), while the rest of the century shows a high proportion (ca. 75\%). Thus, despite the pronominal form, indirect objects show an increasing tendency to occur after direct objects.



The decreasing preference for io\,+\,do could, on the other hand, be an effect of the clause type, for which two factors are of relevance: how common a certain clause type is, and what the preference is within the different clause types.\footnote{As for clause types, it remains to be investigated whether the tendencies found in my sample of \isi{ditransitive} \isi{passive} verbs are true more generally, e.g. if relative clauses generally became more common over time.}  A full account of these two factors would lead us too far, but I will point out some general tendencies.



Clauses with topicalized or relativized direct objects will always have the order do\,+\,io. See examples \REF{ex:falk:15a} and footnote \ref{fn:02:13} above. As already mentioned, \ili{Late Old Swedish} has a comparatively high proportion of topicalized direct objects (12 out of 44 clauses). Setting \ili{Late Old Swedish} aside, topicalized direct objects are quite rare up until 1699 (6–7\%, $N = 153$), after that becoming somewhat more common (13–15\%, $N = 657$), leading to a decrease in the order io\,+\,do. As for clauses with relativized direct objects, they tend generally to become more common over time, from ca. 11\% in \ili{Old Swedish} ($N = 112$) to 33\% on average during the 19\textsuperscript{th} century ($N = 580$). This leads to a further decrease for io\,+\,do.


 Clauses with topicalized or \isi{relativized indirect objects} will naturally count as cases of io\,+\,do order. See example \REF{ex:falk:14a} and footnote \ref{fn:02:13} above. Topicalized indirect objects are not found in the \isi{medieval laws}, probably because fronted objects are generally rare. In the rest of the \ili{Old Swedish} sample, approx. 18\% ($N = 87$) of the \textit{s}{}-passives in the sample have topicalized indirect objects, with no big difference between Early and \ili{Late Old Swedish}. In later periods, the percentage drops to 5–10\% ($N = 742$) with some variation, but without any clear trends between 1526 and 1901. Clauses with \isi{relativized indirect objects} are generally very rare, and do not have any great impact on the general picture.



In clause types in which neither of the objects is topicalized or relativized, the ordering is “free”, in the sense that both orders are possible. In main clauses, both arguments follow the finite verb. Here, the order io\,+\,do is strongly preferred. See examples (\ref{ex:falk:14b}, \ref{ex:falk:15b}) above. Such examples are very common in the laws (14 out of 25 clauses), leading to a high overall proportion of io\,+\,do. Over time, the preferences remain the same, but the type becomes less common, down to 13\% at the turn of the 19\textsuperscript{th} century ($N = 321$). To some extent then, the order io\,+\,do became less common as a consequence of clauses with no \isi{topicalization} of objects, in which the io\,+\,do order was preferred, becoming less common.



In embedded clauses without a \isi{relativized object} either object may occur pre-verbally; see example \REF{ex:falk:16} above. The proportion of the clause type varies over time without any clear tendency, but we find changes in the argument order. Up until 1699, io\,+\,do dominates (63\%, $N = 57$). After that, we find variation, but generally with a preference for do\,+\,io (ca. 20–40\%; $N = 161$).



To summarize, the decreased percentage of io\,+\,do is partly an effect of changes in the \isi{relative frequency} of different clause types: clauses with a relativized \isi{direct object} (always do\,+\,io) become more common, while clauses with a \isi{topicalized indirect object} (always io\,+\,do) and clauses in which both arguments follow the finite verb (strongly preferring io\,+\,do) become less common. However, there is also a growing preference for do\,+\,io in other types of embedded clauses and in object-initial main clauses.



Recall that the argument orders shown in \tabref{tab:falk:4} do not correspond to the syntactic function (passivized io or do), only to their underlying status as indirect or direct objects. Comparing with \tabref{tab:falk:3} above, we see that passivized indirect objects during the 19\textsuperscript{th} century (18\%) are less common than the order io\,+\,do (29\%), which is the proportion of  io + do ($N=170$) with respect to the total ($N=580$). This is due to the argument order io\,+\,do sometimes occurring in clauses with passivized direct objects. This is the case in clauses in which both arguments follow the finite verb and in clauses with topicalized indirect objects. In what follows, the \isi{choice of subject} in the \isi{SPF corpus} in these two clause types will be discussed further.



The patterns found when both arguments follow the finite verb are illustrated in \REF{ex:falk:20}:


\ea%20
    \label{ex:falk:20}
\ea\label{ex:falk:20a}
\gll I  en  not  tilldelas            der  Sara  Widebeck  en  örfil\\
      in  a  note  to.share\textsc{.prs}.\textsc{pass}    there  Sara  Widebeck  a  box.on.the.ear\\
\glt ‘In a note, Sara Widebeck is given a box on the ear’ (SPF, 1840)

\ex\label{ex:falk:20b}
\gll Efter  danske  konungen  Kristian IV:s    nederlag  fråndömdes      hans  son  Fredrik  biskopsdömet  i  Halberstadt\\
after  \ili{Danish}  king.\textsc{def}    Kristian IV.\textsc{poss}  defeat      from-sentence.\textsc{pst}.\textsc{pass}      his   son  Fredrik  bishopric.\textsc{def}  in  Halberstadt\\
\glt ‘After the \ili{Danish} king Kristian IV’s defeat, his son Fredrik was sentenced to forfeit the bishopric in Halberstadt’ (SPF, 1900)

\ex\label{ex:falk:20c}
\gll Det  erböds          dem        mat    ur    kungliga    köket\\
      it      offer.\textsc{pst.sg.pass}    them.\textsc{obj}  food  from  royal      kitchen.\textsc{def}\\
\glt ‘They were offered food from the royal kitchen’ (SPF, 1841)

\ex\label{ex:falk:20d}
\gll snart  räcktes        honom    wärdens        hand  till   ett  redligt  handslag\\
      soon  hand\textsc{.pst.pass}  him\textsc{.obj}  host.\textsc{def.poss}   hand  to   an  honest   handshake\\
\glt ‘Soon the host’s hand was held out to him for an honest handshake’ (SPF, 1816)

\ex\label{ex:falk:20e}
\gll Skulle      der  borta  någon  förmånlig    anställning  erbjudas dig         så   är      du    naturligtvis    fri\\
shall.\textsc{pst}     there  over  some    advantageous  position     offer\textsc{.inf}.\textsc{pass}      you\textsc{.obj}   so  be.\textsc{prs}  you  of.course      free\\
\glt ‘If you should be offered some advantageous position over there, you are of         course free’ (SPF, 1900)

\ex\label{ex:falk:20f}
\gll Först  bjuds            vi      i  prästgår’n      kaffe    och  dopp\\
      first  offer\textsc{.prs}.\textsc{sg.pass}    we.\textsc{sbj}  in  parsonage\textsc{.def}  coffee  and  buns\\
\glt ‘First, we are offered coffee and buns in the parsonage.’ (SPF, 1900)\footnote{The singular form of the verb with a plural subject, as well as the form \textit{går’n} (cf. standard written \textit{gården}) – perhaps also the \isi{passivized indirect object} – indicates vernacular language.}
\z
\z\largerpage

\REF{ex:falk:20a} and \REF{ex:falk:20b} differ minimally in the \isi{definiteness} of the \isi{direct object}. An \isi{indefinite} \isi{direct object} could be construed as an \isi{associate subject}, a possible analysis of \REF{ex:falk:20a} (cf. the discussion around \REF{ex:falk:11b} above). In \REF{ex:falk:20b}, on the other hand, the definite \isi{direct object} points out the \isi{indirect object} as the promoted subject. \REF{ex:falk:20c} shows the \isi{direct object} construed as an \isi{associate subject} in situ. \REF{ex:falk:20d} shows a pronominal \isi{indirect object} in front of a \isi{passivized direct object}, that is, an instance of so-called long \isi{object shift}. (Long) \isi{object shift} is possible only in clauses with a finite main verb; a \isi{passivized direct object} in clauses with an infinite main verb will involve an unambiguous post-verbal \isi{indirect object}, as exemplified in \REF{ex:falk:20e}.{\interfootnotelinepenalty=10000\footnote{Io\,+\,do seems to be almost obligatory, when possible, i.e. after a finite main verb. Only one example has do\,+\,io after a finite main verb:

\ea \gll Då    han  sedan  blef    frisk  och  begärde    äfven  den  tredje  dagen  vägrades \textit{detta}  \textit{honom}\\
         when  he    then  get.\textsc{pst}  well  and  demand.\textsc{pst}  also    the  third  day.\textsc{def}  refuse\textsc{.pst.pass}  this  him.\textsc{obj}\\
         \glt ‘When he later on got well and asked for the third day also, he was refused this.’ (1900; SPF)
\z}} The patterns in (\ref{ex:falk:20}c–d) reflect a discrepancy between argument order (io\,+\,do) and subject choice (\isi{direct object}). In the latest period, though, another pattern is also found, in which the subject form of the pronoun reveals that the \isi{indirect object} is passivized \REF{ex:falk:20f}. In summary, the preferred order remains the same, but in the most recent period investigated, 1898–1901, the \isi{indirect object} is more often \isi{promoted to subject}.


In clauses with a \isi{topicalized indirect object}, it is often impossible to tell which object is passivized (see \ref{ex:falk:8b} above). In \REF{ex:falk:21a}, the case of the \isi{indirect object} reveals that the \isi{direct object} is passivized. Again, we find a discrepancy between linear order (io\,+\,do) and \isi{choice of subject}. However, this pattern is not very common; there are 11 examples in total. Somewhat more common is the alternative in \REF{ex:falk:21b}, with a \isi{passivized indirect object} (19 examples).


\ea%21
    \label{ex:falk:21}
\ea \label{ex:falk:21a}
\gll Broder,  dig    gifves            bilderna      som  en  hälsning\\
      brother,  you.\textsc{obj}  give.\textsc{sg.prs.pass}  pictures\textsc{.def}  as    a  greeting\\
\glt ‘Brother, you are given the pictures as a greeting’ (SPF, 1900)

\ex \label{ex:falk:21b}
\gll jag      ålägges          böter      för    underlåten  bevakning\\
      I.\textsc{nom}  on-lay.\textsc{sg.prs.pass}  pentalty  for    withheld   guard\\
\glt ‘Penalty is laid upon me because of withheld guard’ (SPF, 1880)
\z
\z

A final clause type to discuss is clauses with a relativized underlying \isi{direct object}. There is a strong tendency for the \isi{direct object} also to be \isi{promoted to subject} – thus, the underlying order do\,+\,io will correspond to a \isi{passivized direct object}. As opposed to clauses in which both arguments follow the finite verb, passivized indirect objects did not become more common in this clause type, but remained very low even in the most recent part of the \isi{SPF corpus}. In one respect, there is a difference, though: whereas (pronominal) indirect objects tended to precede the finite verb in earlier periods (as in \ref{ex:falk:22}a–b below), such an order is more or less obsolete in the \isi{SPF corpus}, although a few examples can be found (9 of 190); see (\ref{ex:falk:22}c–d).


\ea%22
    \label{ex:falk:22}
\ea\label{ex:falk:22a}
\gll alt  thz …    som  hanum    giordhis          for    gudz      sculd\\
      all  that {}  that  him\textsc{.dat}    do\textsc{.pst.sg.pass}  for    God.\textsc{gen}  sake\\
\glt ‘everything … that was done to him in the name of God’ (Leg Bil, EOS, p. 119)

\ex\label{ex:falk:22b}
\gll rettferdighet … hwilken  mig    skenkes\\
      justice {}      which    me\textsc{.obj}   give\textsc{.prs.sg.pass}\\
\glt ‘justice that I am given’ (\isi{SAOB} \textit{skänka}, 1709)

\ex\label{ex:falk:22c}
\gll den  arm,  som  erbjudes            er\\
      the  arm  that  offer\textsc{.prs.sg.pass}     you.\textsc{obj}\\
\glt ‘the arm that you are offered’ (SPF, 1849)

\ex\label{ex:falk:22d}
\gll att    ersätta        den  fattige  bonden    all  den  skada    honom till-fogats\\
      to    compensate.\textsc{inf}  the  poor    peasant  all  the  damage  him.\textsc{obj}         to-add\textsc{.sup.pass}\\
\glt ‘to compensate the poor peasant for all the damage that had been inflicted upon       him’ (SPF, 1880)
\z
\z

In the \ili{Old Swedish} example in \REF{ex:falk:22a}, the preposed \isi{dative} is most likely an example of \isi{stylistic fronting}, or alternatively it is an oblique subject. \isi{Stylistic fronting} became very unusual during the first half of the 18\textsuperscript{th} century (see statistics in \citealt{Falk1993}: 326). Therefore, word orders like that in \REF{ex:falk:22b} more probably involve a verb-final \isi{embedded clause}, which was quite a common word order during the 17\textsuperscript{th} century for some authors, and occurred now and then with authors born after 1700 \citep{Platzack1983}; this word order is not unusual among the relative clauses in 1526–1799. In present-day Swedish, word orders like \REF{ex:falk:22d} are no longer possible, whereas a post-verbal \isi{indirect object} as in \REF{ex:falk:22c} is a grammatical alternative to a pre-verbal \isi{passivized indirect object} (see \ref{ex:falk:24} below).\footnote{An anonymous reviewer gives an example from 1920, probably collected from \textit{Äldre svenska romaner}. One other example is found in \textit{Äldre svenska romaner}, from the same novel (Bergman, \textit{Herr von Hancken}):

\ea \gll det  lilla,    som  honom    anförtrotts …\\
        the  small.\textsc{def}  that  him.\textsc{obj}  entrust.to.\textsc{sup.pass}\\
        \glt  ‘the small things that had been entrusted to him’
\ex  \gll alla de  värdigheter  som  mig    rätteligen  tillkommer    men  som  mig    förmenats\\
         all the  honours    that  me.\textsc{obj}  rightly    belong.to.\textsc{prs}  but  that  me.\textsc{obj}  deny\textsc{.sup.pass}\\
       \glt ‘all the honours that belong to me by right but I have been denied’
\z The construction is obviously used for stylistic reasons. Both the (1\textsuperscript{st} person) storyteller and Herr von Hancken are quite precious and ridiculous people.}


The changes presented in this section will be discussed further in \sectref{sec:falk:4} below. First, I will present my analysis of the difference between the asymmetry in older stages of Swedish and the symmetry that emerged during the 19\textsuperscript{th} century.

\section{Analysis}\label{sec:falk:4}


In the analysis that follows, the core idea is that the middle argument position is an \isi{inherent case} position, both in an \isi{object-asymmetrical language} like \ili{Old Swedish} and in an \isi{object-symmetrical language} like present-day Swedish. What has changed is the case-assigning properties of \isi{ditransitive} verbs. Before this analysis is presented, I will go through some basic assumptions (\sectref{sec:falk:4.1}).

\subsection{Basic assumptions}\label{sec:falk:4.1}


In a \isi{ditransitive verb} phrase, three argument positions are found: positions for a (verb phrase internal) subject, an \isi{indirect object}, and a \isi{direct object}. I will take the structure to be a projection of the verb, creating a \isi{complement} position of V (\isi{direct object}), a spec-\isi{VP} position (\isi{indirect object}), and a spec-\isi{vP} position (subject).\footnote{Alternative analyses of \isi{ditransitive} constructions are that v governs a small clause of some kind, projected by an abstract head. Different approaches take this abstract head to be P\textsc{\textsubscript{have}}, giving a reading ‘cause somebody to have something’ (e.g. \citealt{HarleyJung2015}) or an applicative head Appl (e.g. \citealt{Pylkkanen2008}).} Following standard assumptions, I assume that the external argument is suppressed in the \isi{passive voice}.


Furthermore, I will assume that “\isi{structural case}” is a \isi{licensing} structural relation between a head and the closest available DP with matching features in its \isi{c-command} domain. Mono-transitive verbs \isi{probe} a DP in its \isi{complement} position, and T probes the closest DP. I will take the relevant features to be $\varphi $-features and case features. The head probes a DP with $\varphi $-features and the unspecified \isi{case feature} of the DP gets a value (subject or \isi{object case}) from the head.\footnote{The labels of the features are of minor importance in this connection. Rather, the mutual dependency between the relevant head (T or V) and the DP is important: T/V “needs” something from a DP, formalized as unspecified $\varphi ${}-features in T/V probing for specified $\varphi ${}-features in the DP, and the DP “needs” something from T/V, formalized as an unspecified \isi{case feature} getting a specified value from T/V once the \isi{agreement} relationship of $\varphi ${}-features is established. I assume that the $\varphi ${}-features and the case features always occur in combination.} I will further assume that \textsc{epp} features require that the \isi{licensing} relation is established in an overt spec-head configuration. The $\varphi $-features of T have an \textsc{epp} feature in present-day Swedish, thus triggering \isi{movement} of an DP to spec-TP. Alternatively, an inserted \isi{expletive} subject may satisfy the \textsc{epp} feature, if entering an \isi{agreement} relation with a DP in situ (the “\isi{associate subject}”).\footnote{I have no account of the \isi{definiteness effect}, but take it as an indication of the status of an \isi{associate subject}. Cf. \REF{ex:falk:7} above.}



\isi{Lexical case}, as found in \ili{Old Swedish}, is a verb-idiosyncratic property.\footnote{Different labels and characterizations of non-\isi{structural case} have been proposed; see e.g. \citet[181–182]{Thrainsson2001}. “\isi{Lexical case}” should be understood here as a verb-\isi{idiosyncratic case}. It could probably be realized both in a spec-head and head-sister configuration, but I make no more specific assumptions here about \isi{lexical case}.  “\isi{Inherent case}” is used here only as the specific property of an \isi{object case} in spec-\isi{VP}; see further below.} 



The core idea in the analysis that follows is that a DP in spec-\isi{VP} does not depend on a case-\isi{licensing} head. Instead, I will develop the idea that spec-\isi{VP} is a position with \isi{inherent case}. Being a VP-internal case, it is compatible with the \isi{case feature} of V (an \isi{object case}), rather than the case features of T (subject/\isi{nominative} case). The exact nature of this \isi{inherent case} will be explored further below. I will argue that this property of spec-\isi{VP} has not changed in the history of Swedish. What did change, however, was the \isi{feature setup} of \isi{ditransitive} verbs: before the change, \isi{ditransitive} verbs had one set of unspecified $\varphi ${}-features; after the change, they had two sets of unspecified $\varphi $-features. \isi{Passive} verbs have one set of unspecified $\varphi ${}-features less, both before and after the change.\footnote{This corresponds to the idea that \isi{passive} morphology “absorbs” \isi{structural case}.}


\subsection{The case of indirect objects before the change: Analysis}\label{sec:falk:4.2}


Recall that, after the loss of morphological case, the \isi{indirect object} had some properties that are atypical for an argument with \isi{lexical case}: it had no distinctive morphological form and was a verb-type case rather than a case of individual verbs (cf. the \isi{dative} in examples \REF{ex:falk:3} and \REF{ex:falk:5} above). These properties follow straightforwardly if spec-\isi{VP} is a position with an inherent \isi{object case}, as follows.


Before the change, the only option was that the \isi{direct object} passivized, in the sense of changing case from \isi{accusative} to \isi{nominative}. In this section, I will show that this fact follows from an analysis in which spec-\isi{VP} was a position with \isi{inherent case}, and a \isi{ditransitive verb} had one set of unspecified $\varphi ${}-features, probing a DP with $\varphi $-features. An active verb probes the \isi{direct object}, and the \isi{indirect object} is licensed by virtue of the \isi{inherent case} property of spec-\isi{VP}. 



A \isi{passive ditransitive} verb had no unspecified $\varphi $-features before the change. The only unspecified $\varphi $-features in such a structure are found in T. When T probes a DP in its c-commanding domain for $\varphi $-features, the closest DP is in spec-\isi{VP}. This DP is case-licensed but, crucially, only by virtue of its position. As will be outlined in more detail below, it would in principle be possible to escape this position if the \isi{indirect object} DP is probed by an \textsc{epp} feature. However, such a structure is ruled out, since the \isi{direct object} is not case-licensed. The effect will be that the \isi{indirect object} is trapped, so to speak, in spec-\isi{VP}. 



Its status as a position with \isi{inherent case} will make spec-\isi{VP} invisible when T probes in its \isi{c-command} domain. Thus, T may \isi{probe} the \isi{direct object} further down. An \textsc{epp} feature in T will trigger \isi{movement} of the \isi{direct object} to spec-TP, the \isi{subject position}.


\begin{sloppypar}
The proposal accounts for the properties of the \isi{indirect object} before the change, as presented in \sectref{sec:falk:2.2} above, properties that are not normally found with \isi{lexical case}. The “lexical” property of the \isi{indirect object} is not its morphological case (\isi{dative} or \isi{genitive}), but instead the argument structure of the verb, i.e. the very property of being a \isi{ditransitive verb}, which is a verb-type property rather than a verb-idiosyncratic property. The verb type will project a spec-\isi{VP} position, which by assumption is a position with an inherent \isi{object case}. With respect to minimality conditions on forming a relationship between T and a DP further down, the \isi{inherent case} will have the same effect as a \isi{lexical case}: it does not block such a relationship.
\end{sloppypar}

\subsection{The case of indirect objects after the change: Analysis}\label{sec:falk:4.3}


Recall that the case of the \isi{indirect object} in an \isi{object-symmetrical language} like present-day Swedish also has some atypical properties: the \isi{indirect object} may passivize, showing that it does not have a \isi{lexical case}. However, at the same time, it may not be construed as an \isi{associate subject} in situ in the verb phrase, and it does not block the \isi{direct object} from moving to the \isi{subject position}, nor from being construed as an \isi{associate subject} in situ. In these respects, its case resembles a \isi{lexical case}. These facts will be accounted for as follows. 


I propose that \isi{ditransitive} verbs in present-day Swedish have two sets of unspecified $\varphi $-features in the \isi{active voice}, and one in the \isi{passive voice}. I also further explore the properties of spec-\isi{VP}, showing that the facts will follow if spec-\isi{VP} is still an \isi{inherent case} position. Thus, the DP generated in this position will be case-licensed by its position rather than through \isi{agreement} with a case-\isi{licensing} head.



In the \isi{active voice}, the verb in V probes the \isi{direct object} in the \isi{complement} position, and from V, it probes the \isi{indirect object} in spec-\isi{VP}. The verb finds matching $\varphi $-features, and its \isi{case feature} will be compatible with the \isi{inherent case} of spec-\isi{VP}, both being object cases.



From the proposed analysis, the two possibilities in the \isi{passive voice} will follow. A \isi{passive ditransitive} verb has one set of unspecified $\varphi $-features. First, consider the possibility that the verb probes the closest DP downwards from its base position in V, i.e. the \isi{direct object}. Both VP-internal objects are now case-licensed~– the \isi{direct object} by its relationship to the verb in V, the \isi{indirect object} by its position. But since both objects are case-licensed, T will find no DP with matching features: the \isi{case feature} of T carries subject case (\isi{nominative}), whereas the \isi{inherent case} of spec-\isi{VP} is an \isi{object case}. In other words, the \isi{indirect object} cannot be construed as an \isi{associate subject}, due to the feature mismatch. At the same time, the \isi{indirect object} may be attracted by the \textsc{epp} feature on the $\varphi $-features of T. In other words, it may escape its \isi{case position}, ending up as a \isi{passivized indirect object}.



Next, consider the alternative in which the \isi{direct object} is passivized. In this case, the verb probes the closest DP from v. As in the \isi{active voice}, a relationship can be established between the verb and the \isi{indirect object}. Next, T probes a DP with matching features. Just as before the change, due to its \isi{inherent case} property, spec-\isi{VP} will not intervene, and T may establish the \isi{licensing} relationship with the \isi{direct object}. The \textsc{epp} feature in T will trigger \isi{movement} of the \isi{direct object} to spec-TP, or the \isi{direct object} may stay in situ as an \isi{associate subject}.



As outlined above, the case of indirect objects in the \isi{passive voice} has what at first glance seems to be a curious mix of structural and \isi{lexical case} properties. It is “structural” in the sense that it can passivize; it is “lexical” in the sense that it does not block \isi{movement} of the \isi{direct object}. This mix follows from the proposal that spec-\isi{VP} is an \isi{inherent case} position.


\subsection{More on the notion of “inherent case position”}\label{sec:falk:4.4}


The proposed analysis relies on three crucial properties of spec-\isi{VP} of \isi{ditransitive} verbs: it has \isi{inherent case}, a DP in this position may remain in spec-\isi{VP} if probed by a head with compatible case features, and it can escape case if probed by an \textsc{epp} feature. Together, these properties will account for the passivization possibilities.


It is difficult to find any independent evidence for a notion like “\isi{inherent case} position”. There is, however, a possible parallel: an inherent \isi{semantic role} of spec-\isi{VP}. As we have seen, the \isi{indirect object} of a \isi{ditransitive verb} can have different semantics. Many monotransitive verbs may be construed with an \isi{optional indirect object}, and this optional object will always be interpreted as a (potential) receiver/beneficient. This is well known with production verbs like \textit{bygga} ‘build’, \textit{baka} ‘bake’, etc., but an \isi{optional indirect object} may also show up with verbs like \textit{köpa} ‘buy’, \textit{skaffa} ‘procure’. To the extent that we can add an \isi{indirect object} to a verb like \textit{stjäla} ‘steal’, it will be interpreted as the receiver: to \textit{stjäla någon en cykel} ‘steal someone a bike’ means that the person \textit{receives} a bike, not that the bike is stolen \textit{from} the person. 



Also crucial in the analysis is the assumption that the \isi{indirect object} can escape spec-\isi{VP} if attracted by an \textsc{epp} feature. Since case is associated with the position, not the DP, the DP is free to move. Its “lost” case will be compensated for in spec-TP by the case features of T. Again, a comparison with optional indirect objects is illustrative. In principle, such an \isi{optional indirect object} could also move to spec-TP, trigged by the \textsc{epp} feature. But then it would lose its interpretation, and this could not be compensated for in spec-TP. Hence, optional indirect objects cannot be passivized; compare \REF{ex:falk:23a} and \REF{ex:falk:23b}:


\ea%23
    \label{ex:falk:23}
\ea[]{\label{ex:falk:23a}
\gll Pappa  stickade/köpte/stal      mig    en  tröja\\
     Daddy  knit\textsc{.pst}/buy\textsc{.pst/}steal\textsc{.pst}  me.\textsc{obj}  a  sweater\\
\glt ‘Daddy knitted/bought/stole a sweater for me.’}

\ex[*]{\label{ex:falk:23b}
\gll Jag    stickades/köptes/stals                  en  tröja\\
       I.\textsc{subj}  knit\textsc{.pst.pass}/buy\textsc{.pst.pass/}steal\textsc{.pst.pass}  a  sweater\\}
\z
\z

Thus, somewhat indirectly, we find support for the idea that at least spec-\isi{VP} could be connected with position-inherent properties.

\subsection{Accounting for object symmetry}\label{sec:falk:4.5}


In this section, I will compare the proposed analysis with other accounts of \isi{passive ditransitive} verbs in present-day Swedish. The analyses differ in several respects, including basic assumptions about the structure of \isi{double object construction}, as well as the mechanisms and restrictions on \isi{licensing}. A full account of these differences would lead us too far afield – here, I will just point out some similarities and the main differences between the different accounts. The primary focus is on how the analyses account for the \isi{object symmetry} in the \isi{passive voice}, i.e. why both objects may passivize.


In the analysis by \citet{HaddicanHolmberg2019}, a \isi{double object construction} includes a verb-governed phrase, PP\textsc{\textsubscript{have}}, with the \isi{indirect object} as the specifier and the \isi{direct object} as the \isi{complement}. The point of departure for the analysis of present-day Swedish is the observation that bimorphemic \isi{ditransitive} verbs passivize more easily than monomorphemic \isi{ditransitive} verbs (see \sectref{sec:falk:2.1} above). In the \isi{passive voice}, the verb is not a \isi{case assigner} – but the \isi{prefix} of a bimorphemic verb is. The \isi{prefix} may assign case to the closest DP, the \isi{indirect object}. In this way, the \isi{indirect object} is “deactivated” (in the terminology of Haddican \& Holmberg), making the \isi{direct object} accessible from T. A \isi{passivized direct object} will follow. The \isi{prefix} can also transmit its case-assigning capacity downwards to P\textsc{\textsubscript{have}}. P\textsc{\textsubscript{have}} will then case-license the \isi{direct object}, and T will \isi{probe} the \isi{indirect object}, leading to a \isi{passivized indirect object}.



In \ili{Norwegian}, verb class is not significant, and \citet{HaddicanHolmberg2019} propose another analysis to account for this. Since \isi{passive} monomorphemic \isi{ditransitive} verbs are not totally prohibited in Swedish, this alternative will be available (marginally) in Swedish as well. In this proposal, the relevant case-assigning head is not a verbal \isi{prefix}, but instead the abstract head P\textsc{\textsubscript{have}}. P\textsc{\textsubscript{have}} can assign case either to its spec position, the \isi{indirect object}, or to its \isi{complement} position, the \isi{direct object}. The object left without a case will be probed by T, i.e. turn up as the subject.\footnote{In \citet{HolmbergEtAl2019}, the abstract head is labelled Appl. The options – \isi{case assignment} to the specifier position or the \isi{complement} position – are the same.} 



Thus, in both structures, there is a vP-internal \isi{case assigner} in the \isi{passive voice}: either the \isi{prefix} or the abstract head P\textsc{\textsubscript{have}}. \isi{Object symmetry} is obtained through different possibilities for this \isi{case assigner}. 



In my proposal, different possibilities for the \isi{case assigner} are also crucial: a \isi{case assigner} (a head with unspecified $\varphi ${}-features) can assign case (successfully \isi{probe} a DP with matching features) from different positions. The verb can either \isi{probe} the \isi{direct object} in its base position, or the \isi{indirect object} from the v-position. But there is an advantage that only one assignment mechanism is available: assignment (\isi{agreement}) under \isi{c-command}, without alternative case-assignment mechanisms. Furthermore, given that passivization of monomorphemic verbs like \textit{giva} ‘give’ is marginally possible also in Swedish, it is not clear in Haddican \& Holmberg’s analysis why optional passivized indirect objects as in \REF{ex:falk:23b} are decidedly ungrammatical.



Another advantage of my proposal concerns the \isi{definiteness effect}: it follows from my analysis that the \isi{indirect object} cannot be construed as an \isi{associate subject} in situ, since there will be a case clash. As far as I can see, nothing prevents this in Haddican \& Holmberg’s analysis: the \isi{prefix} can transmit its case-assignment capacity to P\textsc{\textsubscript{have}}, and it would be possible for T to form a chain with either object as long as it adheres to the \isi{definiteness restriction}.



\citet{Platzack2005,Platzack2006} proposes a different source for the \isi{object symmetry} in the \isi{passive voice}, namely the properties of the \isi{indirect object} DP. In his analysis, DPs in the \isi{indirect object} position can either have or lack $\varphi ${}-features. In the former case, unspecified $\varphi ${}-features of T get a value from the \isi{indirect object}, which ends up as the subject. If the \isi{indirect object} lacks $\varphi ${}-features, T probes further down, finding the necessary $\varphi ${}-features on the \isi{direct object} instead, the result being a \isi{passivized direct object}. Note that DPs can lack $\varphi ${}-features in Platzack’s account only in the \isi{indirect object} position. The similarities with my proposal are obvious – DPs in the \isi{indirect object} position do not enter into a relationship with T. But instead of locating relevant properties in the DP, I have located them in the position, the \isi{inherent case} status of spec-\isi{VP}. No optional features are needed, and \isi{licensing} is throughout a mutual dependency relationship between a head and a DP. Furthermore, as in Haddican \& Holmberg’s analysis, I cannot see how indirect objects as associate subjects are ruled out in Platzack’s account.


\section{The changes: Discussion and residual questions}\label{sec:falk:5}


The main focus for my investigation has been to trace the change in Swedish from an asymmetrical language, in which only direct objects could passivize, i.e. change case from object to subject case, to a symmetrical language, in which both objects can passivize. I have presented this as a change in the grammar in the 19\textsuperscript{th} century: from a grammar in which \isi{passive ditransitive} verbs did not have any unspecified $\varphi ${}-features with accompanying case features, to a grammar in which passivized \isi{ditransitive} verbs have one set of unspecified $\varphi ${}-features, and therefore have the capacity to case-license an object. A first \isi{question} to discuss is the impelling force behind this change.


My investigations of argument order in earlier stages of Swedish have shown developments prior to the grammatical change: over time the (underlying) \isi{indirect object} more and more often follows rather than precedes the (underlying) \isi{direct object}. A second \isi{question} is why the word order preference changed. 



The preferred subject of \isi{passive} ditransitives in the late 19\textsuperscript{th} century is still the \isi{direct object}. Therefore, a final \isi{question} concerns the situation in present-day Swedish: why is the \isi{passivized indirect object} the default choice today?



I discuss these questions in chronological order, starting with the second one.


\subsection{Changes before 1800}\label{sec:falk:5.1}


\ili{Old Swedish} showed a weak preference for indirect objects (io) to precede direct objects (do) in the \isi{passive}. To a certain extent, \isi{genre} plays a role: in the \isi{medieval laws}, clauses in which both arguments follow the finite verb were very common, and in this clause type the order io\,+\,do has always been preferred, perhaps reflecting the unmarked underlying order. It is less clear why topicalized direct objects are so common in my \ili{Late Old Swedish} sample. It remains to be investigated if this was really the case more generally during this period, or if my collection of data is not fully representative; recall that examples were taken only from a dictionary, not directly from the historical sources. With this in mind, we still see a clear change in preferences over time (in \tabref{tab:falk:4} above). As shown in \sectref{sec:falk:3.8}, this is partly due to the frequency of different clause types: clauses with a relativized \isi{direct object} tend to become more common, resulting in do\,+\,io order, and clauses with both arguments following the finite verb tend to become less common. Clauses with topicalized indirect objects also tend to become less common. But we can also note that do\,+\,io became more common in other types of embedded clauses, and that topicalized direct objects became more common (ignoring the somewhat exceptional figures from \ili{Late Old Swedish}). In both these cases there is a clear difference between the periods before and after 1700: the period 1526–1699 resembles \ili{Early Old Swedish}, whereas the 18\textsuperscript{th} century resembles the 19\textsuperscript{th} century (\sectref{sec:falk:3.8}). This coincides with two other changes in Swedish: the loss of \isi{lexical case} and the introduction of non-referential subjects. The loss of \isi{lexical case} for verbs like \textit{lika} ‘like’, \textit{ångra} ‘regret’, meant that the object/\isi{dative} case was replaced with the subject/\isi{nominative} case. The introduction of non-referential subjects was an effect of stricter conditions on the \isi{licensing} of the \isi{subject position} \citep{Falk1993}. Both these changes are possibly part of the answer as to why the order do\,+\,io gained ground; it would be odd if an infrequent construction like a \isi{ditransitive} \isi{passive} changed all by itself. Both the loss of \isi{lexical case} and the introduction of non-referential subjects led to a requirement for a \isi{nominative} \isi{noun phrase} outside the verb phrase. In clauses with passivized \isi{ditransitive} verbs, this in turn led to a greater preference for do\,+\,io.

\subsection{Changes detected in the SPF corpus (1800–1901)}\label{sec:falk:5.2}


The \isi{question} of the introduction of \isi{object symmetry} during the latter part of the 19\textsuperscript{th} century can be divided into two: a “how” \isi{question} and a “why” \isi{question}.


The “how” \isi{question} concerns the factors that promoted the change. We can imagine that the \isi{reanalysis} was closer at hand for some verbs, and that these verbs paved the way for a general \isi{reanalysis} of the \isi{feature setup} of \isi{ditransitive} verbs. Obvious candidates for this “leading role” in the change are verbs with a \isi{prepositional prefix}: prepositions select DPs. In the analysis assumed here, they have unspecified $\varphi ${}-features together with a \isi{case feature}. This \isi{feature setup} could also be reinterpreted as a \isi{feature setup} when the preposition is part of the verb. From here, a next step could be that other prefixes were also reinterpreted as probes with unspecified $\varphi ${}-features. A more general possibility of passivizing the \isi{indirect object} would then come later. However, as was shown in \sectref{sec:falk:3.7}, this assumed pattern is only partly detectable in the number of first instances of each individual verb collected.



In a preliminary investigation of passivized indirect objects, \citet{Falk1995,Falk1997} concluded that indirect objects with an atypical \isi{semantic role} were attested earlier as subjects in the \isi{passive voice}. However, just as with respect to the formal properties of the verb, the influence of semantic properties is only visible in the collected material to a minor extent (see \sectref{sec:falk:3.7}). It remains to be investigated whether a more fine-grained semantic analysis would reveal a clearer pattern; that would require a larger collection of data than the 30 verbs investigated in the \isi{SPF corpus}.



Another factor in the “how” \isi{question} concerns clause type. As shown in \sectref{sec:falk:3.8}, clauses with a relativized underlying \isi{direct object} show a continued preference for also passivizing the \isi{direct object}. In this clause type, passivized indirect objects occur comparatively late:


\ea%24
    \label{ex:falk:24}
\gll Den    plats,        ni        härmed  erbjudes          på  vårt  kontor\\
    the    employment  you.\textsc{sbj}  hereby    offer\textsc{.prs.sg.pass}    at  our  office\\
\glt ‘The employment at our office that you are offered hereby’ (SPF, 1880)
\z


In clauses where both arguments follow the finite verb, the preference for the order io\,+\,do instead remained, in some cases leading to passivized indirect objects:

\ea%25
    \label{ex:falk:25}
\gll Om  fadren    bevisligen    vore            rubbad    till    sina      sinnen  så skulle folket      sedermera  lätt      kunna    bibringas    farhågan,  att galenskap blefve              sonens          arfvedel\\
    if    father\textsc{.def}  obviously      be.\textsc{pst.sbjv}    deranged  at    \textsc{poss}.\textsc{refl}  senses  so         would people.\textsc{def}   later        easily    can.\textsc{inf}    impart.\textsc{inf.pass}  fear.\textsc{def}    that         madness   become.\textsc{pst.subjv}  son.\textsc{def.poss}  heritage  \\
\glt ‘If it were proved that the father was mentally deranged, fear that madness would become the son’s heritage would possibly be imparted to the people’ (SPF, 1844)
\z

Passivized direct objects still dominate in this clause type at the end of the century, but passivized indirect objects are more common than they are overall (1898–1901: 13 out of 42 examples (almost one-third), compared to 21\% (see \tabref{tab:falk:3}).


Another clause type in which passivized indirect objects are more common than they are overall is clauses with topicalized indirect objects. Examples in the oldest subpart of the \isi{SPF corpus} are rare (2 out of 8 examples). In the subsequent periods, almost half of the topicalized indirect objects are passivized (17 out of 36 examples).



This leads to the \isi{question} of \textit{why} the change took place. To a certain extent, clauses with topicalized indirect objects probably played a role: to construe the fronted object also as the subject is in line with the general pattern in Swedish, where subjects are often fronted. Thus, a \isi{topicalized indirect object} may be seen not only as a favourable context for reinterpretation, but also a cause of the change.



Another part of the answer is probably to be found in the changed argument preferences we have observed during the 18\textsuperscript{th} century. Often, the growing number of topicalized direct objects and do\,+\,io in embedded clauses resulted in word orders that were not optimal for \isi{information structure}, as in the following examples:


\ea%26
    \label{ex:falk:26}

\ea
\gll Små  pillor          och  bekymmer  gifwas        mig    wäl ibland\\
      small  peddling.things  and  trouble     give.\textsc{prs.pl.pass}  me.\textsc{obj}  certainly              sometimes\\
\glt ‘Certainly, I sometimes get troubles and small things to peddle at’ (\isi{Argus}, 1732)

\ex
\gll tå    then  hedern    igenom  Felt-marskalken    Gr.      Dücher böds            mig\\
when  this  honour.\textsc{def}  through  field-marshal.\textsc{def}    count    Dücher      offer.\textsc{pst.sg.pass}    me\textsc{.obj}\\
\glt ‘when I was offered this mark of honour thanks to Field Marshal Count Dücher’ (Reuterholm, 1730–1740)
\z
\z


To choose the \isi{indirect object} as the subject instead will often give a more natural \isi{information structure}.


Finally, even if coordination is quite uncommon in the collected material (with a total of 13 examples), such examples are still worth mentioning. Recall that a normative grammarian like Wellander found passivized indirect objects “smooth” in coordination (see example \REF{ex:falk:17d} above).


\subsection{Changes after 1901}\label{sec:falk:5.3}


In present-day Swedish the default is to passivize the \isi{indirect object}, and bimorphemic verbs passivize more easily than monomorphemic \isi{ditransitive} verbs (\citealt{HolmbergPlatzack1995,Lundquist2004,HaddicanHolmberg2019}). As shown by \citet{Lundquist2004}, a \isi{direct object} is passivized if it is relativized or questioned, or if it is highly topical and the \isi{indirect object} supplies \isi{new information} (i.e. is rhematic).\footnote{This generalization is built on 40 clauses with passivized indirect objects and 40 clauses with passivized direct objects in newspapers from 1965–1998. Four verbs, \textit{erbjuda} ‘offer’, \textit{tilldela} ‘award’, \textit{frånta} ‘deprive of’, and \textit{tillägna} ‘dedicate to’ were investigated.} In addition, a \isi{passivized direct object} is grammatical when both arguments follow a finite main verb in contexts like (\ref{ex:falk:20}c–d) above. 


Lundquist has also argued that it is impossible to topicalize or relativize an \isi{indirect object} across a \isi{passivized direct object} (\isi{judgements} from \citealt{Lundquist2004}):


\ea%27
    \label{ex:falk:27}
\ea[*?]{\gll Den  mannen    har        jobbet    erbjudits\\
       that  man\textsc{.def}    have.\textsc{prs}  job\textsc{.def}  offer\textsc{.sup}.\textsc{pass}\\
}
\ex[*?]{\gll Mannen    som    jobbet    har      erbjudits\\
        man\textsc{.def}    that    job\textsc{.def}  have.\textsc{prs}  offer\textsc{.sup}.\textsc{pass}\\
}
\z
\z

\citet{HolmbergEtAl2019} and \citet{Platzack2006} give formal/structural explanations for this restriction; see also \citet{Lundquist2015}. 


In other words: the language has changed since 1901, when the default was passivized direct objects, indirect objects could be topicalized across a \isi{passivized direct object}, and monomorphemic \isi{passive} verbs were found alongside bimorphemic ones.



I have not investigated how common the corresponding active \isi{ditransitive} verbs are in the \isi{SPF corpus}, and can say nothing about different passivization possibilities (cf. \citealt{HaddicanHolmberg2019}). But as early as in the \isi{SPF corpus}, the usage patterns of present-day Swedish are actually detectable. Firstly, if we look at the \isi{choice of subject} in individual tokens of the investigated verbs (see Appendix~\ref{falk:appendix:2}--\ref{falk:appendix:3}), we see that monomorphemic verbs are about as common as bimorphemic verbs (283 vs. 297), but also that indirect objects passivize more easily with bimorphemic verbs (23\%) than with monomorphemic verbs (12\%). At least the latter fact points out the direction of the development. Secondly, as already noticed, the \isi{indirect object} is hardly ever passivized in clauses with relativized direct objects in the \isi{SPF corpus}. This corresponds to one of the conditions for passivizing the \isi{direct object} in Lundquist’s investigation \parencite{Lundquist2004}. As for the other condition, a highly topical \isi{direct object} in combination with a rhematic \isi{indirect object}, we can take this to be a further development of what I have seen as one of the reasons for the change in the grammar in the first place: the preference for topical elements to precede \isi{new information}. This is possibly also the reason why clauses like \REF{ex:falk:27} are highly marked or even ungrammatical: elements are placed in the first position of the clause, either because they are topical or because they have contrastive focus. Since direct objects are passivized if the \isi{indirect object} provides \isi{new information}, it makes sense that indirect objects are not topicalized across the subject. This would make the restriction pragmatic rather than grammatical. So what about a fronted \isi{indirect object} with contrastive focus? Fronting is hardly better in a context like the following:


\ea%28
    \label{ex:falk:28}
\gll Vad    hände    egentligen  med  jobbet    du      sökte?\\
    what    happened  actually    to    job.\textsc{def}  you.\textsc{sg}  seeked\\

\gll Äsch,  den  där  slöfocken    PELLE  kommer  det  erbjudas\\
    ugh    the  that  dullard.\textsc{def}    Pelle    will      it    offer.\textsc{inf.pass}\\
\glt ‘What about the job you applied for? Ugh, it was offered to that dullard Pelle’
\z



I conclude that the changes from around 1900 until today still await a full account.\footnote{The restriction on fronting the \isi{indirect object} across a \isi{passivized direct object} possibly follows from a general “immobility effect” – the \isi{indirect object} must not leave spec-\isi{VP}, unless attracted from spec-TP.}

\section{Summary}\label{sec:falk:6}


The main results from this investigation are that a major grammatical change took place in Swedish in the second part of the 19\textsuperscript{th} century: it became possible to passivize indirect objects. I have proposed an analysis of the grammar before and after this change, based on new case-\isi{licensing} possibilities of the verb in combination with a preserved property of spec-\isi{VP} as a position with inherent \isi{object case}. I have also argued that this change was at least partly due to a previous change in the preferred argument order, a change which in turn was caused by the introduction of an overtly realized \isi{nominative} in the \isi{subject position}. Moreover, I have suggested that further developments since 1900 were caused primarily by pragmatic factors, but this requires further investigation.

\section*{Acknowledgements}


Cecilia Falk was diagnosed with cancer in the spring of 2021 and passed away on July 21, 2021. She will be sorely missed. When she fell ill, she asked us, the editors of this volume, to help her make the final corrections to this chapter, and we have done so. She also wished to thank three anonymous reviewers for their helpful comments. \hfill--- Ida Larsson and Erik M. Petzell.\hbox{}\pagebreak


\section*{Abbreviations}
\begin{multicols}{2}
\begin{tabbing}
PDS\hspace{1ex} \=    present-day Swedish\kill
EOS \>  \ili{Early Old Swedish}\\
do  \> \isi{direct object}\\
io  \> \isi{indirect object}\\
LOS \>  \ili{Late Old Swedish}\\
OS  \> \ili{Old Swedish}\\
PDS \>  present-day Swedish\\
SUP \>  \isi{Supine}
\end{tabbing}
\end{multicols}

% % % \section{Sources}

\section*{Excerpted EMS authors born 1571–1600}
\begin{description}[font=\normalfont,itemsep=\bibitemsep,leftmargin=\bibhang]\sloppy

\item[Gyllenhjelm, Carl Carlsson (b. 1574).] \textit{Egenhändiga anteckningar af Carl Carlsson Gyllenhjelm rörande tiden 1597–1601} [Memoirs. Written in about 1640. Author’s own manuscript.]. In \textit{Historiska handlingar} 20:2, pp. 258–395. Stockholm, 1905. 

\item[Oxenstierna, Gabriel Gustafsson (b. 1587).] \textit{Gabriel Gustafsson Oxenstiernas bref 1611–1640} [Letters. Written 1611 1640. Author’s own manuscript.]. In \textit{Rikskanslern Axel Oxenstiernas skrifter och brefvexling,} latter part, third volume, pp. 1–71. Edited by P. Sondén Stockholm: Norstedts, 1890. 


\item[Hand, Johan (b. ca 1590).] \textit{Johan Hands dagbok under K. Gustaf II Adolfs resa till Tyskland1620} [Diary. Written in 1620. Author’s own manuscript.]. In \textit{Historiska handlingar} 8:3, pp. 3–39. Stockholm, 1879.


\item[Tungel, Lars Nilsson (b. ca 1590).] \textit{Lars Tungels dagbok 1633} [Diary. Written in 1633. Author’s own manuscript.]. In \textit{Historiska handlingar} 22:1, pp. 79–181. Stockholm, 1907.


\item[Banér, Johan (b. 1596).] \textit{Johan Banérs bref 1624–1630} [Letters. Written 1624–1630. Author’s own manuscript.]. In \textit{Rikskanslern Axel Oxenstiernas skrifter och brefvexling,} latter part, sixth volume, pp. 1–63. Edited by P. Sondén. Stockholm: Norstedts, 1893. 
\end{description}

\section*{Excerpted EMS authors born 1601–1635}

\begin{description}[font=\normalfont,itemsep=\bibitemsep,leftmargin=\bibhang]\sloppy
\item[Rålamb, Claes (b. 1622).] \textit{Diarium under resa till Konstantinopel 1657–1658} [Memoirs. Written 1657–1658. Manuscript copied from the author’s manuscript in about 1700.]. In \textit{Historiska handlingar} 37:3, pp. 33–174. Stockholm, 1963.


\item[Gyllenius, Petrus Magnus (b. 1622).] \textit{Diarium Gyllenianum} [Diary. Written after 1667. Author’s own manuscript.]. Edited by Finska statsarkivet through R. Hausen. Helsinki, 1882.

\item[Dahlberg, Erik (b. 1625).] \textit{Erik Dahlbergs dagbok 1625–1699} [Diary. Written after 1699. Author’s own manuscript.]. Edited by H. Lundström. Uppsala/Stockholm, 1912.

\item[Ekeblad, Johan (b. 1629).] \textit{Johan Ekeblads brev till brodern Claes Ekeblad 1639–1655} [Letters. Written 1639–1655. Author’s own manuscript.]. Edited by S. Allén. Gothenburg, 1965.

\item[Horn, Agneta (b. 1629).] Beskrivning över min vandringstid [Memoirs. Written in about 1657. Author’s own manuscript.]. Edited by G. Holm. Stockholm: Almqvist \& Wiksell, 1959.
\end{description}

\section*{Excerpted EMS authors born 1636–1670}
\begin{description}[font=\normalfont,itemsep=\bibitemsep,leftmargin=\bibhang]\sloppy

\item[Bolinus, Anders (b. 1643).] \textit{En dagbok från 1600-talet} [Diary. Written 1666–1687. Author’s own manuscript.]. Edited by E. Brunnström. Stockholm: Norstedts, 1913.

\item[Swedberg, Jesper (b. 1653).] \textit{Jesper Swedbergs lefvernes beskrifning} [Memoirs. Written 1729. Author’s own manuscript.]. Edited by G. Wetterberg. Lund, 1941.

\item[Hermelin, Olof (b. 1658).] \textit{Bref från Olof Hermelin till Samuel Bark 1702–1709} [Letters. Written 1702–1709. Author’s own manuscript.], pp. 1–51. Edited by C. von Rosen. Stockholm: Norstedts, 1913.

\item[Bark, Samuel (b. 1662).] \textit{Bref från Samuel Bark till Olof Hermelin 1702–1708} [Letters. Written 1702–1708. Author’s own manuscript.], pp. 126–147. Edited by C. von Rosen. Stockholm: Norstedts, 1914.

\item[Stenbock, Magnus (b. 1664).] \textit{Magnus Stenbock och Eva Oxentsierna, en brefväxling} [Letters. Written 1688–1702. Author’s own manuscript.], all letters in Swedish. Edited by C. M. Stenbock. Stockholm: Norstedts, 1913.
\end{description}

\section*{Excerpted EMS authors born 1671–1700}

\begin{description}[font=\normalfont,itemsep=\bibitemsep,leftmargin=\bibhang]\sloppy
\item[Stiernhöök, Olof (b. 1673).] \textit{Drabanten och kaptenen vid lifgardet Olof Stiernhööks journal 1700–1703} [Diary. Written 1700 1703. Author’s own manuscript.]. Edited by S. E. Bring. Lund, 1913.

\item[Reuterholm, Nils (b. 1676).] See Quoted sources below.

\item[Pihlström, Anders (b. 1677).] \textit{Anders Pihlströms dagbok 1708–1723} [Diary. Written 1708–1723. Author’s own manuscript.]. In \textit{Historiska handlingar} 18:4. Stockholm, 1903.

\item[Karl XII (b. 1682).] \textit{Konung Karl Xii:s egenhändiga bref} [Letters. Written 1699–1708. Author’s own manuscript.], pp. 227–360. Edited by E. Carlsson. Stockholm: Norstedts, 1893.
\end{description}


\section*{Excerpted EMS authors born 1701–1735}

\begin{description}[font=\normalfont,itemsep=\bibitemsep,leftmargin=\bibhang]\sloppy
\item[von Linné, Carl (b. 1707).] \textit{Iter Lapponicum} [Travel book/diary. Written 1732. Author’s own manuscript.]. In \textit{Skrifter af Carl von Linné utgifna af kungl. svenska vetenskapsakademien}, pp. 5–119. Edited by Th. M. Fries. Uppsala, 1913.

\item[von Lingen, Johan (b. 1708).] \textit{Reinholdt Johan von Lingens självbiografiska anteckningar} [Memoirs. Written in the 1740s. Author’s own manuscript.], pp. 7–63. Edited by G. Nilzén. Stockholm, 1983.

\item[Tilas, Daniel (b. 1712).] \textit{Curriculum Vitæ} [Memoirs. Written 1751–1757. Author’s own manuscript.]. In \textit{Historiska handlingar} 38:1, pp. 20–149. Stockholm, 1966.


\item[Kalm, Pehr (b. 1716).] \textit{Pehr Kalms brev till samtida II: Pehr Kalms brev till friherre Sten Carl Bielke} [Letters. Written 1741–1746. Author’s own manuscript.], pp. 11–111. Edited by C. Skottsberg. Åbo, 1960.

\item[Ferrner, Bengt (b. 1723).] \textit{Resa i Europa} [Travel book. Written after 1762. Author’s own manuscript.], pp. 1–100. Edited by S. G. Lindberg. Uppsala, 1956.
\end{description}

\section*{Quoted sources}
\begin{description}\sloppy
\item[\textit{Argus}:] Dalin, Olof von (b. 1708). \textit{Then Swänska Argus} [The Swedish \isi{Argus}]. Stockholm, 1732–1734. Available through FTB/\isi{Korp} (text) and LB (facsimile).
\item[Bonniers:] \textit{Bonniersromaner I (1976–1977)} [Novels from 1976–1977]. Available through \isi{Korp}.
\item[Leg:] Stephens, George (ed.). 1847. \textit{Svenska medeltidens kloster- och helgonabok} […] \textit{Ett forn-svenskt legendarium} […] [The Swedish medieval book of monasteries and saints … An \ili{Old Swedish} collection of legends]. Stockholm: Norstedts. Originally written sometime between 1276 and 1307. Manuscripts: Codex Bureanus (Leg Bu), ca. 1350, Codex Bildstenianus (Leg Bil), first half of the 15th century. Available through FTB/\isi{Korp}.
\item[\normalfont Holm, Gösta.] 1952. \textit{Om s-passivum i svenska} [On \isi{s-passive} in Swedish]. Lund: Gleerups.
\item[Reuterholm:] Landahl, Sten (ed.). 1957. \textit{Nils Reuterholms journal} [The journal of Nils Reuterholm]. Auth. b. 1676. (Historiska handlingar 36:2.) Stockholm: Norstedts.
\item[\isi{SAOB}:] \textit{Ordbok över svenska språket, utg. av Svenska Akademien} [Dictionary of the Swedish language, published by The Swedish Academy]. 1893–. Lund. Available here: \href{http://www.saob.se}{{www.saob.se}} 
\item[\isi{SAOL}:] \textit{Svenska Akademiens ordlista} [The Swedish Academy word list], 1\textsuperscript{st} ed. (1874), 7\textsuperscript{th} ed. (1900), 11\textsuperscript{th} ed. (1986). Available here: \url{http://spraakdata.gu.se/saolhist/}
\item[Sdw:] Söderwall, K.F. 1884—1918. \textit{Ordbok öfver svenska medeltids-språket} [Dictionary of the Swedish medieval language]. Vol. I–III. Lund. Available here: \url{https://spraakbanken.gu.se/resurser/soederwall} 
\item[SPF:] Swedish prose fiction 1800–1900. Available through \isi{Korp}. 
\item[\textit{Äldre svenska romaner}] [\ili{Older Swedish} novels]. Available through \isi{Korp}. 
\end{description}

\section*{Electronic corpora}

\begin{description}
\item[FTB:] \isi{Fornsvenska textbanken} [The text bank of \ili{Old Swedish}]: \url{https://project2.sol.lu.se/fornsvenska} 
\item[\isi{Korp}:] \url{https://spraakbanken.gu.se/korp/?mode=all_hist}
\item[LB:] The Swedish literature bank: \url{http://www.litteraturbanken.se}
\end{description}



\appendixsection{Investigated ditransitive s-verbs in the SPF-corpus}\label{falk:appendix:1}\largerpage[2]

\begin{description}
\item[\normalfont *:] not attested with \isi{passivized indirect object} in the SPF-corpus
\item[\normalfont \isi{SAOL}:] \isi{passivized indirect object} judged as incorrect in \isi{SAOL} 1 (1874)
\end{description}

{\ea[]{\textit{anförtro} ‘entrust to’ \jambox*{SAOL}}
\ex[]{\textit{av}{}- ‘off-’  \jambox*{\isi{SAOL} (\textit{avfordra})}}
  \ea[]{\textit{avfordra} ‘off-demand; demand from’}
  \ex[*]{\textit{avkräva} ‘off-demand; demand from’}
  \ex[*]{\textit{avtaga} ‘off-take; take from’}
  \z

\ex[]{\textit{beröva} ‘deprive of’}
\ex[]{\textit{bespara} ‘spare’}
\ex[]{\textit{bevilja} ‘grant’\jambox*{SAOL}}
\ex[]{\textit{bibringa} ‘impart to’}
\ex[]{\textit{bjuda} ‘offer’}
\ex[*]{\textit{delgiva} ‘inform of’\jambox*{SAOL}}
\ex[]{\textit{erbjuda} ‘offer’\jambox*{SAOL}}
\ex[]{\textit{från}{}- ‘from-’ \jambox*{\isi{SAOL} (\textit{fråndöma}, \textit{fråntaga})}}
  \ea[]{\textit{fråndöma} ‘from-judge; deprive of by sentence’}
  \ex[]{\textit{fråntaga} ‘from-take; deprive of’}
  \z
\ex[*]{\textit{före}- ‘before-’\jambox*{\isi{SAOL} (\textit{förelägga})}}
   \ea[]{\textit{förelägga} ‘before-put’; set (a task) to’}
   \ex[]{\textit{föreslå} ‘propose’}
   \ex[]{\textit{förevisa} ‘before-show; show’}
   \z
\ex[*]{\textit{förlåta} ‘forgive for’}
\ex[*]{\textit{förmena} ‘deny’}
\ex[]{\textit{förunna} ‘grant’}
\ex[*]{\textit{förvägra} ‘refuse’  \jambox*{SAOL}}
\ex[*]{\textit{förära} ‘present with’  \jambox*{SAOL}}
\ex[]{\textit{giva} ‘give’}\largerpage
\ex[]{\textit{lova} ‘promise’}
\ex[]{\textit{lämna} ‘leave to’}
\ex[*]{\textit{meddela} ‘inform of’  \jambox*{SAOL}}
\ex[]{\textit{neka} ‘deny’}
\ex[]{\textit{på}{}- ‘on-’} 
  \ea[]{\textit{påtruga} ‘on-press; press upon’}
  \ex[*]{\textit{påtvinga} ‘on-force; force on’}
  \ex[*]{\textit{pålura} ‘on-dupe; trick into’}
  \z
\ex[*]{\textit{räcka} ‘hand to’}
\ex[*]{\textit{servera} ‘serve with’}
\ex[]{\textit{skänka} ‘give’}
\ex[]{\textit{till}{}- (\textit{to}{}-)  \jambox*{\isi{SAOL} (\textit{tilldela}, \textit{tillfoga})}}
   \ea[]{\textit{tilldela} ‘to-share; award’}
   \ex[]{\textit{tillfoga} ‘to-add; inflict on’}
   \ex[*]{\textit{tillskicka} ‘to-send; send to’}
   \z
\ex[]{\textit{unna} ‘grant’}
\ex[*]{\textit{visa} ‘show’}
\ex[*]{\textit{vägra} ‘refuse’\jambox*{SAOL}}
\ex[]{\textit{å}{}- ‘on-’  \jambox*{\isi{SAOL} (\textit{ådöma}, \textit{ålägga})}}
   \ea[]{\textit{ådöma} ‘on-judge; sentence to’}
   \ex[]{\textit{ålägga} ‘on-put; impose on’}
   \z
\z}

\appendixsection{First occurrences with passivized indirect object}\label{falk:appendix:2}
\begin{description}
     \item[Formal properties:] 
       \begin{description}
          \item[]
          \item[prep:] prepositional affix (see footnote \ref{fn:02:15})
          \item[mono:] monomorphemic
          \item[bi:] other bimorphemic 
       \end{description}
     \item[Semantic properties:]
     \begin{description}
          \item[]
          \item[from:] \isi{transfer} from somebody
          \item[to:] \isi{transfer} to somebody
          \item[hindered to:] hindered \isi{transfer} to somebody
      \end{description}
\end{description}

\begin{longtable}{ll lll}
\lsptoprule & & source & formal  & semantics\\\midrule\endfirsthead\midrule & & source & formal  & semantics\\\midrule\endhead\endfoot\lspbottomrule\endlastfoot

1606 & \textit{betala} ‘pay, compensate for’ &     \isi{SAOB} & bi   & to\\
1633 & \textit{anmoda} `request'             & \isi{SAOB}  & bi & to\\
1647 & \textit{av}{}- (‘off-’)               &     \isi{SAOB} & prep & from\\
     & \textit{avskära} ‘separate from’      &          &\\
     & 1669 \textit{avbörda} ‘relieve of’    &          &\\
     & 1779 \textit{avkläda} ‘strip of’      &          &\\
1704 & \textit{på}{}- (‘on-’)                &          & prep  & to\\
     & \textit{påkasta} ‘throw on’           &     \isi{SAOB} &\\
1732 & \textit{bjuda} ‘offer’                &     \isi{SAOB} & mono &  to\\
1819 & \textit{beröva} ‘deprive of’          &     \isi{SAOB} & bi  & from\\
1840 & \textit{bespara} ‘spare’              &      SPF & bi  & hindered to\\
1840 & \textit{lova} ‘promise’               &      SPF & mono & to\\
1841 & \textit{lämna} ‘leave’                &      SPF & mono & to\\
1844 & \textit{bibringa} ‘impart to’         &      SPF & bi   & to\\
1850 & \textit{från}{}- ‘from-’              &          & prep & from\\
     & \textit{fråntaga} ‘deprive of’        &     \isi{SAOB} &\\
1860 & \textit{erbjuda} ‘offer’              &     SPF  & bi   & to\\
1860 & \textit{giva} ‘give’                  &      SPF & mono & to\\
1874 & \textit{anförtro} ‘entrust to’        &     \isi{SAOL} & bi   & to\\
1874 & \textit{bevilja} ‘grant’              &     \isi{SAOL} & bi   & to\\
1874 & \textit{delgiva} ‘inform of’          &     \isi{SAOL} & bi   & to\\
1874 & \textit{före}{}-  ‘before’            &          & prep & to\\
     & \textit{förelägga} ‘set (a task) to’  &     \isi{SAOL} &\\
1874 & \textit{förvägra} ‘refuse’            &     \isi{SAOL} & bi   & hindered to\\
1874 & \textit{förära} ‘present with’        &     \isi{SAOL} & bi   & to\\
1874 & \textit{meddela} ‘inform of’          &     \isi{SAOL} & bi   & to\\
1874 & \textit{till}{}- ‘to-’                &          & prep & to\\
     & \textit{tilldela} ‘award’             &     \isi{SAOL} &\\
1874 & \textit{vägra} ‘refuse’               &     \isi{SAOL} & mono  & hindered to\\
1874 & \textit{å}{}- ‘on-’                   &     \isi{SAOL} & prep &  to\\
     & \textit{ålägga} ‘impose on’           &          &\\
1880 & \textit{förunna} ‘grant’              &      SPF & bi   & to\\
1880 & \textit{neka} ‘deny’                  &      SPF & mono & hindered to\\
1899 & \textit{unna} ‘grant’                 &      SPF & mono & to\\
1900 & \textit{räcka} ‘hand to’              &      SPF & mono & to\\
1900 & \textit{skänka} ‘give’                &      SPF & mono & to\\
1978--1979 & \textit{förlåta} ‘forgive’      & Bonniers & bi  & from\\
1978--1979 & \textit{förmena} ‘deny’         & Bonniers & bi  & hindered to\\
1978--1979 & \textit{servera} ‘serve with’   & Bonniers & mono & to\\
1978--1979 & \textit{visa} ‘show’            & Bonniers & mono & to\\
\end{longtable}

\appendixsection{Tokens of the 30 investigated verbs in the spf-corpus}\label{falk:appendix:3}
\begin{table}[H]
\begin{tabular}{l rrr}
\lsptoprule
& pass.io & pass.do & ambiguous\\\midrule
\textit{anförtro} ‘entrust to’ &  2 & 3  & 1\\
\textit{av}{}- ‘off-’          &  2 & 6  & 0\\
\textit{beröva} ‘deprive of’   & 23 & 9  & 1\\
\textit{bespara} ‘spare’       &  4 & 14 & 1\\
\textit{bevilja} ‘grant’       &  1 & 13 & 1\\
\textit{bibringa} ‘impart to’  &  7 & 3  & 0\\
\textit{bjuda} ‘offer’         &  3 & 37 & 2\\
\textit{delgiva} ‘inform of’   &  0 &  3 & 1\\
\textit{erbjuda} ‘offer’       & 12 & 41 & 0\\
\textit{från}{}- ‘from-’       &  2 & 10 & 1\\
\textit{före}{}- ‘before-’     &  0 & 10 & 0\\
\textit{förlåta} ‘forgive for’ &  0 &  6 & 0\\
\textit{förmena} ‘deny’        &  0 &  1 & 0\\
\textit{förunna} ‘grant’       &  3 & 25 & 0\\
\textit{förvägra} ‘refuse’     &  0 &  2 & 2\\
\textit{förära} ‘present with’ &  0 &  3 & 0\\
\textit{giva} ‘give’           &  7 & 49 & 2\\
\textit{lova} ‘promise’        &  3 &  4 & 0\\
\textit{lämna} ‘leave to’      & 10 & 39 & 3\\
\textit{meddela} ‘inform of’   &  0 & 17 & 1\\
\textit{neka} ‘deny’           &  6 & 11 & 0\\
\textit{på}{}- ‘on-’           &  1 &  3 & 0\\
\textit{räcka} ‘hand to’       &  1 & 25 & 0\\
\textit{servera} ‘serve with’  &  1 & 15 & 1\\
\textit{skänka} ‘give’         &  1 & 26 & 1\\
\textit{till}{}- ‘to-’         &  7 & 38 & 5\\
\textit{unna} ‘grant’          &  1 &  3 & 1\\
\textit{visa} ‘show’           &  0 & 24 & 1\\
\textit{vägra} ‘refuse’        &  0 &  5 & 1\\
\textit{å}{}- ‘on-’            &  5 &  5 & 2\\
Total & 102 & 450 & 28\\\lspbottomrule
\end{tabular}
\end{table}

{\sloppy\printbibliography[heading=subbibliography,notkeyword=this]}
\end{document}
