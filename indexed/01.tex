\documentclass[output=paper]{langscibook}
\ChapterDOI{10.5281/zenodo.5792951}
\author{Ida Larsson\affiliation{Østfold University College, Halden} and Erik M. Petzell\affiliation{Institute for Language and Folklore, Gothenburg}}
\title{Introduction: Morphosyntactic change in Late Modern Swedish}
\abstract{The chapters in this volume are concerned with morphosyntactic change in Late Modern Swedish, i.e. the period from the beginning of the 18\textsuperscript{th} century onwards. Although the period is interesting (considering, for instance, standardization processes), it has previously received fairly little attention in the syntactic literature. The contributions in this volume cover several grammatical domains, including case and verbal syntax, word order and agreement, and grammaticalization in the nominal domain. In this introduction to the volume, we give a background to Late Modern Swedish. We briefly discuss the external factors that are particularly relevant for morphosyntactic change during this period and give an overview of the morphosyntax of Late Modern Swedish. Finally, we provide a summary of the chapters that follow.

\keywords{Late Modern Swedish, morphosyntactic change, standardization, historical corpora, word order}
}
\IfFileExists{../localcommands.tex}{
  \addbibresource{localbibliography.bib}
  % add all extra packages you need to load to this file

\usepackage{tabularx,multicol}
\usepackage{url}
\urlstyle{same}

\usepackage{listings}
\lstset{basicstyle=\ttfamily,tabsize=2,breaklines=true}

\usepackage{langsci-basic}
\usepackage{langsci-optional}
\usepackage{langsci-lgr}
\usepackage{langsci-gb4e}

\usepackage{todonotes}

\usepackage[linguistics]{forest}
\usepackage{soul}
\usepackage{subfigure}
\usepackage{longtable}
\usepackage{enumitem}

  \newcommand*{\orcid}{}
%\newcommand{\keywords}[1]{\textbf{#1}}


\makeatletter
\let\theauthor\@author
\makeatother

\newcommand{\keywords}[1]{\textbf{Keywords:} #1}


\DeclareNewSectionCommand
  [
    counterwithin = chapter,
    afterskip = 2.3ex plus .2ex,
    beforeskip = -3.5ex plus -1ex minus -.2ex,
    indent = 0pt,
    font = \usekomafont{section},
    level = 1,
    tocindent = 1.5em,
    toclevel = 1,
    tocnumwidth = 2.3em,
    tocstyle = section,
    style = section
  ]
  {appendixsection}

\renewcommand*\theappendixsection{\Alph{appendixsection}}
\renewcommand*{\appendixsectionformat}{\appendixname~\theappendixsection\autodot\enskip}
\renewcommand*{\appendixsectionmarkformat}{\appendixname~\theappendixsection\autodot\enskip}
 
  %% hyphenation points for line breaks
%% Normally, automatic hyphenation in LaTeX is very good
%% If a word is mis-hyphenated, add it to this file
%%
%% add information to TeX file before \begin{document} with:
%% %% hyphenation points for line breaks
%% Normally, automatic hyphenation in LaTeX is very good
%% If a word is mis-hyphenated, add it to this file
%%
%% add information to TeX file before \begin{document} with:
%% %% hyphenation points for line breaks
%% Normally, automatic hyphenation in LaTeX is very good
%% If a word is mis-hyphenated, add it to this file
%%
%% add information to TeX file before \begin{document} with:
%% \include{localhyphenation}
\hyphenation{
anaph-o-ra
Dor-drecht
mono-mor-phe-mic
Swed-ish
sche-mat-ic
Viska-da-li-an
An-ders-son
dia-lekt-forsk-ning
dra-ma-språk
bref-vex-ling
Ak-tu-ell
folk-livs-forsk-ning
Þor-björg
Ak-ti-ons-art
Upp-sala
myck-en
}

\hyphenation{
anaph-o-ra
Dor-drecht
mono-mor-phe-mic
Swed-ish
sche-mat-ic
Viska-da-li-an
An-ders-son
dia-lekt-forsk-ning
dra-ma-språk
bref-vex-ling
Ak-tu-ell
folk-livs-forsk-ning
Þor-björg
Ak-ti-ons-art
Upp-sala
myck-en
}

\hyphenation{
anaph-o-ra
Dor-drecht
mono-mor-phe-mic
Swed-ish
sche-mat-ic
Viska-da-li-an
An-ders-son
dia-lekt-forsk-ning
dra-ma-språk
bref-vex-ling
Ak-tu-ell
folk-livs-forsk-ning
Þor-björg
Ak-ti-ons-art
Upp-sala
myck-en
}
 
  \togglepaper[1]%%chapternumber
}{}

\begin{document}
\maketitle 
%\shorttitlerunninghead{}%%use this for an abridged title in the page headers



\section{Introduction}\label{sec:intro:1}


This volume deals with \isi{morphosyntactic change} in \ili{Late Modern Swedish} (LMS). In the traditional \isi{periodization} of the history of Swedish, LMS is the last of four periods. The other three are Early and \ili{Late Old Swedish} (EOS, LOS), and \ili{Early Modern Swedish} (EMS); see \REF{ex:intro:1} below.\footnote{Traditionally, two older periods are also included: \ili{Ancient Nordic} –800 (in Swedish: \textit{urnordiska}) and \ili{Runic Swedish} 800–1225 (in Swedish: \textit{runsvenska}) (\citealt{Wessen1958}: 7–43; \citealt{Bergman1968}: 13–29). More recently, it has been suggested that LMS should be followed by a third period starting around 1880: \textit{modern nysvenska} (lit. ‘\ili{Modern New Swedish}’); cf. the Swedish labels for EMS and LMS: \textit{äldre} (‘Elder’) and \textit{yngre} (‘Younger’) \textit{nysvenska} (‘New Swedish’) (\citealt{Thelander1988}; \citealt{Malmgren2007}). However, \textit{modern nysvenska} has hardly become a standard period label, but is part of an ongoing theoretical discussion of \isi{periodization} in Swedish historical linguistics (see \citealt{Ralph2000}; \citealt{Johansson2007, Johansson2010}). Here, we use LMS in its traditional sense, viz. as a period starting in 1732 and leading up to the present. Throughout this volume, the language of today – i.e. the language that present-day native speakers have intuitions about – is referred to as present-day Swedish (PDS).}


\ea \label{ex:intro:1}
\ea   \ili{Early Old Swedish} (EOS) 1225–1375
\ex  \ili{Late Old Swedish} (LOS) 1375–1526
\ex   \ili{Early Modern Swedish} (EMS) 1526–1732
\ex   \ili{Late Modern Swedish} (LMS) 1732–
\z
\z

The two earliest dates (1225, 1375) are approximations. 1225 represents the introduction of the \ili{Latin} alphabet for writing Swedish, a process which started at the beginning of the 13\textsuperscript{th} century. 1375 represents a period of demographic and political change in Scandinavia due mainly to the devastating effects of the Black Death, and the growing influence of the \isi{Hanseatic League} in northern Europe. The two modern periods (EMS, LMS), on the other hand, both have starting dates that coincide with the year of appearance of an important publication: \textit{Thet Nyia Testamentit på Swensko}, ‘The New Testament in Swedish’ (printed in 1526), and the ground-breaking weekly journal \textit{Then Swänska Argus}, ‘The Swedish \isi{Argus}’ (first issued in 1732), respectively.



Linguistic change can, in other words, be studied in texts going back to the early 13\textsuperscript{th} century (and even further if we include the runic inscriptions). The oldest preserved Swedish text in the \ili{Latin} alphabet is a medieval law, the \textit{Elder Westrogothic law} (\isi{EWL}, in Swedish: \textit{Äldre Västgötalagen}). The \isi{EWL} is the oldest of the \isi{laws of the provinces} (Sw. \textit{landskap}) that later became the Swedish kingdom.\footnote{There are \isi{medieval laws} (as well as other texts) from regions that did not become part of Sweden until the middle of the 17\textsuperscript{th} century (viz. Gotland and Skåne). Traditionally, early texts from these areas have been excluded from the history of Swedish.} It begins with a section on the role of religion in society, the beginning of which is given in \REF{ex:intro:2}.


\ea    \label{ex:intro:2}
\gll  Her   byriarz                 laghbok                     væsgöta Krister               ær           fyrst   i     laghum           warum           þa ær           cristna                       var                  oc   allir cristnir                   konongær.         böndær               oc   allir bocarlær                       biscupær               oc     allir boclærðir                       mæn.               Varþær       barn           til kirkiu                 boret                 oc   beþiz                   cristnu. þa     scal         faþir           ok   moðer           fa guðfæþur         oc    guðmoþor~      oc   salt         oc   uatn. þæt           scal        bæræ      til   kirkiu         þa   scal a   prést                 kallæ   han                 skal               a     kirkiu bole                       boæ.\\
here   begin.\textsc{prs.refl}     law.book.\textsc{m.sg.nom}     westrogoth.\textsc{m.pl.gen} Christ.\textsc{m.sg.nom}   be.\textsc{prs.sg}     first   in   law\textsc{.m.pl.dat}   our\textsc{.m.pl.dat}   then be.\textsc{prs.sg} Christendom\textsc{.f.sg.nom}   our\textsc{.f.sg.nom}   and   all.\textsc{m.pl.nom} Christian.\textsc{m.pl.nom}   king.\textsc{m.sg.nom}  farmer.\textsc{m.pl.nom}    and  all.\textsc{m.pl.nom} resident.man.\textsc{m.pl.nom}    bishop.\textsc{m.sg.nom}   and    all.\textsc{m.pl.nom} book.learn.\textsc{ptcp}.\textsc{m.pl.nom}   man.\textsc{m.pl.nom}   become.\textsc{prs.sg}     child\textsc{.n.sg}   to  church\textsc{.f.sg.gen}   carry\textsc{.ptcp.n.sg}  and ask.\textsc{prs.sg.pass}   christening\textsc{.f.sg.acc.} then   shall\textsc{.prs.sg}   father\textsc{.m.sg.nom}   and   mother\textsc{.f.sg.nom}   get.\textsc{inf} godfather\textsc{.m.sg.acc} and godmother\textsc{.f.sg.acc}   and salt.\textsc{n.sg.acc}   and   water\textsc{.n.sg.acc} it.\textsc{n.sg.acc}   shall\textsc{.prs.sg}   carry\textsc{.inf}     to   church\textsc{.f.sg.gen}   then   shall\textsc{.prs.sg} on priest\textsc{.m.sg.acc}   call\textsc{.inf}   he\textsc{.m.sg.nom}    shall\textsc{.prs.sg}     on   church\textsc{.f.sg.gen} farm.\textsc{n.sg.dat}         live\textsc{.inf}\\
\glt ‘Here begins the law book of the West Goths. Christ is first in our law. Thereafter comes our Christian faith and all Christians, king, farmers and all resident men, bishop, and all learned men. If a child is carried to church, and christening is asked for, then father and mother should get godfather, and godmother, and salt, and water. One should carry that to church. Then one should call for a priest. He should live at the parsonage.’ (\isi{EWL}, 1220s; from FTB)\footnote{Many older Swedish texts are available through \textit{Fornsvenska textbanken}, ‘the text bank of \ili{Old Swedish}’: \url{https://project2.sol.lu.se/fornsvenska/}, which we will refer to as FTB. The same texts can be accessed through the corpus infrastructure \isi{Korp} \citep{BorinEtAl2012}: \url{https://spraakbanken.gu.se/korp/?mode=all_hist}. In the following, we do not provide a page number for examples taken from electronic sources.}
\z



The language of the Swedish \isi{medieval laws} differs from present-day Swedish (PDS) in many ways. Among other things, EOS had a rich \isi{case system} (e.g. \textit{a kirki-u bol-e} ‘at church-\textsc{gen} farm-\textsc{dat’}), post-nominal possessives (\textit{laghbok væsgöta} ‘law book of West Goths’, \textit{laghum warum} ‘laws our’), and lacked \isi{indefinite} articles (\textit{barn til kirkiu boret} ‘child to church carried’, \textit{a prest kalla} ‘on priest call’). Moreover, overt pronominal subjects were quite rare: although referential pronouns were only occasionally omitted (see for instance the overt \textit{han} referring to \textit{prést}, ‘priest’, in the last sentence), there were neither \isi{expletive} nor generic pronouns (e.g. \textit{þa scal a prést kallæ} lit. ‘then [one] shall on priest call). As for the position of verbs, EOS still had \isi{OV order} (\textit{han skal a kirkiu bole boæ} lit. ‘he shall on land of church live’), and \isi{sentence adverbials} followed the finite verb in both main and embedded clauses (although there are no such examples in \REF{ex:intro:2}; see \sectref{sec:intro:3.1.1} below).



Five centuries after the \isi{EWL}, the grammatical system had undergone dramatic change on all levels. Consider the introduction to \textit{Argus} in \REF{ex:intro:3} below, which, as noted, represents the beginning of the \ili{Late Modern Swedish} period.\footnote{In the glosses, \textsc{C} refers to common \isi{gender}; see more in \sectref{sec:intro:3.3} on the change from a \isi{three-gender system} to a system with two genders. \ili{Present-day Swedish} has a morphological distinction between the participial verb form used in perfects, the so-called \isi{supine} form (\textsc{sup}), and \isi{perfect} participles (\textsc{ptcp)} used e.g. in passives, but since this distinction is not yet established in the 18\textsuperscript{th} century, the participles in examples like \REF{ex:intro:3} are glossed as \textsc{ptcp}; see more in \sectref{sec:intro:3.1.2} below.}


\ea        \label{ex:intro:3}
\gll  THEN   SWÄNSKA     \isi{ARGUS}   N:o I.     Ingen         lärer  kunna         neka,     at     ju       sådane   Skriffter hafwa         stor       nytta     med   sig,   som,   på   ett     angenämt   och lustigt         sätt,     föreställa       Lärdomar   och Wettenskaper;   Derföre hafwa         och de     gamla,     under   roliga         Dikter, liufliga       Samtahl         eller   nöysamma   Historier, underwisat Folket           om   Dygden,        och   likasom   skiämtewijs   förehållit dem       alfwarsamma   Sede-Läror.      I  nyare       tider,    och   än     i   dag, se             wi   äfwen, hos kloka     Nationer,   sådane   Skriffter med   mycken     nytta       utgifwas         och   älskas\\
  the.\textsc{c.sg}   Swedish\textsc{.def} \isi{Argus}   no 1  no.one.\textsc{c.sg}   shall\textsc{.prs.sg} be.able.to.\textsc{inf}  deny.\textsc{inf}   that   \textsc{part}  such.\textsc{pl}  writing.\textsc{pl}  have.\textsc{prs.pl}   large.\textsc{c.sg}  benefit  with  \textsc{refl}  that  on    a.\textsc{n.sg}  pleasant.\textsc{n.sg}  and amusing.\textsc{n.sg}   manner  present.\textsc{prs.pl}    learning.\textsc{pl}  and science.\textsc{pl}    therefore have.\textsc{prs.pl}   also the.\textsc{pl}  ancient\textsc{.pl}  during  entertaining.\textsc{pl}  poem.\textsc{pl} delightful.\textsc{pl}  conversation.\textsc{pl}   or    diverting.\textsc{pl}  tale.\textsc{pl}  instruct.\textsc{ptcp} people.\textsc{def.n.sg}  about  virtue.\textsc{def.c.sg}  and  almost  jokingly    impart.\textsc{ptcp} \textsc{3pl.obj}  grave.\textsc{pl}      moral-lesson\textsc{.pl}  in   new.\textsc{comp}  time.\textsc{pl}  and  even to   day see.\textsc{prs.pl}   we   also     at  wise.\textsc{pl}  nation.\textsc{pl}  such.\textsc{pl}   writing.\textsc{pl} with   much\textsc{.c.sg}   benefit   publish.\textsc{prs.pass}   and  cherish.\textsc{prs.pass}\\
\glt ‘The Swedish \isi{Argus} No 1. No one can deny that such writings are indeed beneficial, that, in a pleasant and amusing manner, administer learning and science. So the ancients have instructed the people on virtue through~entertaining poems, delightful conversations and diverting tales, and imparted grave lessons of morality in an almost jocular manner. In more recent times, and even today, we still find writings of this kind, useful and cherished, published in wise nations.’ (\textit{Argus}, 1730s; from FTB)
\z



Here, very little is left of the old \isi{case system} (some forms linger in the \isi{pronominal system}, e.g. \textit{dem} ‘them’, which is an old \isi{dative}, but in \REF{ex:intro:3}, and still today, it functions as a general object form), the modern article system is fully in place (e.g. \textit{ett angenämt och lustigt sätt} ‘a pleasant and amusing manner’), and possessives are prenominal. Moreover, referential \textit{pro}{}-drop was no longer possible, and expletives were increasingly becoming the rule in the 18\textsuperscript{th} century. \ili{Late Modern Swedish} was a VO language (e.g. \textit{underwisat Folket om Dygden} ‘instructed the people about virtue’), and in embedded clauses, the finite verb generally remained in the verb phrase, and it therefore followed \isi{sentence adverbials}. However, some things have remained stable over time. For instance, both EOS and LMS are V2 languages, which means that in main clauses, the finite verb is always inverted with the subject, unless the subject itself is topicalized. This results in either SV- or (X)VS-initial word order, i.e. a surface order where V never comes later than second position. Thus, we find SV (e.g. \textit{han skal} ‘he shall’ in \REF{ex:intro:2}, and \textit{Ingen lärer} ‘No one should’ in \REF{ex:intro:3}) and (X)VS (e.g. \textit{Varþær barn} ‘becomes child’ in \REF{ex:intro:2}, and \textit{Derföre hafwa och de gamla} lit. ‘therefore have also the old’ in \REF{ex:intro:3}) in both the EOS and the LMS texts.\footnote{Many instances of the XVS structure are obscured in \REF{ex:intro:2} due to the lack of overt subjects.}



Many of the substantial grammatical changes that took place in the period between the texts in \REF{ex:intro:2} and \REF{ex:intro:3} have been investigated in the historical records. There are for instance studies of the shift from OV to \isi{VO word order} (e.g. \citealt{Delsing1999}; \citealt{Petzell2011}; \citealt{Sangfelt2019}), the loss of \textit{pro}{}-drop \citep{Hakansson2008} and the emergence of expletives \citep{Falk1993}, changes in embedded word order (\citealt{Platzack1988centralskandinaviska}; \citealt{Falk1993}; \citealt{Hakansson2011}), the \isi{grammaticalization} of auxiliaries (\citealt{Andersson2007}; \citealt{Bylin2013}) and complementizers \citep{Rosenkvist2004}, the loss of \isi{case morphology} (\citealt{Delsing1991,Delsing2014Studier,Delsing1991}; \citealt{Norde1997}; \citealt{Falk1997}; \citealt{Skrzypek2005}) and the \isi{grammaticalization} of (in)definite articles (\citealt{Skrzypek2009}; \citealt{BrandtlerDelsing2010}; \citealt{Stroh-Wollin2016}). These studies typically consider the development of Swedish from the \ili{Old Swedish} period until the early \ili{Late Modern Swedish} period (i.e. the middle of the 18\textsuperscript{th} century). They also concern changes that to a large extent can be observed in all of the \ili{Mainland North Germanic} languages (see e.g. \citealt{HolmbergPlatzack1995}).



As is specified in \REF{ex:intro:1}, LMS continues into the present. Thus, in a way, it indicates the end of the history of Swedish. It is, of course, an absurd idea that history should have an endpoint. Nevertheless, the impression that LMS is too close to the present to be of interest or importance for historical linguistics has indeed shaped the output of this discipline in Sweden. Its main focus has always been on the earliest stages of the language (\citealt{Wollin1988}; \citealt{Haapamaki2010}). Naturally, such an inclination towards the archaic is understandable when the main objective is the \isi{reconstruction} of a proto-language. However, even diachronic research set in a generative framework, where the age of the linguistic source is irrelevant, has shown a strong tendency towards addressing the grammatical structure of old rather than recent Swedish, despite the fact that the latter is much more robustly documented in texts of various types. An important reason for this is simply that, as we have already seen, many interesting (and quite dramatic) things happened in the grammar of Swedish towards the end of the Middle Ages.



History clearly did not end in 1732, but to date we know considerably less about \isi{morphosyntactic change} from the middle of the 18\textsuperscript{th} century onwards than we do about earlier periods. The present volume aims to remedy this. As we will see in the following, there were changes in word order, in the abstract \isi{case system}, in the distribution of adverbials, and so on, that took place in the \ili{Late Modern Swedish} period, and there are \isi{grammaticalization} processes that continue into the present day. Moreover, the \ili{Late Modern Swedish} period is interesting for a number of reasons. This was when Swedish was established as a national \isi{standard language}. New genres emerged, and the written language became more generally available to all speakers. We also sometimes find diverging developments in the different \ili{Mainland North Germanic} languages, and some of the much-discussed differences between \ili{Danish}, \ili{Norwegian} and Swedish (e.g. in \isi{argument placement}, passives, particles, and \isi{participle morphology}) were established during this period. In addition, during the 19\textsuperscript{th} and 20\textsuperscript{th} centuries, the traditional dialects underwent more dramatic changes than ever. 



This anthology contains a collection of papers that all discuss \isi{morphosyntactic change} in \ili{Late Modern Swedish}. Some of the articles aim to complete our knowledge of previously studied phenomena, addressing the last remnants of a medieval system (e.g. \isi{lexical case} or \isi{verbal agreement} in archaic dialects). Others instead focus on changes that began in \ili{Early Modern Swedish} or even later (e.g. the development of quantifiers). However, the authors of both of these sets of articles engage in the task of analysing linguistic developments that are still ongoing, reflected in unstable and varying present-day usage. The papers shed new light on both internal and external factors in language change; we will see effects of morphological change and of \isi{standardization} processes, as well as of syntactic economy principles.



In this introduction, we provide some background to the \ili{Late Modern Swedish} period. The main aim is to set the stage for the papers in the volume, but since there is currently no accessible overview of \ili{Late Modern Swedish}, we also briefly provide some details about \ili{Late Modern Swedish} grammar. \sectref{sec:intro:2} gives an overview of some of the external factors that are relevant for \isi{morphosyntactic change} during the period. In \sectref{sec:intro:3}, we present some central aspects of the morphosyntax of \ili{Late Modern Swedish}. \sectref{sec:intro:4} gives an overview of the papers in the volume.


\section{External factors in Late Modern Swedish }\label{sec:intro:2}


As noted above, the outer prerequisites for Swedish changed in the \ili{Late Modern Swedish} period. In this section we give an overview of the external factors that have affected the development of Swedish morphosyntax during the period. \sectref{sec:intro:2.1} is concerned with the \isi{standardization} of Swedish. In \sectref{sec:intro:2.2}, we briefly discuss the inter- and intra-individual variation that can be observed in the 18\textsuperscript{th} and 19\textsuperscript{th} centuries. \sectref{sec:intro:2.3} gives a short overview of the sources on \ili{Late Modern Swedish}, with particular focus on the available electronic corpora that are used by the authors in this volume.


\subsection{Standardization and education}\label{sec:intro:2.1}


As shown in \sectref{sec:intro:1}, there are Swedish texts written with the \ili{Latin} alphabet from the 13\textsuperscript{th} century onwards. However, the development of a \isi{standard language} came much later (for a thorough description of this process, see \citealt{Teleman2002}). It was not until the publication of the New Testament in Swedish in 1526 and its natural continuation with the entire \isi{Bible} in 1541, that one uniform way of writing Swedish reached all the parishes of the realm. Spreading the \isi{Bible} in Swedish by employing the Gutenbergian printing technique was an important item on the agenda of the centralized Swedish state, which began to be implemented at the beginning of the 16\textsuperscript{th} century. In order to consolidate the emerging \isi{nation state}, clearly distinguishing it from other similar states (in particular from Denmark, with which Sweden had been in a union since the late 14\textsuperscript{th} century), the governing elite (with King Gustav Vasa in the lead) resorted to both practical and spiritual means. In a way, the new \isi{Bible} embodied both strategies. By adopting the reformed religion, the state gained full control of the church, which had to cut its ties to Rome, including all its institutions and, not least, its long-established presence in all local communities: men of the church now answered to the king in Stockholm, not to the pope in Rome. Of course, the primary objectives of the state when distributing a printed \isi{Bible} were not linguistic – the point was rather to execute and demonstrate the power of a new and centralized Sweden (see \citealt{Kouri1994} for discussion). Nevertheless, the linguistic consequences for the written language were immense \parencite{Stahle1970}.



As late as the early 18\textsuperscript{th} century, the \isi{Bible} of 1541 was still the most relevant prototype for written Swedish. There had been new editions of the \isi{Bible} issued both in 1618 and in 1703, but the form of the original was kept more or less intact, with only minimal revisions (see \citealt{Platzack2005gamla}). By contrast, the state took quite radical measures in the domain of civil law towards the end of the EMS period, resulting in a new, albeit linguistically quite archaic \citep{Wessen1965} code of law for the nation – \textit{Sveriges Rikes lag}, ‘the law of the kingdom of Sweden’ – from 1734, which is still in effect in parts. However, neither law nor \isi{Bible} came to play any significant role in the shaping of the written standard during the 18\textsuperscript{th} century. Instead, new genres that emerged through the Age of Enlightenment arose as preferred models for \ili{Standard Swedish}. Although this new and secular standard was eventually codified in dictionaries (e.g. Sahlstedt’s \textit{Swensk Ordbok}, ‘Swedish Dictionary’, from 1773) and normative pamphlets (e.g. Leopold’s \textit{Afhandling om svenska stafsättet}, ‘Treatment of the Swedish orthography’, from 1801), it was through the distribution of new texts that the modern way of writing Swedish reached a wider audience. Consequently, the \isi{productive} publishers of the time had a massive impact on the spread of linguistic norms; one of the most prominent was Lars Salvius, whose efforts are described at length by \citet{Santesson1986}.



Exactly how many people had direct access to written texts during the Late Modern period is hard to say. Although text consumption certainly increased during the 18\textsuperscript{th} century, most people lived in the countryside and were probably relatively unaffected by the development of intellectual life in the city. But, based on the detailed census conducted by the church (by means of the so-called \textit{husförhör}, lit. ‘house interrogations’), most people were already listed as “being able to read” by the end of the 17\textsuperscript{th} century. However, this was probably a very rudimentary form of \isi{literacy}, comprising reading from the \isi{Bible} (or perhaps reciting it by heart) but not writing (\citealt{Johansson1981}; \citealt{Berg1994}). In the early years of the 19\textsuperscript{th} century, general education programmes were launched, resulting in free \isi{schooling} for all children from 1842, but school did not become obligatory until 1882.



Still, the mastering of the written code by the many was very much a matter of the 19\textsuperscript{th} and especially the 20\textsuperscript{th} century. Even with mandatory primary education, many left school only partially literate (as in the 1600s). For example, in a recording from the 1950s, part of which is transcribed in \REF{ex:intro:4} below, an old \ili{Viskadalian}\footnote{Viskadalen is a \isi{dialect} area in the southwest of Sweden, along the southern reaches of the River Viskan (see \citealt{Petzell2017,Petzell2018} and \citeyear{chapters/07} [this volumes] for more details).} woman (A) tells the interviewer (S) of her time in school at the end of the 19\textsuperscript{th} century. She recalls that they would read various religious texts, but she never did learn how to write – it was simply not on the curriculum.


\ea \label{ex:intro:4}
\ea
\gll A:    Vi   fingem                  läsa             i     testamentet            å     i     kattjesen å     i     bibelska […] men att   Dyber […] han lärde inte   å   {skriva […]}\\
     {}     we   get.\textsc{pst.1pl}  read.\textsc{inf} in   testament.\textsc{def} and   in   catechism.\textsc{def} and   in   Biblical   {}       but   that Dyberg   {}    he   teach.\textsc{pst} not   to   write\textsc{.inf}\\
    \glt `We got to read from the Testament, and the Catechism and the Biblical […] but Dyberg, he didn’t teach us to write’
\ex
\gll S:     Vem   lärde       det   då?\\
     {}  who    teach.\textsc{pst}  that  then\\
    \glt `Who taught you that, then?’

\gll A:   Nä   ja   kan           inte     skriva     nöe […]\\
  {} No  I  can\textsc{.prs.sg}   not  write.\textsc{inf}   anything  \\
    \glt `No, I can’t write anything’ (Öxn)
\z
\z


As we will see in the chapters that follow, \isi{standardization} had consequences for \isi{morphosyntactic change} in the \ili{Late Modern Swedish} period. The papers by Valdeson and Kalm suggest that \isi{schooling} may have played a role in the development of double object constructions and \isi{adverbial} infinitives, respectively. Using the \isi{non-standardized variety} of \ili{Övdalian} as a point of comparison, Kalm argues that the elaboration of the written language led to the development of new grammatical possibilities. \isi{Standardization} processes also clearly affected the direction of change, as well as the spread and establishment of new patterns (e.g. the new word order in \isi{particle} constructions discussed in the paper by Larsson \& Lundquist). The spread of the \isi{standard language} also had consequences for the dialects. For instance, in his chapter on \isi{morphosyntactic variation} in \ili{Viskadalian} Swedish, Petzell argues that verbal \isi{person agreement} was reanalysed as part of \isi{tense}, and that one of the driving forces behind this \isi{reanalysis} was the introduction of the new standard word order in embedded clauses, which was incompatible with richly agreeing verbs. Finally, \isi{standardization} naturally came with \isi{normative grammarians} promoting or advising against certain constructions (see \citealt{Teleman2002, Teleman2003Tradis}, and references therein, for a discussion of \isi{language planning} and policy in LMS). In her chapter on \isi{passive} ditransitives, Falk relates the actual usage of these constructions to contemporary recommendations in normative dictionaries. 


\subsection{Variation}\label{sec:intro:2.2}


Grammatical change in the development from Old to \ili{Late Modern Swedish} led to considerable linguistic variation both within and across speakers, on all linguistic levels. Since there was no fully established standard yet, there was still considerable variation even in the written language at the beginning of the \ili{Late Modern Swedish} period. As for the spoken language, the late 1700s and early 1800s stand out as a pinnacle of dialectal diversity. However, already towards the end of the 19\textsuperscript{th} century, \isi{dialect} levelling and the spread of a spoken standard had more or less wiped out the varying linguistic landscape of old in just a few generations (\citealt{NilssonPetzell2015}).



In the development of the \isi{standard language}, the spoken language of the upper classes in the area around Stockholm (\isi{Central Sweden}) played an important role. The 17\textsuperscript{th} century author Georg Stiernhielm states explicitly that he prefers this variety to other dialects, and in his treaty on Swedish, Sven \citet{Hof1753} makes similar comments (see \citealt{Widmark2000}: 26). Language change in Early and \ili{Late Modern Swedish} can also often first be observed in informal texts by authors of Central Swedish origin. Many innovations have early attestations in the memoirs of Agneta Horn (born 1629), an upper-class woman without formal education. For instance, she is the first to show evidence of a change in the word order in \isi{particle} constructions, discussed by \textcitetv{chapters/04}. Moreover, she has a stronger preference for the \isi{auxiliary} \textit{ha} ‘have’ (rather than \textit{vara} ‘be’) in \isi{participle} constructions with \isi{unaccusative} verbs than many of her contemporaries (see \citealt{Larsson2009} and below); in the written standard, \textit{ha} became established as the norm in the second half of the 18\textsuperscript{th} century (see \citealt{Johannisson1945}; \citealt{Larsson2009}: 247, Table 7.4). With respect to clause structure, Horn is also fairly modern. In her memoirs, there are only sporadic occurrences of the old \isi{OV order} \citep{Petzell2011}, and subordinate clauses generally have the modern order of finite verb and sentence \isi{adverbial} (\citealt{Falk1993}; see also \sectref{sec:intro:3.1.2} below). In addition, one of the earliest examples of an inverted \isi{expletive} (as in PDS), indicating true subjecthood, comes from her \citep[268]{Falk1993}.\footnote{The example, quoted from \citet[268]{Falk1993}, is given in (i) with the inverted \isi{expletive} in italics.

\ea\gll Och  när    thet  snöga,    så  lågh \textit{thet} stora  snödrifwan    i    kamaren\\
        and  when  it      snow.\textsc{pst},  so  lie.\textsc{pst}  \textsc{expl}  big    snowdrift.\textsc{def} in  chamber.\textsc{def}\\
   \glt ‘And when it was snowing, there was a big snowdrift inside the room.’
   \z}



Texts like Horn’s memoirs give us good insights into the contemporary spoken language. The variation also shows up in theatre plays from the time (see e.g. \citealt{Widmark1970} and below). In the play \textit{Några mil från Stockholm} ‘A few miles from Stockholm’ by Adolf Fredrik Ristell (1787), we can, for instance, observe that the subject form of the third person plural pronoun is \textit{de} in the stage directions, in line with the PDS written standard, but in the dialogue, the form \textit{di} is used. The object forms of the first and second person singular pronouns are \textit{mäj} ‘me’ and \textit{däj} ‘you’ in the dialogue – as in the present-day spoken standard – and not \textit{mig} and \textit{dig}, which is the written standard. Moreover, forms like \textit{trägåln} ‘garden’ for PDS \textit{trädgården} and \textit{Drånningholm} for the name ‘Drottningholm’ (‘the queen’s islet’), as well as \textit{vanlia} ‘usual’ for PDS \textit{vanliga}, reflect the pronunciation of the upper classes in \isi{Central Sweden} at the time. Assimilated forms like \textit{drånning} rather than \textit{drottning} ‘queen’ and \textit{trägål} rather than \textit{trädgård} ‘garden’, used by Ristell (as well as Horn a century before), are considered highly dialectal in the present-day language. During the 20\textsuperscript{th} century the unmarked pronunciation has changed to one that is closer to the written language. Today, the \textit{ttn}{}-sequence is pronounced as two segments ([t]+[n]) rather than one ([n:]), and the -\textit{rd}{}-sequence is pronounced as a retroflex \textit{d} ([ɖ]) rather than a retroflex \textit{l} ([ɭ]).



According to the guide to Swedish pronunciation by Lyttkens \& Wulff from 1889, many of the forms used by Horn and Ristell that are perceived today as highly dialectal (or rural) were still considered unmarked in the late 1800s.\footnote{Thus, assimilating various dental clusters into geminate \textit{nn} (as \textit{drottning} → \textit{drånning} ‘queen’, \textit{vändning} → \textit{vänning} ‘turn’) was considered perfectly natural. However, the \textit{l}{}-pronunciation of the sequence \textit{rd} (indicated by the spelling \textit{trägå}\textbf{\textit{l}}) appears to have been outdated in the spoken language of educated people already by the 18\textsuperscript{th} century \citep{Hof1753}.} However, during the 20\textsuperscript{th} century, a new spoken standard emerged. Lacking access to the prestigious spoken language of high society in Stockholm, primary school teachers had started to promote a way of speaking that was very close to the written letter, and was thus easy to acquire. Previously, such written-like speech, as it were, had been reserved for public announcements \citep{Widmark1970}, and was considered unfit for everyday conversations by contemporary intellectuals (\citealt{Cederschiold1897}; \citealt{Noreen1903}). Nevertheless, the strategy of the schoolteachers was successful \citep{Widmark2000}. At the same time, they discouraged the use of traditional dialects, aiming at a common spoken code for all in the modern and equal social-democratic Sweden. As a consequence, spoken Swedish of the late 20\textsuperscript{th} century was probably less varied than it had ever been before; see \citet{NilssonPetzell2015} for more details (including comparison with \ili{Norwegian} and \ili{Danish}). However, dialectal diversity was by no means eliminated altogether, as can be seen in the papers by Kalm and Petzell, where archaic dialects of today (or at least of a quite recent yesterday) that deviate considerably from the \isi{standard language} play an important role.


\subsection{New genres and more data}\label{sec:intro:2.3}


The \ili{Late Modern Swedish} period offers substantially more data for the historical linguist than earlier periods: old genres remain, new emerge, and more texts are preserved. In letters and diaries from the 18\textsuperscript{th} century, we can observe the linguistic variation of the time. In addition, there are, as noted, a growing number of plays containing dialogue that attempts to mimic the spoken language (see e.g. \citealt{Widmark1970,Widmark2000,Thelander2007}). The project \textit{Swedish drama dialogue over three centuries} (\citealt{MarttalaStromquist2001}) has collected a corpus of 45 plays from the period 1725–2000, divided into intervals of 25 years, with a total of more than 800,000 words. For the study of \isi{morphosyntactic change}, it is clearly useful to have access to sources which reflect the spoken language as closely as possible (see \citealt{Magnusson2007}: 69–74 for discussion), and the corpus of Swedish drama dialogue provides us with perhaps the best possible sample. Several of the papers in this volume use this corpus.



The production of non-fiction flourished during the 18\textsuperscript{th} century, with texts about science, gardening, cooking, and so on. In the 19\textsuperscript{th} century, the production of fiction underwent a veritable explosion. Some of these texts can be accessed in the corpus of Swedish prose fiction 1800–1900 (the \isi{SPF corpus}). This corpus includes all Swedish original novels and short stories published in separate editions in the years 1800, 1820, 1840, 1860, 1880 and 1900, and includes altogether more than 16 million tokens.\footnote{Texts from a few additional years are also included in the corpus, specifically 1841–1844, 1898–1899, and 1901.}  The \isi{SPF corpus} can be accessed through the language infrastructure \isi{Korp} \citep{BorinEtAl2012}, which also contains letters, newspaper prose, and older laws, as well as other older and modern corpora (of varying quality) with older fiction and non-fiction.\footnote{\url{https://spraakbanken.gu.se/korp}}  Here there is, for instance, a corpus of 56 novels from the period 1840–1930 (\isi{Äldre svenska romaner}, ÄSv, ‘older Swedish novels’). The corpora make new methods available, the possibilities of which are explored in \textcitetv{chapters/03}. Valdeson uses the \isi{Korp} infrastructure to investigate the frequencies of double object constructions at different times. Among other things, he uses a measure of \isi{productivity}, referred to as \textit{lexical variation}, which considers how many different verbs occur in the double object constructions and how many different objects can occur with a specific verb.



Another important source for Early and \ili{Late Modern Swedish} is the Swedish Academy Dictionary (\isi{SAOB}),\footnote{\isi{SAOB} can be accessed here: \href{http://www.saob.se}{{www.saob.se}}} which provides a thorough description of the Swedish vocabulary from 1526 to the present. In this volume, data from \isi{SAOB} are used in Falk’s discussion of passivization of \isi{ditransitive} verbs, as well as in Delsing’s account of the \isi{grammaticalization} of the \isi{quantifier} \textit{mycket} ‘much’. We also give examples taken from \isi{SAOB} in \sectref{sec:intro:3} below.



In addition to the written sources of \ili{Late Modern Swedish}, there are various types of sources to the spoken language. There is a large number of recordings of traditional dialects from the middle of the 20\textsuperscript{th} century: the Institute for Language and Folklore has approximately 25,000 hours of \isi{dialect} audio from all over Sweden, the Americas (mostly from the North), Finland, and \isi{Estonia}. In the Americas, Swedish is a \isi{heritage language} (see e.g. \citealt{Larsson2015}). In \isi{Estonia} it was a \isi{minority language} until the 1940s (see \citealt{Rosenkvist2018}), and in Finland it still is a \isi{minority language}. Most of these early recordings are digitized, but they are only sporadically transcribed and therefore searchable to a very limited extent \citep{BergEtAl2019}. In this volume, Petzell investigates word order in recordings of \ili{Viskadalian} Swedish from the 1940s, 1950s, 1960s simply by listening to the audio files. The Institute also harbours a vast collection of phonetically precise, handwritten \isi{dialect} texts from the late 1890s and early 1900s (\citealt{Sellberg1993}: 431–432; \citealt{SOU1924/27}: 30–33). In this introduction, we give some examples from such a \isi{dialect} text to illustrate (among other things) morphosyntactic archaisms (see \sectref{sec:intro:3.4}). Recently, promising attempts have been made to \isi{transfer} the handwritten texts to a digital (and therefore searchable) format by employing so-called HTR (\isi{handwritten text recognition}) techniques (see \citealt{Petzell2019,Petzell2020}).


\section{Late Modern Swedish Morphosyntax}\label{sec:intro:3}


In this section, we give a very brief overview of the morphosyntax of \ili{Late Modern Swedish}, as a backdrop to the studies in the following chapters. We focus on central aspects of the grammar, particularly on phenomena that are relevant in the chapters that follow, and on phenomena that have previously been shown to change during the \ili{Late Modern Swedish} period. Among other things, we will not discuss \isi{V2 order} or binding of reflexives, which seem to have been stable throughout the \ili{Modern Swedish} period.\footnote{\citet{Tingsell2010} shows that there is some inter- and intra-individual variation in the distribution of reflexives in 18\textsuperscript{th} century Swedish, and that the variation is very similar to what is found in multilingual urban settings in present-day Swedish.}  For clarity, we employ a fairly standard model of phrase structure, where the clause is divided into three domains: the verbal domain (\isi{VP}), the \isi{inflectional domain} (\isi{IP}), and the \isi{C-domain} (\isi{CP}), where features relating to \isi{finiteness}, clause type, and \isi{illocutionary force} are found (see \citealt{Platzack2010}; \citealt{Faarlund2019}). 


\subsection{Verbal morphology and verb placement}\label{sec:intro:3.1}


\sectref{sec:intro:3.1.1} is concerned with verb placement in embedded clauses and subject-\isi{verb agreement}. In \sectref{sec:intro:3.1.2}, we give a summary of some changes in the Swedish \isi{tense system} that took place partly in the \ili{Late Modern Swedish} period. \sectref{sec:intro:3.1.3} gives a brief overview of the history of Swedish infinitival structures headed by \textit{att}.


\subsubsection{Embedded word order and agreement}\label{sec:intro:3.1.1}


As stated above, the \isi{V2 word order} of Swedish main clauses has been stable for many centuries (see also \citealt{Alving1916}). By contrast, in subordinate contexts, the position of the finite verb has changed since the Middle Ages. In EOS, the verb preceded \isi{sentence adverbials} as in \REF{ex:intro:5a} below, indicating verb \isi{movement} out of the \isi{VP} to somewhere in the \isi{I-domain}. In present-day Swedish, the finite verb of subordinate clauses instead remains in the \isi{VP}, where it is preceded by \isi{sentence adverbials} as in \REF{ex:intro:5b}. Today, V can move out of the \isi{VP} in a \isi{subordinate clause} only in a limited set of \textit{that}{}-clauses where the \isi{complementizer} can take an entire \isi{CP} as its \isi{complement}. Such embedded \isi{V-to-C} \isi{movement} is possible only when the content of the \isi{subordinate clause} can be interpreted as being asserted by the \isi{speaker}, as in \REF{ex:intro:5c}; see \citet{Julien2015}, \textcitetv{chapters/07} and \textcitetv{chapters/06} for more details.


\ea\label{ex:intro:5}
\ea\label{ex:intro:5a}
\gll  ther     the \textit{mågho} \textit{äy} aff   gånga                \\
where   they   may.3\textsc{pl}   not   off   go\textsc{.inf}\\\jambox*{\isi{V-to-I} (EOS)}
\glt ‘from where they must not deviate’ (K-styr, 14\textsuperscript{th} c.)


\ex\label{ex:intro:5b}
\gll  huset       där     vi \textit{inte} \textit{ville} {bo}\\
        house.\textsc{def}     where   we     not   want.\textsc{pst}  live.\textsc{inf}\\\jambox*{\isi{V in situ}  (PDS)}
\glt        ‘the house where we didn’t want to live’


\ex\label{ex:intro:5c}
\gll  hon   sa   att   han   \textit{ville}       \textit{inte}   äta     den    \\
        she         say.\textsc{pst} that he   want.\textsc{pst}  not   eat.\textsc{inf}  it\\\jambox*{emb. \isi{V-to-C}  (PDS)}
\glt         ‘she said that he did not want to eat it’
\z
\z

According to \citet[176]{Falk1993}, the modern \isi{subordinate clause} word order of \REF{ex:intro:5b} becomes the dominant order in Swedish texts towards the end of EMS, reaching over 80\% with the generation of authors who were born during the last decades of the 16\textsuperscript{th} century. Then, in LMS, the proportion stabilizes above 90\%.



Starting with \citet{Kosmeijer1986}, many scholars have argued that the order in \REF{ex:intro:5a} (which is still the normal order in \ili{Icelandic}) is dependent on the presence of \isi{agreement} morphology on finite verbs (which \ili{Icelandic} has); this is usually labelled the \isi{Rich Agreement Hypothesis} (\isi{RAH}). As for \ili{Old Swedish}, the \isi{RAH} makes the correct prediction (\isi{V-to-I} should occur), since finite verbs agreed in both person and number with their subjects: for instance, a weak verb like \textit{läsa} ‘read’, had four forms in the present \isi{tense} in OS: \textit{läser} (\textsc{sg}), \textit{läsum} (1\textsc{pl}), \textit{läsin} (2\textsc{pl}), and \textit{läsa} (3\textsc{pl}).



In the dominant Swedish dialects (i.e. those surrounding Stockholm), the \isi{person distinction} seems to have been lost towards the end of the 15\textsuperscript{th} century \citep{Neuman1925}, and the number distinction during the 17\textsuperscript{th} century \citep{Larsson1988}. In other words, the rapid spread of the modern word order reported by \citet{Falk1993} coincides with the final loss of (number) \isi{agreement} in the spoken language of most Swedish writers. Consequently, Falk takes \isi{number agreement} to be a necessary prerequisite for \isi{V-to-I} \isi{movement}. In addition, and drawing on earlier work by \citet{Platzack1985} and \citet{PlatzackHolmberg1989}, she ties the loss of (number) \isi{agreement} to two other syntactic changes that took place towards the end of EMS, namely the loss of \isi{stylistic fronting} and the loss of verbal \isi{licensing} of null expletives.\footnote{\citet{AlexiadouFanselow2002} instead argue that \isi{stylistic fronting} of adverbials was reanalysed as an instance of the new word order. Such a \isi{reanalysis} was originally proposed by \citet{Pettersson1988}, whose paper (which is in Swedish) Alexiadou and Fanselow were clearly unaware of.} 



Others, most notably \citet{Rohrbacher1999} and \citet{KoenemanZeijlstra2014}, have maintained that the \isi{number agreement} of EMS would not have been enough to trigger \isi{V-to-I}. Based on what we know of other varieties that employ \isi{V-to-I}, the crucial threshold is expected to be the loss of \isi{person agreement}. If \isi{V-to-I} was lost with \isi{person agreement}, the old word order that still lived on until the 17\textsuperscript{th} century must have been derived by some other mechanism. \citet{KoenemanZeijlstra2014} suggest that an increased use of embedded \isi{V-to-C} (as in \ref{ex:intro:5c}) could have been one way of holding on to the old word order when it was no longer possible to generate it by moving V to I (cf. \citealt{HeycockEtAl2010} for a similar approach to archaic word order in \ili{Faroese}).



However, as stressed by \citet{Gartner2019}, in order for the \isi{V-to-C} analysis to be more than just an ad hoc solution to save the \isi{RAH}, it must be shown that embedded \isi{V-to-C} was a much more widespread phenomenon in EMS than it is today. After all, in PDS, embedded \isi{V-to-C} is possible only in a subset of all subordinate clauses, whereas \isi{V-to-I} came with no such restrictions. In fact, embedded \isi{V-to-C} was indeed possible in EMS in contexts where it is completely ungrammatical today (see \sectref{sec:intro:3.2.1} below for examples). This lends support to \citegen{KoenemanZeijlstra2014} proposal that the old word order lived on for quite some time in a new guise.



In this volume, Petzell argues that the 19\textsuperscript{th} century development of the southern \ili{Viskadalian} \isi{dialect} is a mirror image of the development of \ili{Standard Swedish}. In this \isi{dialect}, the original \isi{person agreement} is still intact. Nevertheless, the standard word order has been the dominant order since the 19\textsuperscript{th} century. At first, this appears to falsify the \isi{RAH}. However, what has happened (according to Petzell) is that the \isi{agreement} morphology has been reanalysed as part of \isi{tense}. Thus, elements of the old grammar are preserved, but within a new category.


\subsubsection{ The Swedish perfect}\label{sec:intro:3.1.2}


\ili{Present-day Swedish} has the rather exotic possibility of omitting finite forms of the temporal \isi{auxiliary} \textit{ha} ‘have’ in all types of non-V2 clauses (see e.g. \citealt{Julien2002}; \citealt{AndreassonEtAl2004}; \citealt{Backstrom2019}, and references therein); an 18\textsuperscript{th} century example is given in \REF{ex:intro:6}, where the position of the missing \isi{auxiliary} is marked by ∅. As far as is known, this option of \isi{auxiliary} omission is impossible in all of the other North \ili{Germanic} languages.


\ea\label{ex:intro:6}
\gll  Det     ser         nu   så rasande   förnämt   ut     i   Swerige […] sedan   det ∅ blifwit     ont     efter penningar\\
it     look.\textsc{prs.sg}   now   so terribly   pretentious out in   Sweden ~ since   it  ~ become\textsc{.ptcp}  pain    after money\\
\glt ‘It now looks so terribly pretentious in Sweden, since there has been a shortage of money’ (\textit{Argus}; from \citealt{Johannisson1945}: 172)
\z


There are a few scarce examples of \textit{ha}{}-omission from the 15\textsuperscript{th} century, but it did not become common until the 17\textsuperscript{th} century. It is frequent in texts from the beginning of the \ili{Late Modern Swedish} period, and in some texts (e.g. \textit{Argus}) a large majority of the relevant examples lack an overt finite \isi{auxiliary} (see \citealt{Johannisson1945}: 184; \citealt{Backstrom2019}: 87). In a recent study by \citet{Backstrom2019}, \textit{ha}{}-omission is viewed as a \isi{syntactic loan} from \ili{German} (cf. \citealt{Johannisson1945}, and see \citealt{Breitbarth2005} on older \ili{German}). \citet{Larsson2009} and \citet{Sangfelt2019} tie the possibility of \textit{ha}{}-omission to the loss of \isi{V-to-I} \isi{movement} (cf. the change in embedded word order above).



There were a couple of other changes in the temporal-aspectual system during the \ili{Late Modern Swedish} period which also resulted in differences between Swedish and the other North \ili{Germanic} languages. Firstly, Swedish developed a particular participial form only used with the \isi{auxiliary} \textit{ha} ‘have’ to form the \isi{perfect} \isi{tense}; this form is often referred to as the \textit{supine} (Sw. \textit{supinum}). In present-day Swedish, strong verbs have \isi{supine} forms that are \isi{morphologically distinct} from the \isi{neuter} singular of past participles; compare the \isi{passive} in \REF{ex:intro:7a} with the \isi{perfect} in \REF{ex:intro:7b}. The \isi{supine} form was gradually established in the 18\textsuperscript{th} century, but it was not fully in place in the written language until the 19\textsuperscript{th} century (see \citealt{Platzack1981}; \citealt{Larsson2009}: 422–423; \citealt{Backstrom2019}). Not all present-day dialects make the distinction between \isi{supine} and \isi{past participle} morphology.


\ea\label{ex:intro:7}
\ea\label{ex:intro:7a}
\gll  det    var \textit{skrivet} \\
it   was write.\textsc{ptcp.n.sg}\\
\glt    ‘it was written’ (ÄSv, 20\textsuperscript{th} century)


\ex\label{ex:intro:7b}
\gll  fastt       du   inte   har \textit{skrivit}  på   så   länge\\
      although   you   not   have   write.\textsc{sup}   for   so long\\
\glt    ‘although you haven’t written for so long’ (SPF, 19\textsuperscript{th} century).
\z
\z


In the 17th century, the possibility of having \textit{vara} ‘be’\,+\,a \isi{participle} of an \isi{unaccusative} verb became more restricted (see \citealt{Johannisson1945}; \citealt{Larsson2009,Larsson2015}; cf. \citealt{McFaddenAlexiadou2005} for a similar development in \ili{English}). This possibility still remains in \ili{Norwegian} and \ili{Icelandic}, and in \ili{Danish} the construction has grammaticalized into a \textit{be}{}-\isi{perfect} (see \citealt{Larsson2021Emergence}). \textit{Vara} ‘be’, was still (marginally) grammatical with some groups of unaccusatives at the beginning of the 18\textsuperscript{th} century; examples from the 16\textsuperscript{th} and 17\textsuperscript{th} centuries are given in \REF{ex:intro:8}. In present-day Swedish, \textit{ha} ‘have’ is the only option with active participles of all types of verbs (although subject to some dialectal variation, see e.g. \citealt{Larsson2014HAVE}, and \sectref{sec:intro:3.4} below). Both \textit{ha} and \textit{vara} were possible with unaccusatives in older Swedish; the examples in \REF{ex:intro:9} have \textit{ha}.

\ea\label{ex:intro:8}
\ea\label{ex:intro:8a}
\gll  Så seer   man   här \textit{tilgått} \textit{wara} \\
so   sees   one     here   about.go\textsc{.ptcp}     be\\
\glt ‘one sees that things have happened in this way here’ (16\textsuperscript{th} century; from  \citealt{Larsson2009}: 156)


\ex\label{ex:intro:8b}
\gll  Jost Cursel […] och andra   lifländare mera,   som \textit{vore} aff\\
      Jost Cursel  {}   and other   Livonians more   who   were   of \\

\gll  godh   villia medh \textit{redne}\\
      good   will   with   ride.\textsc{ptcp}\\
    \glt `Jost Cursel […] and several other Livonians who had ridden along out of free will’ (17\textsuperscript{th} century; from \citealt{Larsson2009}: 262)
\z
\ex\label{ex:intro:9}
\ea\label{ex:intro:9a}
\gll  Kan   man   wel   merkia huru thå \textit{haffuer} \textit{tilgått} \\
can       one   well   notice   how   then     has         about.go\textsc{.ptcp}\\
\glt      ‘One can well notice how things then have happened’ (16\textsuperscript{th} century; from  \citealt{Larsson2009}: 156)

\ex\label{ex:intro:9b}
\gll \textit{haffver} herr   Nils   Bielke […] \textit{rididt} till   herttigen\\
      has              sir   Nils   Bielke   {}  ride.\textsc{ptcp}   to    duke.\textsc{def}\\
    \glt `has Sir Nils Bielke ridden to the duke’ (17\textsuperscript{th} century; from \citealt{Larsson2009}: 263)
\z
\z

\subsubsection{ Infinitivals}\label{sec:intro:3.1.3}


In PDS, the \isi{infinitive marker} \textit{att} is more or less restricted to \isi{control infinitives}; it does not occur in \isi{ECM} contexts and only with a limited set of \isi{raising} verbs (\citealt{TelemanEtAl1999}/3: 572, 597).\footnote{For instance, with \textit{verka} ‘seem’, \textit{förefalla} ‘appear’, as well as with passives like \textit{ses} ‘see.\textsc{pass}’ and \textit{förmodas} ‘presume.\textsc{pass}’, there is never an \textit{att} involved; see (i). However, with \textit{börja} ‘start’ (see ii), \textit{att} is optional, and with \textit{se ut} ‘look like’, \textit{att} is even mandatory (iii).

\ea \gll Hon   verkade/föreföll/sågs/förmodades     (*att)   springa   i    den   riktningen\\
        she   seemed/appeared/see\textsc{.pst.pass}/presume\textsc{.pst.pass} to     run     in   that   direction\textsc{.def}\\

\ex \gll Det   har börjat (att)   sitta   fåglar   på staketet     där\\
         it   has begun to   sit     birds   on fence\textsc{.def}   there\\
         
\ex \gll Det   ser     ut   *(att)   regna   vid   horisonten\\
    it   sees     out to     rain     by   horizon.\textsc{def}\\
\z} 
This restriction appears to hold for earlier stages as well, although there are sporadic examples from OS and EMS where \textit{att} occurs in the \isi{complement} of a modal verb (\citealt{Lagervall2014}: 149--157; \citealt{Kalm2016Satsekvivalenta}: 133–134), as well as in \isi{ECM} constructions (\citealt{Kalm2016Satsekvivalenta}: 136–137). However, compared to the other North \ili{Germanic} languages, the implicature \textit{att}\,+\,infinitive → control infinitive seems fairly robust in Swedish. By contrast, in \ili{Icelandic}, some modals are obligatorily constructed with \textit{að} (the \ili{Icelandic} equivalent to \textit{att}), and in \ili{Norwegian}, both \isi{raising} constructions and \isi{ECM} constructions involve \textit{å} (the \ili{Norwegian} equivalent); see \citet[45]{Kalm2016Satsekvivalenta} and Faarlund (2019: 248–251) for more comparative details.



In OS, \textit{att} (often spelled \textit{at}) formed a tight unit with the infinitive, presumably cliticizing to the left of the verb \citep{Falk2010Studier}. This unit could be preceded as well as followed by other elements, as shown in \REF{ex:intro:10}. Here, the object (\textit{gest} ‘guest’) precedes the infinitival complex (\textit{at=husla}), and the comparative \isi{adverbial} (\textit{sum bondæ} ‘like a farmer’) follows it.


\ea \label{ex:intro:10}
\gll præster   ær   skyldugher  gest \textit{at} \textit{husla} sum  bondæ \\
  priest  is  required    guest   to  give.communion  as  farmer  \\
\glt ‘it is the duty of the priest to give the communion to a guest as he does to a local farmer’ (\isi{EWL}; from \citealt{Falk2010Språkhistoria}: 33)
\z


In early EMS, it became increasingly common for elements to intervene between \textit{att} and the verb, as in the example in \REF{ex:intro:11} below, where there is a PP between \textit{att} and the infinitive. Such interventions can be seen as an indication that \textit{att} had been reanalysed as heading a non-finite clause rather than an infinitival phrase. This development from \isi{infinitive marker} to non-finite \isi{complementizer} appears to affect the realization of \textit{att}: in OS, \textit{att} was optional in many contexts where it is mandatory today. The stricter demand for an overt \textit{att} started manifesting itself in texts at around the same time as the \isi{reanalysis} from \isi{proclitic} to \isi{complementizer} would have taken place \citep[35]{Falk2010Studier}.


\ea \label{ex:intro:11}
\gll  lustigt \textit{att} om     sommersz     tydh \textit{spaszera}. \\
  amusing  to   in   summer.\textsc{poss}  time  stroll\\
\glt ‘amusing to stroll in the summer time’ (17\textsuperscript{th} century; from \citealt{Kalm2016Satsekvivalenta}: 144)
\z


However, the categorical status of \textit{att} is hardly the only factor determining whether it can be omitted or not. Neither in \ili{Danish}, where \textit{att} is still a \isi{proclitic}, nor in \ili{Norwegian}, where \textit{att} can be either a \isi{proclitic} or a \isi{complementizer}, is \textit{att} optional. Compared to these two languages, PDS is very liberal when it comes to \textit{att}{}-omission (see \citealt{Faarlund2019}: 248–251 for details). For the better part of the EMS period, \isi{proclitic} and \isi{complementizer} \textit{att} co-existed (just as they still do in \ili{Norwegian}). According to Kalm, the \isi{proclitic} \textit{att} (as in \ref{ex:intro:10}) became obsolete towards the end of EMS \citep[145]{Kalm2016Satsekvivalenta}.



A recent development in the history of Swedish \isi{control infinitives} regards the possibility of embedding the infinitival structure under a preposition. Such embedding did occur already in OS under the \isi{directional preposition} \textit{till}, at least partly in order to reinforce a \isi{purposive} reading (\citealt{Kalm2016Satsekvivalenta}: 204–208); eventually, \textit{till} developed into an alternative \isi{infinitive marker} \citep[210]{Kalm2016Satsekvivalenta}. However, it was not until LMS that \isi{control infinitives} started combining with different kinds of prepositions, thus conveying a wide variety of \isi{adverbial} meanings. The emergence of these \isi{adverbial} infinitives in LMS is the topic of Kalm’s paper in this volume.\footnote{There were also other changes in infinitival constructions in the \ili{Modern Swedish} period. In particular, \isi{ECM} constructions appear to have had a wider distribution in older Swedish than they do in the present day. However, this remains to be investigated further.}


\subsection{ Argument placement}\label{sec:intro:3.2}


In this section, we look in turn at the placement of subjects (\sectref{sec:intro:3.2.1}) and objects (\sectref{sec:intro:3.2.2}) in Early and \ili{Late Modern Swedish}.


\subsubsection{Subject placement}\label{sec:intro:3.2.1}


As argued by \textcite{Hakansson2008}, spec-\isi{IP} has been a dedicated \isi{subject position} since \ili{Late Old Swedish}, although it is not until the end of \ili{Early Modern Swedish} that spec-\isi{IP} is obligatorily filled (\citealt{Falk1993}, \citeyear{chapters/02} [this volume]). However, EMS subjects could still surface in a position where we do not find them anymore, viz. after the finite verb, resulting in surface VS order in certain embedded contexts; see \REF{ex:intro:12} below. In \REF{ex:intro:12a}, the VS string appears in the second conjunct of a \textit{that}{}-clause, and in \REF{ex:intro:12b}, it follows a relative pronoun. Both these uses of VS are ungrammatical today. The difference is presumably linked to a more liberal use of embedded \isi{V-to-C} in EMS (see \sectref{sec:intro:3.1.1} above). The present-day system was established during the beginning of the Late Modern period \citep{Petzell2013}.


\ea\label{ex:intro:12}
\ea\label{ex:intro:12a}
\gll  [han] sade at     hon   nu   har     någ    råt     om   migh och \\
he   said   that she   now has   enough   care.\textsc{ptcp}   about   me     and\\

\gll  \textit{skule}   \textit{hon} nu   inte längre inbila       sig   något herewäle  öfwer   mig.\\
    should   she   now   not longer   imagine.\textsc{inf}   \textsc{refl}   any   dominance   over   me \\
\glt `he said that she has now cared for me enough and she shouldn’t imagine that she could dominate me any longer.’ (Horn, b. 1629)

\ex\label{ex:intro:12b}
\gll  Hwilket \textit{skall}   \textit{Mahomet} \textit{2:[secun]dus} hafwa   giort \\
      which   shall   M         second       have       done\\
    \glt ‘which Mahomet the second is supposed to have done’ (Rålamb, b. 1622, p. 125)
\z
\z

We move on now to the ordering of subjects and \isi{sentence adverbials}. Here, the order in earlier \ili{Late Modern Swedish} varied, much as it does in the present-day language: non-initial subjects could either precede or follow a sentence \isi{adverbial}; the order subject–\isi{adverbial} is often referred to as involving \textit{subject shift} (see e.g. \citealt{Holmberg1993}; \citealt{Svenonius2002}; \citealt{Andreasson2007}). Weak pronominal subjects almost always shifted across the \isi{adverbial}:


\ea\label{ex:intro:13}
\gll  Herr Baron,   kiänner \textit{I} \textit{intet} Lars Lustig? \\
Sir      Baron  know    you not     Lars Lustig\\
\glt ‘Baron, don’t you know Lars Lustig?’ (Gyllenborg, b. 1679)
\z


However, on occasion, weak pronouns could follow negation; \REF{ex:intro:14a} has a non-referential \textit{det} after negation, and \REF{ex:intro:14b} has a non-shifted generic \isi{pronominal subject}. Examples like these are admittedly rare in the historical texts, and they hardly occur in present-day Swedish (but see \citealt{Bentzen2014} on \isi{dialect variation} in present-day North \ili{Germanic}).


\ea\label{ex:intro:14}
\ea\label{ex:intro:14a}
\gll  så börjar   Frökne-namnet     låta       så flatt så flatt,   at   I     gifwa   Er ingen ro,     för     än   I       fått       byta     bårt det samma, är \textit{icke} \textit{det} så\\
  so starts   maiden-name.\textsc{def}    sound   so flat   so flat   that you   give   \textsc{refl} no       peace   before than you   get\textsc{.sup} change \textsc{part}   it same is   not     it     so\\
\glt ‘so the maiden-name starts to sound so flat, so flat, that you give yourself no peace until you have exchanged it, isn’t it so’ (Gyllenborg, b. 1679)


\ex\label{ex:intro:14b}
\gll Fredric: En   sjö   utan     fiskar å     en skog     utan     foglar, maschär   mär ja   vill   inte  gå   på Opran.\\
    Fredric:  a     lake   without   fish   and   a   forest   without   birds ma.chère   mère I   want not  go   to Opera\textsc{.def}\\

\gll Clas: jo,   ja   vill   si     dä   där   regne   som \textit{inte} \textit{man} blir   våt   åf. \\
    Clas: yes   I   want   see   that   there   rain     that   not   one   is     wet   by\\

\glt `Fredric: A lake without fish and a forest without birds, ma chère mère, I don’t want to go to the Opera. Clas: But I want to see the rain that you don’t get wet from.’ (Ristell, b. 1744)
\z
\z

Non-pronominal subjects, on the other hand, could either precede or follow the \isi{adverbial}:


\ea
\ea
\gll  Men   min Gu-Far,       har \textit{intet} \textit{Fru} \textit{Lotta} orsak   at   wara swartsiuk \\
but   my   god-father   has   not   Madam Lotta reason to be   jealous\\

\glt ‘But my godfather, doesn’t Madam Lotta have reason to be jealous?’ (Gyllenborg, b. 1679)

\ex
\gll  så   är \textit{Juncker} \textit{Torbiörn} \textit{intet} mas \\
    so is    nobleman    Torbjörn  not    miser\\
    \glt ‘Noble Torbjörn is not a miser’ (Gyllenborg, b. 1679)
\z
\z


The frequency of \isi{subject shift} with non-pronominal subjects varies somewhat between texts, and it fluctuates over time \citep{LarssonLundquist2021}. However, in the present-day language, there is still variation between speakers and texts (see e.g. the data in the \isi{Nordic Word Order Database}; \citealt{LundquistEtAl2019}; cf. \citealt{Andreasson2007}).\footnote{The \isi{Nordic Word Order Database} is available here: \url{https://tekstlab.uio.no/nwd}} 


\subsubsection{Object placement}\label{sec:intro:3.2.2}


As shown by \citet{Delsing1999}, the change from OV to \isi{VO word order} had already started during EOS. However, we still find quite a lot of OV examples well into the 18\textsuperscript{th} century (\citealt{Petzell2011}; \citealt{Sangfelt2019}). In fact, even the very symbol of modernity, \textit{Argus}, sometimes displays \isi{OV order}. The text in \REF{ex:intro:3} above continues like this:


\ea
\gll Men  fast      än   hwarken de gamla \textit{sådane} \textit{Läro-sätt} \textit{skulle} \textit{älskat} eller   nyare     frägdade   Folckeslag \textit{dem} \textit{älska}\\
but     although   than   neither   the ancient  such       lessons    would  love\textsc{.ptcp} or  newer  esteemed people.\textsc{pl}  them love\\
\glt ‘But although neither the ancient nor newer esteemed peoples would have loved such lessons’ (\textit{Argus})
\z

These rather late examples of \isi{OV order} do not indicate an underlying OV structure in the \isi{VP} (as has been suggested for EOS; see \citealt{Delsing1999}: 189, 211–214). Instead, the position of the object to the left of the entire verbal complex in a \isi{subordinate clause} suggests that it has moved out of the V-domain into the \isi{I-domain} \citep{Petzell2011}. 



Whereas the possibility of creating OV by moving O over a finite verb in situ (in V) disappeared during the 18\textsuperscript{th} century, weak pronominal objects and reflexives can still shift across a sentence \isi{adverbial} today when the main verb is in C; this is commonly referred to as \textit{object shift}.\footnote{Non-pronominal
    \isi{object shift} across negation is not possible in present-day Swedish, and it does not seem to have occurred in older Swedish either, with a small number of exceptions in \ili{Old Swedish} (Falk p.c.). One rare example is given in (i).

    \ea
    \gll  For thy at   the \textit{thz} \textit{första} \textit{bodhordh} \textit{ekke} hioldo\\
        for that that   they   the first   commandment   not   kept\\
        \glt ‘because they didn’t keep the first commandment’ (15\textsuperscript{th} c.; from Falk p.c.)
    \z
    }
The EMS word orders in \REF{ex:intro:17} below are, in other words, acceptable even in present-day Swedish (see e.g. \citealt{Holmberg1986}; \citealt{Andreasson2008}; \citealt{Bentzen2014}; \citealt{Erteschik-ShirJosefsson2017}).


\ea\label{ex:intro:17}
\ea\label{ex:intro:17a}
\gll  Ammiral, Jag   kiände \textit{Er} \textit{intet}.\\
    Admiral   I       knew   you       not\\
    \glt ‘Admiral, I didn’t recognize you.’ (Gyllenborg, b. 1679)


\ex\label{ex:intro:17b}
\gll  jag tror \textit{dig} \textit{intet},   för     än   jag   får smakat \\
    I     believe you     not     before   than   I     get taste.it\\
    \glt ‘I don’t believe you until I get to taste it’ (Modée, b. 1698)
\z
\z


Contrasted pronominal objects and objects with a non-nominal (or type) antecedent do not shift (see \citealt{Andreasson2008}):


\ea\label{ex:intro:18}
\ea\label{ex:intro:18a}
\gll  Utan     vidare ceremonier,   herr   öfverste,   gif   mig   åter min fostersyster. Hon   tillhör \textit{inte} \textit{er}.\\
without   further ceremonies   sir   colonel     give me   back my foster.sister she   belongs.to   not   you\\
\glt ‘Without further ceremonies, Colonel, give me back my foster sister. She does not belong to you.’ (Jolin, b. 1818)


\ex\label{ex:intro:18b}
\gll  Nej, maschär   mär,   ja vill \textit{inte} \textit{dä}\\
      no    ma.chère   mère  I  want  not   it\\
 \glt       ‘No, ma chère   mère, I don’t want that’ (Ristell, b. 1744)
\z
\z


As in present-day Swedish, \isi{object shift} in LMS is not completely obligatory even with weak pronouns (see e.g. \citealt{Erteschik-ShirJosefsson2017}): 

\ea
\gll  MAGISTERN: dä     sant     att  Baron  Fredric  ha  slaje    gåssen. \\
  teacher\textsc{.def}  it  true   that  Baron  Fredric  has  beaten  boy.\textsc{def}\\

\gll  FREDRIC: Ja, men   hvar    före   lydde    han \textit{inte} \textit{mäj}, när      ja   befalte.\\
 Fredric    yes but   where  fore    obeyed   he    not  me   when    I   commanded\\
 
\glt ‘The teacher: is it true that Baron Fredric has beaten the boy? Fredric: Yes, but why didn’t he obey me when I commanded?’ (Ristell, b. 1744)
\z


In this volume, \isi{object shift} is discussed in the paper by Larsson \& Lundquist. They show that although there is perhaps some variation both within and across texts, there is no change in the distribution of \isi{object shift} across negation, with the exception of \isi{object shift} in \isi{particle} constructions: objects could shift across verb particles in older Swedish (as they do in the other North \ili{Germanic} languages), but this is no longer a possibility.  



Unlike the other North \ili{Germanic} languages, present-day Swedish has the possibility of shifting a weak object pronoun or \isi{reflexive} across a non-\isi{pronominal subject} to a position immediately after the finite verb; this is often referred to as \textit{long object shift} (see e.g. \citealt{Holmberg1986}; \citealt{Heinat2010}). Long \isi{object shift} mostly occurs with reflexives \REF{ex:intro:20a}, but at least some speakers allow long \isi{object shift} with pronouns if they have a distinct object form. For instance, the first person singular pronoun \textit{mig} ‘me’ can shift across the subject \REF{ex:intro:20b}, but the third person \textit{dom} ‘they, them’ cannot, since it does not have a distinct object form; consequently, in \REF{ex:intro:20c}, \textit{dom} can only be interpreted as the subject. Long \isi{object shift} is impossible across a \isi{pronominal subject}; cf. \REF{ex:intro:20d}.


\ea \label{ex:intro:20}
\ea[]{\label{ex:intro:20a}
\gll  I     morse   rakade \textit{sig} \textit{Kalle}. (PDS)\\
in    morning  shaved   \textsc{refl}    Kalle\\
\glt `This morning, Kalle shaved.’}

\ex[]{\label{ex:intro:20b}
\gll  Idag     erbjöd \textit{mig} \textit{Lisa} en glass.\\
    today    offered   me   Lisa  an ice.cream\\
\glt `Today, Lisa offered me an ice cream.’}

\ex[]{\label{ex:intro:20c}
\gll  Idag   erbjöd   \textit{dom} \textit{Lisa}   en   glass.\\
    today offered   they/them   Lisa   an   ice.cream  \\
\glt `Today, they offered Lisa an ice cream.’\\NOT: ‘Today Lisa offered them an ice cream.’}

\ex[*]{\label{ex:intro:20d}
\gll I morse   rakade mig   jag.\\
      in morning shaved me     I\\}

\z
\z


Long \isi{object shift} is attested with both reflexives and pronouns throughout the \ili{Modern Swedish} period (and it occurs also in \ili{Old Swedish}; Falk p.c.). Examples from \textit{Argus} are given in \REF{ex:intro:21}.


\ea \label{ex:intro:21}
\ea
\gll  på rätt   grundar \textit{sig} \textit{ett} \textit{folks} \textit{Sällhet} \\
    on right founds    \textsc{refl}   a   people\textsc{.poss}   bliss\\
    \glt `A people’s bliss is founded on righteousness’ (\textit{Argus})

\ex
\gll  Anledningen     gaf \textit{mig} \textit{Herr} \textit{Ehrenmenvet}  \\
       possibility.\textsc{def}  gave   me   Mr  Ehrenmenvet\\
    \glt `Mr Ehrenmenvet gave me the possibility’ (\textit{Argus})
\z
\z

\subsection{Double objects and passives}\label{sec:intro:3.3}


In this section, we first look briefly at double object and \isi{benefactive} constructions (in \sectref{sec:intro:3.3.1}), and then, in \sectref{sec:intro:3.3.2}, we turn to passives.


\subsubsection{Double objects}\label{sec:intro:3.3.1}


Swedish has the well-known alternation between a construction with two objects on the one hand, and a double \isi{complement} construction (with object\,+\,PP-\isi{adverbial}) on the other. In the present-day language, few verbs require double objects, and many alternate, depending e.g. on whether the recipient/\isi{benefactive} argument is pronominal or not. The same type of alternation can be observed throughout the history of Swedish; examples of the verb \textit{giva} ‘give’ from the 18\textsuperscript{th} and 19\textsuperscript{th} centuries (taken from \citealt{Valdeson2016}) are given in \REF{ex:intro:22} below.


\ea \label{ex:intro:22}
\ea \label{ex:intro:22a}
\gll  Hon har gifvit \textit{mig} \textit{den} \textit{aftalta} \textit{vinken} \\      
     she   has given   me   the   agreed.upon  wave.\textsc{def}\\
     \glt ‘She has waved at me as we agreed.’ (19\textsuperscript{th} century; from \citealt{Valdeson2016}: 280) 

\ex\label{ex:intro:22b}
\gll  Och hwilken Fader som gifwer \textit{sin} \textit{Dotter} \textit{til} \textit{en} \textit{Man} \textit{som} \textit{hon} \textit{icke} \textit{kan} \textit{tåla}\\
      and   what    father   who gives     \textsc{poss.refl}  daughter     to   a   man that she   not   can  stand\\
\glt  `And what a father, who gives his daughter to a man that she can’t stand’ (18\textsuperscript{th} century; from \citealt{Valdeson2016}: 284)


\ex \label{ex:intro:22c}
\gll  om   vi     skulle   ge \textit{rum} \textit{åt} \textit{vår} \textit{vän}\\
      if     we   would   give   room for   our   friend\\
    \glt `if we would give room to our friend’ (18\textsuperscript{th} century; from \citealt{Valdeson2016}: 280)
\z
\z

In \REF{ex:intro:22a}, \textit{giva} takes a pronominal \isi{indirect object} and a non-pronominal \isi{direct object}. In \REF{ex:intro:22b}, there is a non-\isi{pronominal object} and a PP introduced by the preposition \textit{til} ‘to’. In \REF{ex:intro:22c}, there is also a non-\isi{pronominal object} and a PP, but here the preposition is (the less common) \textit{åt} ‘to, toward’.



Since EOS, the alternative with a PP has gradually gained ground. In EMS and LMS, the choice between the different constructions depends on lexical and information structural factors, as in the present-day language. However, as is clear from \textcitetv{chapters/03}, the use of the construction with two objects changed in the \ili{Late Modern Swedish} period, with the \isi{double object construction} becoming both less frequent and lexically more restricted. In present-day Swedish, the construction with a \isi{direct object}\,+\,PP is often preferred. 



Swedish still has the possibility of so-called free benefactives, as in \REF{ex:intro:23}. In the present-day language, free benefactives are rather restricted, and they are not always possible even with verbs of production, bringing, or ballistic motion (see \citealt{Lundquist2014Double} and references therein).


\ea\label{ex:intro:23}
\gll  Jag   stickade  henne   en   tröja. \\
I         knitted   her     a     sweater\\
\glt ‘I knitted her a sweater.’
\z


The group of verbs that could take two nominal objects has gradually grown smaller in the \ili{Late Modern Swedish} period (see \citealt{Valdeson2017}). For instance, fewer \isi{verbs of communication} (e.g. \textit{berätta} ‘tell’; see \citealt{Silen2005}) can now occur with double objects, and \isi{verbs of hindrance} (e.g. \textit{hindra} ‘hinder’) no longer do; compare the LMS examples in \REF{ex:intro:24} with the present-day Swedish ones in \REF{ex:intro:25}.


\ea\label{ex:intro:24}
\ea[]{
\gll  berätta henne    det samma \\
tell  her  the same \\
\glt ‘tell her the same’ (Gyllenborg, b. 1679)}

\ex[]{
\gll  Republiquen ville     hindra   honom   det\\
      republic\textsc{.def} wanted   hinder   him     that\\
    \glt `The Republic wanted to hinder him from that.’ (18\textsuperscript{th} century; from \isi{SAOB})}
\z
\ex \label{ex:intro:25}
\ea[*]{
\gll  berätta   henne   något         –   berätta   något       för henne   (PDS)\\
tell     her  something    {}  tell      something  for her\\}

\ex[*]{
\gll hindra honom  det   –   hindra   honom   från   det\\
 hinder him    that  {}  hinder   him     from   that\\}
\z
\z

Moreover, there have also been changes in the word order possibilities in double object constructions. In present-day Swedish, the order of the indirect and the \isi{direct object} is invariable, with few exceptions: the \isi{indirect object} always precedes the \isi{direct object} in the verb phrase. In OS and EMS, the opposite order was also possible, as in the example in \REF{ex:intro:26}. This possibility largely disappeared around the end of EMS; \citet{Valdeson2016} finds no examples in texts from the 18\textsuperscript{th} century onwards. However, with a small number of verbs, there is still some variability, as with \textit{tillägna} ‘dedicate’ in \REF{ex:intro:27}.


\ea\label{ex:intro:26}
\gll  ok   gaf   gul             ok   self           fatøco     folke. \\
  and gave   gold\textsc{.n.sg.acc}   and  silver\textsc{.n.sg.acc}   poor\textsc{.dat}   people\textsc{.n.sg.dat}\\
    \glt `and gave gold and silver to poor people’ (EOS; from \citealt{Valdeson2016}: 280)
\ex\label{ex:intro:27}
\ea
\gll  Stevie Wonder   tillägnade konserten     sin         hustru. (PDS)\\
Stevie Wonder  dedicated  concert.\textsc{def}     \textsc{poss.refl}  wife  \\

\ex
\gll  Stevie Wonder   tillägnade   sin       hustru   konserten.\\
        Stevie Wonder  dedicated  \textsc{poss.refl}  wife     concert\textsc{.def}  \\
\glt        ‘Stevie Wonder dedicated the concert to his wife.’ \citep[137]{Lundquist2014Double}
\z
\z

In present-day Swedish, either of the objects can be \isi{promoted to subject} in passives, but the \isi{indirect object} is most often chosen (see \citealt{Lundquist2004}). Examples are given in \REF{ex:intro:28}. In \REF{ex:intro:28a}, the \isi{indirect object} has been \isi{promoted to subject} in the \isi{passive}, whereas in \REF{ex:intro:28b} the \isi{direct object} has been \isi{promoted to subject}.


\ea\label{ex:intro:28}
\ea\label{ex:intro:28a}
\gll  Hon   erbjöds         ett jobb. (PDS)\\
she   offer\textsc{.pst.pass}   a    job\\
\glt ‘She was offered a job.’

\ex\label{ex:intro:28b}
\gll  Jobbet   erbjöds         henne. \\
  job.\textsc{def}     offer.\textsc{pst.pass} her\\
    \glt `The job was offered to her.’
\z
\z



In older Swedish, only the \isi{direct object} could be \isi{promoted to subject}. The possibility of passivizing the \isi{indirect object} (as in \ref{ex:intro:28a}) arose in the \ili{Late Modern Swedish} period. This development is the topic of the paper by Falk in this volume.


\subsubsection{Passives}\label{sec:intro:3.3.2}


Swedish has two ways of forming passives, both of which already existed in \ili{Old Swedish}. Firstly, there is a \isi{periphrastic passive} with \textit{vara} ‘be’ (this gives a stative \isi{passive}) or \textit{varda/bli} ‘become’ (yielding an eventive \isi{passive})\,+\,a \isi{passive} \isi{participle}. The eventive \isi{passive} \isi{auxiliary} was \textit{varda} until around the \ili{Early Modern Swedish} period, when \textit{bli} (a \isi{loan} from \ili{Low German}) gradually took over. From the second half of the 18\textsuperscript{th} century, \textit{bli} was the rule in the \isi{standard language}, but some dialects still use \textit{varda} in the past \isi{tense} (i.e. \textit{vart}). In addition, Swedish has a morphological \isi{passive} formed with the verbal suffix -\textit{s}. The example in \REF{ex:intro:29} includes both a \isi{periphrastic passive} and a morphological \isi{passive}.


\ea\label{ex:intro:29}
\gll  Men  ehwad   flit         och   möda     här     wid \textit{fordras}, och ehwad öde   Wårt   Arbete nu \textit{blir} \textit{underkastat}\\
but   what   diligence and   hardship   here   by    require\textsc{.prs.pass} and what  destiny our     work   now   is     subjugate\textsc{.ptcp}\\
\glt ‘but what diligence and hardship is hereby required, and what destiny our work is now subject to’ (\textit{Argus})
\z

\ili{Norwegian} and \ili{Danish} have both periphrastic and morphological passives too, but the distribution varies between the languages (see e.g. \citealt{Engdahl2006}; \citealt{Laanemets2012}; \citealt{Faarlund2019}). In short, the \textit{s}{}-\isi{passive} has a wider range of uses in present-day Swedish than in the other languages. To some extent, this also holds for older stages. In \ili{Norwegian} and \ili{Danish}, the \textit{s}{}-\isi{passive} is for instance generally not possible in the past \isi{tense}. In Swedish, preterite forms with \isi{passive} morphology can be found early on, as shown in \REF{ex:intro:30}.


\ea\label{ex:intro:30}
\gll j     samu   stund \textit{førþes} døþ     vt     af staþenom\\
in   same     moment bring\textsc{.pst.pass}   dead     out   of town.\textsc{def.m.sg.dat}\\
\glt ‘in the same moment [a woman] was brought dead out of the town’ (Leg, EOS, p. 151)
\z


In present-day Swedish, the morphological \isi{passive} can be used in all tenses, including in the \isi{perfect}. This possibility first emerged in \ili{Early Modern Swedish} (see e.g. \citealt{Holm1952}; \citealt{Platzack1989}; \citealt{Larsson2009}: 412); one example is given in \REF{ex:intro:31}. In the 18\textsuperscript{th} and 19\textsuperscript{th} centuries, \isi{normative grammarians} still disapproved of this use of the \textit{s}{}-\isi{passive} (see \citealt{Platzack1989}).


\ea\label{ex:intro:31}
\gll när     waran       har \textit{fördts} in, så ha   namne      kom[m]it in mä.\\
  when product\textsc{.def}  has bring.\textsc{ptcp.pass}     in  so has name\textsc{.def} come\textsc{.ptcp} in too\\
\glt ‘when the product has been brought in, the name has come too’ (17\textsuperscript{th} century; from \citealt{Larsson2009}: 412)
\z


With respect to the \isi{periphrastic passive}, it appears to have been less restricted in older Swedish than in the present-day language. The examples in \REF{ex:intro:32} below (from Falk p.c.) show that the 17\textsuperscript{th} century edition of \textit{Nils Mattsson Kiöpings resa}, ‘The journey of N. M. K.’, has periphrastic passives (see \ref{ex:intro:32a}), whereas the 18\textsuperscript{th} century edition of the text, revised by the influential publisher Lars Salvius (see \sectref{sec:intro:2.1} above), instead has the \textit{s}{}-\isi{passive} (see \ref{ex:intro:32b}). In the present-day language, an \textit{s}{}-\isi{passive} would indeed be used in this context; the \textit{s}{}-\isi{passive} tends to be the unmarked choice (see e.g. \citealt{Engdahl2006}).


\ea\label{ex:intro:32}
\ea \label{ex:intro:32a}
\gll  Then   yterste     Barcken   är grå, \textit{blifwer}     \textit{aff-skurin} och \textit{bort-kastat}: Then innerste     är askefärgader, \textit{blifwer} uthi   fyrkantige   stycken \textit{skuren},   och   sädan \textit{sammanrullader}\\
the   outermost bark\textsc{.def}  is  grey becomes  off-cut\textsc{.ptcp}  and   away\textsc{{}-}throw.\textsc{ptcp} the     innermost   is ash.coloured   becomes  in    square     pieces cut\textsc{.ptcp} and   then     together.roll\textsc{.ptcp}\\
\glt  ‘The outermost bark is grey, is cut off and thrown away: the innermost is ash-, is cut in square pieces and then rolled together’ (Kiöping, b. 1621)

\ex \label{ex:intro:32b}
\gll  [barken] \textit{rensas} först bort och \textit{kastas} sin       kos. Den inre … \textit{skäres} i   fyrkantige   stycken,   hvilka sedan \textit{rullas} tilhopa\\
    bark.\textsc{def} clear.\textsc{prs.pass}   first away and throw.\textsc{prs.pass}   \textsc{poss.refl}   way the   inner {} cut.\textsc{prs.pass} in   square     pieces   which then     roll\textsc{.prs.pass} together\\

\glt `The bark is first cleared and thrown away. The inner … is cut in square pieces, which are then rolled together’ (Salvius, b. 1706).
\z
\z

The distribution of the different passives in older Swedish has not been thoroughly investigated (but see \citealt{Kirri1975}). For an extensive study of the \textit{s}{}-\isi{passive} in \ili{Old Swedish} and the Swedish dialects, see \citet{Holm1952}.


\subsection{Nominal morphology and the noun phrase}\label{sec:intro:3.4}


\ili{Present-day Swedish} has both definite and \isi{indefinite} articles, and generally requires so-called \isi{double definiteness} marking in modified noun phrases (see \citealt{Julien2005}). This system is fully in place in \ili{Late Modern Swedish}; see the examples from \textit{Argus} given in \REF{ex:intro:33}. Determiners are prenominal, and this also includes possessives, except occasionally with some kinship terms; compare \REF{ex:intro:34a} and \REF{ex:intro:34b}.


\ea \label{ex:intro:33}
\ea \label{ex:intro:33a}
\gll  den högmodiga     efter-tankan \\
the  haughty   after-thought\textsc{.def}\\
\glt ‘the haughty after-thought’ (\textit{Argus})

\ex \label{ex:intro:33b}
\gll en så     oskyldig   afsikt\\
    a   such  innocent   intention\\
\glt `such an innocent intention’ (\textit{Argus})
\z
\ex \label{ex:intro:34}
\ea \label{ex:intro:34a}
\gll  min Läsare \\
my   reader \\
\glt        ‘my reader’ (\textit{Argus})

\ex \label{ex:intro:34b}
\gll Har Swåger       min   ingen Pinne-Skog   til sitt       Bruk\\
 has  brother\_in\_law  my  no    stick-forest    to  \textsc{poss.refl}  cultivation\\
\glt        ‘Has my brother-in-law no stick-forest [poor forest] for his cultivation’ (\textit{Argus})
\z
\z

In \ili{Late Modern Swedish}, adjectives are no longer inflected for case, but attributive adjectives \isi{agree} with the noun in \isi{gender}, number, and \isi{definiteness}; see \REF{ex:intro:35} below.\footnote{In PDS, the adjectival ending in definite or plural noun phrases is generally \textit{{}-a}; \textit{{}-e} is sometimes used with reference to male human beings (see \citealt{Bylin2016}). In older Swedish, \textit{{}-e} had a wider use (depending on author and text); see \citet{Larsson2004} and references therein.}  \isi{Predicative} adjectives also show \isi{agreement} in number and \isi{gender}, as shown in \REF{ex:intro:36}.\footnote{PDS has non-agreeing predicatives in so-called pancake sentences (e.g. \citealt{Josefsson2009}), as in (i).

\ea
\gll  Pannkakor   är   gott. \\
Pancake\textsc{.pl}  is   good\textsc{.n.sg}\\
\glt ‘Pancakes are good.’
\z

Simplifying somewhat, the subject here looks semantically like a small clause, and it refers to the eating of pancakes (rather than to a plurality of pancakes). According to \citet{Faarlund1977} and \citet{Josefsson2014}, the possibility of pancake sentences arose around 1900, but \citet{HaugenEtAl2019} find Swedish examples from the 1850s; an early example is given in (ii).

\ea \gll mjölgröt     är södt\\
        flour.porridge\textsc{.c.sg}   is sweet\textsc{.n.sg}\\
        \glt ‘Flour porridge is sweet’ (Sw. 1850s; from \citealt{HaugenEtAl2019}: 252)
\z Pancake sentences are not attested before the 19\textsuperscript{th} century, and it seems clear that the possibility arose in the LMS period.} No \isi{gender} distinctions are made in the plural, or in definite attributive adjectives (setting aside the marginal use of -\textit{e} described in footnote 18).


\ea\label{ex:intro:35}
\ea\label{ex:intro:35b}
\gll  ett       godt         ord\\
a.\textsc{n.sg}  good.\textsc{n.sg}  word[\textsc{n]}\\
\glt ‘a good word’

\ex\label{ex:intro:35a}
\gll  en       stor           Bok \\
a.\textsc{c.sg}  great.\textsc{c.sg}  book[\textsc{c]}\\
\glt ‘a great book’

\ex\label{ex:intro:35c}
\gll  många~  kloka    ord\\
    many\textsc{.pl}  wise\textsc{.pl}  word[\textsc{n.pl]}\\
    \glt ‘many wise words’

\ex\label{ex:intro:35d}
\gll  utländske   Böcker\\
    foreign.\textsc{pl}  book.\textsc{pl}\\
    \glt  ‘foreign books’

\ex\label{ex:intro:35e}
\gll  den     stora       hopen\\
    the.\textsc{c.sg} large.\textsc{def}   group.\textsc{def.c.sg}\\
\glt    ‘the large group’

\ex\label{ex:intro:35f}
\gll  det       magra       hufwudet\\
    the\textsc{.n.sg} meagre\textsc{.def}    head\textsc{.def.n.sg}\\
\glt    ‘the meagre head’
    (examples from \textit{Argus})\\
\z
\ex \label{ex:intro:36}
\ea \label{ex:intro:36a}
\gll  Menniskian       är högmodig. \\
man.\textsc{def.c.sg}   is   conceited.\textsc{c.sg}\\
\glt ‘Man is conceited.’

\ex \label{ex:intro:36b}
\gll  om   en       Satz       är   falsk       eller   sann\\
    if     a.\textsc{c.sg}     sentence[\textsc{c}] is   false.\textsc{c.sg}   or     true\textsc{.c.sg}\\
\glt `if a sentence is false or true’

\ex \label{ex:intro:36c}
\gll  detta       omdömet         är ädelt\\
    this.\textsc{n.sg}   opinion\textsc{.def.n.sg}    is noble\textsc{.n.sg}\\
\glt `this opinion is noble’

\ex \label{ex:intro:36d}
\gll  fast       de   sielwa   icke   äro       ostraffbare\\
    although   they   self.\textsc{pl}   not   be\textsc{.prs.pl}  unpunishable.\textsc{pl}\\
\glt `although they are not unpunishable themselves’ (examples from \textit{Argus})
\z
\z

The \isi{three-gender system} of \ili{Old Swedish} was gradually lost in the period 1500–1900 (see \citealt{Davidson1990}: 48–50). In the 18\textsuperscript{th} century, a system with two genders (\isi{neuter} and common \isi{gender}) dominated in all genres, and this is also what we find in Dalin’s \textit{Argus}. However, remnants of the old system are preserved in many dialects, and this can most often be seen in the \isi{pronominal system} (rather than in the \isi{inflection} of determiners and adjectives); see \sectref{sec:intro:3.4} below for examples. Whereas \textit{Argus} has a pronoun \textit{den} ‘it’ referring back to inanimate entities with common \isi{gender} (see \ref{ex:intro:37}), other LMS texts use \textit{han} to refer to \isi{masculine} inanimates and \textit{hon} to refer to nouns with grammatical \isi{feminine} \isi{gender}. Examples are given in \REF{ex:intro:38}.


\ea\label{ex:intro:37}
\ea\label{ex:intro:37a}
\gll  \textit{annan} \textit{lärdom},     än \textit{den} som   kunde   wärkställas \\
  other.\textsc{c.sg}  learning[\textsc{c}] than   \textsc{c.3sg}   which   could   execute\textsc{.inf.pass}\\
    \glt `other learnings than that which could be executed’ (\textit{Argus})

\ex\label{ex:intro:37b}
\gll \textit{En} \textit{ting} wände   hon   bort  med annat   tal,     när \textit{den}\\
 one.\textsc{c.sg}   thing[\textsc{c}] turned   she   away   with other   speech   when   \textsc{c.3sg}\\

 \gll bracktes       på   bahnen\\
    bring.\textsc{pst.pass}   on   course.\textsc{def}\\

\glt ‘One thing she always diverted by talking of something else, when it was brought up’ (\textit{Argus})
\z
\ex \label{ex:intro:38}
\ea\label{ex:intro:38a}
\gll  Tag   repet         och drag \textit{kälken},             så tror       far,      att \\
take   rope\textsc{.def.n.sg}  and pull    sledge.\textsc{def.c/m.sg}   so believes father   that \\

\gll \textit{han}     är din.\\
    \textsc{m.3sg}  is  yours\textsc{.c/m.sg}  \\
    \glt `Take the rope and pull the sledge, so Father will believe that it is yours.’ (19\textsuperscript{th} century; from \isi{SAOB})

\ex\label{ex:intro:38b}
\gll  tenckia   alt wel   om \textit{sin} \textit{öfwerhet},        tala    wel om \textit{henne}\\
think   all well  about \textsc{poss.refl.c/f.sg} authority[\textsc{c/f}]  speak well of  \textsc{f.3sg}\\
\glt ‘think only well of their authority, speak well of it’ (18\textsuperscript{th} century; from \isi{SAOB})
\z
\z

There is considerable \isi{dialect variation} in nominal morphosyntax in North \ili{Germanic}; see e.g. \citet{Delsing2003} and \citet{Dahl2015}.


\subsection{A concluding remark on dialect variation}\label{sec:intro:3.5}


As of today, most of the LMS morphosyntax described in \sectref{sec:intro:3} has spread to the entire Swedish speaking area, which (setting aside the heritage varieties in the Americas) includes parts of Finland as well as Sweden. However, there is still variation, and this was the main objective of \isi{ScanDiaSyn} (Scandinavian \isi{Dialect} Syntax), a collaborative project that was initiated in the early 2000s, involving participants from all the Nordic countries, and which resulted in a number of digital resources (\citealt{JohannessenEtAl2009}; \citealt{LindstadEtAl2009}).



Going back a century or so, the dialectal variation within Sweden was substantial (see \sectref{sec:intro:2.2} above). Dialects on the peripheries – from a Stockholm perspective – often deviated substantially from \ili{Standard Swedish}, both because they had held on to archaic traits long gone in \isi{Central Sweden}, and because they had undergone separate developments, either unique or shared with neighbouring dialects or languages.



Let us consider, for instance, the \isi{traditional dialect} of \isi{Orust} in the southwestern province of Bohuslän. In a collection of \isi{Orust} narratives, phonetically transcribed by dialectologists around 1900 (see \sectref{sec:intro:2.3} above), much of the morphosyntax is reminiscent of what we find in texts from the early 18\textsuperscript{th} century. As can be seen in \REF{ex:intro:39} below,\footnote{The phonetic font is simplified here (see \citealt{Petzell2019,Petzell2020} for details).} this goes for the \isi{tense} and \isi{gender} systems, the realization of non-referential subjects, and the syntax of verbal particles. In \REF{ex:intro:39a}, the \isi{auxiliary} is \textit{vara} ‘be’ rather than \textit{ha} ‘have’, as in PDS (see \sectref{sec:intro:3.1.2} above). The \isi{indefinite article} \textit{e} (in \textit{e gran} ‘a pine tree’) indicates \isi{feminine} \isi{gender} (distinct from \isi{masculine} \textit{en} and \isi{neuter} \textit{ett}; see \sectref{sec:intro:3.3}). As shown in \REF{ex:intro:39b}, anaphoric pronouns also \isi{agree} in \isi{gender} with their antecedents: the \isi{feminine} \isi{clitic} \textit{ner} refers to \textit{kuärna} ‘mill.\textsc{def}.\textsc{fem}’. As for the syntax of verbal particles, both the position of arguments and the lack of \isi{particle} \isi{incorporation} place the \isi{Orust} \isi{dialect} closer to EMS than to PDS (see the paper by Larsson \& Lundquist in this volume for EMS data). Thus, in \REF{ex:intro:39c} the \isi{pronominal object} comes before the \isi{particle} (\textit{henne ud}) rather than after, whereas in PDS it comes after, and in \REF{ex:intro:39d}, the \isi{particle} (\textit{fram}) follows the \isi{participle} (\textit{sätt}) rather than being attached to its left as in PDS (\textit{fram-satt}).


\ea \label{ex:intro:39}
\ea \label{ex:intro:39a}
\gll å     sö lå   där    e     grân,     sum   vâ   blåst       ikôll\\
      and    so lay     there  a.\textsc{f}  pine.tree[\textsc{f}]    that     was blow\textsc{.ptcp}   down\\
\glt       ‘also, there was a pine tree lying on the ground, which the       wind had blown down’ (\isi{Orust} 22:2)

\ex \label{ex:intro:39b}
\gll  kuärna,       um   i       vell       sälje=ner\\
    mill\textsc{.def.f.sg}   if     you\textsc{.pl}   want.\textsc{prs}   sell.\textsc{inf}=her.\textsc{cl}\\
\glt `the mill, if you want to sell it’ (\isi{Orust} 27:9)

\ex \label{ex:intro:39c}
\gll  velle       nara     henne   ud\\
    want.\textsc{pst}   lure.\textsc{inf}  her     out\\
\glt `wanted to lure her out’ (\isi{Orust} 27:3)

\ex \label{ex:intro:39d}
\gll  se bLe         där   sätt       fram   ett     feskefâd\\
    so become.\textsc{pst}   there   put.\textsc{ptcp}   forth   a.\textsc{n.sg}   fish.plate[\textsc{n}]\\
\glt `then, a fish plate was put on the table’ (\isi{Orust} 27:2)
\z
\z


Many of the morphosyntactic peculiarities of the \isi{Orust} \isi{dialect} could also be taken to reflect the fact that \isi{Orust} is situated in the peripheral west, closer to both Denmark and Norway than to Stockholm. Up until 1658, Bohuslän was a \ili{Norwegian} province, dominated by Denmark from the late Middle Ages (as was the rest of Norway). As noted, the \isi{Orust} \isi{particle} syntax, the \isi{tense system}, and the \isi{three-gender system} is reminiscent of EMS, but much of it is also very similar to what we find in present-day \ili{Norwegian}. Furthermore, the \isi{expletive} subject is \textit{där} in \REF{ex:intro:39a} and \REF{ex:intro:39d}, just like \ili{Danish} \textit{der}, and the periphrastic formation of the \isi{passive} with \textit{bli} in \REF{ex:intro:39d} would be infelicitous in PDS, where the \textit{s}{}-\isi{passive} is preferred, but fine in both \ili{Danish} and \ili{Norwegian}. In fact, the preference for the \textit{bli} \isi{passive} in the \isi{Orust} sample lacks any correlate in the history of \ili{Standard Swedish} (see \sectref{sec:intro:3.3.2}). By contrast, \isi{expletive} \textit{där} varied with \textit{det} for quite some time in EMS \citep{Falk1993}.



Naturally, many of the traditional dialects exhibit developments of their own, innovations that are not attested in any other variety (standard or non-standard). One example of this comes from the Swedish \isi{dialect} of the Estonian island of \isi{Nuckö}, described by \citet{Vesterdahl2018}. In this variety, the \isi{case system} of \ili{Old Swedish} is all gone, much like in \ili{Standard Swedish} and in most dialects. Nevertheless, the old nominative-\isi{accusative} distinction on adjectives lives on, but with a new function. According to Vesterdahl, the \isi{Nuckö} speakers have reanalysed the distinction as a predicative-attributive distinction, operating within a still intact \isi{three-gender system} (as in \isi{Orust}). Consequently, we get pairs like \textit{storan båt} ‘big boat’, and \textit{båten är storor} ‘the boat is big’; the \isi{adjective} has the old \isi{masculine} \isi{accusative} ending (\textit{storan}) when it modifies the noun inside the DP, but the old \isi{masculine} \isi{nominative} (\textit{storor}; cf. OS \textit{storer}) when the \isi{adjective} is used predicatively.



\ili{Late Modern Swedish} is obviously not the end of history, either. New cases of variation of course arise continually, in the \isi{standard language} as well as in the dialects. Some of this new variation can be observed as a difference between older and younger speakers in the \isi{ScanDiaSyn} investigations. For instance, with respect to measureless quantificational exclamatives, \citet{Vangsnes2014} observes that younger speakers in Sweden more often accept a split structure with the \textit{wh}{}-word in initial position but the rest of the phrase in the base position (\textit{vad det var bilar här!}, lit. ‘what it was cars here!’), whereas older people often only accept fronting of the whole phrase (\textit{vad bilar det var här!}, lit. ‘what cars it was here’). Some recent changes are a consequence of \isi{language planning} and policy (such as the introduction of a new gender-neutral pronoun with human reference, \textit{hen}; see \citealt{LedinLyngfelt2013}). Other examples involve familiar types of grammatical change, like the \isi{grammaticalization} of discourse markers (like \textit{bara/ba} lit. ‘only’, discussed by \citealt{Eriksson1995} among others). In fact, several of the changes discussed in the following chapters are possibly still on-going. For instance, the changes in the use of double object constructions discussed by Valdeson will most likely continue in the future, and the relatively new use of \isi{adverbial} infinitives with a \isi{concessive} meaning observed by Kalm will possibly gain ground in the coming decades. As pointed out by Falk, the preferences for \isi{choice of subject} in ditransitives have also shifted recently, and it might be that this change has not yet reached its conclusion. Furthermore, Larsson and Lundquist suggest that there are recent shifts in the preferences for \isi{particle} constructions (e.g. with modified particles), which need to be investigated further in the present-day language.

\section{The papers in this volume}\label{sec:intro:4}


This volume includes six full-length articles and one squib. The contributions cover different grammatical domains, including case and verbal syntax, word order and \isi{agreement}, and \isi{grammaticalization} in the \isi{nominal domain}. 



Firstly, the paper by Cecilia Falk discusses the possibility of promoting an \isi{indirect object} to subject in a \isi{passive}; this is referred to as the passivization of an \isi{indirect object}. Falk shows that only direct objects could be passivized in Swedish before the 17\textsuperscript{th} century, and that a major change in the grammar took place in the second part of the 19\textsuperscript{th} century. She proposes that the \isi{indirect object} is merged in an \isi{inherent case} position both in older and present-day Swedish, but that the featural make-up, and, crucially, the case assigning properties of \isi{ditransitive} verbs have changed. She assumes that before the change, there was no phi-\isi{agreement} between the \isi{indirect object} and the verb, whereas after the change, a \isi{ditransitive verb} carried two sets of phi-features. This difference accounts for the different possibilities in passives. Falk furthermore suggests that the change in passivization possibilities is related to the emergence of a dedicated and obligatory \isi{subject position} in the \isi{I-domain} (cf. \sectref{sec:intro:3.2.1} above).



Fredrik Valdeson’s paper, too, is concerned with double object constructions, albeit from a different theoretical point of view. Valdeson investigates the use of verbs with double objects from a constructional perspective and argues that changes in the \isi{double object construction} provide evidence for a constructional network where higher and lower levels (more or less abstract constructions) can change in similar ways, but also partly independently. He observes that the \isi{double object construction} becomes less frequent in the period from the beginning of the 19\textsuperscript{th} century to the present. It also occurs with fewer verbs; in Valdeson’s terms there is less \textit{lexical variation}. He looks more closely at the most frequent verbs and shows that some of them also show less \isi{lexical variation} – they occur with fewer different types of objects. However, some verbs become less frequent in the double object constructions, but still show high \isi{lexical variability} with respect to object types. Valdeson therefore concludes that \isi{productivity} is not necessarily dependent on \isi{text frequency}.



Ida Larsson and Björn Lundquist study the development of a strict order between verbal particles and objects. Up until the middle of the 17\textsuperscript{th} century, Swedish had the same word order possibilities in \isi{particle} constructions as, for instance, modern \ili{English} and \ili{Norwegian}: pronominal objects typically preceded verbal particles, whereas non-pronominal objects could either precede or follow the \isi{particle}. \ili{Present-day Swedish}, on the other hand, differs from all the other \ili{Germanic} languages by requiring all objects to follow a \isi{particle}. Larsson and Lundquist show that the change started in the 17\textsuperscript{th} century and that the modern word order was largely established around the end of the 18\textsuperscript{th} century. However, not all \isi{particle} constructions were affected at the same time, and there is ongoing development into the present day. The authors suggest that the variability in older Swedish had to do with the status of the \isi{particle} as a phrasal modifier, in combination with the principles of the linearization of phrases. The change, they argue, was due to a \isi{reanalysis} of the \isi{particle} from phrase to head; this is not an unexpected development given economy principles such as the \isi{Head Preference Principle} (\citealt{van_Gelderen2004}).



Mikael Kalm discusses the emergence of different kinds of \isi{adverbial} infinitival clauses. In \ili{Old Swedish}, the only type of \isi{adverbial} \isi{infinitival clause} that is attested expresses purpose, and other types do not seem to become possible until the 17\textsuperscript{th} century; temporal and instrumental \isi{adverbial} infinitivals are rare in Kalm’s sources, and they are not attested before the 19\textsuperscript{th} century. Kalm ties this development partly to the \isi{grammaticalization} of the \isi{infinitival marker} \textit{att}. As in many other \ili{Germanic} languages, this marker started out as a preposition, but it was not until the 18\textsuperscript{th} century, Kalm argues, that it lost all prepositional content, in effect preparing the ground for the wide assortment of \isi{adverbial} infinitives that we have today. The development, Kalm suggests, is a consequence of contact-induced \isi{grammaticalization}, as well as so-called \textit{Verschriftlichung} and language \textit{Ausbau} (\citealt{Hoder2009, Hoder2010}). In other words, the use of \isi{adverbial} infinitival clauses depends on the development and elaboration of the written code, and the Swedish written code is influenced by other languages. To test this hypothesis, Kalm compares present-day \ili{Standard Swedish} with translations into \ili{Övdalian}, which, unlike \ili{Standard Swedish}, was not codified until recently. The use of \isi{adverbial} infinitival clauses is therefore expected to be restricted or even non-existent in \ili{Övdalian}. Kalm shows that although some of the \isi{adverbial} infinitivals in the Swedish original text are translated with infinitival clauses, \ili{Övdalian} prefers other constructions (coordination, embedded finite clauses, etc.). Temporal and instrumental infinitival clauses seem to be avoided in the \ili{Övdalian} translations.



Adrian Sangfelt studies word order in complex VPs, and investigates the possibility of having adverbials (and other constituents) between the separate verbal heads. In general, OV languages (e.g. \ili{German}) do not allow intervening material in such contexts, whereas VO languages (e.g. \ili{English}) do (see \citealt{Haider2010,Haider2013}). Sangfelt investigates verbal clusters in Swedish during the period 1725–1850. During this time, the final remains of the old OV system disappeared, and given the cross-linguistic patterns, it is therefore expected that it will be increasingly possible to have material intervening between the verbal heads. Interestingly, Sangfelt’s results suggest that Swedish contradicts the generalization: intervening elements become increasingly uncommon. In the end, the link between OV and mandatory \isi{clustering} turns out to be indirect. More specifically, the only verbal sequence that appears never to be broken is main verb\,+\,\isi{auxiliary} (VAux). Although VAux is restricted to OV languages, \isi{OV word order} does not need to involve VAux.



Erik Petzell tests the \isi{Rich Agreement Hypothesis} (\isi{RAH}), recently revitalized by Koeneman \& Zeijlstra (2014; see also \citealt{Tvica2017}), on data from \ili{Viskadalian} Swedish. In this \isi{dialect}, there is no \isi{V-to-I} \isi{movement}, although finite verbs are inflected for all persons in the plural (i.e. rich \isi{agreement} by any standard). However, the \isi{RAH} still stands, Petzell maintains, as long as the person endings are analysed as part of [\isi{tense}], an account that is independently supported by the emergence of the 2SG \isi{clitic} \textit{(s)tä}. Petzell further argues that both the \isi{reanalysis} of \isi{agreement} as part of \isi{tense} (i.e. [\isi{tense}]-[agr]→ [\isi{tense}\textsubscript{agr}]), and the \isi{clitic} development, where the former \textsc{2sg} suffix \textit{{}-(s)t} becomes part of the pronominal \isi{clitic} \textit{ä}, represent instances of \isi{syntactic grammaticalization} in the sense of \citet{RobertsRoussou1999, RobertsRoussou2003}. In both cases, the \isi{agreement} morpheme climbs upwards in the syntactic tree, as it were, as it becomes associated with T (the locus of \isi{tense} as well as of subjects), rather than with a lower functional head.



\isi{Grammaticalization} is discussed in the final contribution as well, where Lars-Olof Delsing studies how the gradable adjectives \textit{mycket} ‘much’ and \textit{lite} ‘little’ developed into quantifiers, and the concomitant loss of \isi{agreement} morphology. Delsing shows that non-agreeing \textit{lite} spread from the 17\textsuperscript{th} century onwards, and that the development of the \isi{quantifier} \textit{mycket} took place mainly during the 18\textsuperscript{th} and 19\textsuperscript{th} centuries. He further argues that weak forms of \textit{lite} and \textit{mycket} have been reanalysed further, and that they are polarity items in present-day Swedish. There are to date few detailed studies of \isi{grammaticalization} within the \isi{nominal domain} in Swedish, and one important contribution of Delsing’s squib is to point to questions for future work.


\section*{Acknowledgements}


Thanks to Cecilia Falk, Elisabet Engdahl, Kari Kinn, and two reviewers for their helpful comments and constructive criticism on an earlier version of this introduction. We are also grateful to Dick Claésson and Paulina Helgeson for their translation of the introduction to Dalin’s \textit{Argus}. Any remaining flaws and inconsistencies are, of course, our responsibility. Our work was supported by the Institute for Language and Folklore in Gothenburg, Østfold University College, the University of Oslo, and the Research Council of Norway (through the project \textit{Variation and Change in the Scandinavian Verb Phrase}, grant no. 250755).

\section*{Abbreviations}

\begin{multicols}{2}
\begin{tabbing}
\isi{RAH}\hspace{1ex} \= \isi{Rich Agreement Hypothesis}\kill
\isi{ECM} \>  Exceptional case-marking    \\
EOS \>  \ili{Early Old Swedish}\\
EMS \>  \ili{Early Modern Swedish}\\
HTR \>  \isi{Handwritten Text Recognition}\\
LMS \> \ili{Late Modern Swedish}\\
LOS \>  \ili{Late Old Swedish}\\
OV  \> Object–Verb order \\
OS  \> \ili{Old Swedish}\\
PDS \>  present-day Swedish\\
\isi{RAH} \>  \isi{Rich Agreement Hypothesis}\\
SUP \>  \isi{supine}\\
V2  \> \isi{Verb second order}\\
VO  \> Verb–Object order\\
VS  \> Verb–Subject order
\end{tabbing}
\end{multicols}

\section*{Sources}

\begin{description}[font=\normalfont]\sloppy
\item[\textit{ÄSv}:] \isi{Äldre svenska romaner} 1840–1930 [\ili{Older Swedish} novels 1840–1930]. Available through \isi{Korp}.
\item[\textit{Argus}:] Dalin, Olof von (b. 1708). \textit{Then Swänska Argus} [The Swedish \isi{Argus}]. Stockholm, 1732–1734. Available through FTB\slash \isi{Korp} (text) and LB (facsimile).
\item[\isi{EWL}:] Collin, Hans Samuel \& Carl Johan Schlyter (eds.). 1827. \emph{Samling af Sweriges gamla lagar. Första bandet. Westgötalagen} [Collection of the old laws of Sweden. Volume one. The Westrogothic law]\emph{. \textup{Stockholm: Z. Haeggström}. \textup{Originally written in the 1220s.} }Available through FTB\slash \isi{Korp}.
\item[Gyllenborg:] Gyllenborg, Carl (b. 1679). \textit{Swenska sprätthöken} [The Swedish dandy]. Stockholm, 1740. Available through LB.
\item[Horn:] Horn, Agneta (b. 1629). \textit{Beskrivning över min vandringstid} [Description of my life]. Edited by Gösta Holm. Stockholm: Almqvist \& Wiksell, 1959. Available through FTB\slash \isi{Korp}.
\item[Jolin:] Jolin, Johan (b. 1818). \textit{Barnhusbarnen eller Verldens dom} [The children of the orphanage or the judgement of the world]. Stockholm, 1849. See \citet{MarttalaStromquist2001}. Available through LB.
\item[K-styr:] Moberg, Lennart (ed.). 1964. \textit{En nyttigh bok om konnunga styrilse och höfdinga} [A useful book on the royal rule]. Facsimile of 1634 edition by Johannes Bureus. Uppsala. Original from the early 14\textsuperscript{th} century. Available through FTB\slash \isi{Korp}.
\item[Kiöping:] Kiöping, Nils Mattson (b. 1621). Nils Matssons Reesas korta Beskriffning [The short description of the journey of Nils Mattsson Kiöping]. In \textit{Een kort Beskriffning Uppå Trenne Reesor och Peregrinationer, sampt Konungarijket Japan}. Printed by Johan Kankel in Wisingsborgh, 1674. Available through \isi{Korp}.
\item[Leg:] Stephens, George (ed.). 1847. \textit{Svenska medeltidens kloster- och helgonabok} […] \textit{Ett forn-svenskt legendarium} […] [The Swedish medieval book of monasteries and saints … An \ili{Old Swedish} collection of legends]. Stockholm: Norstedts. Originally written sometime between 1276 and 1307. Earliest manuscript (Codex Bureanus) from ca. 1350. Available through FTB\slash \isi{Korp}.
\item[Modée:] Modée, Reinhold Gustaf (b. 1698). \textit{Håkan Smulgråt} [Håkan Cheapskate]. Stockholm, 1739. See \citet{MarttalaStromquist2001}. Available through LB.
\item[\isi{Orust}:] \isi{Dialect} texts (IOD, old accession numbers 22:1–3, 27:1–9) from 1897–1901 from the Island of \isi{Orust}, kept at the Institute for Language and Folklore in Gothenburg.
\item[Öxn:] Recordings (accession numbers ULMA6804–6806) from 1956 from the parish of Öxnevalla, kept at the Institute for Language and Folklore in Uppsala.
\item[Ristell:] Ristell, Adolf Fredrik. (b. 1744). \textit{Några mil från Stockholm} [A few miles from Stockholm]. Manuscript from 1787. Edited by Gösta Langenfeldt \& Bo Thörnqvist. Stockholm: Department of Scandinavian languages, 1974. See \citet{MarttalaStromquist2001}.
\item[Rålamb:] Callmer, Christian (ed.). 1963. \textit{Diarium under resa till Konstantinopel 1657–1658} [`Diary during a journey to Constantinople 1657–1658’, undertaken by Claes Rålamb (b. 1622)]. (Historiska handlingar 37:3.) Stockholm: Norstedts.
\item[Salvius:] Salvius, Lars (b. 1706). \textit{Beskrifning om en resa genom Asia, Africa och många andra hedna länder, som är Giord af Nils Matson Kiöping för detta Kongl. Maj:ts skeps lieutnant} [A description of a journey through Asia, Africa, and many other pagan countries, which is made by Nils Matson Kiöping, former lieutenant of the Royal Navy]. Printed in Stockholm, 1743. Available through \isi{Korp}.
\item[\isi{SAOB}:] \textit{Ordbok över svenska språket, utg. av Svenska Akademien} [Dictionary of the Swedish language, published by The Swedish Academy]. 1893–. Lund. Available here: \href{http://www.saob.se}{{www.saob.se}}
\item[SPF:] Swedish prose fiction 1800–1900. Available through \isi{Korp}.
\end{description}

\section*{Electronic corpora}

\begin{description}[font=\normalfont]
\item[FTB:] \isi{Fornsvenska textbanken} [The text bank of \ili{Old Swedish}]: \url{https://project2.sol.lu.se/fornsvenska} 
\item[\isi{Korp}:] \url{https://spraakbanken.gu.se/korp/?mode=all_hist}
\item[LB:] The Swedish literature bank: \url{http://www.litteraturbanken.se}
\end{description}

{\sloppy\printbibliography[heading=subbibliography,notkeyword=this]}
\end{document}
