\documentclass[output=paper]{langscibook}
\ChapterDOI{10.5281/zenodo.5792959}
\author{Mikael Kalm\affiliation{Stockholm University}}
\title{The emergence of adverbial infinitives in Swedish}
\abstract{Prepositional infinitives are commonly used in present-day Swedish to express a range of adverbial notions, including purposive, abessive, substitutive, temporal, and instrumental meanings. However, only purposive infinitives are attested in Old Swedish. The aim of this article is to give an account of the emergence of adverbial infinitives in the history of Swedish. Other adverbial infinitives emerged from the late 17\textsuperscript{th} century onwards, possibly as a result of contact-induced (replica) grammaticalization. It is furthermore argued that the emergence of adverbial infinitives should be seen as a result of the increasing importance and consequent demands for precision of the written language, and as part of the establishing of a written norm, separate from the spoken language. This assumption is supported by data from the traditional dialect of Övdalian, where adverbial infinitives are not used.

\keywords{adverbial infinitives, prepositional infinitives, Swedish, Övdalian, language Ausbau, grammatical replication, contact-induced change, Verschriftlichung}
}
\IfFileExists{../localcommands.tex}{
  \addbibresource{localbibliography.bib}
  % add all extra packages you need to load to this file

\usepackage{tabularx,multicol}
\usepackage{url}
\urlstyle{same}

\usepackage{listings}
\lstset{basicstyle=\ttfamily,tabsize=2,breaklines=true}

\usepackage{langsci-basic}
\usepackage{langsci-optional}
\usepackage{langsci-lgr}
\usepackage{langsci-gb4e}

\usepackage{todonotes}

\usepackage[linguistics]{forest}
\usepackage{soul}
\usepackage{subfigure}
\usepackage{longtable}
\usepackage{enumitem}

  \newcommand*{\orcid}{}
%\newcommand{\keywords}[1]{\textbf{#1}}


\makeatletter
\let\theauthor\@author
\makeatother

\newcommand{\keywords}[1]{\textbf{Keywords:} #1}


\DeclareNewSectionCommand
  [
    counterwithin = chapter,
    afterskip = 2.3ex plus .2ex,
    beforeskip = -3.5ex plus -1ex minus -.2ex,
    indent = 0pt,
    font = \usekomafont{section},
    level = 1,
    tocindent = 1.5em,
    toclevel = 1,
    tocnumwidth = 2.3em,
    tocstyle = section,
    style = section
  ]
  {appendixsection}

\renewcommand*\theappendixsection{\Alph{appendixsection}}
\renewcommand*{\appendixsectionformat}{\appendixname~\theappendixsection\autodot\enskip}
\renewcommand*{\appendixsectionmarkformat}{\appendixname~\theappendixsection\autodot\enskip}
 
  %% hyphenation points for line breaks
%% Normally, automatic hyphenation in LaTeX is very good
%% If a word is mis-hyphenated, add it to this file
%%
%% add information to TeX file before \begin{document} with:
%% %% hyphenation points for line breaks
%% Normally, automatic hyphenation in LaTeX is very good
%% If a word is mis-hyphenated, add it to this file
%%
%% add information to TeX file before \begin{document} with:
%% %% hyphenation points for line breaks
%% Normally, automatic hyphenation in LaTeX is very good
%% If a word is mis-hyphenated, add it to this file
%%
%% add information to TeX file before \begin{document} with:
%% \include{localhyphenation}
\hyphenation{
anaph-o-ra
Dor-drecht
mono-mor-phe-mic
Swed-ish
sche-mat-ic
Viska-da-li-an
An-ders-son
dia-lekt-forsk-ning
dra-ma-språk
bref-vex-ling
Ak-tu-ell
folk-livs-forsk-ning
Þor-björg
Ak-ti-ons-art
Upp-sala
myck-en
}

\hyphenation{
anaph-o-ra
Dor-drecht
mono-mor-phe-mic
Swed-ish
sche-mat-ic
Viska-da-li-an
An-ders-son
dia-lekt-forsk-ning
dra-ma-språk
bref-vex-ling
Ak-tu-ell
folk-livs-forsk-ning
Þor-björg
Ak-ti-ons-art
Upp-sala
myck-en
}

\hyphenation{
anaph-o-ra
Dor-drecht
mono-mor-phe-mic
Swed-ish
sche-mat-ic
Viska-da-li-an
An-ders-son
dia-lekt-forsk-ning
dra-ma-språk
bref-vex-ling
Ak-tu-ell
folk-livs-forsk-ning
Þor-björg
Ak-ti-ons-art
Upp-sala
myck-en
}
 
  \togglepaper[1]%%chapternumber
}{}

\begin{document}
\SetupAffiliations{mark style=none}
\maketitle 
%\shorttitlerunninghead{}%%use this for an abridged title in the page headers



\section{Introduction}\label{sec:kalm:1}\largerpage
The development of Swedish into a literary language is to a large extent characterized by intensive contact with other languages: Latin and Low German in the Middle Ages and High German and French in the Early Modern era. The first steps towards a literary and somewhat standardized written language were already being taken during the late Middle Ages in the monasteries and convents, most importantly at the abbey of Vadstena. The Latin influence on these texts is considerable, not least at a syntactic level, causing a split between written Old Swedish and the spoken varieties. This has been described as a process of language Ausbau (see \citealt{Hoder2009, Hoder2010}). A more homogeneous written language took shape after the Reformation, with the translation of the Bible into Swedish during the first half of the 16\textsuperscript{th} century as an important landmark. It was not until the end of the 18\textsuperscript{th} century, however, that the standardization process was completed (more or less). As mentioned above, this post-Reformation period is characterized by contact between Swedish and High German and French (as well as Latin, but to a much lesser extent than before). Traces of this influence are clearly visible in the vocabulary even today (see \citealt{Teleman2003Swedish,Teleman2003Tradis}), but we know considerably less about the syntactic influence from these languages. 



This paper investigates a syntactic innovation from the dynamic post-Ref\-or\-ma\-tion period in the history of written Swedish, namely the emergence of adverbial infinitives, i.e. prepositional infinitive clauses functioning as adverbial adjuncts. While present-day Swedish has a range of such prepositional clauses, they are not attested in Old Swedish. A few examples of the adverbial infinitives discussed in this article are given in \REF{ex:kalm:1} below, all from present-day Swedish.\largerpage  



\ea
\label{ex:kalm:1}
\ea Purposive\label{ex:kalm:1a}\\
\gll Han gick ut för att få sig lite luft.\\
he went out for \textsc{im} get.\textsc{inf} \textsc{refl} some air\\
\glt ‘He went out to get some air.’

\ex Abessive\label{ex:kalm:1b}\\
\gll Hon sålde företaget utan att fråga de anställda.\\
she sold company.\textsc{def} without \textsc{im} ask.\textsc{inf} the employees\\
\glt ‘She sold the company without asking the employees.’

\ex Substitutive\label{ex:kalm:1c}\\
\gll Jag satt hemma istället för att gå ut.\\
I sat home instead for \textsc{im} go.\textsc{inf} out\\
\glt‘I stayed home instead of going out.’

\ex Temporal\label{ex:kalm:1d}\\
\gll Hon läste medicin efter att ha kommit tillbaka från Berlin.\\
she read medicine after \textsc{im} have.\textsc{inf} come back from Berlin\\
\glt ‘She studied medicine after having returned from Berlin.’

\ex Instrumental\label{ex:kalm:1e}\\
\gll Han gjorde sig oumbärlig genom att alltid vara steget före.\\
he made \textsc{refl} indispensable through \textsc{im} always be.\textsc{inf} step.\textsc{def} before\\
\glt ‘He made himself indispensable by always being one step ahead.’
\z
\z

Adverbial infinitives (as well as adverbial constructions more generally) have not attracted much attention from Scandinavian scholars. One exception is \citet{Holm1967}, who makes several claims about adverbial infinitives in his influential survey of prose styles in the history of Swedish. Firstly, focusing on present-day Scandinavian, Holm suggests that adverbial infinitives belong to a written register and are not used in “uneducated” (Sw. \textit{obokligt}) speech, nor in the traditional dialects. He also claims that adverbial infinitives emerged with inspiration from other languages, such as High German and French, and that they are not attested in Old Swedish \citep[27]{Holm1967}. Holm does not present any evidence to support his claims. 



The aim of this article is to give a more detailed account of adverbial infinitives in the history of Swedish, and to compare standard Swedish with the non-standardized variety of Övdalian (see \sectref{sec:kalm:4.3}) with regard to the use of prepositional infinitives to express adverbial notions. I will also discuss whether the emergence of adverbial infinitives in Swedish could be an example of contact-induced grammaticalization (see \citealt{HeineKuteva2003, HeineKuteva2005}) and if this, in turn, can be understood as part of an ongoing language Ausbau process (in the sense of \citealt{Kloss1967}). If so, adverbial infinitives should originally have been part of a written register rather than a spoken register. Important questions are therefore: (1) when adverbial infinitives emerged in Swedish, and (2) whether there is support for the claim that they first appeared as part of a written register. In order to answer the first question, I have investigated Early Modern and Late Modern Swedish texts (ca. 1500–2000) from the digital corpora the Old Swedish Text Bank (Sw. \textit{Fornsvenska textbanken}) and the Swedish Literature Bank (Sw. \textit{Litteraturbanken}), both available through the corpus infrastructure Korp \citep{BorinEtAl2012}.\footnote{\url{https://spraakbanken.gu.se/korp/}}  In addition to this, the corpus of Swedish drama dialogue (Sw. \textit{Svensk dramadialog}; \citealt{MarttalaStromquist2001}), covering the period 1725–2000, has also been investigated in its entirety. To shed light on the second question (i.e. whether adverbial infinitives first appeared as part of a written register), I have conducted a contrastive study of Swedish and the traditional dialect of Övdalian, using translations from Swedish into Övdalian.



The article is organized as follows. In \sectref{sec:kalm:2}, I give a brief introduction to the concept of language Ausbau in relation to the emergence of standardized written Swedish. \sectref{sec:kalm:3} provides a background to the empirical study, with a survey of the history of the Germanic infinitive and a typological perspective on adverbial infinitives in European languages. In \sectref{sec:kalm:4}, I present my empirical findings, and \sectref{sec:kalm:5} provides a theoretical discussion. Finally, \sectref{sec:kalm:6} contains some concluding remarks. 


\section{Functional expansion and formal elaboration of written Swedish}\label{sec:kalm:2}



In the process of codifying a traditionally spoken language or dialect, i.e. transforming it into a written language, the written language will soon differ in certain regards from the spoken varieties. Over time, the use of the written language is normally expanded into new domains. This expansion itself leads to an elaboration of the written language, as it is shaped in more or less deliberate ways to meet certain (communicative or aesthetic) needs. This should be understood as an ongoing process, and is identified by \citet{Kloss1967} as a diachronic process of language Ausbau. It can also be understood as a process of Verschriftlichung (see \citealt{Fischer2007}: 37–38), i.e. the elaboration of the written language will lead to register-specific developments, such as more complex syntactic constructions. The functional expansion and subsequent formal elaboration of the written language lead to a split between the spoken and written varieties of a language (see \citealt{Hoder2009}). In societies in which a large part of the population has access to the written language (as in most modern Western societies), phenomena originally considered only as part of the written register can spread to the spoken language \parencites[37]{Fischer2007}[]{Weis2004}. 



In Swedish, the language Ausbau process was characterized by influence from other languages. The earliest longer texts in Old Swedish written with the Latin alphabet are the provincial laws (from the 13\textsuperscript{th} century onwards), but it was mainly in the context of Vadstena Abbey that the shaping of Old Swedish into a literary language took place. Since Latin was the predominant language in this environment, the Swedish texts produced there were often heavily influenced by Latin, for instance regarding relativization and other syntactic patterns (see further \citealt{Hoder2009, Hoder2010, Wollin1981, Wollin1983}). The bulk of the texts were also translations or paraphrases of Latin originals (i.e. not original works in Swedish). Many of these Latin-inspired constructions survived into Early Modern Swedish through the Reformation Bible translations, which were influenced by the Vadstena tradition (see \citealt{Stahle1970, Teleman2002, Teleman2003Swedish}). Outside of the religious context, another language was in more direct contact with Swedish, namely Low German. During the Middle Ages, the Hanseatic League had a strong impact on the Scandinavian countries, in particular in urban settings, where a large part of the population were Low German merchants. Unlike Latin, the Low German influence was not limited to the written language, but also affected the spoken language of the cities (see \citealt{Braunmuller1997,Braunmuller2005,}). The influence can be observed on all linguistic levels, including numerous grammatical loans, for example conjunctions such as \textit{men} ‘but’, \textit{samt} ‘and’, auxiliaries such as \textit{måste} ‘must’, \textit{bliva} ‘become’, and derivational morphology such as \textit{be}{}-, \textit{för}{}-. (See for example \citealt{Braunmuller2005} for an overview of Low German (and Latin) influence on Swedish.)


During the Early Modern era (i.e. post-Reformation), written Swedish was used in more contexts (for instance in private letters, diaries, science reports, novels, etc.) and no longer mainly in religious or administrational contexts. The Reformation in the first half of the 16\textsuperscript{th} century marked the end of the text production within the monasteries, since they were closed down. Instead, original production in Swedish increased, and the Latin influence decreased. Simultaneously, the importance of the Hanseatic League diminished, thus leading to less contact between Swedish and Low German. Instead, High German became an important language in Sweden, affecting the emerging written standard, not least with regard to the Bible translation in 1541. The High German influence was not only orthographic and lexical, but also syntactic. For example, it has been suggested that the increasing OV frequency of the time \citep{Petzell2011} and so-called finite \textit{ha}{}-drop in subordinate clauses (see \citealt{Johannisson1945}, who first made the claim, and \citealt{Backstrom2019}, who puts it to the test) are both due to contact with High German (see also \citealt{Braunmuller2005} for an overview). High German continued to be an important language in Sweden throughout the Early and Late Modern Swedish period, but from the 17\textsuperscript{th} century onwards, Swedish was also affected by French. The French influence, however, was not as far reaching as the High German one, affecting mainly the lexicon (see \citealt{Gellerstam2005}). By the end of the 18\textsuperscript{th} century, a more or less standardized written Swedish was in place (\citealt{Teleman2002}, see also \citetv{chapters/01}).


\section{Background}\label{sec:kalm:3}


In this section, I give a short introduction to the historical origins of the infinitive and the infinitive marker with special focus on North Germanic. Thereafter, I discuss adverbial infinitives from a typological perspective, comparing present-day Swedish with some other European languages, primarily Romance languages and German. I also summarize the historical development of adverbial infinitives in Romance languages. 


\subsection{The Germanic infinitive: A short introduction}\label{sec:kalm:3.1}



In Swedish, as in many other European languages, the infinitive is considered the basic form of the verb, used for example as the dictionary entry. Unlike in English, where the infinitive is identical to the verb stem, the Swedish infinitive is an inflected form, at least when the verb stem is consonantal.\footnote{There are Scandinavian dialects in which the verb stem is identical to the infinitive, as in English (see \citealt{Delsing2014Verbsystemet}).}  The Swedish infinitive is always vocalic; if the stem ends with a consonant, an \textit{a}{}-suffix is added. However, if the verb stem is vocalic, there is no formal difference between the stem and the infinitive, as illustrated in \REF{ex:kalm:2} below. 


\ea\label{ex:kalm:2}
\begin{tabular}[t]{llll}
{} & stem & infinitive & {}\\
a. & \textit{läs} & \textit{läs-a} & ‘read’\\
{} & \textit{köp} & \textit{köp-a} & ‘buy’\\	
b. & \textit{sy} & \textit{sy}{}-${\emptyset}$ & ‘sew’\\
{} & \textit{hoppa} & \textit{hoppa}{}-${\emptyset}$ & ‘jump’\\
\end{tabular}
\z

From a historical perspective, the verbal status of the infinitive is less obvious. The characteristic \textit{a}{}-ending of the Swedish infinitive is the remnant of an Old Germanic derivational suffix -\textit{an(a)} used in the formation of verbal nouns (\citealt[193]{FalkTorp1900}). Originally, the infinitive was thus nominal rather than verbal, and as such it took case endings like any other nominal element. Examples of this can be found in West Germanic, but not in Ancient Nordic \citep[205]{Prokosch1939}. Over time, the deverbal noun was reanalysed as a verb, as the derivational affix was grammaticalized into an inflectional ending. The -\textit{n} was lost at an early stage in Scandinavian languages but is preserved in, for example, German and Dutch (see \citealt[636]{Noreen1898}). The infinitive is thus common to all the Germanic languages. There are infinitives in other Indo-European languages as well, for example in Romance, but these are etymologically distinct from their Germanic counterparts (see \citealt[205]{Prokosch1939}, \citealt[193]{FalkTorp1900}).



In Germanic (and also in Romance), the infinitive came to be associated with prepositions that were later grammaticalized into infinitive markers. While there is a common Germanic infinitive, the infinitival preposition is different in the North Germanic languages than in the Continental Germanic languages (including English). The Nordic preposition is \textit{at} (‘at’, ‘by’), identical to the present-day Swedish preposition \textit{åt}, and with the same semantics (see \citealt{Hellquist1948}), and the Proto-Germanic is *\textit{tō} (‘at’, ‘by’), thus corresponding to Dutch \textit{te}, Eng. \textit{to}, Ger. \textit{zu}, etc. Note that the infinitive markers in Romance languages have similar semantics, for example French \textit{de} and \textit{à} (see \citealt[15]{Beckman1934}).\footnote{In Romance, the infinitive markers seem to retain more of their prepositional status than in Germanic. See \sectref{sec:kalm:3.2} below.}  With the deverbal noun as a complement, the prepositional phrase was used to express allative or locative meanings. As the nominal status of the infinitive was gradually lost, and it was reanalysed as a verb, the prepositional infinitives came to express purposive meanings instead. This can be understood as a semantic change in which the prepositional phrase first expressed a concrete, spatial goal and later a more abstract goal (see \citealt{Los2005}). In Gothic, for example, there is a regular system in which the infinitival preposition \textit{du} (optionally) can be used in purposive infinitives, while non-purposive infinitives are always bare, i.e. appear without the infinitival preposition. A few examples are given in \REF{ex:kalm:3} below. The distribution of the Old High German \textit{ze} seems to be similar to that of Gothic \textit{du} \citep{Haspelmath1989}. 

\ea
\label{ex:kalm:3}
\ea Purposive\\\label{ex:kalm:3a}
\gll sat du aíthron (Gothic)\\
sit.\textsc{pst.3sg} to pray.\textsc{inf}\\
\glt ‘he sat down to pray’ (from \citealt{Wright1954}: 193–194) 

\ex Object\\\label{ex:kalm:3b}
\gll othedun faríhnan ina (Gothic)\\
fear\textsc{.pst.3pl} ask.\textsc{inf} him.\textsc{acc}\\
\glt ‘they feared to ask him’ (from \citealt{Wright1954}: 193–194)

\ex Subject\\ \label{ex:kalm:3c}
\gll ni godh ist niman hláif barne (Gothic)\\
not good be.\textsc{prs.3sg} take.\textsc{inf} bread.\textsc{acc} child.\textsc{pl.gen}\\
\glt‘it is not good to take the bread of the children’ (from \citealt{Wright1954}: 193–194)
\z
\z

Over time, the infinitival preposition spread from purposive to non-purposive contexts in all of the Germanic languages. The results of this process are similar across Germanic languages, but there is also variation between the languages, even within the North Germanic branch (see \citealt{Haspelmath1989,Los2005,Kalm2016Prepositioner, Kalm2016Satsekvivalenta,Kalm2019}). The final stages of the spread of infinitival prepositions to non-purposive contexts can partly be traced in Early Old Swedish (EOS, ca. 1225–1375). In many of the earliest Swedish texts (i.e. the provincial laws from the 13\textsuperscript{th} century), \textit{at} is regularly used in purposive infinitives, while its use outside of these contexts is highly irregular (\citealt{Kalm2016Satsekvivalenta}: 186; see also \citealt{Kalm2019} for a comparison with Old Danish and Old Gutnish). A few examples are given in \REF{ex:kalm:4}. Note that present-day Swedish requires the infinitive marker in all of the contexts exemplified.\pagebreak


\ea
\label{ex:kalm:4}
\ea Purposive\\\label{ex:kalm:4a}
\gll Combær han til at köpæ iorþ\\ 
comes he there \textsc{im} buy.\textsc{inf} soil\\
\glt‘He comes there to buy land’ (EWL, EOS) 

\ex Complement to adjective\\\label{ex:kalm:4b}
\gll siþæn ær han skyldughær kunungær wæræ\\
 then is he obliged king be.\textsc{inf}\\
\glt ‘then he is obliged to be king’ (UL, EOS)
 
\ex Subject\\\label{ex:kalm:4c}
\gll Nu ær þæt klokkarans skuld. bæra bok ok stol i sokn\\
now is it clerc.\textsc{def.gen} obligation carry.\textsc{inf} book and stole in parish.\textsc{def}\\
\glt ‘Now it is the liability of the clerk to bring the Bible and the stole to the parish’ (ÖgL, EOS)
\z 
\z 



The use of the infinitival preposition/infinitive marker stabilized during the Old Swedish period, and by the beginning of the Early Modern period (the 16\textsuperscript{th} century onwards), its distribution was generally the same as in present-day Swedish (see however \citealt{Hellquist1902}: 194–195 for some exceptions to this). The prepositional status of the infinitive marker was gradually lost during the Old Swedish period (see further \citealt{Kalm2016Satsekvivalenta}: 195–199). 


\subsection{Typological and diachronic perspectives on adverbial infinitives}\label{sec:kalm:3.2}

As mentioned earlier, the historical development of adverbial constructions has not been a major focus in Scandinavian linguistics. In part, this is probably due to the difficulties associated with defining the adverbial category as such; adverbial notions can be expressed in various ways (see \citealt[3]{Van_der_auwera1998}). From a typological point of view, it can be noted that adverbial clauses (finite and non-finite) are common in the languages of Europe, although there is substantial variation with regard to the exact syntactic expression (see \citealt{Hengenveld1998} for a comprehensive overview). Focusing on adverbial infinitives, we can note that many of the adverbial notions that are expressed using prepositional infinitives in Swedish correspond to prepositional participles in English (see the examples in \REF{ex:kalm:1} above and their respective translations into English). Prepositional infinitives can be used in, for instance, German and French to express some of these notions. See \REF{tab:kalm:1} for examples. 



\ea Examples of prepositional infinitives in French and German.\label{tab:kalm:1}
\ea Purposive 
\ea German\\
   \gll Sie   isst   viel,   um   Gewicht   zu   gewinnen.\\
        She  eats  a.lot   for   weight     \textsc{im}   win.\textsc{inf}\\
   \glt ‘She eats a lot in order to gain weight’\\
\ex French\\
    \gll Elle   mange   beaucoup pour   prendre   du   poids.\\
    she   eats   a.lot for   gain.\textsc{inf}   \textsc{prep.art}   weight \\
\glt ‘She eats a lot in order to gain weight.’\\
\z
\ex Abessive 
\ea German\\
\gll Er   ging,   ohne     etwas     zu   sagen.\\
     he   left   without   something   \textsc{im}   say.\textsc{inf}\\
\glt ‘He left without saying anything.’
\ex French\\
\gll Il   est   sorti   sans     rien     dire.\\
     he   is   left   without   nothing   say.\textsc{inf}\\
\glt ‘He left without saying anything.’
\z
\ex Substitutive
\ea German\\
\gll Er   blieb   vor     dem   Fernseher,   anstatt mit   seinen   Freunden   aus-zu-gehen.\\
     he   stayed   in.front.of   \textsc{art}   TV     instead.of with   his   friends   out-\textsc{im}{}-go.\textsc{inf}\\
\glt ‘He stayed in front of the TV instead of going out with his friends.’
\ex French\\
\gll Il   est   resté   devant     la   télé au   lieu   de   sortir     avec   ses   amis.\\
     he   is   stayed   in.front.of   \textsc{art}   TV in   stead   of   going.out.\textsc{inf}    with   his   friends\\
\glt ‘He stayed in front of the TV instead of going out with his friends.’
\z
\ex Temporal (anteriority)
\ea German\\
    –
\ex French\\
\gll Après   avoir     réfléchi,   elle   lui   répondit.\\
after   have.\textsc{inf}   reflected   she   him   answered\\
\glt ‘After having reflected, she answered him.’\\
\z
\z
\z


Note that the prepositional infinitives in French normally do not contain the equivalent of an infinitive marker (e.g. \textit{att} in Swedish). In German, however, \textit{zu} is used in a similar way to its Swedish counterpart. This is probably due to the fact that the infinitive markers of Romance have retained their prepositional status to a greater extent than in the Germanic languages. As is evident from the examples above, there is no perfect overlap between the languages, i.e. there are differences in the range of prepositional infinitives available (see \citealt{Hengenveld1998}). Unlike in Swedish, temporal posteriority may be expressed using a prepositional infinitive in French (see \ref{ex:kalm:5a} below). In Swedish, this would correspond to a finite clause. In some Romance languages, such as Spanish and Portuguese, there is also a concessive infinitive. This would also normally correspond to a finite clause in Swedish. Examples of Romance prepositional infinitives expressing temporal posteriority and concessive meaning are given in \REF{ex:kalm:5a} and \REF{ex:kalm:5b} respectively. 


\ea
\label{ex:kalm:5}
\ea French\label{ex:kalm:5a}\\ 
\gll Il faut réfléchir avant de parler.\\
\textsc{indf} must reflect.\textsc{inf} before \textsc{prep} speak.\textsc{inf}\\
\glt ‘One must reflect before speaking.’ (from \citealt{Grevisse1993}: 1298) 

\ex Spanish\label{ex:kalm:5b}\\ 
\gll Ambos procesos requieren de las mismas técnicas a pesar de ser totalmente diferentes.\\
both processes require \textsc{prep} \textsc{art} same techniques \textsc{prep} despite \textsc{prep} be.\textsc{inf} totally different\\
\glt ‘Both processes require the same techniques even though they are totally different.’ (from \citealt{Schulte2007What}: 533)
\z 
\z 



Conversely, the Swedish instrumental infinitive corresponds to other constructions in both Romance languages and German. In French, a contrastive study shows that \textit{genom} ‘through, by’\,+\,\textit{att}\,+\,infinitive (i.e. what I have called instrumental infinitive; see \REF{ex:kalm:1e} above) generally corresponds to the use of the \textit{gérondif} \citep[128]{Hellqvist2015}. It is thus quite evident that the system of using a prepositional infinitive to express certain adverbial notions is common in many languages, but the exact range of prepositional infinitives available varies quite a lot between different languages. If \citet{Holm1967} is correct in assuming that adverbial infinitives in Swedish developed on the basis of French and/or High German models, it would be the construction as such (i.e., the use of prepositional infinitives as adverbial adjuncts) that was replicated, rather than the individual components. 



Adverbial infinitives are not attested in Latin, but are innovations in the vernaculars. In his investigation of adverbial infinitives in the history of Spanish, Portuguese, and Romanian, \citet{Schulte2007Prepositional,Schulte2007What} shows that prepositional infinitives emerged independently in each of the three languages. In spite of this, the developments show striking similarities, as the relative chronological order in which the different prepositional infinitives appeared is identical. The order is shown in \REF{ex:kalm:6} below. 


\ea \label{ex:kalm:6}
purposive > abessive > temporal > substitutive > concessive
\z

The development of adverbial infinitives in Romance can be understood as a gradual process in which one prepositional construction is attested after the other. It should be noted that Romanian has not yet developed a concessive infinitive, in contrast with Portuguese and Spanish. \citet{Schulte2007Prepositional} argues that his results show that the emergence of adverbial infinitives is in line with general tendencies in how adverbial categories evolve in languages (cf. \citealt{Cristofaro2005}; see also \sectref{sec:kalm:5.1}). With regard to Swedish, it can be noted that all of the adverbial infinitives in \REF{ex:kalm:6} are attested, with the exception of concessive infinitives. At least, this is the impression given by the Swedish Academy grammar (\citealt{TelemanEtAl1999}/3: 589–591), in which no mention of concessive infinitives is made. Nevertheless, it is possible to find examples of what appear to be concessive infinitives, mainly in informal Swedish. The following examples are excerpted using the present-day Swedish corpora in the corpus infrastructure Korp \citep{BorinEtAl2012}. 


\ea
\label{ex:kalm:7}
\ea  \label{ex:kalm:7a}
\gll jag blir (sur) på nära och kära som trots att ha verkat vara okej med {instruktionen […]} kräver att jag ska vara anträffbar\\
I become sulky at close and dear that despite \textsc{im} have.\textsc{inf} seemed be.\textsc{inf} okay with instruction.\textsc{def} demand that I shall be available\\ 
\glt ‘I get (sulky) at my loved ones who even though they seem to be okay with the instruction […] demand that I be available’ (Blogg 2007)

\ex  \label{ex:kalm:7b}
\gll Men, det känns ändå kul att belönas med drygt 65\% trots att ha ridit en ritt jag inte var nöjd med.\\
but it feels still fun to reward.\textsc{inf.pass} with fully 65\% despite \textsc{im} have.\textsc{inf} ridden a ride I not was pleased with\\ 
\glt ‘But, it is still fun to be rewarded with slightly more than 65\% even though I had ridden a ride I was not satisfied with.’ (Blogg 2016)
\z 
\z 



To sum up, adverbial infinitives of different sorts exist in many European languages, including French and German. It has been suggested that the emergence of adverbial infinitives in Romance follows a general pattern of how adverbial categories evolve in languages. An interesting question is therefore whether this analysis can also be extended to Swedish. In the next section, the empirical investigation of adverbial infinitives in Swedish and Övdalian will be presented. 


\section{Adverbial infinitives in Swedish}\label{sec:kalm:4}
In this section, I present the results of the empirical investigation of adverbial infinitives in the history of Swedish and in the traditional dialect Övdalian. Since purposive infinitives have a somewhat separate history from the other adverbial infinitives, as they are attested even in Old Swedish, they are discussed separately, in \sectref{sec:kalm:4.1}.


\subsection{Purposive infinitives}\label{sec:kalm:4.1}
Purposive infinitives are common throughout the history of Swedish, but their form has varied substantially over time. In present-day Swedish, purposive infinitives are normally introduced with the preposition \textit{för} ‘for’. In specific contexts, where the purposive reading of the infinitive is implied, it is possible to omit the preposition. An example of such a context is given in \REF{ex:kalm:8a} below, where the matrix verb is \textit{skicka} ‘send’. In \REF{ex:kalm:8b}, the preposition is obligatory since it is not a “purposive” context (see \citealt{Kalm2016Satsekvivalenta}: 137–138). Note that both constructions are purposive, the difference being that the preposition \textit{för} is obligatory only in \REF{ex:kalm:8b}.


\ea Present-day Swedish\label{ex:kalm:8}
\ea \label{ex:kalm:8a}
\gll Vi skickade honom (för) att köpa mjölk.\\
we sent him \textsc{prep} \textsc{im} buy.\textsc{inf} milk\\
\glt ‘We sent him to buy milk.’

\ex  \label{ex:kalm:8b}
\gll Jag satte mig på andra sidan gången för att inte vara i vägen.\\
 I sat \textsc{refl} at other side.\textsc{def} aisle.\textsc{def} \textsc{prep} \textsc{im} not be.\textsc{inf} in way.\textsc{def}\\
\glt ‘I sat down on the other side of the aisle in order not to be in the way.’
\z 
\z 


In earlier stages of Swedish, the form of the purposive infinitive varied both diachronically and synchronically (depending on the semantic context of the infinitive clause). In the earliest Old Swedish texts (i.e. the provincial laws from the 13\textsuperscript{th} century), purposive infinitives were normally not prepositional (see \citealt{Kalm2016Satsekvivalenta}: 120). As mentioned in \sectref{sec:kalm:3.1} above, the infinitive marker \textit{at} spread from purposive to non-purposive contexts in the 13\textsuperscript{th} century. A possible explanation as to why purposive infinitives were not prepositional in the earliest or most archaic time might be that \textit{at} itself was considered a marker of the purposive reading and therefore a preposition was not needed. Given the etymological status of \textit{at} as a preposition, it is also possible that it maintained its prepositional status at the time, and that it could not be introduced by yet another preposition. 



Leaving the categorical status of the infinitive marker aside, the use of \textit{at} in purposive infinitives was very common in the earliest stages of Old Swedish. It was not completely obligatory, however, and it is possible to find examples of bare infinitives with a purposive reading.\footnote{It can be noted that in earlier stages of Germanic languages, the infinitival ending was a case suffix, representing the accusative. Since the accusative itself could denote a goal, neither the infinitival preposition (the infinitive marker) nor a preposition was necessary (see \citealt{Haspelmath1989,Jeffers1975}).} Interestingly, this use of the bare infinitive seems to be restricted to contexts in which the purposive reading of the infinitive clause is implied, typically in combination with verbs of motion, i.e. similar to the contexts in which the preposition \textit{för} ‘for’ can be omitted in present-day Swedish.\footnote{The bare purposive infinitive in combination with verbs of motion is also attested in Estonian Swedish (\citealt{Lagman1958}: 88–89; see also \citealt{Jorgensen1970}: 38).}  Below, \REF{ex:kalm:9a} is a typical example of a purposive \textit{at}-infinitive from the 13\textsuperscript{th} century, and \REF{ex:kalm:9b} shows a bare infinitive with a purposive reading. Note that the verb \textit{sænda} ‘send’ in \REF{ex:kalm:9b} is semantically identical to the present-day Swedish verb in \REF{ex:kalm:8a} above.\largerpage

\ea
\label{ex:kalm:9}
\ea \label{ex:kalm:9a}
\gll han ær i. sokn farin siukum. at hiælpæ.\\
 he is in parish gone sick \textsc{im} help.\textsc{inf}\\ 
\glt ‘he went to the city to help the sick’ (EWL, EOS)

\ex \label{ex:kalm:9b}
\gll han sænde sina dicipulos viþa vm væruldena pradica.\\ 
 he sent his disciples wide around world.\textsc{def} preach.\textsc{inf}\\
\glt ‘he sent his disciples around the world to preach’ (Leg, EOS)
\z 
\z 

In the Late Old Swedish period (LOS, ca. 1375–1526), it became increasingly common to explicitly mark the purposive semantics of the infinitive using the preposition \textit{til} ‘to’, possibly with inspiration from Low German (see also \sectref{sec:kalm:5.1}). The increasing use of prepositional purposive infinitives seems to correlate with the distributional expansion of \textit{at} through which the connection between \textit{at} and the purposive meaning is weakened. Two examples of \textit{til}{}-governed purposive \textit{at}{}-infinitives from the Late Old Swedish period are given in \REF{ex:kalm:10} below. 

\ea
\label{ex:kalm:10}
\ea  \label{ex:kalm:10a}
\gll Til mykla oglädhi war thu här komin aff franz til at göra nakan wanhedher konungx döttrom\\
for much unhappiness were you here come of France to \textsc{im} make.\textsc{inf} some dishonor king.\textsc{gen} daughters\\
\glt ‘to our great displeasure, you came here from France to dishonor the daughters of the king’ (KM, LOS)

\ex\label{ex:kalm:10b}
\gll jak nidherfoor til Colne til ath thär faa reliquias\\
 I down.traveled to Cologne to \textsc{im} there get.\textsc{inf} relics\\
\glt ‘I travelled to Cologne to get relics there’ (Måns, LOS, p. 332)
\z 
\z 


During the Early Modern period (ca. 1526 onward), there is considerable variation in the form of the purposive infinitive.\footnote{There is also substantial variation in the form of purposive infinitives in German, both synchronically and diachronically (see \citealt{Demske2011}).}  In addition to the Old Swedish patterns, some purposive infinitives are also introduced by \textit{til} ‘to’ alone, as in \REF{ex:kalm:11a} below, or by \textit{til} ‘to’ in combination with the preposition \textit{för} ‘for’, as in \REF{ex:kalm:11b}. This is a consequence of the grammaticalization of \textit{til} (\textit{at}) as an infinitive marker during the 16\textsuperscript{th} and 17\textsuperscript{th} centuries (see \cites[]{Kalm2014}[]{Kalm2016Prepositioner}[203–221]{Kalm2016Satsekvivalenta}). In \REF{ex:kalm:11c} there is an example of \textit{til at} as a complex infinitive marker in a purposive infinitive introduced by the preposition \textit{för} ‘for’. Not until the late 18\textsuperscript{th} century was the present-day system with a \textit{för}{}-introduced \textit{at}{}-infinitive established. An early example is given in \REF{ex:kalm:11d}. 


\ea
\label{ex:kalm:11}
\ea  \label{ex:kalm:11a}
\gll Men nu ha vi komme hit te gratulera\\
but now have we come here \textsc{prep}/\textsc{im} congratulate.\textsc{inf}\\
\glt ‘But now we have come to congratulate’ (2cKUSINE, 1791) 
\ex  \label{ex:kalm:11b}
\gll Ä de inte Ni, som kom te mej för te berätta hur Ni har hört…\\
is it not you that came to me for \textsc{im} tell.\textsc{inf} how you have heard\\
\glt ‘Is it not you, who came to me, to tell me how you had heard…’ (2cKUSINE, 1791)

\ex  \label{ex:kalm:11c}
\gll Häldre ger jag ut min plåt för sådant, än jag går på Operan och knuffas, för til at få se et par illa uphängda Gudar träta ur Dis dure med hwarandra.\\
rather give I out my ticket for such than I go to Opera.\textsc{def} and jostle for \textsc{im} \textsc{im} get.\textsc{inf} see.\textsc{inf} a pair poorly hanged gods argue from Dis dure with each.other\\
\glt ‘I would rather give away my ticket than go to the Opera and hustle in order to see a pair of poorly hanging gods arguing in D-sharp major with each other.’ (2aSTERBH, 1776)

\ex \label{ex:kalm:11d}
\gll Sedan Danske skeep flotten had liggat söder i Callmars sundh och blockuerat i 2 veckor så gick han åth Gottland för att forsorga sigh medh wedh och proviant.\\
since Danish ship fleet.\textsc{def} had layed south in Kalmar.\textsc{gen} channel and blocked in 2 weeks so went he to Gotland for \textsc{im} support.\textsc{inf} \textsc{refl} with firewood and provisions\\ 
\glt ‘Since the Danish fleet had been in the south, blocking the channel of Kalmar for two weeks, he went to Gotland to get supplies of firewood and provisions.’ (Bol, ca. 1697, p. 78)
\z 
\z 


\begin{sloppypar}
The form of the purposive infinitive has evidently varied substantially throughout the history of Swedish: it has been bare, and introduced by different infinitive markers (\textit{at}, \textit{til at}, \textit{til}) and prepositions (\textit{til} ‘to’\textit{, för} ‘for’). In \tabref{tab:kalm:2}, I give an overview of the proportions of the purposive constructions in the history of Swedish. 
\end{sloppypar}


\begin{table}
\caption{Distribution of purposive infinitives in the history of Swedish, based on data from \citet{Kalm2016Satsekvivalenta} and the corpus of Swedish drama dialogue ($N = 1230$).\label{tab:kalm:2}}
\begin{tabular}{lrrrrrr} 
\lsptoprule
& bare inf. & \textit{at}  & \textit{til at}  & \textit{til}  & \textit{för til at}  & \textit{för at} \\\midrule
1225–1300 & 22\% & 78\% & – & – & – & –\\
1301–1375 & 8\% & 85\% & 8\% & – & – & –\\
1376–1450 & 2\% & 84\% & 14\% & – & – & –\\
1451–1526 & 5\% & 70\% & 23\% & 2\% & – & –\\
1527–1600 & – & 25\% & 74\% & 1\% & – & –\\
1601–1675 & 3\% & 66\% & 30\% & 1\% & – & 1\%\\
1725–1750 & – & 43\% & 55\% & – & – & 2\%\\
1775–1800 & – & 15\% & 1\% & 8\% & 2\% & 74\%\\
1825–1850 & – & 6\% & – & – & – & 94\%\\
1875–1900 & – & 2\% & – & – & – & 98\%\\
1925–1950 & – & 1\% & – & – & – & 99\%\\
1975–2000 & – & 2\% & – & – & – & 98\%\\
\lspbottomrule
\end{tabular}
\end{table}


The data in \tabref{tab:kalm:2} show that the purposive infinitive most commonly takes the form of a non-prepositional \textit{at}{}-infinitive in Early Old Swedish (i.e. 1225–1300 and 1301–1375). There are also examples of bare infinitives and, from the 14\textsuperscript{th} century onwards, \textit{til}{}-introduced \textit{at-}infinitives. The latter became more common over time, and in the 16\textsuperscript{th} century, this was the dominant pattern.\footnote{This is partly due to the fact that \textit{til at} was grammaticalized into a complex infinitive marker during the 16\textsuperscript{th} century. The use of \textit{til at} was not restricted to purposive contexts, but used more generally by certain writers in the 16\textsuperscript{th}–18\textsuperscript{th} centuries. See \textcites[]{Kalm2016Prepositioner}[203–221]{Kalm2016Satsekvivalenta} for further details.} The variation in form increased during the 17\textsuperscript{th} and early 18\textsuperscript{th} centuries, but decreased during the second half of the 18\textsuperscript{th} century, with the rather sudden shift from \textit{til} ‘to’ to \textit{för} ‘for’ as the general purposive preposition. In the early 18\textsuperscript{th} century, \textit{för} ‘for’ was quite rare in purposive contexts, while it dominated the purposive infinitives in the latter part of the century. It is not quite clear what motivation there was for this shift in purposive preposition.\footnote{As pointed out by an anonymous reviewer, one possibility could be influence from the purposive \textit{pour} ‘for’\,+\,infinitive construction in French.}  Other constructions (such as \textit{för te}\,+\,infinitive or \textit{för til at}\,+\,infinitive) must be considered quite marginal in comparison with these more frequent patterns. From the 19\textsuperscript{th} century onwards, only \textit{för}{}-introduced \textit{at}{}-infinitives and bare \textit{at}{}-infinitives are attested with purposive meaning. 


\subsection{Other adverbial infinitives}\label{sec:kalm:4.2}

Adverbial infinitives are prepositional in present-day Swedish, with the above-mentioned exceptions of certain purposive infinitives (see example \REF{ex:kalm:8} above), and the infinitive marker \textit{att} is obligatory. Prepositional infinitives were already common in Old Swedish, but the prepositions were then part of the lexical construction of a governing verb, noun, or adjective, i.e. they did not introduce adverbial adjuncts. The earliest examples of adverbial (adjunct) infinitives appear in texts from the second half of the 17\textsuperscript{th} century. First attested are abessive infinitives in a text from 1657, and substitutive infinitives in a text from 1675; see \REF{ex:kalm:12} below. 


\ea
\label{ex:kalm:12}
\ea Abessive\\\label{ex:kalm:12a}
\gll[Då] kom wijd Minans springande en fransos flygandess och ståendes på föttren       mitt ibland Officerarne utan at wara på något sätt skadder.\\
then came at mine.\textsc{def.gen} exploding a Frenchman flying and standing at foot.\textsc{pl.def} midst among without \textsc{im} be.\textsc{inf} at any way hurt \\ 
\glt ‘[Then] came when the mine exploded a Frenchman flying and standing on his feet in the midst of the officers, without being in any way hurt.’ (Rålamb, 1657)

\ex Substitutive\\\label{ex:kalm:12b}
\gll i stället för at fruckta dem, utbrast han i desse ord: Ju flere Fijender ju meer ähra.\\
 in place for to fear.\textsc{inf} them exclaimed he in these words the more enemies the more honor\\
\glt ‘instead of fearing them, he exclaimed these words: The more enemies, the more glory.’ (Mål-roo, 1675)
\z 
\z 


Both abessive and substitutive infinitives seem to have been used only sporadically during the late 17\textsuperscript{th} century, but they rapidly become common in texts from the 18\textsuperscript{th} century, especially the abessive infinitives (see \citealt[129–130]{Kalm2016Satsekvivalenta}). The first examples of temporal and instrumental infinitives date from much later. The earliest instance of a temporal infinitive is from 1779, and the first instrumental infinitive appears in 1829; see \REF{ex:kalm:13} below.  

\ea
\label{ex:kalm:13}
\ea Temporal\label{ex:kalm:13a}\\
\gll Då jag efter att ha likafullt gått {ut […]} åter tillbaka kom holt jag före att man borde i så critiqva omständigheter ej våga dröja längre utan strax gjöra Revolten.\\
as I after \textsc{im} have.\textsc{inf} nevertheless gone.\textsc{sup} out again back came held I for that one should in such critical circumstances not dare wait longer but soon make revolt.\textsc{def}\\
\glt ‘As I, after nevertheless having gone out, came back, I meant that one should, in such critical conditions, not dare to wait any longer but immediately begin the revolt.’ (Ehrensvärd, 1779, p. 6)

\ex Instrumental\label{ex:kalm:13b}\\
\gll {[…] konsten} segrar icke genom att trotsa, utan genom att följa naturen\\
art.\textsc{def} prevails not through \textsc{im} defy.\textsc{inf} but through \textsc{im} follow.\textsc{inf} nature.\textsc{def}\\
\glt ‘Art will not prevail by defying, but by following nature’ (von Unge, 1829)
\z 
\z 


In a similar way to the situation in Romance (\citealt{Schulte2007Prepositional,Schulte2007What}), the development of adverbial infinitives in Swedish thus seems to have been a gradual process. Over time, more adverbial notions came to be expressed with prepositional infinitives. In \REF{ex:kalm:14} below, I have summarized the order of first appearance for the adverbial infinitives investigated in this article. A comparison with the corresponding process in Romance (see \REF{ex:kalm:6} above) shows that the processes are similar, but not identical. As is the case in Romance, purposive infinitives are the first to be attested, followed by abessive infinitives. While substitutive infinitives are attested earlier than temporal infinitives in Swedish, it is the opposite order in Romance. The instrumental infinitive has no equivalent in Romance, and there are no examples of concessive infinitives in the Swedish corpora investigated for this study.\footnote{See, however, the examples from present-day Swedish in \REF{ex:kalm:7} above.}     

\ea  \label{ex:kalm:14}
\gll purposive >  abessive  > substitutive > temporal > instrumental\\
(1225) {} (1657) {} (1675) {} (1779) {} (1829)\\
\z 



In conclusion, we have seen that it became possible to express certain adverbial notions with prepositional infinitives from the second half of the 17\textsuperscript{th} century onwards. The emergence of such adverbial infinitives seems to have been a gradual process in the sense that abessive and substitutive infinitives are attested about a hundred years before the first example of temporal infinitives. Instrumental infinitives are attested during the first half of the 19\textsuperscript{th} century. The process shows similarities with the grammaticalization of adverbial infinitives in Romance (see \citealt{Schulte2007Prepositional,Schulte2007What}). 


\subsection{The case of Övdalian}\label{sec:kalm:4.3}


As mentioned above, \citet[27]{Holm1967} claims that adverbial (prepositional) infinitives are not attested in traditional dialects. If this is correct, it means that other linguistic resources need to be used to express these adverbial notions. As already mentioned, Holm does not present any data to support his claim, and it can also be difficult to find the relevant data. When comparing standardized languages with each other, it can be fruitful to work contrastively with translations between the two languages. This method allows us to find specific constructions in a language and see how they are translated into another language. Dialects and other non-standard varieties are normally spoken, and rarely written down, and the comparison between the variety and the standard language is therefore more difficult to conduct, at least when you are interested in the usage of a specific construction. From a Swedish point of view, there is one important exception to this, namely Övdalian. 

Övdalian (Sw. \textit{älvdalska}) is traditionally considered a Swedish dialect, but in recent times there have been attempts to get it acknowledged as a minority language in Sweden because of its distance, linguistically speaking, from standard Swedish. Övdalian is spoken in the northern parts of the province of Dalarna by around 2,500 people \citep[27]{Garbacz2009}. The variety is characterized both by many archaic traits (such as the persistence of a partial case system and subject-verb inflection) and by innovations (for example, secondary diphthongs). It has received considerable attention from linguists in the last decades, with the result that it is fairly well described (see for instance \citealt{Garbacz2009,BentzenEtAl2015}). There is also an interest among the speakers of Övdalian in preserving the variety.\footnote{There is, for example, an association for the preservation of Övdalian, Ulum Dalska, founded in 1985.} As a consequence, Övdalian has been codified in dictionaries and grammars in recent years, unlike most other Scandinavian dialects or non-standard varieties. An orthographic norm is in the process of being established, and there are at the moment a fair number of texts available in Övdalian, both original works and translations. 

It is possible that Övdalian is going through a process of both standardization and Verschriftlichung at the moment, but given its relative isolation historically, it has not been influenced by other languages, and not even by standard Swedish to any great extent, it seems. At present, however, it is likely that Övdalian is being affected by Swedish, not least since all of its speakers are bilingual. 

For this investigation, I have chosen to use two Övdalian translations from Swedish: the novel \textit{Hunden/Rattsjin} ‘The dog’ and the Gospel of John (in Sw. Johannesevangeliet, Joh.; in Övd. Juanneswaundsjilą, Jua.). I have excerpted all of the adverbial infinitives in the Swedish texts and then compared them with their respective translations. The aim is to establish whether Övdalian uses prepositional infinitives or other linguistic resources to express the notions that are expressed with prepositional infinitives in Swedish. My hope is that this contrastive study will give us clues as to how adverbial notions might be expressed in varieties that have not gone through a process of Verschriftlichung or language Ausbau, including earlier stages of Swedish. That being said, it is important to remember that the Övdalian texts might be affected by the Swedish originals. 

There are in total 79 adverbial infinitives in the two Swedish texts, evenly distributed between them. An overview of the Swedish data is given in \tabref{tab:kalm:3}. 

\begin{table}
\caption{Adverbial infinitives in the Swedish texts.\label{tab:kalm:3}}
\begin{tabular}{lrrr} 
\lsptoprule
& \textit{Hunden} & \textit{Johannesevangeliet} & Total\\\midrule
Purposive & 28 & 35 & 63\\
Abessive & 10 & 1 & 11\\
Instrumental & 1 & 3 & 4\\
Temporal & 0 & 1 & 1\\
Total & 39 & 40 & 79\\
\lspbottomrule
\end{tabular}
\end{table}


The purposive infinitives are by far the most common. There are only 16 examples of other adverbial infinitives in the two texts; substitutive infinitives are not attested. All of the purposive infinitives are introduced by the preposition \textit{för} ‘for’ in the Swedish texts. These are translated using several different constructions in Övdalian, including prepositional infinitives. An overview is given in \tabref{tab:kalm:4}. 



\begin{table}
\caption{Övdalian translations of purposive infinitives.\label{tab:kalm:4}}
\begin{tabular}{lrrr} 
\lsptoprule
& \textit{Rattsjin} & \textit{Juanneswaundsjilą} & Total\\\midrule
\textit{fer te}{}-inf. (‘for to’) & 7 & 19 & 26\\
\textit{og} coordination (‘and’) & 11 & 12 & 23\\
\textit{so} clause (‘so’) & 5 & 2 & 7\\
\textit{fer} clause (‘for’) & 5 & 0 & 5\\
bare inf. & 0 & 1 & 1\\
other\footnote{In this one instance, the translation is far from the Swedish original for some reason, and it is difficult to say what part of the translation would correspond to the purposive infinitive.} & 0 & 1 & 1\\
\lspbottomrule
\end{tabular}
\end{table}

In the reference grammar of Övdalian (\citealt{AkerbergNystrom2012}), the authors note that the purposive \textit{för att}{}-infinitive (‘for to’) is somewhat difficult to translate to Övdalian. Apparently, Övdalian would prefer to use a construction with the conjunction \textit{og} ‘and’\,+\,infinitive instead (\citealt{AkerbergNystrom2012}: 539). However, in the translations I have investigated, prepositional (purposive) infinitives are common (26 instances; see \tabref{tab:kalm:4}), and they are used much like in the Swedish original. Note that the Övdalian infinitive marker is \textit{te} (see \sectref{sec:kalm:4.1} above). In the purposive infinitives in \REF{ex:kalm:15} and \REF{ex:kalm:16} below, both Swedish and Övdalian have infinitives introduced with the preposition \textit{för/fer} ‘for’ and the infinitive marker (\textit{att/te}).


\ea
\label{ex:kalm:15}
\ea Present-day Swedish\\\label{ex:kalm:15a}
\gll Nu fick han lyfta benen högt för att komma fram i snösörjan.\\
now got he lift.\textsc{inf} leg.\textsc{pl.def} high for \textsc{im} forward in slush.\textsc{def} come.\textsc{inf}\\ 
\glt ‘He now had to raise his leg up high in order to make his way through the slush.’ (Hunden, p. 26)

\ex Övdalian\\\label{ex:kalm:15b}
\gll Nu wart an lypt fuätum og fer te tågå sig framm i wåtsn\k{i}uäm.\\
now was he lift feet and for \textsc{im} come.\textsc{inf} \textsc{refl} forward in slush.\textsc{def}\\
\glt ‘He now had to raise his leg up high in order to make his way through the slush.’ (Rattsjin, p. 26)
\z 
\ex
\label{ex:kalm:16}
\ea Present-day Swedish\\\label{ex:kalm:16a}
\gll Detta sade de för att sätta honom på prov och få något att anklaga honom för.\\
this said they for \textsc{im} put.\textsc{inf} him at test and get something \textsc{im} accuse.\textsc{inf} him for\\
\glt ‘They were using this question as a trap,{~}in order to have a basis for accusing him.’ (Joh. 8:6)

\ex Övdalian\\\label{ex:kalm:16b}
\gll Edar sagd dier bar fer tä frest an og fer tä fǫ nod tä klågå ǫ an fer.\\
this said they just for \textsc{im} tempt.\textsc{inf} him and for \textsc{im} get.\textsc{inf} something \textsc{im} complain.\textsc{inf} at him for\\
\glt ‘They were using this question as a trap,{~}in order to have a basis for accusing him.’ (Jua. 8:6)
\z 
\z 

\textit{Og}-introduced structures are also common, but they are far from always being non-finite. Rather, \textit{og} commonly introduces a finite clause from which the subject has been omitted, as in example \REF{ex:kalm:17} below: \textit{slätsjer} ‘lick’ is the singular form of the present tense of the verb. Instead of a purposive reading, the construction emphasizes the simultaneity of the two verb actions. In \REF{ex:kalm:18}, however, the prepositional infinitive in Swedish is translated using \textit{og}\,+\,the infinitive: \textit{dsjärå} ‘do, make, create’ is an unambiguous infinitive. Because of the Övdalian tendency to drop vocalic endings (see \citealt{AkerbergNystrom2012}: 532), it is sometimes difficult to establish whether the verb is finite or non-finite. An example of this can be found in \REF{ex:kalm:19}. There is no way of knowing whether it is the infinitival \textit{a}{}-ending or the past tense third person plural \textit{e}{}-ending that is omitted from the form \textit{myöt} ‘meet’. 


\ea
\label{ex:kalm:17}
\ea Present-day Swedish\label{ex:kalm:17a}\\
\gll Ibland stannar han för att slicka på skaren.\\
sometimes stops he for \textsc{im} lick.\textsc{inf} at snow.crust.\textsc{def}\\
\glt ‘Sometimes he stops to lick the snow crust.’ (Hunden, p. 13)

\ex Övdalian\label{ex:kalm:17b}\\ 
\gll Millumað stanner an og slätsjer skårån liteð.\\
Sometimes stop he and lick.\textsc{pres} snow.crust.\textsc{def} little\\
\glt ‘Sometimes he stops to lick the snow crust a little bit.’ (Rattsjin, p. 13)
\z 
\ex
\label{ex:kalm:18}
\ea Present-day Swedish\label{ex:kalm:18a}\\
\gll Skulle jag annars säga att jag går bort för att bereda plats för er?\\
would I otherwise say.\textsc{inf} that I go away for \textsc{im} prepare.\textsc{inf} place for you\\
\glt ‘If that were not so, would I have told you that I am going there{~}to prepare a place for you?’ (Joh. 14:2)

\ex Övdalian\label{ex:kalm:18b}\\
\gll Edd ig sagt ellest, ig far dait og dsjärå ruom ad id?\\
had I said otherwise I go there and make.\textsc{inf} room at you\\
 \glt ‘If that were not so, would I have told you that I am going there{~}to prepare a place for you?’ (Jua. 14:2)
\z 
\ex
\label{ex:kalm:19}
\ea Present-day Swedish\label{ex:kalm:19a}\\
\gll När folk hörde att han hade gjort detta tecken drog de ut för att möta honom.\\
when people heard that he had made this sign went they out for \textsc{im} meet.\textsc{inf} him\\
\glt ‘Many people, because they had heard that he had performed this sign,{~}went out to meet him.’ (Joh. 12:18)

\ex Övdalian\label{ex:kalm:19b}\\
\gll So mes fuotsjed fingg är, an add gart ed teckned, fuor dier aut og myöt onum.\\
so when people got hear he had done that sign went they out and meet.\textsc{inf/pres} him\\
\glt ‘Many people, because they had heard that he had performed this sign,{~}went out to meet him.’ (Jua. 12:18)
\z 
\z 


There are also twelve instances in which Övdalian has a subordinate clause instead of a prepositional infinitive; ten of these examples are found in Rattsjin. According to \citet[491]{AkerbergNystrom2012}, both \textit{fer} and \textit{so} are purposive subjunctions (but \textit{so} can also be used to introduce resultative clauses). Two examples are given in \REF{ex:kalm:20} and \REF{ex:kalm:21}.  

\ea
\label{ex:kalm:20}
\ea Present-day Swedish\label{ex:kalm:20a}\\
\gll En gång hade han plumsat i för att försöka nå dem.\\
one time had he splashed in for \textsc{im} try.\textsc{inf} reach them\\ 
\glt ‘Once, he had splashed in in an attempt to reach them.’ (Hunden, p. 51)

\ex Övdalian\label{ex:kalm:20b}\\
\gll Iesn add an pulsað åv auti fer an willd biuäð til kum að diem.\\
once had he splashed off into for he wanted try to come at them\\
\glt ‘Once, he had splashed in in an attempt to reach them.’ (Rattsjin, p. 51)
\z 
\ex
\label{ex:kalm:21}
\ea Present-day Swedish\label{ex:kalm:21a}\\ 
\gll Själva stannade de utanför, för att inte bli orena\\
self stayed they outside for \textsc{im} not become.\textsc{inf} unclean\\
\glt ‘To avoid ceremonial uncleanness they did not enter the palace’ (Joh. 18:28)

\ex Övdalian\label{ex:kalm:21b}\\
\gll og siuover dsjingg dier it in i hlotted, so dier uld it werd \k{u}oriener.\\
and self went they not in in palace.\textsc{def} so they would not become unclean\\
\glt ‘and to avoid ceremonial uncleanness they did not enter the palace,{~}because they wanted to be able to eat the Passover.’ (Jua. 18:28)
\z 
\z\largerpage[2]


Finally, there is one example of a bare infinitive with a purposive meaning, reproduced as \REF{ex:kalm:22} below. Interestingly, the bare infinitive appears in what I referred to above as a purposive context, i.e. in combination with a verb phrase expressing motion. In these contexts, we also find bare infinitives in Old Swedish.\footnote{This also seems to be the case in Estonian Swedish (\citealt{Lagman1958}: 88–89).} 


\ea
\label{ex:kalm:22}
\ea Present-day Swedish\label{ex:kalm:22a}\\
\gll När Maria hörde det, steg hon strax upp och gick för att möta honom.\\
when Mary heard that went she soon up and went for \textsc{im} meet.\textsc{inf} him\\
\glt ‘When Mary heard this, she got up quickly and went to meet him.’ (Joh. 11:29)

\ex Övdalian\label{ex:kalm:22b}\\ 
\gll Snjäst Mari fick är ed, raitt ǫ upp sig og fuor stad myöt onum.\\
when Mary got hear that raised she up \textsc{refl} and went along meet.\textsc{inf} him\\
\glt ‘When Mary heard this, she got up quickly and went to meet him.’ (Jua. 11:29)
\z 
\z 


Abessive infinitives are relatively common in the Swedish texts, especially in the novel \textit{Hunden}. They are generally not translated with a prepositional construction in Övdalian. Instead of using embedding, as in Swedish, the Övdalian translators have normally chosen to use a coordinate negated structure instead, as in (\ref{ex:kalm:23}--\ref{ex:kalm:24}) below. On one occasion, the translator used a finite subordinate structure instead; see \REF{ex:kalm:25} below.\largerpage[-1]


\ea
\label{ex:kalm:23}
\ea Present-day Swedish\label{ex:kalm:23a}\\
\gll Han följde dem alltid med blicken men låg kvar utan att känna upphetsning.\\
he followed them always with eye.\textsc{def} but stayed put without \textsc{im} feel.\textsc{inf} excitement\\
\glt ‘He followed them always with his eye, but stayed put without feeling excitement.’ (Hunden, p. 52)\\

\ex Övdalian\label{ex:kalm:23b}\\ 
\gll An fygd ǫ ðiem min ogum olltiett og låg kwer og wart it ekster.\\
he followed at them with eyes always and stayed put and become.\textsc{pst} not excited\\
\glt ‘He followed them always with his eye, but stayed put without feeling excitement.’ (Rattsjin, p. 52)
\z 
\ex
\label{ex:kalm:24}
\ea Present-day Swedish\label{ex:kalm:24a}\\
\gll Han som låg under vindfället lyssnade utan att förstå.\\
he who laid under windfall.\textsc{def} listened without \textsc{im} understand.\textsc{inf}\\
\glt ‘He who was under the windfall listened without understanding.’ (Hunden, p. 82)\\
\ex Övdalian\label{ex:kalm:24b}\\ 
\gll An so låg under windfellę lydd men bigript it noð.\\
he who laid under windfall.\textsc{def} listened but understand.\textsc{pst} not anything\\
\glt ‘He who was under the windfall listened without understanding.’ (Rattsjin, p. 82)
\z
\ex
\label{ex:kalm:25}
\ea Present-day Swedish\label{ex:kalm:25a}\\
\gll Hunden gick i skogskanten, långa sträckor utan att synas alls.\\
dog.\textsc{def} went in forest.edge.\textsc{def} long distances without \textsc{im} see.\textsc{inf.pass} at.all\\
\glt ‘The dog wandered by the edge of the forest, long distances without revealing himself at all’ (Hunden, p. 109)\\
\ex Övdalian\label{ex:kalm:25b}\\ 
\gll Rattsjin dsjikk laungg strettsjur i skuägkantem, so int an syndes noð.\\
dog.\textsc{def} went long distances in forest.edge.\textsc{def} so not he see.\textsc{pst.pass} at.all\\
\glt ‘The dog wandered by the edge of the forest, long distances without revealing himself at all.’ (Rattsjin, p. 109)
\z 
\z 


There are two examples (one from each text) in which the translators use a prepositional infinitive just like in the Swedish original. This may be the result of a slavish translation from Swedish. The example from the Gospel of John is given in \REF{ex:kalm:26} below.\pagebreak


\ea
\label{ex:kalm:26}
\ea Present-day Swedish\label{ex:kalm:26a}\\
\gll Ty den som Gud har sänt talar Guds ord; Gud ger Anden utan att mäta.\\
for it who God has sent speaks God.\textsc{gen} word God gives spirit.\textsc{def} without \textsc{im} measure.\\
\glt ‘For the one whom God has sent speaks the words of God, for God gives the Spirit without limit.’ (Joh. 3:34)\\

\ex Övdalian\label{ex:kalm:26b}\\
\gll Fer an so Gud ar stsjickad, an glemer Gudes uord, fer Gud dsjäv Andan autǫ tä mela.\\
for he who God has sent he speaks God.\textsc{gen} word for God gives spirit.\textsc{def} without \textsc{im} measure.\textsc{inf}\\
\glt ‘For the one whom God has sent speaks the words of God, for God gives the Spirit without limit.’ (Jua. 3:34)
\z
\z 


We can therefore conclude that abessive meanings are normally not expressed through prepositional infinitives in Övdalian, but by other means. The most frequent pattern used in the texts investigated is a coordinated finite structure containing a negation, instead of a non-finite structure as in the Swedish original. Even though this is clearly an attempt to convey the same meaning in the translation, it must be noted that the translation has a slightly different meaning than the Swedish original. 



Let us turn next to the four instrumental infinitives found in the Swedish texts. Three of them have been translated using a finite subordinate structure, as in (\ref{ex:kalm:27}–\ref{ex:kalm:28}). In \REF{ex:kalm:27}, the Swedish instrumental infinitive has been translated with a temporal subordinate clause introduced by the subordinator \textit{dar}, and in \REF{ex:kalm:28} the infinitive is translated with a \textit{bar}-introduced (‘only’, ‘just’) subordinate clause. In \REF{ex:kalm:29}, the infinitive has been translated with a relative clause instead. These examples suggest that instrumental infinitives are not used in Övdalian and that there might not be any specialized means to express instrumental meaning. 


\ea
\label{ex:kalm:27}
\ea Present-day Swedish\label{ex:kalm:27a}\\
\gll Mycket folk följde efter, därför att de såg de tecken han gjorde genom att bota de sjuka.\\
 much people followed after because that they saw the signs he did through \textsc{im} cure.\textsc{inf} the sick\\
\glt ‘And a large crowd was following him, because they saw the signs he performed on the sick.’ (Joh. 6:2)\\

\ex Övdalian\label{ex:kalm:27b}\\
\gll Ed war mitsjid fuok so fygde, fer dier såg tecknę an garde, dar an buoted diem so war kliener.\\
 it was much people who followed for they say signs.\textsc{def} he did when he cured those who were sick\\
\glt ‘And a large crowd was following him, because they saw the signs that he was doing by curing the sick.’ (Jua. 6:2)
\z
\ex
\label{ex:kalm:28}
\ea Present-day Swedish\label{ex:kalm:28a}\\
\gll Det hade hänt mellan några valpar och en grovröstad karl som kunde bli hälften så stor genom att ställa sig på knä.\\
 it had happened between some puppies and a gruff.voiced man who could become half so big through \textsc{im} stand.\textsc{inf} \textsc{refl} on knee\\
\glt ‘It had happened between a couple of puppies and a man with a gruff voice who could halve his size by kneeling.’ (Hunden, p. 86)\\

\ex Övdalian\label{ex:kalm:28b}\\ 
\gll Eð add ennt millǫ nogum wepum og ienum gruävröstaðum kalle so wart elptn so stur bar an stelld sig ǫ kninę.\\
it had happened between some puppies and a gruff.voiced man who became half so big only he stood \textsc{refl} on knees\\
\glt ‘It had happened between a couple of puppies and a man with a gruff voice who could halve his size by kneeling.’ (Rattsjin, p. 86)
\z 
\ex
\label{ex:kalm:29}
\ea Present-day Swedish \label{ex:kalm:29a}\\
\gll Men dessa har upptecknats för att ni skall tro att Jesus är Messias, Guds son, och för att ni genom att tro skall ha liv i hans namn.\\
but these have record.\textsc{sup.pass} for that you shall believe that Jesus is Messiah God.\textsc{gen} son and for that you through \textsc{im} believe.\textsc{inf} shall have life in his name \\
\glt ‘But these are written so that you may believe that Jesus is the Christ, the Son of God, and that by believing you may have life in his name.’ (Joh. 20:31)

\ex Övdalian\label{ex:kalm:29b}\\ 
\gll Men ittad ar uort skrievt, so ulid truo att Iesus ir Messias, Gudes Sun, og so ulid åvå laived i namnę onumes, dar truoid.\\
 but this has become written so you believe that Jesus is Messiah God.\textsc{gen} son and so you have life.\textsc{def} in name his who believes\\
\glt ‘But these are written so that you may believe that Jesus is the Christ, the Son of God, and that by believing you may have life in his name.’ (Jua. 20:31)
\z
\z 

Finally, the only example of a temporal infinitive has been translated with a temporal subordinate clause: 


\ea
\label{ex:kalm:30}
\ea Present-day Swedish\label{ex:kalm:30a}\\ 
\gll Efter att ha sagt detta fortsatte han: (…)\\
 after \textsc{im} have.\textsc{inf} said this continued he\\
\glt ‘After having said this, he continued: (…)’ (Joh. 11:11)

\ex Övdalian\label{ex:kalm:30b}\\
\gll Mes an add sagt edar, fuortsett an: (…)\\
 as he have.\textsc{pst} said this continued he\\
\glt ‘As he had said this, he continued: (…)’ (Jua. 11:11)
\z
\z 


To sum up, the contrastive study shows that Swedish adverbial infinitives are normally translated using other constructions in Övdalian. Purposive infinitives, by far the most common construction in the Swedish originals, are translated in several different ways. Often, a prepositional infinitive is used, much like in Swedish, but another common strategy is to use constructions with the coordinator \textit{og} instead. To a lesser degree, finite subordinate clauses are used, and in one instance a bare infinitive (in combination with a verb phrase expressing motion). Abessive infinitives are most often translated with a finite coordinate clause containing a negation. This means that the translators are trying to convey the abessive meaning using another linguistic construction (coordination instead of embedding). In the case of instrumental infinitives, the translation uses temporal or other subordinate clauses, which leads to a slightly different (and less specific) meaning. The only example of a temporal infinitive has been translated using a temporal finite clause in Övdalian. 

The results partly support \citegen[27]{Holm1967} claim that prepositional infinitives are not found in traditional dialects. In any case, they are not as frequent as in Swedish, and not all types occur. Since adverbial infinitives are not attested in Old Swedish, it is possible that similar (or the same) constructions to those that we find in Övdalian might have been used in earlier stages of Swedish to express the various adverbial notions. 


\subsection{Summary of empirical findings}\label{sec:kalm:4.4}


This investigation of adverbial infinitives in the history of Swedish takes purposive infinitives as its point of departure since they are the only type attested as early as Old Swedish. In the earliest stages of Swedish, purposive infinitives are not introduced by a preposition, like their present-day counterparts. Over time there is an increasing tendency for purposive infinitives to be prepositional, and by the second half of the 18\textsuperscript{th} century, almost all of them are. The connection between a specific form and a specific meaning is thus strengthened over time. Furthermore, there is a rather sudden change of the purposive preposition from \textit{til} ‘to’ to \textit{för} ‘for’ during the second half of the 18\textsuperscript{th} century. The reasons for this shift remain unclear. 



Other adverbial infinitives are attested from the second half of the 17\textsuperscript{th} century onwards in a gradual process whereby more adverbial notions are expressed with prepositional infinitives over time. Abessive infinitives are attested first in a text from 1657, and slightly later (in a text from 1675) we find the first example of a substitutive infinitive. Temporal infinitives are not attested until the late 18\textsuperscript{th} century (first example from 1779), and the first instance of an instrumental infinitive is found in a text from 1829. This order of appearance is similar (although not identical) to the corresponding emergence of adverbial infinitives in Romance. 



In addition to the historical survey of adverbial infinitives, I have presented a contrastive investigation of Övdalian translations of Swedish texts containing adverbial infinitives. It has been claimed that traditional dialects, such as Övdalian, do not use prepositional infinitives as adverbial adjuncts (see \citealt{Holm1967}: 27). This investigation gives some support to these claims, since the adverbial infinitives in the Swedish originals are often translated with other constructions in the Övdalian versions. This is especially evident in the case of abessive infinitives, as the translators use a coordinated negated clause (with a few exceptions). It is possible that Övdalian can give us clues as to how these adverbial notions were expressed in earlier stages of Swedish, i.e. before the emergence of adverbial infinitives. 


\section{Discussion}\label{sec:kalm:5}

In this section, the empirical results from the investigation are discussed. I first discuss the role of language contact with regard to adverbial infinitives in Swedish. Thereafter, I turn to the question of whether adverbial infinitives should be seen as part of a process of specialization and elaboration of the written language (i.e. language Ausbau). 


\subsection{The role of language contact}\label{sec:kalm:5.1}

Whereas purposive infinitives are already attested in Early Old Swedish, other adverbial infinitives did not appear until the Early and Late Modern Swedish period. Purposive infinitives were normally introduced by the infinitive marker \textit{at} in Early Old Swedish, and since \textit{at} etymologically is a preposition (see \sectref{sec:kalm:3.1} above), it could be argued that the (purposive) \textit{at}{}-infinitive should be regarded as a prepositional phrase at this time (see \citealt{Kalm2016Satsekvivalenta}). Non-purposive infinitives were often bare (i.e. not introduced by \textit{at}) in Early Old Swedish, but later on, the use of the \textit{at}{}-infinitive spread at the expense of the bare infinitive. By the Late Old Swedish period, \textit{at} thus seems to have lost its purposive semantics and its prepositional status, and the \textit{at}{}-infinitive was used in purposive as well as non-purposive contexts. As a consequence of the lost relationship between the \textit{at}{}-infinitive and the purposive meaning, it became increasingly common for purposive \textit{at}{}-infinitives to be introduced by the preposition \textit{til}, possibly as a way of explicitly marking the purposive reading of the infinitive. There is, however, also a general tendency (especially during the Early Modern Swedish period) for infinitive clauses to be introduced by \textit{til}, and it has been argued that \textit{til at} was grammaticalized as a complex infinitive marker during this time (see \citealt{Kalm2014,Kalm2016Satsekvivalenta}). It is not unlikely that the promotion of \textit{til} during this time was facilitated by Low German influence in much the same way as \citet{Nesse2002} suggests that the infinitive marker \textit{te} of the Norwegian Bergen dialect emerged as a result of language contact with Low German. The Low German \textit{te} had a double function as both infinitive marker and preposition (just like in present-day German and English).

The first signs of the modern system of prepositional adverbial infinitives appeared by the end of the Early Modern Swedish period. By this time, the Low German influence on Swedish had already come to an end and it is thus not likely that the appearance of the first adverbial infinitives was due to contact with Low German. Instead, \citet{Holm1967} suggests that the constructions emerged as a result of language contact with High German and/or French. Of the two languages, at least High German seems like a possible candidate for such an influence on Swedish, since we know of other syntactic constructions that arose due to influence from High German (see \sectref{sec:kalm:2} above). (As noted, the French influence on Swedish that we know of today is lexical rather than syntactic.) If the emergence of adverbial infinitives in Swedish was due to language contact, this could be understood as a case of contact-induced (or replica) grammaticalization (see \citealt{HeineKuteva2003, HeineKuteva2005}) in much the same way as \citet{Nesse2002} accounts for the appearance of the infinitive marker \textit{te} in the Norwegian Bergen dialect. This would mean that the (foreign) system of using prepositional infinitives to express adverbial notions was replicated in Swedish. The fact that the instrumental infinitive is specifically Swedish (i.e. not attested in High German) is not an argument against this analysis, since it is the initiation of the grammaticalization process that is contact-dependent, not the entire process. It is thus expected that you would find language-specific development even in cases of contact-induced grammaticalization. 



Even though it cannot be ruled out that the emergence of adverbial infinitives in Swedish should be understood as a contact phenomenon, it is equally possible that they emerged independently in Swedish in much the same way as has been argued for the Romance languages (\citealt{Schulte2007Prepositional, Schulte2007What}). \citet{Schulte2007Prepositional, Schulte2007What} shows that new adverbial infinitives appeared in the same relative order in Spanish, Portuguese, and Romanian. The relative order is very similar (albeit not identical) to the corresponding development in Swedish. Compare the Romance cline in \REF{ex:kalm:6} and the Swedish in \REF{ex:kalm:14}, repeated as \REF{ex:kalm:31} and \REF{ex:kalm:32} below. 

\ea
\label{ex:kalm:31}
purposive > abessive > temporal > substitutive > concessive
\ex
\label{ex:kalm:32}
purposive > abessive > substitutive > temporal > instrumental
\z

In Romance, as in Swedish, purposive infinitives are attested first, followed by abessive infinitives. Thereafter, the order differs with regard to temporal and substitutive infinitives, and while Swedish developed an instrumental infinitive, Romance has a concessive infinitive. Interestingly, there are examples of concessive infinitives in (informal) present-day Swedish as well (see the examples in \REF{ex:kalm:7} above). Since it is not likely that the appearance of concessive infinitives in present-day Swedish is due to contact with Romance languages, it rather seems like the development follows some general pattern of how adverbial constructions evolve in a language. This is also what \citet{Schulte2007Prepositional, Schulte2007What} suggests with regard to Romance, and he relates the order to \citegen{Cristofaro2005} typological hierarchy of “deranked” (roughly: non-finite) adverbial clauses. According to Cristoforo, there is a general tendency for deranked constructions to be associated first with purposive meaning (see also \citealt{Haspelmath1989}) and then later with other semantics. The overall result of the study is that there are cross-linguistic patterns as to how adverbial (non-finite) clauses evolve in languages, and \citet{Schulte2007Prepositional, Schulte2007What} shows that Romance languages fit into this description. This paper suggests that the development of adverbial infinitives in Swedish has followed the same general pattern, starting with purposive infinitives and ending with concessive infinitives in a similar way to the Romance languages.



To sum up, it is possible that language contact has affected the emergence of adverbial infinitives in Swedish, but it cannot be ruled out that we are dealing with a “natural” evolution of adverbial categories. It is likely that the use of \textit{til} in combination with purposive (and other) infinitival clauses is due to contact with Low German, but when it comes to other adverbial infinitives, it is less clear that they should be understood as contact phenomena. Since the two explanations are not mutually exclusive, it is possible that the grammaticalization was initiated by language contact, but that the continuation of the process was independent, following general patterns of how adverbial non-finite constructions evolve. In the next section, I will discuss the adverbial infinitives in relation to the ongoing specialization of the written language during the Late Old and Early Modern Swedish period. 


\subsection{Adverbial infinitives as an Ausbau phenomenon}\label{sec:kalm:5.2}


Simultaneously with the expansion into new domains, a written language normally undergoes formal elaborations of different kinds, visible for instance in syntactic complexity (see \cites[38--39]{Fischer2007}[]{Kloss1967}). During the Middle Ages, when the bulk of the text production in Swedish consisted of translations from other languages, the elaboration of Old Swedish into a literary language was conducted under more or less direct influence from the source language, predominantly Latin (see \citealt{Wollin1981,Wollin1983,Hoder2009, Hoder2010}). This caused a split between the spoken and written varieties of Old Swedish, since the spoken language was not affected by Latin. In the emerging cities, it was instead influences from Low German that shaped (the spoken and written) Swedish (see \sectref{sec:kalm:2} and \sectref{sec:kalm:5.1}). The increasing use of participle constructions in Late Old Swedish, for example, is considered to be the result of such a Latin influence \citep{Ahlberg1942,Hoder2010}. Some of these participle constructions seem to be used to express adverbial notions that today could be expressed using prepositional infinitives. In \REF{ex:kalm:33} below, the negated participle has an abessive meaning and would correspond to \textit{utan} ‘without’\,+\,\textit{att}\,+\,infinitive in present-day Swedish. (Note that English uses a construction similar to the one found in Old Swedish.)


\ea 
\label{ex:kalm:33}
 \gll far {iak …} til iherusalem eyg vitande hwat mik skal ouir koma\\ 
travel I to Jerusalem not know.\textsc{ptcp} what me shall over come.\textsc{inf}\\
 \glt ‘I travel … to Jerusalem not knowing what shall come over me’ (ApG, ca. 1385, p. 164; from \citealt{Ahlberg1942}: 163)
\z


Although these Latin-inspired constructions were exclusively part of a written register, i.e. they were not used in the spoken language of the time, many of the participle constructions survived in the 16\textsuperscript{th} century Reformation Bible translations and remained part of a religious register for centuries \citep[17–19]{Stahle1970}. For the most part, they have not survived into present-day Swedish (neither spoken nor written). 

After the Reformation, text production in Swedish increased rapidly, but not predominantly in religious contexts, as opposed to the situation in the Middle Ages. Texts were also original works in Swedish to a much greater extent than before, i.e. not translations or paraphrases. This means that the elaboration of Swedish during this period was somewhat less directly influenced by other languages than was the case during the late Middle Ages, when the first steps towards a more homogeneous, standardized Swedish were taken. As we have seen, a new abessive construction appeared during the late 17\textsuperscript{th} century. Instead of participles like the one in \REF{ex:kalm:33}, prepositional infinitives were used to express the same adverbial notion, as in \REF{ex:kalm:34} below. The fact that specifically abessive constructions appeared during periods (and in contexts) in which the written language was being elaborated might suggest that they are part of an ongoing process of Verschriftlichung. While the first of these processes (during the late Middle Ages) was interrupted by the Reformation, the second process has continued. 


\ea
\label{ex:kalm:34}
 \gll Skulle det wäl wara möjeligit, at jag fått en stiufmor, utan at weta deraf.\\
should it well be.\textsc{inf} possible that I get.\textsc{sup} a stepmother without \textsc{im} know.\textsc{inf} thereof\\
\glt ‘Would it be possible that I had got a stepmother without knowing anything of it?’ (1dSMUL, 1738)
\z


The emergence of constructions such as the one in \REF{ex:kalm:33} above is an example of how the demands of new genres and text types lead to a specialization of the language. In this case, it is quite obvious that the construction in question has emerged with inspiration from Latin, but as we have seen in \sectref{sec:kalm:5.1}, this is not necessarily the case with adverbial infinitives. In spite of this, both the Latin-inspired constructions and the adverbial infinitives can be understood as part of the same process of formal elaboration and specialization that a language undergoes when it is being used in new functional contexts. An important piece of evidence supporting the assumption that the development of adverbial infinitives is part of a written-language specific development is the fact that prepositional adverbial infinitives (with the exception of purposive infinitives) do not seem to be used in the traditional, spoken dialect of Övdalian. Since the linguistic distance between Swedish and Övdalian is substantial, and has been for a long time, Övdalian has not been influenced by written Swedish in the same way as other regional and local varieties of spoken Swedish have been. Its isolated location also means that contact with other languages has been minimal. Until recently, the dialect has not been used in writing to any large extent, which means that it has not been through processes of Verschriftlichung or language Ausbau. The empirical investigation showed that the Övdalian translators very often used other constructions to convey the meaning expressed by adverbial infinitives in the Swedish originals. Many of the purposive infinitives were expressed with finite clauses, both coordinated and subordinated. The abessive infinitives, on the other hand, were almost consistently translated with a negated coordinated structure, like the one in \REF{ex:kalm:24} above, here repeated as \REF{ex:kalm:35}.


\ea
\label{ex:kalm:35}
\ea Present-day Swedish \\
\gll Han som låg under vindfället lyssnade utan att förstå.\\
 he who laid under windfall.\textsc{def} listened without \textsc{im} understand.\textsc{inf}\\
\glt ‘He who was under the windfall listened without understanding.’ (Hunden, p. 82)

\ex Övdalian\\
\gll An so låg under windfellę lydd men bigript it noð.\\
he who laid under windfall.\textsc{def} listened but understand.\textsc{pst} not anything\\
\glt ‘He who was under the windfall listened without understanding.’ (Rattsjin, p. 82)
\z
\z


Even though both constructions convey roughly the same meaning, the Swedish prepositional infinitive can be understood as a grammaticalized construction expressing abessive meaning, while the coordinated structure in Övdalian is not exclusively used with this meaning. Prepositional infinitives thus offer a more precise way of expressing adverbial notions, which in earlier stages of the language might have been expressed as in Övdalian. This can be understood as a kind of grammaticalization where a specific meaning (in this case abessive) is connected to a specific construction (in this case a prepositional infinitive). When it comes to the abessive meaning, there thus exists a more or less one-to-one relationship between form and function in Swedish (in the sense that the construction \textit{utan} ‘without’\,+\,\textit{att}-infinitive always has abessive meaning), but this is not the case in Övdalian. This becomes even clearer when we compare instrumental infinitives in Swedish with their Övdalian translations. In these instances, it rather seems like Övdalian lacks the ability to express this precise meaning. Instead, a construction with a temporal meaning is used, as in example \REF{ex:kalm:27b} above. 



There are thus reasons to assume that adverbial constructions (for example expressing abessive meaning) are part of a process of elaboration and specialization of the written language when it is being used in new domains. Abessive constructions emerged in the late Middle Ages, then clearly under the influence of Latin, as well as in the Early Modern period. Both periods are characterized by an increasing use of the written language. The comparison with Övdalian shows that specialized adverbial constructions are not used in the same way as in present-day Swedish. Since the variety has remained mostly spoken (at least until recently), this supports the assumption that adverbial infinitives emerged as part of a written register.  


\section{Summary and conclusions}\label{sec:kalm:6}


This article traces the development of prepositional infinitives functioning as adverbial adjuncts throughout the history of Swedish, and investigates how present-day Swedish adverbial infinitives are translated into the traditional dialect of Övdalian. The results show that the emergence of adverbial infinitives is a gradual process in many ways similar to the corresponding development in Romance languages. While purposive infinitives were already attested in Old Swedish, other adverbial infinitives emerged during the post-Reformation era, starting with abessive and substitutive infinitives in the 17\textsuperscript{th} century, followed by temporal and instrumental infinitives in the late 18\textsuperscript{th} and early 19\textsuperscript{th} centuries, respectively. This was a dynamic period in the history of Swedish, characterized by a substantial functional expansion of the written language and the subsequent formal elaborations of the written form (i.e. language Ausbau). The contrastive investigation of Övdalian translations of Swedish texts suggests that adverbial infinitives are not used in the dialect, with the exception of purposive infinitives. Instead, other strategies are used in order to express the same adverbial notions. As opposed to Standard Swedish, Övdalian has not existed as a literary (written) language until recently. I have taken the absence of adverbial infinitives in Övdalian as an indication that they originally were part of a written rather than spoken register. I take the emergence of adverbial infinitives in Swedish to be part of a process of Verschriftlichung and language Ausbau. 



The article also explores the possibility that adverbial infinitives are the result of language contact with High German and French, as suggested by \citet{Holm1967}. Since we know of other instances of syntactic influence from High German on Swedish, but of mainly lexical influence from French, High German stands out as the more probable candidate in this regard. Moreover, it is quite clear that some of the Old Swedish participle constructions which are used in much the same way as present-day Swedish adverbial infinitives emerged under the influence of Latin. It is, however, not evident that the same is true for the later emergence of infinitival constructions. Adverbial infinitives are common in European languages, and even though it cannot be ruled out that the Swedish constructions are (partly) due to language contact, there is also the possibility that they emerged independently as a result of the elaboration and specialization of the written language during the Early Modern period. The empirical findings of this investigation show that the relative order in which the adverbial infinitives appeared appears to convey a more general pattern of how adverbial constructions tend to evolve cross-linguistically (especially with regard to the initial and final stages of the process). Even if the emergence of the first adverbial infinitives is the result of language contact (in that case an example of replica grammaticalization), the subsequent grammaticalization of other adverbial infinitives could still be independent. In other words, the possibility of language contact does not exclude that the subsequent grammaticalization process was independent and in line with cross-linguistic patterns. 


\section*{Abbreviations}
\begin{tabbing}
EOS\hspace{1ex}\= Early Old Swedish\kill
EOS \> Early Old Swedish\\
IM  \> infinitive marker\\
LOS \> Late Old Swedish\\
\end{tabbing}


\section*{Quoted texts}
\begin{description}[font=\normalfont]\sloppy
\item[1dSMUL:] Modée, Reinhold Gustaf (b. 1698). \textit{Håkan Smulgråt} [Håkan Cheapskate]. Stockholm, 1739. See  \citet{MarttalaStromquist2001}. Available through LB.
\item[2aSTERBH:] Kexél, Olof (b. 1748). \textit{Sterbhus-kammereraren Mulpus eller Caffe-huset i Stora Kyrkobrinken} [The chief accountant of the estate Mulpus or the coffee house in the main church hill]. Stockholm, 1776. See \citet{MarttalaStromquist2001}. Available through LB.
\item[2cKUSINE:] Envallsson, Carl (b. 1756). \textit{Kusinerna, eller: Fruntimmers-sqvallret} [The cousins or the gossip of the women]. Stockholm, 1807. See \citet{MarttalaStromquist2001}. Available through LB.
\item[ApG:] Apostla Gerningar [Acts of the Apostles]. 1385. In \textit{Klosterläsning}. Edited by G. E. Klemming. (Samlingar utgivna av Svenska fornskriftssällskapet 22.) Stockholm, 1877–1878. Available through FTB\slash Korp.
\item[Blogg:] Bloggmix [A selection of Swedish blogs]. 1998–2017. Available through Korp.
\item[Bol:] Bolinus, Andreas (b. 1642). \textit{En dagbok från 1600-talet} [A diary from the 17\textsuperscript{th} century]. Edited by E. Brunnström. Stockholm, 1913. 
\item[Leg:] Stephens, George (ed.). 1847. \textit{Svenska medeltidens kloster- och helgonabok} […] \textit{Ett forn-svenskt legendarium} […] [The Swedish medieval book of monasteries and saints … An Old Swedish collection of legends]. Stockholm: Norstedts. Originally written some time between 1276 and 1307. Earliest manuscript (Codex Bureanus) from ca. 1350. Available through FTB\slash Korp.
\item[Ehrensvärd:] \textit{C. A. Ehrensvärds brev} [The letters of C. A. Ehrensvärd]. Auth. b. 1745. Edited by Gunhild Bergh. Stockholm, 1916.
\item[Hunden:] Ekman, Kerstin. 1986. \textit{Hunden} [The dog]. Stockholm: Bonnier. 
\item[Joh.:] \textit{Johannesevangeliet} [The gospel of John]. In \textit{Bibel 2000}. Gothenburg: Bokförlaget Cordia AB, 1999. 
\item[Jua.:] \textit{Juanneswaundsjila: Johannesevangeliet på älvdalska} [The gospel of John in Övdalian]. Knivsta: Lars Steensland, 1989.
\item[KM:] Karl Magnus saga [The story of Karl Magnus]. End of 14\textsuperscript{th} century. In \textit{Prosadikter från Sveriges medeltid}. Edited by G. E. Klemming. (Samlingar utgivna av Svenska fornskriftsällskapet 28.) Stockholm, 1887–1889. Available through FTB\slash Korp.
\item[Mål-roo:] Columbus, Samuel (b. 1642). \textit{Mål-Roo eller roo-mål} [Language amusement or amusing language]. Ca. 1675. Edited by Bengt Hesselman. (Nordiska texter och undersökningar 6.) Stockholm: Hugo Geebers förlag, 1935. Available through FTB\slash Korp.
\item[Måns:] \textit{Peder Månssons brev på svenska från Rom till Vadstena kloster 1508–1519} [Peder Månsson’s letters in Swedish from Rome to Vadstena monastery 1508–1519]. Auth. b. ca. 1460. Edited by Robert Geete. (Småstycken på forn-svenska. Andra serien.) Stockholm, 1915.
\item[Rattsjin:] Ekman, Kerstin. 2000. \textit{Rattsjin} [The dog]. Älvdalen: Juts böcker. 
\item[Rålamb:] Callmer, Christian (ed.). 1963. \textit{Diarium under resa till Konstantinopel 1657–1658} [`Diary during a journey to Constantinople 1657–1658’, undertaken by Claes Rålamb (b. 1622)]. (Historiska handlingar 37:3.) Stockholm: Norstedts.
\item[UL:] Schlyter, Carl Johan (ed.). 1834. \emph{Samling af Sweriges gamla lagar. Tredje bandet.} \textit{Uplands-Lagen} [Collection of the old laws of Sweden. Volume three. The law of Uppland]. Stockholm: Norstedts. Originally written in 1296. Available through FTB\slash Korp.
\item[von Unge:] von Unge, Otto Sebastian (b. 1797). \textit{Vandring genom Dalarne, jemte Författarens Resa söderut} [Walk through Dalecarlia, as well as the author’s journey to the south]. Stockholm: Z. Hæggström, 1829. Available through LB.
\item[EWL:] Collin, Hans Samuel \& Carl Johan Schlyter (eds.). 1827. \emph{Samling af Sweriges gamla lagar. Första bandet. Westgötalagen} [Collection of the old laws of Sweden. Volume one. The Westrogothic law]. Stockholm: Z. Haeggström. Originally written in the 1220s. Available through FTB\slash Korp.
\item[ÖgL:] Collin, Hans Samuel \& Carl Johan Schlyter (eds.). 1830. \emph{Samling af Sweriges gamla lagar. Andra bandet.} \textit{Östgöta-Lagen} [Collection of the old laws of Sweden. Volume two. The Ostrogothic law]. Stockholm: Norstedts. Originally written in the 1280s. Available through FTB\slash Korp.
\end{description}


\section*{Electronic corpora}
\begin{description}[font=\normalfont]
\item[FTB:] Fornsvenska textbanken [The text bank of Old Swedish]:  \url{https://project2.sol.lu.se/fornsvenska} 
\item[Korp:] \url{https://spraakbanken.gu.se/korp/?mode=all_hist}
\item[LB:] The Swedish literature bank: \url{http://www.litteraturbanken.se}
\end{description}

{\sloppy\printbibliography[heading=subbibliography,notkeyword=this]}
\end{document}
